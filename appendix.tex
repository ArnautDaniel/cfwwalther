
\makeatletter
\patchcmd{\@addmarginpar}%
    {\box \@marbox}%
    {\hbox{%
        \ifmpar@rule@rside
        \hskip-\mparrulefactor\marginparsep\mparrule
        \hskip\mparrulefactor\marginparsep
        \fi
                          \box \@marbox
        \ifmpar@rule@lside
        \hskip\mparrulefactor\marginparsep\mparrule
        \fi}%
     \global\mpar@rule@lsidefalse
     \global\mpar@rule@rsidefalse
    }%
    {\typeout{*** SUCCESS ***}}{\typeout{*** FAIL ***}}

\patchcmd{\@addmarginpar}%
    {\global\setbox\@marbox\box\@currbox}%
    {\global\setbox\@marbox\box\@currbox
     \global\mpar@rule@lsidetrue
     \else
     \global\mpar@rule@rsidetrue
    }%
    {\typeout{*** SUCCESS ***}}{\typeout{*** FAIL ***}}

\newif\ifmpar@rule@lside
\newif\ifmpar@rule@rside
\makeatother

\usepackage{color}

% \marginparrule generates the \vrule but should use no space horizontally
% using color is just for fun ...
\newcommand\mparrule{\textcolor{orange}
    {\hskip-2pt\vrule width 4pt\hskip-2pt}}

% placement factor: .5 places the rule midway in the space made available 
% by \marginparsep
\newcommand\mparrulefactor{.4}

\def\divider{\par
  \vskip 1em
  \centerline{\hbox to 0.5\hsize{\hrulefill}}
  \vskip 1em
}

\usepackage{tikz}
\usetikzlibrary{backgrounds}
\makeatletter

\tikzset{%
  fancy quotes/.style={
    text width=\fq@width pt,
    align=justify,
    inner sep=1em,
    anchor=north west,
    minimum width=\linewidth,
  },
  fancy quotes width/.initial={.8\linewidth},
  fancy quotes marks/.style={
    scale=8,
    text=white,
    inner sep=0pt,
  },
  fancy quotes opening/.style={
    fancy quotes marks,
  },
  fancy quotes closing/.style={
    fancy quotes marks,
  },
  fancy quotes background/.style={
    show background rectangle,
    inner frame xsep=0pt,
    background rectangle/.style={
      fill=gray!25,
      rounded corners,
    },
  }
}

\newtcolorbox{fancyquotes}{%
    enhanced jigsaw, 
    breakable,      % allow page breaks
    frame hidden,   % hide the default frame
    top=.5cm,
    left=1.25cm,       % left margin
    right=1.25cm,      % right margin
    overlay  unbroken={%
        \node [scale=6,
            text=black,
            inner sep=0pt,] at ([xshift=.75cm,yshift=-1.35cm]frame.north west){``}; 
        \node [scale=6,
            text=black,
            inner sep=0pt,] at ([xshift=-.8cm,yshift=.2cm]frame.south east){''};  
    },
    % if you wish to have the look different for page-broken boxes, adjust the following
    overlay first={%
    \node [scale=6,
            text=black,
            inner sep=0pt,] at ([xshift=.75cm,yshift=-1.35cm]frame.north west){``}; 
    },
    overlay middle={},
    overlay last={%
        \node [scale=6,
            text=black,
            inner sep=0pt,] at ([xshift=-.8cm,yshift=.2cm]frame.south east){''};  
    },
    % paragraph skips obeyed within tcolorbox
    parbox=false,
}



\let\clipbox\relax % PSTricks (used by PSVectorian) already defines a \clipbox, so we need this workaround
\usepackage{adjustbox} % Adjustbox to rescale the ornaments (scalebox breaks titlesec for some reason...)

\newcommand{\otherfancydraw}{% Defining a command to shorten things
\begin{adjustbox}{max height=0.9\baselineskip}% Rescaling to have height of 0.5\baselineskip
  \raisebox{-0.25\baselineskip}{
  \rotatebox[origin=c]{0}{% And rotating 90 degrees
                             \includegraphics{lutherose.png}% Ornament n° 26 (http://melusine.eu.org/syracuse/pstricks/vectorian/psvectorian.pdf)
  }}%
\end{adjustbox}%
}

% A command to create a rule centered vertically on the text (from: https://tex.stackexchange.com/questions/15119/draw-horizontal-line-left-and-right-of-some-text-a-single-line/15122#15122)
\newcommand*\ruleline[1]{\par\noindent\raisebox{.8ex}{\makebox[\linewidth]{\hrulefill\hspace{1ex}\raisebox{-.8ex}{#1}\hspace{1ex}\hrulefill}}}

\titleformat% Formatting the header
  {\chapter} % command
  [block] % shape - Only managed to get it working with block
  {\normalfont\bfseries\sc\huge} % format - Change here as needed
  {\centering Chapter \thechapter\\} % The Chapter N° label
  {0pt} % sep
    {\centering \ruleline{\otherfancydraw}\\ % The horizontal rule
  \centering #1} % And the actual title
