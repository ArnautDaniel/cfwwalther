\documentclass[
     12pt,                     % font size
     a4paper,                  % paper format
     BCOR=5mm,                % binding correction
     DIV=12,                   % stripe size for margin calculation            %
     openright,
     twoside,
     headsepline
]{book}

\usepackage{marginnote}
\usepackage{csquotes}
\usepackage[usenames, dvipsnames]{xcolor}
\usepackage{fontspec}
\usepackage{multicol}
\usepackage{dashrule}
\usepackage{comment}
\usepackage[sc,compact,explicit]{titlesec}
\usepackage{auto-pst-pdf}
\usepackage{psvectorian}
\usepackage{afterpage}
\usepackage{microtype}
\usepackage{cabin}
\usepackage{suffix}
\usepackage{tabulary}
\usepackage{changepage}
\usepackage[many]{tcolorbox}
\usepackage[absolute,overlay]{textpos}
\usepackage{textcomp}
\usepackage[papersize={8.5in,11in},lmargin=2cm,rmargin=5cm,marginparwidth=3.5cm,top=2cm,bottom=2cm,marginparsep=2em]{geometry} 
\usepackage{layouts}
\usepackage{etoolbox}
\usepackage{libertineotf}

\makeatletter
\patchcmd{\@addmarginpar}%
    {\box \@marbox}%
    {\hbox{%
        \ifmpar@rule@rside
        \hskip-\mparrulefactor\marginparsep\mparrule
        \hskip\mparrulefactor\marginparsep
        \fi
                          \box \@marbox
        \ifmpar@rule@lside
        \hskip\mparrulefactor\marginparsep\mparrule
        \fi}%
     \global\mpar@rule@lsidefalse
     \global\mpar@rule@rsidefalse
    }%
    {\typeout{*** SUCCESS ***}}{\typeout{*** FAIL ***}}

\patchcmd{\@addmarginpar}%
    {\global\setbox\@marbox\box\@currbox}%
    {\global\setbox\@marbox\box\@currbox
     \global\mpar@rule@lsidetrue
     \else
     \global\mpar@rule@rsidetrue
    }%
    {\typeout{*** SUCCESS ***}}{\typeout{*** FAIL ***}}

\newif\ifmpar@rule@lside
\newif\ifmpar@rule@rside
\makeatother

\usepackage{color}

% \marginparrule generates the \vrule but should use no space horizontally
% using color is just for fun ...
\newcommand\mparrule{\textcolor{orange}
    {\hskip-2pt\vrule width 4pt\hskip-2pt}}

% placement factor: .5 places the rule midway in the space made available 
% by \marginparsep
\newcommand\mparrulefactor{.4}

\def\divider{\par
  \vskip 1em
  \centerline{\hbox to 0.5\hsize{\hrulefill}}
  \vskip 1em
}

\usepackage{tikz}
\usetikzlibrary{backgrounds}
\makeatletter

\tikzset{%
  fancy quotes/.style={
    text width=\fq@width pt,
    align=justify,
    inner sep=1em,
    anchor=north west,
    minimum width=\linewidth,
  },
  fancy quotes width/.initial={.8\linewidth},
  fancy quotes marks/.style={
    scale=8,
    text=white,
    inner sep=0pt,
  },
  fancy quotes opening/.style={
    fancy quotes marks,
  },
  fancy quotes closing/.style={
    fancy quotes marks,
  },
  fancy quotes background/.style={
    show background rectangle,
    inner frame xsep=0pt,
    background rectangle/.style={
      fill=gray!25,
      rounded corners,
    },
  }
}

\newtcolorbox{fancyquotes}{%
    enhanced jigsaw, 
    breakable,      % allow page breaks
    frame hidden,   % hide the default frame
    top=.5cm,
    left=1.25cm,       % left margin
    right=1.25cm,      % right margin
    overlay  unbroken={%
        \node [scale=6,
            text=black,
            inner sep=0pt,] at ([xshift=.75cm,yshift=-1.35cm]frame.north west){``}; 
        \node [scale=6,
            text=black,
            inner sep=0pt,] at ([xshift=-.8cm,yshift=.2cm]frame.south east){''};  
    },
    % if you wish to have the look different for page-broken boxes, adjust the following
    overlay first={%
    \node [scale=6,
            text=black,
            inner sep=0pt,] at ([xshift=.75cm,yshift=-1.35cm]frame.north west){``}; 
    },
    overlay middle={},
    overlay last={%
        \node [scale=6,
            text=black,
            inner sep=0pt,] at ([xshift=-.8cm,yshift=.2cm]frame.south east){''};  
    },
    % paragraph skips obeyed within tcolorbox
    parbox=false,
}



\let\clipbox\relax % PSTricks (used by PSVectorian) already defines a \clipbox, so we need this workaround
\usepackage{adjustbox} % Adjustbox to rescale the ornaments (scalebox breaks titlesec for some reason...)

\newcommand{\otherfancydraw}{% Defining a command to shorten things
\begin{adjustbox}{max height=0.9\baselineskip}% Rescaling to have height of 0.5\baselineskip
  \raisebox{-0.25\baselineskip}{
  \rotatebox[origin=c]{0}{% And rotating 90 degrees
                             \includegraphics{lutherose.png}% Ornament n° 26 (http://melusine.eu.org/syracuse/pstricks/vectorian/psvectorian.pdf)
  }}%
\end{adjustbox}%
}

% A command to create a rule centered vertically on the text (from: https://tex.stackexchange.com/questions/15119/draw-horizontal-line-left-and-right-of-some-text-a-single-line/15122#15122)
\newcommand*\ruleline[1]{\par\noindent\raisebox{.8ex}{\makebox[\linewidth]{\hrulefill\hspace{1ex}\raisebox{-.8ex}{#1}\hspace{1ex}\hrulefill}}}

\titleformat% Formatting the header
  {\chapter} % command
  [block] % shape - Only managed to get it working with block
  {\normalfont\bfseries\sc\huge} % format - Change here as needed
  {\centering Chapter \thechapter\\} % The Chapter N° label
  {0pt} % sep
    {\centering \ruleline{\otherfancydraw}\\ % The horizontal rule
  \centering #1} % And the actual title


\def\blankpage{%
      \clearpage%
      \thispagestyle{empty}%
      \addtocounter{page}{-1}%
      \null%
      \clearpage}
    
\begin{document}


\pagenumbering{gobble}

\begin{titlepage}

    \begin{center}
    \vspace*{\baselineskip}
    \rule{\textwidth}{1.6pt}\vspace*{-\baselineskip}\vspace*{2pt}
    \rule{\textwidth}{0.4pt}\\[\baselineskip]
    {\LARGE Dr. C.F.W. Walther \\[0.3\baselineskip] as Theologian}\\[0.2\baselineskip]
    \rule{\textwidth}{0.4pt}\vspace*{-\baselineskip}\vspace{3.2pt}
    \rule{\textwidth}{1.6pt}\\[\baselineskip]
    \scshape
    A modern rendition of Pieper's masterful work \\
    \rule{\textwidth}{0.4pt}\vspace*{-\baselineskip}\vspace{3.2pt}
    \\[\baselineskip]
    Translation by: Prof. Wallace H. McLaughlin \\
    Written by: Dr. Franz Pieper \\
    Typeset by: J. Lucas \\
    [\baselineskip]
    
    \vfill

    {\footnotesize \copyright  Jack Lucas (Idees Fait) --  All rights reserved -- 2018\\ISBN 9780359219247}
    \end{center}
  \end{titlepage}
  \frontmatter
\tableofcontents

  \chapter*{Typesetter's Preface}
\addcontentsline{toc}{chapter}{Preface}

There are a few small notes for how to read this book most effectively.  I have designed it foremost for the layman reader to parse through,  which means I have removed any parenthetical insertions of Hebrew words {\footnotesize  (of which amounts to about 6 occurrences)}.  I have made slight changes to the text without removing words.  For instance on page 8 the description of the \textsc{Meditatio} contains four quotes.  In the original text ``\textit{according to Luther}'' and ``\textit{adds Walther}'' were used as connectives but I have pushed those statements to the margins because the entire quotation is easier to read that way.  \par  I have kept any changes extremely minimal like this in order to stay true to the original translation in my efforts to modernize it's format.  Other simple rules I would like to state for your benefit include:

\begin{itemize}
\item Quotations range from fancy gray boxes to simple in-line italics.  In general,  almost all quotations are Walther's.  If a quotation is not Walther's I have \textbf{bolded} the last name of the person giving the quote,  unless the text makes it easily apparent who is about to speak.  This is to {\footnotesize (hopefully)}  keep things minimal yet functional as this book is extremely quote heavy,  especially when it comes to Walther.  There are also several times where pseudo-quotes are used to show a point or set up a fictional discussion.  In general these will look: {\footnotesize like this line.}
\item Verses citations are used quite heavily in this book and unless expressly required by the text they will either be in the footnotes or sometimes in the margins depending on context.\footnote{John 3:16}  \marginpar{\scriptsize John 3:16} {\scriptsize\textsc{John 3:16}}.  Focus is once again placed on helping you read this like a normal book as opposed to an ``\textit{academic text}''  so word-flow is given priority.
\item I have found it most helpful as an avid reader to be able to remember something I've read by remembering the page it was from and what the page looked like.  Not everyone experiences this,  but in the interest of experimentation I have tried to make each page look unique to help you remember {\footnotesize (or mentally index)} it.  If anything else,  I at least hope it will keep things lively.
\item As the original book is reaching 150+ years old the standard vocabulary in use at the time somewhat differs today.  In this case I have attempted to place definitions in the margins to save you the time of having to go look it up {\footnotesize (which would interrupt your focus)} and in a few cases I have placed a short description of a person or biblical concept to add to the context.
\end{itemize}

And with that you should be well on your way.  This book was generated by \LaTeX\  and will hopefully find it's way into a reprinting someday {\footnotesize (after some more work)} since it was mostly designed to look good for print.\divider
I pray this book blesses you as much as it did me.  Enjoy! \par\hfill \textbf{J.  Lucas}
\mainmatter
\include{parts/part1}
\chapter{How to be a Theologian}
\hrule
\vspace{.30cm}
We have seen that Walther understood by theology the sufficiency to lead sinners to salvation by means of the Word of God. How then, is this sufficiency obtained, or: How does one become a theologian?
\vspace{.10cm}
\hrule
\vspace{1.25cm}
Walther answered this question repeatedly in his writings. And each time he had to answer it for the theological students in the class-room, he spent considerable time upon it.  Theology for Walther is a wisdom from above. And this not only in the sense that the theologian derives everything that he teaches only and alone from the divine revelation, but also specifically that the competence to know the divine revelation, to impart it, and thereby to lead men to salvation, is one wrought only by the Holy Spirit.  Just as no man can discover the material with which theology deals by way of speculation, so also no man can induce in himself the competence rightly to treat and to evaluate this material through human power and art, as, for instance, by following a specific "\textit{scientific method}."

The theological habitude, Walther says, ``\textit{is a supernatural one, not to be attained by human power and diligence}." \begin{displayquote}``\textit{There are certain natural gifts which serve the holy office: keen judgment, eloquence, etc. But these do not belong to the specific gifts of office which make a minister of the Church. St. Paul enumerates the latter in {\scriptsize\textsc{1 Corinthians 12}} and in {\scriptsize\textsc{Romans 12}}: Wisdom, knowledge, faith, discerning of spirits, prophecy, teaching, exhortation, ruling, etc.}" \end{displayquote}The Holy Ghost, who revealed the divine Truth in the Scripture, must Himself through this Truth create for Himself also the instruments who shall know it and communicate and apply it to others unto salvation. ``\textit{Only the Holy Ghost makes D.D.s\footnote{Doctors of Theology}}" remarks Walther in commenting on Luther's dictum\footnote{To the Christian Nobility}, as to how Doctors of the Holy Scripture come into existence in distinction from ``\textit{Doctors of Arts, of medicine, of Laws, of the Sentences}," etc.

Hence Walther also declares that in Luther's sentence ``\textit{oratio, meditatio, tentatio faciunt Theologum}" ``\textit{the only correct theological methodology}" is given.

In his \textsc{Pastoral Theology} he remarks, on page 6: \begin{displayquote}``\textit{To attain to the theological habitude three things are requisite, which are contained in Luther's well-known axiom:}\textbf{ Oratio, meditation, tentatio faciunt theologum}."\end{displayquote}

The \textsc{Oratio} is the humble and earnest prayer that God would give us by His Holy Spirit the right understanding of the Scripture and not let us plunge into it with our reason. For ``\textit{although the grammatical sense of Scripture is clear, yet the Holy Ghost must open up for us the living and salutary understanding of the Scriptures}," and the ``\textit{beginning}" of all theology is to despair of all one's own wisdom, unconditionally to subject one's own opinion to the Word of God, and to be willing to derive all knowledge in spiritual things from the Word of God.

But this no man is able to do according to his own natural disposition. Therefore one must persist with the Oratio, and so much the more in proportion as learning and natural gifts are the greater. \\\begin{fancyquotes}Competent knowledge and rich gifts are a grand endowment. But it should never be forgotten, the greater the knowledge and gifts, the greater the danger that one becomes self-confident, also in theology!\end{fancyquotes}

The \textsc{Meditatio}, that is the constant study of the Scripture, \begin{displayquote}``\textit{the delving deep into God's Word},"
\begin{displayquote}``\textit{to occupy one's self with God's Word in every way},"
\begin{displayquote}``\textit{not in the heart alone, but also externally work on and apply the oral speech and the lettered words in the Book}\marginpar{\scriptsize according to Luther}, \begin{displayquote}``\textit{as one rubs aromatic herbs that they may give forth their own precious scent}''.\marginpar{\scriptsize adds Walther}\end{displayquote}\end{displayquote}\end{displayquote}\end{displayquote}
\divider
That the \textsc{Tentatio} also belongs to ``\textit{theological methodology}" is established, for instance, by {\scriptsize\textsc{2 Cor. 1:3}} When Luther says: \begin{displayquote}``\textit{As soon as the Word of God blooms forth through you, the devil will visit you, and make a real doctor of you, and by his affliction will teach you to seek and love God's Word}'',\end{displayquote}  Walther adds, that is indeed a ``\textit{strange promotion to the doctorate}." But God observes this method: ``\textit{hence no student of theology should grieve if God sends him all manner of temptation}." He is intent on holding fast to this ``\textit{methodology}," although He is well aware that many smile over it as insufficient for our times.
\divider
The \textbf{oratio}, \textbf{meditatio}, \textbf{tentatio} of which Luther speaks, however, are to be found only in the regenerate. And so Walther further insists most emphatically that only one who has first become a true Christian can become a theologian. He writes: \\\begin{fancyquotes}No unbeliever, no natural man, no slave of sin, no non-Christian, no hypocrite, but only a believer, a regenerate and sanctified person, in short, only a true Christian can be a true theologian; as the Christian presupposes the man, so the theologian presupposes the Christian, and as faith includes knowledge, so theology includes faith.\end{fancyquotes}

``\textit{The Holy Scriptures}," he continues,\\\begin{fancyquotes}states this clearly and plainly. The apostle, speaking of the office of the Word, cries out: `\textit{Who is sufficient}'\footnote{2 Corinthians 2:16} and answers: \begin{displayquote}`\textit{Not that we are sufficient of ourselves to think anything as of ourselves; but our sufficiency is of God, who also hath made us able ministers of the New Testament}'\footnote{2 Corinthians 3:5-6}.\end{displayquote} As surely, therefore, as the sufficiency for office is bestowed by God alone, so surely is also the theological habitude, which alone renders one competent for the exercise of the office, bestowed only by God. The holy apostle says further: \begin{displayquote}`\textit{The natural receiveth not the things of the Spirit of God\footnote{Does not perceive and accept what is of the Spirit of God, or the revealed mysteries of faith} for they are foolishness unto him: neither can he know them, because they are spiritually discerned. But he that is spiritual judgeth all things}\footnote{1 Corinthians 2:14-15}'.\end{displayquote}

As surely, therefore, as a natural man does not understand spiritual matters, and can have no correct judgment concerning them, so surely can a natural man be no true theologian, whose chief concern is to judge concerning spiritual matters. Only a truly spiritual man can be a true theologian. An unconverted man can indeed carry theology, as teaching, in his understanding and memory as in a book, and also impart it to others; but, although he can convert others, yet he is himself by virtue of head-knowledge and oral profession no more a true theologian than a book which contains the doctrine of theology comprised in letters and words; he is nothing else than what the apostle says of such, ``\textit{a sounding brass and a tinkling cymbal}"\footnote{1 Corinthians 13:1}. While he teaches others the pure truth unto salvation, it is to himself still a closed book, a mystery which he does not understand, yea, a foolishness. While he preaches to others, he himself is a castaway\footnote{1 Corinthians 9:27}. He does not hold the mystery of faith in a pure conscience\footnote{1 Timothy 3:9}. He still belongs to the world, and hence cannot receive the Spirit of Truth.\end{fancyquotes}
\par

``\textit{Godliness}," remarks Walther elsewhere with reference to the same subject, ``\textit{is not merely advantageous for the theologian, but a conditio sine qua non}\marginpar{\scriptsize\textit{conditio sine qua non\\} the condition without which he is not a theologian}." He refers to {\scriptsize\textsc{1 Timothy 3:1-7 and Titus 1:5-9}}, where, in the description of a true theologian ``\textit{the gifts of office and of sanctification are taken together}." In one category with the ``\textit{apt to teach}" stand ``\textit{vigilant, sober, of good behavior, given to hospitality}." In the assertion that there is no illumination without conversion Walther sides with the Pietists against a few of the later ``\textit{orthodox}."

Walther then demonstrates by the various activities which are incumbent upon a theologian that these can be performed only by one who stands in living faith. ``\textit{It is indeed}," he says, \begin{fancyquotes}a exceedingly important doctrine of our church that the Word of God is of itself quick and powerful and does not first become quick and powerful through the piety of those who preach it. But from this is does not follow that it is a matter of indifference whether one who occupies the office of the ministry is a godly man. Especially on account of the proper distinction of the Law and the Gospel, which is so necessary in the sermon and in the private cure of souls\footnote{Privatseelsorge} it is indispensably necessary that the preacher himself possess true faith of the heart and has himself had spiritual experience.\end{fancyquotes}

In his \textsc{Pastorale} Walther cites \textbf{Luther's} words: \begin{displayquote}`\textit{I experience it myself, and see daily also in others, how difficult it is to distinguish the doctrine of the Law and the Gospel. The Holy Ghost must here be Master and Teacher, or no man on earth will ever be able to understand or teach it. Therefore, no pope, no false Christian, no fanatic\footnote{Schwaermer} is able to divide these two from one another}.'\end{displayquote}

In this connection he notes: \marginpar{\scriptsize\textit{doctrine de discrimine legis et evangelii} \\The Proper Distinction between Law and Gospel}  \begin{fancyquotes}The \textit{doctrine de discrimine legis et evangelii} can indeed be correctly grasped by the intellect without a living faith, but then one goes astray in the application. Furthermore, the unconverted preacher who in his inner heart seeks only bread, honor, and a good living, not the salvation of the souls entrusted to him, will neglect to reprove sins, since he fears that thereby he would make enemies and thereby lose the treasures which he seeks to gain. The unconverted preacher dare not draw clear a picture of a true or false Christian on the basis of God's Word, for he has to fear lest his hearers will say, `\textit{You yourself are not like that!}' or `\textit{That's just like you!}'\par In an unconverted preacher the faithfulness, the zeal, the daily care, and in his preaching the real spiritual fire will be lacking. No office has such great temptations to unfaithfulness as the ministry. The pastor can lounge about for six days, if he wishes, and often the congregation observes with pleasure that the pastor does not '\textit{come around}.' If he has good gifts he can still, with all his laziness preach in such a way that the people will think they are hearing something wonderful. The unconverted preacher chooses such subjects as he can easily treat, and avoids the difficult ones, regardless of how necessary it may be to treat them.\end{fancyquotes} 

Hence Walther as a theological professor always took pains not only to present the Christian doctrine clearly, but also to edify the hearts and sharpen the consciences of his seminarians. Probably the majority of his students will testify that they experienced rich advancement in their spiritual life through Walther's theological instruction. His entire presentation was both instructive and edifying. Individuals among his students first came to a living faith in Christ in his theological lecture hall.

But however strongly Walther, on the one hand, emphasized and reiterated to his seminarians the truth that only ``\textit{one standing in grace, only a regenerate man}" can be a theologian, he, on the other hand, also warned against the abuse of this truth on the part of the sects and fanatics. He said, ``\textit{One can also abuse the teaching that theology is a habitus practicus}," namely in the direction of contempt for thorough theological study, or to indifference and negligence in study. ``\textit{The Methodists imagine that as soon as they are converted they can be preachers}." Every theologian is a Christian, but not every Christian is a theologian! The theological habitude is bestowed by God alone, but by way of diligent study.
 Walther cited the words of \textbf{L. Hartmann}: \begin{fancyquotes} What Tertullian once correctly said of the Christians, that Christians not born but made\footnote{\textit{Christiani non nascuntur, sed fiunt}}, is also true of faithful ministers and teachers of the church, who have need of a long preparation and intensive study if they are to be competent to enter upon their exalted office. For here mere personal reputation or earnestness and holiness of life are not sufficient, much rather is theological knowledge required.\footnote{Pastorale ev., Nuremberg, 1697, p. 237}\end{fancyquotes} In this connection Walther remarks, \begin{displayquote}``\textit{Only the regenerate can become theologians, but theology is not, like the spiritual life, bestowed in a moment}.''\end{displayquote}

As Walther, therefore, strove to impart the most thorough theological training, and that particularly on account of the peculiar circumstances in which the Church of the Reformation is placed in this country, so he also to spur the students on to the greatest diligence in study. He was accustomed to call to their minds that men like \textbf{Chemnitz}, \textbf{Gerhard}, \textbf{Calov}, yea, even \textbf{Luther}, became the great theologians they were ``\textit{not through their great gifts, but through the tireless diligence which they applied}.''

Among the notes which have been made available to the present writer are found also the following, which we present in their original aphoristic form, as they make clear the thoughts which Walther developed for his students: \begin{displayquote}``\textit{Be wary of time! - Read with the pen! - Make excerpts! - Schedule your studies! - Divide up the day and the week! - Read only quality! - Don't read trivia! - Review everything from time to time! - Index things! - Priorities: first the Necessary, then the most useful, then the useful! - Read with a theological interest! - Do not cram for exams! - Don't read trash at all!}"\end{displayquote} Walther warned the candidates of theology not to set too modest a goal. No one should permit himself to be misled by the thought that he has only mediocre gifts, so as to content himself from the beginning with mediocre accomplishments. ``\textit{To be modest in setting your goal is a sinful modesty}.''

This is the way Walther understood his statement: \begin{displayquote}``\textit{Theology is a habitude wrought by the Holy Ghost, and drawn from the Word of God by means of prayer, study, and trial}."\end{displayquote} We cannot surrender this definition of the concept. It is the Lutheran definition, the one taken from God's Word. The danger that we should fall into a fanatical line of thinking, and imagine that every Christian is without further training capable and called to teach publicly, is rather remote.

Also the sects within recent years have at least partially recovered from this delusion and insist upon theological training. But by God's grace we must also bear in mind that mere training makes no theologians, but that rather the basis and beginning of all theological knowledge and ability is living faith in Christ, a genuine conversion.

Only young men who are in a state of spiritual life are capable of studying theology; only truly believing pastors are competent to administer their office. The orthodox Lutheran Church in this country is still greatly in need of pastors. The prospects are that this need will in the immediate future grow not less but greater. But the need can never become so great that we allow manifestly unconverted persons to be called into the ministry, contrary to the Biblical and Lutheran principle that only converted Christians should be preachers or can be the right kind of preachers.

That the orthodox Lutheran synods of this country may look back upon such richly blessed activity comes also from the fact that God has endowed them with pure doctrine and also a ministerium of true believers. If God grants and preserves to them this gift also in the future, then their blessed fellowship will remain and prosper. If we through our ingratitude and heedlessness should lose this gift, should we get a ministerium in large part spiritually dead, then that the fresh and happy activity in our fellowship should soon cease and also the external apostasy from the right doctrine would soon follow.

%%% Local Variables:
%%% mode: latex
%%% TeX-master: "../main"
%%% End:

\chapter{Inspiration \& Open Questions}

\hrule
\vspace{.30cm}
In Zöckler’s ``\textit{Handbuch der Theologischen Wissenschaften}”\footnote{2nd Ed., III}, besides the Reformed theologians, \textbf{Kohlbruegge}, \textbf{Gaussen}, and \textbf{Kuyper}, Walther in St. Louis is named as a champion of the old church doctrine of inspiration  ``\textit{on the Lutheran side}”.  As proof for this reference is made to an article in ``\textit{Lehre und Wehre}” which later appeared in pamphlet form, which was indeed not written by Walther the article: ``\textit{Was lehren die neueren orthodox sein wollenden Theologen von der Inspiration?}”\footnote{Lehre und Wehre, 1871, p. 33ff}; but the statement of the ``\textit{Handbuch}” is nevertheless correct.  Walther not only championed the old church doctrine of inspiration with fullest conviction throughout his whole career as a teacher of the church, but he also designated the yielding up of this doctrine as an apostasy in principle from the Christian religion.
\vspace{.10cm}
\hrule
\vspace{1.25cm}

In the very first volume of ``\textit{Lehre und Wehre}''\footnote{Lehre und Wehre, 1855, p. 248}, in a review of Kahnis’ work: ``\textit{Der innere Gang des deutschen Protestantismus}'', etc., reference is made to the following words which occur in this writing:

\begin{displayquote}``\textit{Protestantism stands and falls with the principle of the sole authority of Scripture.  But this principle is independent of the doctrine of inspiration taught by the old dogmaticians.  To take it up again as it was could be done only with hardening against the truth}.''\end{displayquote} Kahnis took a better position then than he did later.  At that time his name was still in good repute in the Lutheran Church.  Yet Walther even then remarked on the words just cited:

\begin{fancyquotes}We must confess that when we read these words we were terrified in our very heart.  Who wants to go along with a new theology which introduces itself as a development of the old Lutheran theology and then departs from the doctrinal type of our old church just in the very doctrine of the formal principle of theology, in the doctrine of the Holy Scripture, specifically of the \textit{ratio formalis scripturae}, of that which makes Scripture the Holy Scripture?\end{fancyquotes}\marginpar{\scriptsize \textit{ratio formalis scripturae}\\ the formal grounds for something; the essential attributes of matter as they appear in the mind or in a definition.} So wrote Walther in the first volume of ``\textit{Lehre und Wehre}''.  He also treats inspiration in the last Foreword written by him, in the Foreword to the 32nd volume of “\textit{Lehre und Wehre}”\footnote{Lehre und Wehre, 1886}.

                What doctrine of inspiration Walther held for the correct one he briefly shows in “\textit{Lehre und Wehre\footnote{Lehre und Wehre, 1875, p. 257 f.}}'' in three short citations from \textbf{Baier} and \textbf{Quenstedt}.  But he always treated this subject very exhaustively in the regular lectures, and finally in the academic year 1885--1886 in evening lectures.  Walther’s doctrine of inspiration may be briefly summarized as follows: Holy Scripture does not merely contain God’s Word, but is according to its entire compass God’s Word in the proper sense, because God spoke through the sacred writers or gave them\footnote{denselben eingegaben hat} the matters and words, so that now in Holy Scripture not the slightest error can occur, either in dogmatical or even in historical, geographical, and other such matters.  So one must believe, says Walther, concerning Holy Scripture, if one accepts ``\textit{what Holy Scripture says of itself}”.\marginpar{\scriptsize 2 Timothy 3:16; 2 Peter 1:20-21; 1 Corinthians 2:13; John 10:35;}  He regards the concept of the inspiration of Scripture as having been given up by all those \begin{displayquote}“\textit{who acknowledge only an inspiration of the ‘what’ and not the ‘how’ of the matters and not also of the words of Holy Scripture, or who assume degrees of inspiration giving precedence to one book before another, or who grant that Scripture may contain any error, that it condescends not only to the comprehension of simple people but also to their false notions}”.\footnote{Lehre und Wehre, 13, 100.}\end{displayquote}  With regard to those who confuse inspiration with enlightenment and transform inspiration into a mere preservation from error, so that we would still have at least an errorless Scripture, Walther remarked: \begin{fancyquotes}That seems harmless enough, and yet thereby the entire doctrine of inspiration is given up.  We need not merely truth, but divine truth.  We must have a word which has passed through the mouth of God, and consequently is glowing with divine power and penetration, immersed, so to speak, in the mind of God.  The simple truth works through the power of persuasion; not so the Word of God.\end{fancyquotes}  With regard to the expressions of the Church Fathers and the old Lutheran teachers, to the effect that the Holy Writers were like manuas, calami, notarii, tabelliones of the Holy Ghost, Walther remarks: \begin{displayquote}“\textit{Though more recent positive theologians {\scriptsize\textsc{(die Neuglaeubigen)}} may ridicule these expressions yet they express the teaching of Holy Scripture}”.\end{displayquote}  The variations of style which are found in the various books of Scripture Dr. Walther explained, together with the great majority of the old teachers, by the fact that the Holy Ghost used His instruments as He found them; for the “\textit{essence of inspiration lies not in new words but in the fact that words, which may indeed have been otherwise in common use, passed through the mouth of God, that God made them His own words}.”  Whether the Hebrew vowel points which appear in the current pointed Hebrew text were written in the text from the beginning, as the majority of the old Lutheran teachers supposed, Walther declared to be not a dogmatical but a critical question.  He, for his own person, held with Luther, who declared the traditional Hebrew system of vowel points to be the product of a later age.
\divider
We shall here give just one example of the way Walther refuted the objections raised against the church’s doctrine of inspiration.  It is well known that the recent theologians assert they gave up the old doctrine of inspiration in order to rescue the “\textit{divine-human character of Scripture, which the earlier theologians had overlooked.}\footnote{Handbuch der theologischen Wissenschaften, l.c.}  Walther said:
\begin{fancyquotes}Among the many objections which modern believing theologians raise against the doctrine of inspiration as taught by our old dogmaticians one of the most common is that this doctrine in its emphasis on the divine character of Holy Scripture does not do justice to its human side, yes, entirely abolishes this aspect.  As in the Apostolic age the sect of the Docetists denied that in Christ God had become a true man, and taught that the apparent human in Christ was only an appearance, in like manner, it is now said, the old Lutheran dogmatics did with the Bible; the old dogmatics, they assert, makes everything human in the Bible a mere appearance. \par  All this is simply not true.  Also the old dogmatics indeed acknowledges a human side of the Bible in a certain sense.  It acknowledges that the Bible was not, like the Ten Commandments, written directly by God’s own finger, but through men, namely, the apostles and prophets.  Also the old dogmatics further acknowledges that the Bible does not speak the language of heaven, of which St. Paul says he heard unspeakable words, but that the Word of God has clothed itself in our human language and human writing.  Yes, the old dogmatics admits that the Bible was written by the holy writers not in a state of ecstasy but with full consciousness, and that the Holy Spirit accommodated Himself to the language and the human style of each apostle and prophet. \par   The old dogmatics, however, and we with it, teaches that in Christ the Son of God became a true man, but without sin, and thus also in the Bible the Word of God became true human speech, but without error.  As therefore a man for the reason that he is without sin is still not a mere appearance of a man, but a real man, so also human speech which is without error is not for that reason a mere appearance of human speech, but truly human speech. \par  For what purpose then is the cry that the old dogmatics does not do justice to the human side of the Scripture?  The intention is none other than this:  Our error is to consist in the fact that we do not ascribe errors to Holy Scripture as to every other human writing, but that we hold it, among all books, to be the Book of Truth.\footnote{Evening Lecture on the Doctrine of Inspiration, December 18, 1885. (Manteufel Translation)}\end{fancyquotes}

                For what reason did Walther hold so firmly to the church’s doctrine of inspiration?  Before all else because this is the clear teaching of Scripture concerning itself.  But then also because, as already suggested, with the surrender of the church doctrine of inspiration also the truth that Scripture alone is the source and norm of Christian doctrine is surrendered.  It is inconceivable how a man like Kahnis, who has been labeled a “\textit{thinker}”, can put forward the proposition that the principle of Protestantism concerning the sole authority of Scripture is “\textit{independent}” of the old church doctrine of inspiration, that is, from the teaching that Scripture is the perfectly inerrant Word of God.  Everyone will at once be constrained to agree with Walther when he ever and again declares: \begin{fancyquotes}We must absolutely hold fast to the doctrine of inspiration taught by our old dogmaticians.  If we grant that in the Bible even the least error can occur, then man must undertake to separate truth from error.  Thereby man is placed over the Scripture and Scripture has therewith ceased to be the source and norm of faith.  Human reason is made the norma of truth and Scripture sinks to the level of a norma normata.  The slightest deviation from the old doctrine of inspiration introduces a rationalistic germ into theology and leavens the entire structure of doctrine.\footnote{In a lecture 1874-1875}\end{fancyquotes}

On the same subject Walther said, with reference to the controversy over the doctrine of inspiration recently provoked by the Dorpat professors \textbf{Volck} and \textbf{Muehleau}: \begin{fancyquotes} With the doctrine of inspiration stands and falls the truth, certainty, and divine authority of Holy Scripture and therefore of the entire Christian religion and church.  This is not just one doctrine alongside of others, but upon it rest all other doctrines as upon their foundation.  If Holy Scripture is not inspired of God, but brought forth by the will of man, then it is also no divine but a human Scripture.  But if one says: \begin{displayquote}{\footnotesize In all which Scripture reports and declares concerning the earning and attainment of eternal salvation it is of divine origin and therefore infallible in this respect; only in that which stands no necessary connection therewith, in the non-essential and incidental matters is it of human character and therefore not entirely errorless,}\end{displayquote} -- this does not help matters.  For then, by the assertion that human error is mingled with the divinely true content of Scripture, not a part but the whole of Scripture is rendered unsteady and unreliable and the reader is made the superior judge as to which component parts of Scripture contain the essential and which the non-essential, which the divine and which the human, which contain truth and which contain error or at least could contain error.  Then it would be a gigantic hoax and gross deceit that the Christian Church hitherto has always regarded Holy Scripture as the formal principle or as the pure source of all her Christian knowledge, as the inerrant rule and standard of all faith and life, and as the supreme and ultimate arbiter in all controversies concerning faith and religion.  Then one could no longer admonish a Christian as often as he opens his Bible to pray with Samuel \begin{displayquote}‘\textit{Speak, Lord; for Thy servant heareth}’,\end{displayquote} but would rather have to warn every Bible-reader not to surrender himself to Scripture with entire confidence, and admonish him to read Scripture with great caution and constant discrimination and to devote himself to the task of culling the divinely true from the midst of human error.\footnote{Evening Lecture on the doctrine of inspiration, November 27, 1885.}\end{fancyquotes}

                Hence Walther exclaims: “\textit{God have mercy upon His poor Christendom in this last age of distress and danger}”\footnote{L.u.W. 32, p. 77}, in which the Christians have their Bible taken from them, “\textit{the lamp unto their feet and the light unto their path to eternity, their rod and staff in the dark valley of tribulation, in short, God’s Word, and therefore their comfort in the anguish of sin, their hope in the night of their dying hour}”.\footnote{L.c. p. 76.}

                It is his desire therefore that “\textit{Lehre und Wehre}” shall also in future warn against the deniers of the inspiration of Holy Scripture ”\textit{as the worst false prophets of our time}.''

                He writes: \begin{fancyquotes}It is time indeed for every believing theologian, as he values his soul’s salvation, with the utmost earnestness to get into the fight for the highest treasure of Christians which God has given to men after the bestowal of His Son.

                Woe unto him who wants to be reckoned among the theologians and yet will not acknowledge that this above all is his vocation, to preserve unto the common Christian that upon which his faith, and hence also his salvation and blessedness, rests, the ‘\textit{foundation of the apostles and prophets, Jesus Christ himself being the chief cornerstone}’. \par Woe unto him who wants to be reckoned among the theologians and on the other hand  imagines that just for that reason he must as such contend above all that its full freedom remain assured unto science!  Just in this lies the deepest ground for the ever more complete apostasy of modern theology from the revealed divine truth and for the complete transformation of the Christian religion into a human science, namely, that modern theology no longer wishes to be a \textit{habitus practicus}, but the ‘\textit{scientific self-consciousness of the church}’ or ‘\textit{the ecclesiastical science of Christianity}’.\footnote{L.c. p. 6.}\end{fancyquotes}

%%% Local Variables:
%%% mode: latex
%%% TeX-master: "../main"
%%% End:

\chapter{Inspiration \& Open Questions II}

\hrule
\vspace{.30cm}
Dr. Walther had the same object in mind, namely, the guarding of the principle of Scripture, or holding fast to the truth that Holy Scripture alone is the source and norm of  Christian doctrine, also in the controversy on “\textit{open questions}”.  As human reason or science is made the norm for Christian doctrine through denying the church doctrine of inspiration, so “\textit{the church}” with its doctrinal decisions takes the place of Holy Scripture through the modern theory of open questions.  For in what sense did Pastor Loehe, the Iowans, and the authors of the Dorpat theological opinion\footnote{Gutachten} speak of “\textit{open questions}”?  As open questions they desired to consider such doctrines as, although revealed in Scripture, have not yet been decided by the Church in her Symbols or concerning which no agreement has yet been reached among orthodox theologians.\footnote{For the fact that those named really spoke on this sense of open questions, evidence is offered, e.g. in “L.u.W.” 14, 129ff.  Later indeed the Iowans declared that it never entered their mind to speak of open questions in this way.}  \par Among the doctrines declared to be such were the doctrine of the \textbf{Church}, of the \textbf{Ministry} and \textbf{Power of the Keys}, of a \textbf{Millennial Kingdom} still to be expected, of a future twofold Visible \textbf{Advent of the Lord}, and of a twofold bodily \textbf{Resurrection}, of \textbf{Sunday}, etc.

\vspace{.10cm}
\hrule
\vspace{1.25cm}


Also Walther acknowledged the existence of “\textit{open questions}”.  But in an entirely different sense.  He wishes to have the term “\textit{open questions}” used as synonyms with “\textit{theological problems}.”  Hence open questions are to him such as God’s Word leaves open questions which indeed arise in connection with the discussion of the Christian articles of faith, “\textit{but which find no solution in God’s Word}.\footnote{L.u.W., 14, 33.}  Walther insists most strenuously that open questions in this sense be acknowledged, and this for the very purpose that the Scripture principle may remain inviolate.  For if one should wish to “\textit{close}” a question which God’s Word leaves open then one would be adding to the Scripture.  He writes: \begin{displayquote}``\textit{What is not contained and decided in God’s Word must also not be equated with God’s Word and thus added to God’s Word.  But this would take place if orthodoxy should be made dependent upon any doctrine not contained in God’s Word and the denial of it should be given church-divisive significance.  Open questions in this sense are therefore all doctrines which are neither positively nor negatively decided by God’s Word, or such by the affirmation of which nothing which Holy Scripture denies is affirmed, and by the denial of which nothing which Holy Scripture affirms is denied.\footnote{L.u.W., 14,33.}''}\end{displayquote}
Among such open questions Walther, with the older theologians, reckons, among others, also the following:\begin{itemize}\item Whether Mary gave birth to other children after Christ\footnote{the Semper virgo};\item Whether the soul is imparted to every man through propagation from his parents, as flame from flame\footnote{per traducem, traducianism}, or through creative infusion\footnote{creationism};\item Whether the visible world will pass away on the last day according to its substance or only according to its attributes, etc.\footnote{L.u.W., 14, 34}\end{itemize}  On the other hand Walther insists most strenuously that nothing shall be declared an open question and treated as such which is clearly taught in God’s Word and thus decided by God’s Word.
\divider
                And in this connection it makes no difference whether the doctrine in question is fundamental or non-fundamental.  For here the Scripture principle comes into question, namely, whether all which God prescribes to men in Scripture to be believed is to be received by men in faith.  Walther writes: \begin{displayquote}“\textit{We can regard and treat no doctrine which is clearly taught in God’s Word or which contradicts God’s clear Word as an open question, no matter how subordinate or how far removed from the center of saving doctrine upon the periphery it may appear to be or actually is}”.\footnote{L.u.W., 14, 66}\end{displayquote}  And shortly thereafter: \begin{fancyquotes}We assert that in the orthodox church no justification can be conceded to any error against God’s clear Word, that in the orthodox church it may not be made optional to depart even in the least point from God’s clear Word, be it negatively or positively, directly or indirectly, and that every such departure from God’s clear Word, though it should consist in nothing more that the denial that Balaam’s ass spake, demands action on the part of the orthodox church against it, and that when all instructions, admonitions, warnings, and threats, and all exercises of patience have proved fruitless and ineffective to induce the person or group concerned to give up their contradiction against God’s clear Word, finally nothing else than expulsion or a separation can result.\footnote{l.c., p. 68.}\end{fancyquotes}

                Walther further explains how the Scripture principle comes into question here as follows: \begin{fancyquotes}What else is the assertion that such doctrines as are clearly contained in God’s Word could belong to the open questions than an assertion that one can indeed ‘diminish from’ God’s Word, need not always go according ‘\textit{to the law and to the testimony}’, that ‘\textit{a little leaven}’ of false doctrine does no harm and is therefore to be tolerated, that the Scripture can now and then ‘\textit{be broken}’, and one need not exactly ‘\textit{believe all that the prophets have spoken}’, that all Scripture is not so necessary and ‘\textit{profitable}’, and that it is indeed permitted to ‘\textit{break}’ much which is contained in the Scripture?  And yet more: suppose that all the passages cited\footnote{Deuteronomy 4:2; Deuteronomy 12:32; Isaiah 8:20; Revelation 22:19; Galatians 5:9; John 10:35; Luke 24:25; 2 Timothy 3:13,17; Matthew 5:18,19} and similar ones were not found in Holy Scripture, who would not even then, if he only really holds God’s Word to be God’s Word, have to find that theory unacceptable?  For if the Bible is God’s Word, then all the utterances contained therein are decisions of the exalted divine Majesty Himself.  Is it not terrible to declare that which the great God has decided to be still undecided? \par -- When the great God has spoken, to give man the liberty to contradict Him? -- where the great God has given His final judgement, to speak of the right of any creature to pas another judgement? -- to undertake a sifting of that which the eternal Wisdom and the eternal love has revealed for the salvation of men, and to say: This you must believe, confess, and teach, but that you may reject?\footnote{L.u.W., 14, 69.}\end{fancyquotes}
                
{\color{Black} \par If then, anyone says that doctrines are to be regarded and treated as still open because the orthodox church has not yet rendered her decisions upon them in her Symbols, or because there is not yet complete agreement concerning them among the teachers of the orthodox church, the Scripture principle of the Lutheran Church is thereby openly surrendered and crass papism is adopted.  Walther exclaims:} \begin{fancyquotes}From their point of view, then, any one has the liberty to accept or reject what God has revealed and decided in His Word as long as the Church has not yet spoken and rendered her decision; but as soon as the Church has spoken, all liberty has come to an end!}”\footnote{L.u.W., 14, 162.  Trans.: C.T.M., X, 8,588} \par It substitutes the Church for Scripture, man and his decisions for God and His divine decision.  And this substitution surrenders the foremost principle of true Protestantism and ascribes to our Church the principle of the antichristian Papacy, with all its errors and abominations.}”\footnote{l.c., p. 163.  Trans.: l.c., p. 589, corrected.}\end{fancyquotes}

{\color{Black} \par  The question whether a doctrine revealed in God’s Word is first raised to the dignity of a publicly acknowledged article of faith through the Symbolical decision of the Church, coincides with the question whether dogmas are gradually formed, or whether doctrines of the Word of God first become dogmas when they have passed through an ecclesiastical controversy and have become “\textit{symbolically fixed}”.  Walther’s utterance on this point takes account of the exact status controversiae and concedes what must be conceded:}
\begin{fancyquotes}It is true that the Word of God prophecies, and the history of the Church confirms, that the Church does not always stand before us in the same brilliant light of pure public preaching, that it rather, to use the figure of ancients, in this respect decreases and increases like the moon, that it experiences times of special gracious visitation and then again declines.

                \par But it is an error to say that the Church from century to century accumulates an ever-growing fund of divine teachings and according to the law of historical development arrives at constantly enhanced depths and riches of knowledge.  We admit that the Church all the time, through \begin{displayquote}{\footnotesize `men that arise in its midst and who speak perverse things to draw away disciples after them,\footnote{Acts 20:30} is compelled to formulate with increasing precision the pure doctrine which it possesses in order that the fraudulent errorists may be unmasked and false teachings be kept from creeping into it through ambiguous phraseology'}\end{displayquote} -- but this does not imply that the number of its dogmas grows; they are through this activity merely safeguarded ever more carefully against the danger of becoming perverted. 
                \par That Christ is with the Father, that the union of the divine and human nature in Christ took place, that “\textit{in, with, and under}” the bread and wine of the Lord’s Supper Christ’s body and blood are actually present, are given, and are orally received by worthy and unworthy communicants, -- these are, it is true, dogmatic expressions which were not found in the orthodox Church till the days of Arius, Nestorius, Eutyches, and Zwingli; but they are not new dogmas. \par Furthermore, we do not deny that through continued searching of the Scriptures by the Church some things are by and by cleared up which before, through imperfect acquaintance with the languages and history, had been unknown; we admit that in this manner the content of the various doctrines of faith at times is set forth and unfolded in a higher degree than before and that from this point of view we may indeed speak of a progress in knowledge.  But this by no means implies the gradual origin and increase of dogmas which modern theology teaches; we must rather say that through this course that which already is known receives new confirmation.\footnote{L.u.W., 14, 137. Translated: C.T.M, X, 7, 510, 511.}  In the first place it is not true that our dogmas come into existence gradually and that hence there are articles of faith \begin{displayquote}{\footnotesize `which are still in the process of formation, and others which as yet have either not at all or merely by way of beginning been drawn into the stream of events in which dogmas take shape’.}\end{displayquote}  It is not true that some articles of faith ‘\textit{have come down to us as undecided, unfinished question, incomplete structures, as open questions}’, because concerning these things ‘one does not yet find unanimous agreement’ in the Lutheran Church.  This theory, held and advocated with more or less emphasis by almost all modern theologians, though entirely unknown to the old orthodox theologians of our Church; as we view it, it is merely a daughter of Rationalism appearing in Christian dress, a sister of Romanism hiding behind a Protestant mask, and a fruitful mother of large families of heresies.  With respect to the Rationalists it is well known that they were the first to describe dogmas not as the unchangeable, divine, fundamental truths of Christianity but as doctrinal opinions which has arisen in a scientific process or which had been elevated by the various or which had been elevated by the various denominations to the position of ecclesiastical teaching and were considered authoritative in the respective age.
\par For this reason they strictly distinguished between doctrines of the Church and of the Bible... No proof is needed to show that Roman Catholics also teach the gradual rise of dogma; but a few years ago we beheld the spectacle of the present Pope’s declaring the teaching of the Virgin Mary’s immaculate conception, which before had been considered an open question, to be a dogma and now binding for all ‘\textit{believers}’, and just now\footnote{1868},  according to reports, the alleged heir of Peter’s episcopal throne is preparing to enrich his Church again through a new dogma by decreeing his own infallibility.  While modern Lutheran theologians are far removed from the position which would vindicate the right of the Roman Church or even the Pope to create new articles of faith, their theory that dogmas come into existence gradually, that on certain points a ‘\textit{unanimous consensus}’ arises, or that the Church has finally ‘\textit{pronounced}’ and ‘\textit{decided}’ with respect to such matters, is nothing but a sister of Romanism, having put on a Protestant mask. \footnote{L.u.W., 14, 133-136.  Trans.: C.T.M., X, 7, 507 and 508.}\end{fancyquotes} 

Of particular importance is the axiom championed by Dr. Walther: \begin{displayquote}“\textit{Every doctrine of the Bible is a doctrine of the Church.  He who hears the Scripture even from the humblest layman, he hears the Church, because the Church knows and confesses nothing else than the truth revealed in the Scripture.}\end{displayquote}  Walther writes:\begin{fancyquotes} What struggles it cost Luther to attain to this knowledge is well known...  Latter Luther finally realized that he had then really heard the Church when the humblest layman had convinced him with the Scripture.  Our modern Lutherans have returned to the condition of the Christians before the Reformation.  No matter what clear Scripture is brought them by a common Christian, they look upon this, in the language of {\scriptsize\textsc{(the theological faculty of)}} Dorpat, as merely \begin{displayquote}{\footnotesize ‘private and individual Christian convictions, however well grounded they may be, and the results, for the time, of conscientious and believing searching of the Scriptures’,}\end{displayquote} and await the decision of the Church, \begin{displayquote}{\footnotesize ‘because as yet there is no acknowledged standard for their ecclesiastical validity and the question of their Scripturalness is still an undecided point of contention’.}\end{displayquote}  Scripturalness is for them something to be decided not from the Scripture but by the Church.  That they should be hearing the Church when a miserable Missourian brings Scripture is to them a ridiculous idea.  For them the hearing of the Church requires first of all that the learned come together, discuss, dispute, and finally decide. \footnote{L.u.W., 14, 209.}
\end{fancyquotes} 

                Thus therefore Walther emphatically rejected the suggestion that only that is “\textit{Lutheran Church doctrine}” upon which our Church expresses herself in her Symbols.  No, every true Bible doctrine is Lutheran Church doctrine, even if it is not Lutheran Symbolical doctrine.  The Lutheran Church confesses in her Symbols by no means only those doctrines which, because of certain circumstances, she specifically mentions therein, but the entire Holy Scripture and all doctrines contained in Scripture.  \begin{fancyquotes}In regard to a heterodox Church that has set up a false principle and does not accept the Word of God as it reads, but insists on interpreting the Word either according to reason or according to tradition, the following statement cannot be upheld: \begin{displayquote}‘\textit{For her every doctrine of the Bible is a doctrine of the Church}’.\end{displayquote}  But this statement can be made of the truly orthodox Church and hence also of our dear Evangelical Lutheran Church.\end{fancyquotes}

                Hereupon Walther adduces passages of the Lutheran Confessions in which it is asserted that whosoever brings the Scripture, the Word of the prophets and apostles, causes the voice of the Christian Church to be heard.\footnote{L.u.W., 14, 208. Trans.: C.T.M, September, 1939, pp. 663-664} \begin{fancyquotes}That which truly belongs to the Church is always Biblical, and that which is truly Biblical always belongs to the Church, with a different (besonderen) faith; she does desire to be a part of the Church of the apostles and prophets, a part of the Bible Church.  She has indeed written Confessions and defined doctrines, not because they should contain her whole body of doctrine, nor because she had reached a decision only on those doctrines found in her Symbols, but because false churches and false teachers forced her to make clear-cut statements on certain doctrines.  Up to the present time she has seen no necessity for writing special Symbols on other doctrines.  All that she believes therefore is not found in her Symbols, but only in the Bible.  Her Symbols are not so much ‘\textit{the landmarks of spiritual development}’ as the boundary line separating her from certain falsehood.\footnote{L.c, p. 210.  Trans. L.c. pp. 664, 665}\end{fancyquotes}\divider \begin{fancyquotes}If  our Church makes claim only to Symbolical and not at the same time to canonical unity, as Gerhard calls it, i.e., to Biblical unity, then our Church is, we repeat it, not an orthodox Church, but a miserable sect, which does not bind itself to accept the whole Word of God but only certain doctrines thereof.  No matter how dear and valuable the incomparable Confessions of his Church are to every Lutheran, he does not permit them to become the Lutheran Bible, in which the whole faith of his Church is contained, while all other Biblical doctrines are nothing more than matters of ‘\textit{private and individual Christian conviction, however well grounded they may be}.’\footnote{L.c., p. 211. Trans.: l.c., p. 666}\end{fancyquotes}

                “\textit{It is indeed strange}”, Walther adds, \begin{displayquote}“\textit{that men who constantly speak against placing the Confessions above the Bible declare themselves bound as Lutherans only by those doctrines which are fixed Symbolically.  This fact makes it quite evident who those men are that actually stand on Scripture and believe in its supreme authority as well as in its clarity, and those who do not.}”\end{displayquote}\textbf{Pastor Hochstetter}, who took part in the colloquy arranged with the Iowa Synod in 1867 at Milwaukee, writes: \begin{fancyquotes}It was then first really clear to me \footnote{Pastor Hochstetter had recently come from the Buffalo Synod to the Missouri Synod} that the strength of the Missourian teachers lay not so much in their dependence upon the Symbols, as rather in their reverence for God’s Word!\footnote{Isaiah 66:2} There the maxim was: \begin{displayquote}{\footnotesize Everything is Church doctrine which is Bible doctrine, whether it is contained and established in the Symbols or not, if only it is in Holy Scripture}’.\footnote{Geschichte der Missouri-Synode, page 288}\end{displayquote}\end{fancyquotes}

%%% Local Variables:
%%% mode: latex
%%% TeX-master: "../main"
%%% End:

\include{parts/part4}
\include{parts/part5}
\chapter{The Church I}

\hrule
\vspace{.30cm}

It remains for us to characterize Walther’s position in certain individual doctrines which came into controversy.

\vspace{.30cm}
\hrule
\vspace{1.25cm}

                In the first place, however, we must take note that Walther was not a theologian who cherished and cultivated certain favorite doctrines and for their sake neglected other doctrines which are just as clearly revealed in God’s Word.  That has indeed been the habit of not a few men who have become famous in the church.  Thereby they revealed that they indeed stood at the head of a sect, but could not work in a truly churchly manner. \par No, Walther was a true church theologian, who with the greatest faithfulness sought really to teach and to maintain all which is entrusted to the Church in Holy Scripture.  Hence, although he, on the one hand, well knew how to distinguish between the individual doctrines with regard to their absolute necessity for the engendering and preservation of faith, yet, as his teaching in the theological seminary testifies, he held to all the doctrines of the Christian faith with the greatest diligence.\footnote{Cf. Pastorale, p.90 f}  Circumstances, nevertheless, brought it about that Walther had to devote very particular attention and labor to certain individual doctrines.  And to Walther’s position in these doctrines we shall turn our attention in what follows.

                The doctrine which, immediately after their arrival, not only occupied the attention of the Saxon immigrants but became a most vital question for them is the doctrine of the Church.  “\textit{We are no longer a church}”, was the thought in the hearts of many, when the man whom most of them followed with the utmost confidence as their leader and bishop, fell away, and thereby as with one blow the structure of the church which they had hitherto regarded as the true Church was demolished.  In this situation it was principally Walther who convincingly answered from the Scripture, the Confessions, and the writings of Luther, the question, what the Church is, and thus effectively warded off the confusion which threatened to disrupt the little congregation.
\divider
How Walther and the \textit{Missouri Synod} came to the doctrine of the Church, as it is set forth, for instance, in Walther’s book, “\textit{Die Stimme unserer Kirche in der Frage von Kirche und Amt}”, is a matter concerning which quite false views are current still today in Germany.  It is said that Walther fashioned the doctrine according to democratic American conditions.  But the exact opposite is the case. \par In the first place, the immigrants were still very little acquainted with “\textit{American church conditions}'', at the time when the question of Church and Minister was already decided among them.  And when at a later time they came into closer contact with these “\textit{American}” conditions, then it was not these which exercised a decisive influence upon them, but it was they who exerted a deciding influence upon the conditions.  Says Walther: \begin{displayquote}“\textit{We set ourselves with all our might against the abuses prevailing in American church circles. In many circles we succeeded in doing away with the hiring of pastors and the absolute power of the congregation}”.\footnote{We again call attention to the fact the we are citing Walther according to manuscript notes wherever we do not make specific reference to a printed writing.}\end{displayquote} To be sure, the conditions into which God placed the little flock of immigrants were the occasion which led to their recognizing the doctrine of the Church which they now championed as the true doctrine.  But this doctrine itself is not derived from the circumstances, but in time of intense temptation and great tribulation was achieved through the study of the Word of God, the Confessions, and especially the writings of Luther. \par Walther himself writes in the Foreword to “\textit{Kirche und Amt}”\footnote{Church and Ministry}:
\begin{fancyquotes}Willingly as we grant that the conditions under which we live here in America were of decisive influence in leading us to the vital recognition of the doctrine of Church and Ministry laid down in this book, so that we hold it fast as a precious treasure and now confidently confess it before the whole world: we must nevertheless decidedly reject the charge that we have bent and fashioned the holy pure doctrine of our Church in the interest of the conditions and circumstances surrounding us.  \par Since we are here living not under inherited ecclesiastical conditions, but are rather in a position which requires that we lay the foundation for such, and in which also we are able to lay it unhindered by anything already existing, these circumstances have therefore the rather impelled us with great earnestness to search for the principles upon which according to God’s Word and Confessions of our Church the polity of a truly Lutheran fellowship must rest, and according to which such polity must be formulated. \par The less the question arose: \begin{displayquote} {\footnotesize What can we retain without sin?}\end{displayquote} And the more we were occupied with the question: \begin{displayquote}{\footnotesize Who should it be in accordance with God’s Word and the principles expressed and demonstrated in our Church’s Confession?}\end{displayquote} -- so much the more urgent for us was the need of coming into the clear and arriving at a firm assurance of faith concerning the principles of the doctrine of the \textbf{Church}, \textbf{Ministry}, \textbf{Power of the Keys}, \textbf{Church Ordinances}, and the like.  We have not fashioned the doctrine of our Church according to our conditions, but have ordered these according to the doctrine of our Church.  To anyone who doubts this we confidently issue the summons: \begin{displayquote}\textit{Come and see!}\end{displayquote}  And he who with astonishment finds principles presented by us as principles and doctrines of the Lutheran Church which he has hitherto abominated as fanaticism, -- him we can confidently directly to the references which we have adduced in proof, and leave him the choice of either granting us the praise of Lutheran orthodoxy or denying it to the entire cloud of faithful witnesses from Luther down to a Baier and a Hollaz.\footnote{Kirche und Amt, 3rd Edition, Foreword, VIII; 4th Edition, Foreword VIII, IX.}\end{fancyquotes}

Over against the assertion that the doctrine of Church and Ministry expressed in our Confessions is “\textit{still undeveloped and unclear}” Walther says in the same Foreword: \begin{fancyquotes} We are of the firm conviction that the reason Lutherans are now divided over the important doctrines of Church and Ministry and all which is directly connected therewith is that they have disregarded and turned aside from the doctrine laid down in the public Confessions of our Church and developed in the private writings of her orthodox teachers.\par  We are of the firm conviction that our Church has not left the doctrines of Church and Ministry unexamined, so that they now still await development; much less has she in any manner obscured these doctrines or assigned them an unfitting place in the entire structure of doctrine, so that they must now still be readjusted. \par We are of the firm conviction that the great decisive conflict of the Reformation which our Church fought in the Sixteenth Century against the Papacy revolved about these very doctrines of the Church and Ministry which have now again come into question among us, and that the pure clear doctrine on this subject is a precious spoil which our Church won in that conflict.\footnote{Kirche u. Amt, V, VI}\end{fancyquotes}
\divider
                What is the Church in the proper sense?  Walther in his instruction in Dogmatics designated this \textit{a priori} as the main question and the determinative point in the entire \textit{locus doctrinae} concerning the Church and all that is connected therewith.  “\textit{The main thing is to know what the Church is properly and essentially}”.

                What the Church is, is something which was not known in the Papacy before the Reformation, nor was this knowledge desired.  A man who knew it and spoke out about it was burned at Constance.\footnote{See the citations from Aegidius Hunnius and Luther in Walther’s Baier III, 614, 619}  Through Luther it again became known what the Church is, and so well known that Luther could write in the Smalcald Articles\footnote{Smalclad Articles -- (Part III, Art. XII; Mueller, p. 324; Triglotta, pg 499}: \begin{displayquote}“\textit{Thank God, a child seven years old knows what the Church is, namely, the holy believers and lambs who hear the voice of their Shepherd.  For the children pray thus: ‘{\small I believe in one holy Christian Church}’}.”\end{displayquote}  In our time this children’s wisdom has again become almost as unknown to many who hear the Lutheran name as it was under the papacy.  To the question as to what the Church is, even such as have considerable reputation in the Lutheran Church give the most various answers, only not the simple and only correct answer, that the Christians are the Church. As essential parts of which the Church is supposed to consist the following are mentioned: \begin{itemize} \item Christ \item the means of grace \item godly and ungodly \item the office of the means of grace or  the order of teachers and learners \item the order of those who rule and \item those who obey in a definite ecclesiastical constitution.\footnote{Cf. The extracts from the writings of recent theologians, L.u.W., 16, 162 f.}\end{itemize}
                From these and other parts men constructed for themselves the “\textit{Church}”. To the most the Church is an “\textit{outward polity}”, an “\textit{institution}”, in which the Christians form a more or less essential component part, only that they are not themselves the Church. --  It is obvious that, with the existent confusion with regard to the concept of the Church, especially with the conception of the Church an “\textit{institution}”, the much lamented evils of the church cannot be rectified.  How shall one help the church if one does not know what the Church properly is.  If the Church were held to be what it is, the congregation of believers, then care would be directed principally to that thereby believers, children of God, are born and preserved, namely, the preaching of the pure doctrine, and that whereby faith is hindered and destroyed, namely, false doctrine, would be decisively opposed and removed.  \par But since the Church is held to be essentially an institution and a sum of ordinances and relations, care for the welfare of the church is consequently exhausted in the care for the maintenance or restoration of ordinances; yes, in this way everything which could disturb the ecclesiastical “\textit{institution}” {\scriptsize\textsc{(or establishment)}} is anxiously avoided.

                According to Walther the Church is the totality of believers, nothing more and nothing less.  Nothing more: for to the Church belongs no unbeliever or unregenerate person, even though such a one may be in the outward fellowship of the Church, yea, even occupy the highest offices in the same.  Not less: for all believers on the whole earth belong to the Church, whether they are in the visible fellowship of the orthodox Church, or are held under the sects and the Papacy.\footnote{Lutheraner, XI, 17,18}; also those who have been wrongly excommunicated, if they have faith, belong to the Church, as well as those who have not yet been formally received into the Church by Baptism, if they have already come to faith through the Gospel.  In short, only living faith in Christ is decisive of membership in the Church.  \par In Walther’s work, “\textit{Die Stimme unserer Kirche}”, the first two Theses “\textit{of the Church}” read this:\begin{displayquote}``\textit{The Church, in the proper sense of the term, is the communion of saints, that is, the sum total of all those who have been called by the Holy Spirit through the Gospel from out of the lost and condemned human race, who truly believe in Christ, and who have been sanctified by this faith and incorporated into Christ.  To the Church in the regenerated, no heretic}.''\end{displayquote}  Walther proves this with texts such as {\scriptsize\textsc{Ephesians 1:22-23; Ephesians 5:23-27}}, where Christ is called the Head of the Church and the Church is called Christ’s body, where the Church is described as “\textit{subject unto Christ}” and “\textit{sanctified}” and “\textit{cleansed}” by Him.  He remarks on {\scriptsize\textsc{Ephesians 1:22-23}}: \begin{fancyquotes}Since Christ, according to this text, is the Head of the Church and the latter is His body, the true Church, properly so called, is the sum total of all those who are united with Christ as the members of a body are with their head; and on {\scriptsize\textsc{Matthew 16:18}}: \begin{displayquote}`\textit{Upon this rock I will build My Church; and the gates of hell shall not prevail against it}'\end{displayquote} The Church, then, in the proper sense of the term, is built, as regards its members, on the rock of Christ and His Word.  Upon this rock, however, only he is built who by a living faith makes it his foundation.  -- Thus writes St. Paul\footnote{Romans 8:9;  Translation appears to be partially paraphrased.}: \begin{displayquote}`\textit{If any man have not the Spirit of Christ, he is none of His}’.  Now if a person does not belong to Christ, neither is he a member of the true Church, which is His spiritual body.\footnote{Kirche und Amt, 4th Edition, pp. 1,2, and 10.}\end{displayquote}\end{fancyquotes}
\divider
                To designate the relation in which the godless stand to the Church Walther liked to use the expression of \textbf{Gerhard}: “\textit{The godless are indeed in the Church \footnote{according to external fellowship} but not of the Church}”, and \textbf{Calov’s} word: \begin{displayquote}“\textit{Although the hypocrites are in that multitude in which the Church is, yet they are not properly in that multitude which is the Church}.”\end{displayquote}  Between believers and hypocrites, even if they are externally in the same fellowship, there remains always as great a difference as between Christ’s Kingdom and the devil’s kingdom.  According to Walther the reason why people make Christ, the means of grace, the office of the ministry, etc., essential component parts of the Church is because one represents that which is necessarily connected with the Church as the Church itself.  Against this “\textit{error so widely current in our time}” Walther extracts\footnote{L.u.W., 9, 284} the following from the “\textit{Mecklenburgische Theologische Zeitschrift}”: \begin{fancyquotes} That which cannot be separated from the Church, without which the Church cannot exist, which therefore in some way necessarily belongs to the Church, still is not included in the Lutheran concept of the Church as such, does not belong to that which makes up the Church, the communion of saints, Christendom, as such. \par Thus man cannot live without air and daily bread, but air and daily bread do not belong to the concept of man; the human race cannot exist without the earth upon which it lives, and without the heaven which arches itself above it, and without the sun which rises upon it with its light and warmth, nevertheless the concept of the human race is distinct from all this, does not coincide with the concept of the universe.  Christ, the Head of the Church, is inseparable from the Church, which is His body; the existence of the Church would be eliminated with her separation from the Head, from the Lord who dwells in her and works in her through the means of grace, yet Christ does not belong to the concept of the Church, which is the body of Christ, and as such distinct from the Head. \par The same holds true of the means of grace, of the Word and Sacraments.  Through them the Church receives her life from the Head, and without them the Church lacks the basis of her existence; nevertheless they do not belong within the scope of the Lutheran concept of the Church; inseparable from her they are yet distinct from her.  The means of grace have been given to the Church by the Lord, the Church has them, uses them, lives by them, in the Church they are administered in the service of the Lord, that the working of the Lord, through them, increasing and perfecting the Church, may ever go forward, but they themselves are not in any respect the Church.  Therefore the means of grace, rightly administered, are also designated as the \textit{notae} of the true Church.  They are called such not because in them a part of the Church, as it were, emerges into visibility, but because it is assured to faith according to the Word of God that the means of grace where they are rightly administered will not remain without fruit.  For Lutheran doctrine the questions: \begin{displayquote}{\footnotesize What is the Church?}\end{displayquote} And: \begin{displayquote}{\footnotesize Who belongs to the Church?}\end{displayquote}  Are indistinguishable; for the Church is the communion of believers.\end{fancyquotes}

                Since the Church is essentially the communion of believers, it is invisible.  Walther refers to the following texts: \begin{displayquote}“\textit{The Kingdom \marginpar{\scriptsize Luke 17: 20-21}of God cometh not with observation: neither shall they say, Lo here! Or, lo, there! For behold, the Kingdom of God is within you}.”  \\\par “\textit{the Church \marginpar{\scriptsize 1 Peter 2:5}is a spiritual house in which spiritual priests offer up spiritual sacrifices, acceptable to God; and hence is invisible}.”\footnote{Die evangelisch-lutherische Kirche die ware sichtbare Kirche, p. 11.} \\\par “\textit{the Lord alone\marginpar{\scriptsize 2 Timothy 2:19} knows them that are His; now, only those who are the Lord’s constitute the true Church; hence no man can see the Church}”.\footnote{Kirche und Amt, p. 15.}\end{displayquote}  Walther writes in the first volume of the “\textit{Lutheraner}”\footnote{Lutheraner, p. 83}: \begin{fancyquotes} The Church is not a visible institution like a state, but an invisible kingdom, a spiritual building erected by the Spirit of God in the hearts of men... \par  ...It is indisputable\footnote{John 18:36; Luke 17:20-21} that the true Church of Christ is, properly speaking, never visible.  It cannot be otherwise.  For since only truly believing regenerate Christians are members of the Church, no one can say: these or those people are the Church; for everyone should and can become and be sure, as concerns himself, that he is in Christ and Christ in him; but no one can be infallibly sure concerning another man whether he is a child of God, whether he is a living stone of the spiritual house of God or the Church.  Even as Solomon says: \begin{displayquote}‘\textit{God only knoweth the hearts of the children of men}.’\footnote{2 Chronicles 6:30}\end{displayquote} Hence we confess: \begin{displayquote}‘I believe a Church’, ``\textit{but faith is the substance of things hoped for, the evidence of things not seen}’.\footnote{Hebrews 11:1}”\end{displayquote}  And though the men who form the Church can be seen, yet, since they are seen as bodily men, not as spiritual men who belong to the house of the Church\footnote{1 Peter 2:5}, it still remains true that the Church, as a spiritual house built up of spiritual men, is invisible.\footnote{Kirche und Amt, p. 22; Lutheraner, I, 21.}  Hence the holy Christian Church here on earth is invisible at all times, not only in times when the Papacy ruled, but also in times when the light of the Gospel shines brightly upon the nations.\footnote{Kirche und Amt, p. 21.}

                Through the preaching of the Word and the administration of the Sacraments the Church is indeed recognized in its presence but not visible in its essence, even as the soul clearly manifests its presence in the body, but without itself becoming visible.\footnote{Lutheraner, VI, 9; I, 83; VIII, 42.}\end{fancyquotes}

%%% Local Variables:
%%% mode: latex
%%% TeX-master: "../main"
%%% End:

\include{parts/part6b}
\include{parts/part7a}
\chapter{The Ministry II}

\hrule
\vspace{.30cm}
With this doctrine of the origin of the office of the ministry \textit{in concreto} and its implications many who want to be Lutherans have found themselves unable to agree.  And a reason for such disagreement they have pointed to the very expression “\textit{conferred}” as objectionable.  Walther never insisted on this expression as a \textbf{Shibboleth}\marginpar{{\scriptsize Shibboleth:\\A word, especially seen as a test, to distinguish someone as belonging to a particular nation, class, profession etc. of the correct doctrine.\\Typesetter's Note: See Judges 12:5-6}}  He showed, on the one hand, that this expression was not new but had been used by the old orthodox teachers.  On the other hand, he is willing to acknowledge as orthodox on this point everyone who holds that the congregation originally possess the office, and that it is not conferred by one minister upon another, but comes through the election and call of the congregation.
\vspace{.30cm}
\hrule
\vspace{1.25cm}
He remarked on this point in 1873: \begin{fancyquotes}It is continually objected against us, even on the part of best-intentioned critics, as by Pastor Lohrmann in Müden, that we seem to make a particular ‘\textit{form of the conferral theory our Shibboleth, and thereby threaten to decline into a peculiar separatistic position over against all the rest of the Lutheran Church upon earth}’. \par But, thank God, it is not so!  In whatever form other Lutherans may speak of the office and of its conferral we will offer them the hand of church fellowship if only they confess with us the office of the keys as it is laid down, over against the papacy, in our Confession, particularly in the Smalcald Articles, and thus do not deny that not the office-holders but the Church, originally possesses the keys or the office and confers it through the call, so that the pastoral office is not a privileged self-perpetuating order which exists alongside of the Church. \par But whoever denies this, or, although he makes a pretence of admitting it, nevertheless declares our doctrine to be fanatical, while he, for instance, hides behind the invisible Church as a whole, and thus shows that he fundamentally still holds an essentially different doctrine to be correct, with such indeed we cannot work together.\footnote{L.u.W., 19:366 f.}\end{fancyquotes}

        Against the matter itself it has been contended that one becomes involved in contradictions by the doctrine of a conferral of the office of the ministry on the part of the congregation.  It has been put this way: \begin{displayquote}{\footnotesize If the Christians confer the office of the ministry as something which they had before and which the minister is to conduct in their stead, they must all previously have been minister or pastors.}\end{displayquote}  This oft repeated objection is not exactly very clever.  For it ignores the most ordinary analogies. The American citizens through their vote confer the presidency of the United States upon a particular individual without any necessity of their having previously been presidents themselves.  But let us hear Dr. Walther.  He writes: \begin{displayquote}“\textit{We also assert that the calling Christians are not pastors but simply the priestly generation of the New Testament, in whom all ecclesiastical power of office originally rests, through the conferring of which upon certain persons for the public exercise of the same according to God’s ordinance these persons become something which the Christians are not, namely pastors; even as free citizens possessing the right of suffrage are not civic officials but simply the free citizens in whom all civic power of office originally rests, through the conferring of which upon certain persons for the public exercise of the same these persons likewise become something which the citizens are not, namely, civic officials}”.\footnote{L.u.W., 19: 365f.}\end{displayquote}

                Another form of this objection is as follows: \begin{displayquote}{\footnotesize Since the Christians are supposed to possess the office of the keys through Baptism and faith, they could not get rid of the office of the keys without the necessity of “\textit{washing away their Baptism}” and “\textit{rooting out their faith}”.}\end{displayquote}  Besides, the circumstance that the Christians bear the Gospel upon their lips would indicate that they still had the office of the keys.  Otherwise a division of the office of the keys would have to be assumed.  Then the question would arise: “\textit{According to what proportion and relation” the division should take place}.  Walther answers: \begin{fancyquotes}The solution of all the above named difficulties and contradictions in which the doctrine of conferral is supposed to involve its adherents lies simply in the fact that the ministers are servants of the congregation.  As the mistress of the house is not ‘\textit{stripped}’ of her power when she engages servants to whom she commits its exercise, so also the Church of the believers is not deprived of anything; with this difference, that, whereas it is at the option of the mistress of the house whether she chooses to engage such servants, the Church has a \textbf{mandatum divinum} {\scriptsize\textsc{(divine order)}} to this effect.  The question ‘\textit{according to what proportion and relation}’ the Christian has and holds the office over against the minister is answered by the Fourteenth Article of the Augsburg Confession.\footnote{L.u.W. 16:182.}\end{fancyquotes}

Concerning the relation of the office of the ministry to the ministry to other offices in the Church Walther teaches: \begin{displayquote}“\textit{The ministry is the highest office in the Church, from which, as its stem, all other offices of the Church issue}”.\footnote{K.u.A., Thesis VIII, p. 342. Walther and the Church, p. 27}\end{displayquote}  The correctness of this Thesis, which is found \textit{verbotenus} {\scriptsize (word-for-word, Ed.)} also in the Lutheran Confession\footnote{Apology, Art XV., Müller, p. 213. Triglotta, p. 327}, is clear already from the fact that the office of the ministry has the public administration of the keys of the Kingdom of Heaven, which comprise in themselves all ecclesiastical power.  So there can be no office in the Church which stands above the office of the ministry.  Rather is every other office in the Church merely an auxiliary office, which stands at the side of the office of the ministry, whether it be the office of such elders as do not labor in the Word and doctrine {\scriptsize\textsc{(1 Timothy 5:17)}}, or the office of ruling {\scriptsize\textsc{(Romans 12:8)}}, or the diaconate {\scriptsize\textsc{(office of serving, in the narrower sense)}}, or whatever other offices in the Church may be committed to certain persons for their particular administration.  \par Hence those who administer the office of the holy ministry in the Church are called in Scripture elders, bishops, overseers, stewards, etc., and the holders of a subordinate office are called deacons, i.e., servants, not only of God, but also of the congregation and of the bishop, and only of the latter in particular is it said that they take care of the Church of God and watch over all souls as they that must give account \marginpar{{\scriptsize 1 Timothy 3:1, 5,7; 5:17;\\ 1 Corinthians 4:1;\\ Titus 1:7;\\ Hebrews 13:17.}}  Thus also there can be no \textit{jure divino} {\scriptsize\textsc{(by divine right)}} superiority and subordination among those who hold the office of the ministry, but all are on the same level.  Any superiority or subordination is only of human right.\footnote{ K.u.A. p. 342 f.}

                With regard to the rights of the office of the ministry it is to be said that reverence and unconditional obedience is due to this office when the minister speaks God’s Word.  Upon this Walther most urgently insists.  He has been accused of having made the ministers servants of men, with whom the congregations could deal according to their own pleasure, through his teaching concerning the relation of the office of the ministry to the Christian estate.  This accusation is completely unjustified.  Walther from the beginning until his end never surrendered a jot or tittle of the rights which God’s Word ascribes to the office of the ministry.  But let us hear his own words: \begin{fancyquotes}Although the incumbents of the public ministry do not form a more holy order, distinct from the ordinary order of Christians, but merely exercise the universal rights of Christians, with the public and orderly administration of which they have been commissioned, still they are not servants of men on that account.  The principal efficient cause of the ordinance of the public office of preaching is God, the Most High, Himself.\par  This ordinance is not an arrangement which men in their wisdom have instituted for propriety’s sake and for salutary reasons, but it is an institution of the Triune God, The Father, the Son, and the Holy Ghost.  Therefore, when official authority has been conferred on a person by the congregation by means of a regular, legitimate call, that person has been placed over the congregation by God Himself, although it was done through the congregation.\footnote{1 Corinthians 12:28; Ephesians 4:11; Acts 20:28.}  The person installed is henceforth not only a servant of the congregation but at the same time a servant of God, an ambassador in Christ’s stead, by whom God exhorts the Christian congregation.\footnote{1 Corinthians 4:1; 2 Corinthians 5:18-20.} \par Accordingly, when a preacher is ministering God’s Word in his congregation, whether he be teaching or admonishing, reproving or comforting, publicly or privately, the congregation hears from his mouth Jesus Christ Himself and owes him unconditional obedience as to a person by whom God wants to make known His will to them and guide them to eternal life.  The more faithfully the preacher discharges his office, the greater must be the reverence of which the congregation deems him worthy.\footnote{K.u.W., p. 360f. “Walther and the Church, p. 80.}\end{fancyquotes}  Therefore also Walther from the beginning protested against the calling of ministers until further notice {\scriptsize\textsc{(auf Kündigung)}} which had become a rather general custom in America.  This he denounced as a shameful contempt of the divine order of the ministerial office and a degrading of the ministers to the position of servants of men.  The congregation can and should depose a minister from his office only when it is evident that the principal cause of the office of the public ministry, namely God Himself, has deposed him from office, that is, in cases where the minister has become guilty of false teaching or offensive life.  Walther says on this subject: \begin{displayquote}“\textit{Moreover the congregation has no right to take away his office from such a faithful servant of Jesus Christ; if it does so it thereby rejects Christ Himself in whose name he presided over the congregation.  Only then can the congregation remove an incumbent of the office from his office when it is evident from God’s Word that the Lord Himself has deposed him as a wolf or hireling}”.\end{displayquote}  In his “\textit{Pastorale}”\footnote{Pastorale, p. 41f} Walther treats in detail of the usage obtaining specifically in America, \begin{displayquote}“\textit{that the ministers are called only temporarily, that is either with the provision that they may be dismissed at will, or only for a specific term of one or more years, or ‘until notice’, so that at a specified interval from the day the notice is given they are to withdraw from office}”.\end{displayquote}  Walther’s judgement is that a congregation has neither the right to issue such a call nor is a preacher authorized to accept it. \begin{fancyquotes}Such a call conflicts, in the first place, with the divinity of a rightful call to an office of ministry in the Church, which is clearly attested in God’s Word\footnote{Acts 20:28; Ephesians 4:11; 1 Corinthians 12:28}.  For if God is really the one who calls the ministers, then the congregations are only the instrumentalities for the selection of persons for the work to which the Lord has called them\footnote{Acts 13:2}.  When this selection has now taken place, then the minister stands in God’s service and office, and no creature can depose God’s servant from his office or dismiss him unless it can be proved that God Himself has deposed him from his office and dismissed him \footnote{Jeremiah 15:19 compared with Hosea 4:6}, in which case the congregation does not really depose or dismiss the minister, but only carries out God’s deposition or dismissal which has become evident to it.
\par
        If the congregation nevertheless does this at its own pleasure, it then makes itself, instead of God’s instrument, a mistress of the office and usurps God’s own rule and economy...  But the minister who gives to a congregation the right thus to call and dismiss him at its pleasure {\scriptsize\textsc{(discretion)}} thereby makes himself an hireling and a servant of men.  Such a call conflicts also with the \begin{displayquote}“\textit{honor and obedience, which the hearers are to render to the holders of the divine office of the ministry in accordance with God’s Word;\footnote{Luke 10:16; Hebrews 13:17; etc.} for if the hearers really possessed that assumed fullness of power, then it would stand entirely in their own power to release themselves from the rendering of that honor and which God requires of them}.''\end{displayquote}

To be sure, the enjoining and commanding on the part of the ministers and the obedience on the part of the congregation extend only as far as God’s Word.  For anything which is not commanded in God’s Word the preacher may demand no obedience.  If he does this he usurps a lordship in the Church for his own person and overthrows the cardinal principle that the Christians are subject only to Christ but among themselves are brethren.

{\color{Black} Hence also the so-called \textit{constitutive ecclesiastical power}, that is, the power to arrange matters of indifference, belongs not to the minister, but to the entire congregation, that is, to the minister with the congregation.\footnote{Pastorale, p. 365 ff.}\end{fancyquotes}
\par The demand on the part of the preacher that by virtue of the Fourth Commandment he is entitled to obedience also beyond the Word of God is papistical error.  Walther sets up the Thesis in his “\textit{Kirche und Amt}”: \begin{displayquote}“\textit{The preacher may not dominate over the Church; he has accordingly no right to make new laws and to arrange indifferent matters and ceremonies arbitrarily}”.\end{displayquote}  In the “\textit{Proof from the Word of God}” he cites the passages, {\scriptsize
  \textsc{Matthew 20:25-26; Matthew 23:8; John 18:36}}, and continues:}


\begin{fancyquotes}We see from this that the Church of Jesus Christ is not a dominion of such as command and such as obey, but it is one great, holy brotherhood in which no one can dominate and exercise force. Now, this necessary equality among Christians is not abolished by the obedience which they render to the preachers when these confront them with the Word of Jesus Christ; for in this case, in obeying the preachers, they do not obey men but Christ Himself. \par Just as certainly, however, this equality of believers would be abolished and the Church would be changed into a secular state if a preacher would demand obedience also when he presents to the Christians, not the Word of Christ, who is his and all Christians’ Lord and Head, but something which by virtue of his own understanding and experience he considers good and appropriate.\par  Hence the moment there is a discussion in the Church about matters indifferent, that is, such as are neither commanded nor forbidden in God’s Word, the preacher may never demand unconditional obedience for something which appears best just to him.  In such a case it is rather the business of the entire congregation, of the preacher together with the hearers, to decide the question whether what has been proposed should be accepted or rejected.  It is, however, due the preacher, by reason of his office of teacher, overseer, and watchman, to guide the deliberations that have been instituted, to instruct the congregation regarding the matter, to see to it that in settling indifferent matters and arranging order and ceremonies of the church nothing is done in a trifling manner and nothing harmful is adopted.\footnote{K.u.A., p. 370f. – Walther and the Church, p. 81 f.}\end{fancyquotes}  The holy apostles forbid the preachers to lord it over the people, that is, the congregations: {\scriptsize\textsc{1 Peter 5:1-3; 2 Corinthians 8:8; 1 Corinthians 7:35}}. \begin{fancyquotes}When the holy apostles, notwithstanding these statements, among other things write this: \begin{displayquote}‘\textit{The rest will I set in order when I come}’\footnote{1 Corinthians 11:34},\end{displayquote} it is evident from the foregoing that they made arrangements in regard to indifferent matters not by way of commands but by offering their advice and with the consent of the entire congregation.\end{fancyquotes}  As is well known, the recent Romanizing Lutherans ascribe to the ministers the power to make ordinance in the Church on their own authority alone, for which they appeal partly to passages such as {\scriptsize\textsc{Hebrews 13:17}}: “\textit{Obey them that have the rule over you, and submit yourselves}” \par So \textbf{Grabau}: \begin{fancyquotes}Lutheran Christians know that when God’s Word says, \begin{displayquote}‘\textit{Obey them that have the rule over you, and submit yourselves}’,\end{displayquote} it deals not only with preaching, but with all good Christian matters and occasions which God’s Word brings with it and requires, and which pertain to the good government of the churches and also to Christian welfare in life and work, and that honor, love, and obedience according to the Third and Fourth Commandments is demanded... Here the required obedience is in every respect a matter of conscience; but through the Holy Ghost also a willing and cheerful obedience on account of the believing recognition of what is good in the grace of Jesus Christ.\footnote{Colloqium, p. 20}\end{fancyquotes} ---  and in part adduce such passages as {\scriptsize\textsc{1 Peter 2:13}}: \begin{displayquote}“\textit{Submit yourselves to every ordinance of man for the Lord’s sake}”.\footnote{So Superintendent Münchmeyer.  L.u.W., 16, 184.}\end{displayquote}  With regard to the first passage Walther says with the Apoligy: \begin{displayquote}``\textit{Here nothing is said of the ordinances of men, but of teaching the Word of God. So also this passage does not establish a rulership apart from the Gospel}”.\footnote{K.u.A., p. 373.}\end{displayquote}  With regard to the application of {\scriptsize\textsc{1 Peter 2:13}} Walther says: \begin{displayquote}“\textit{To understand under ‘ordinances of man’ in this place the arrangements made by a preacher is a perversion which exceeds all bounds}”.\footnote{L.u.W., 16, 184.}\end{displayquote} This passage speaks of the ordinances of civil government in secular affairs!
\divider
In de-limiting the sphere of authority between congregation and office of the ministry Walther thoroughly examined in particular two points.  They are the questions: \begin{displayquote}{\footnotesize\textit{“To whom belongs the right to impose excommunication?”}} {\footnotesize \&} {\footnotesize\textit{“Who has the right to pass judgement on doctrine?”}}\end{displayquote}  Both questions had to be discussed in connection with the controversy with Pastor Grabau.\footnote{Cf. Buffalo Colloquy, p. 21,22.}

        With regard to the first question Walther insists: \begin{displayquote}“\textit{The preacher has no right to impose and execute excommunication alone, without a previous verdict of the entire congregation}”.\footnote{Thesis IX, C.; K.u.A., p.383. Walther & the Church., p. 83.}\end{displayquote}  Walther, as is characteristic of him, first gives fitting emphasis to the rights of the ministerial office.  For him it is certain “\textit{that the power of the keys in the narrower sense, namely, the power to loose and to bind”, and then hence according to the Word of the Lord and His sacred ordinances the public execution of excommunication belongs to, and must remain with, the incumbent of the public ministry}”.\par  Nevertheless, “\textit{according to the express prescription and order of the same Lord, the investigation preceding the execution of excommunication and the final judicial verdict must come from the entire congregation, that is, from the teachers and hearers}”.\footnote{Matthew 18:15-20.} \par  After citing this passage Walther continues: \begin{fancyquotes}Evidently here Christ, as our Confessions put it, gives the highest jurisdiction to the church, or congregation, and wants a sinner in the congregation to be regarded as an heathen man and a publican, and the awful judgement of excommunication to be executed upon him, only after several fruitless private admonitions and after he has been admonished in vain also publicly, in the presence of, and by, the whole congregation, and therefore his expulsion from their fellowship has been unanimously resolved upon by them and has been executed by the preacher of the congregation. \par In accordance with this procedure, then, even Paul would not excommunicate the incestuous person at Corinth without the congregation, but, in spite of his having declared this great sinner worthy of excommunication, he wrote the congregation that this must be done by them `\textit{when they were gathered together}’.\footnote{1 Corinthians 5:4}\footnote{K.u.A. p. 384. \& Walther and the Church. P. 83}\end{fancyquotes}  Hence Walther also passes the judgment: “\textit{An excommunication which has been resolved by a mere majority to the exclusion of the minority, not unanimously, with even the silent consent of all members, is illegitimate and invalid}”.\footnote{Pastorale, p. 348.}

        But also here Walther is very careful not to go beyond the rightful bounds.  An excommunication which has been imposed by a presbytery or consistory with the knowledge and consent of the people he declares to be valid and legitimate.  He remarks\footnote{ K.u.A., & Walther and the Church, l.c.}: \begin{displayquote}“\textit{It will go without saying that what the congregation through ‘many’ and ‘before all’\footnote{2 Corinthians 2:6; 1 Timothy 5:20} did at the time of the apostles can be validly and legitimately done also where the ruling congregation is represented by a presbytery or consistory, composed of clergymen and laymen, so that the presbytery or consistory alone renders the verdict of excommunication, provided only that is done with the knowledge and consent of the people}”.\end{displayquote}  And yet Walther most decidedly advises against the introduction of this arrangement in our American congregations.  And this he does also for the reason that the right to exclude impenitent sinners may not in this manner get away from the congregations altogether, as it has for the most part come about in the State Churches.  As concerns the right to judge doctrine, “\textit{no-proof}”, says Dr. Walther, “\textit{is needed}” that also this belongs to the office of the public ministry. ``\textit{According to divine right the function of passing judgment on doctrine belongs to the ministry of preaching}”.  Indeed, without this function the preachers could not at all discharge their office.  It is certainly the duty of the office of the public ministry not only to present the correct doctrine, but also to expose, refute, and warn against the false doctrine, if it is to achieve its purpose of leading souls, in spite of all sorts of seduction, unto final salvation.  But by the establishment of the special office for passing judgment on doctrine this right has not by any means been taken away from laymen.\footnote{Loehe and Grabau wanted to grant a seat and voice in ecclesiastical tribunals and councils (synods) to pastors only.  The latter says: “\textit{You shall leave the judgment of doctrine to those to whom according to the Twenty-eighth Article (?) of the Augsburg Confession it properly belongs}”. (Zweiter Synodalbrief. Colloqium, p. 22.)}  \par Rather does Scripture make the exercise of this right their most sacred duty.  This is proved, first, by all those passages of Holy Scripture in which this judging is enjoined also upon ordinary Christians.  For instance, thus writes the holy Apostle Paul: \begin{displayquote}“\textit{I speak as to wise men’ judge ye what I say.  The cup of blessing which we bless, is it not the communion of the blood of Christ}?”\footnote{1 Corinthians 10:15-16.}\end{displayquote}  Again: \begin{displayquote}“\textit{Try the spirits whether they are of God}”\footnote{1 John 4:1.  Cf. 2 John 10-11; 1 Thessalonians 5:21}\end{displayquote} The proof is furnished, furthermore, by all those passages in which Christians are exhorted to beware of false prophets, such as {\scriptsize\textsc{Matthew 7:15-16; John 10:5}}, in such passages in which they are praised for their zeal in testing doctrine {\scriptsize\textsc{(Acts 17:11)}}.  \par Lastly, we have an account in the Acts of the Apostles stating that at the first apostolic council laymen were not only present but also spoke, and that the decisions reached on this occasion were made by them as well as by the apostles and elders and were sent in their name as well as that of the apostles.  Hence there is no doubt that laymen have a seat and voice in church jurisdiction and at synods with the public ministers of the Church.\footnote{K.u.A., p. 298 f. Walther and the Church, p. 85 f.} To take away or even to diminish this right of the laymen is an accursed church-robbery and has as its consequence that it becomes impossible any longer to withstand the intrusion of false doctrine.\footnote{K.u.A. p. 400 f.}


\chapter{Church Government}

\hrule
\vspace{.30cm}
Concerning the Church and Church-Government Walther teaches: “\textit{With the keys of the kingdom of heaven every Evangelical Lutheran local congregation has the entire church power which it needs, that is, the power and authority to perform everything that is requisite for its government}”.\footnote{Die rechte Gestalt, etc., p. 24, Walther and the Church, p.21}
\vspace{.30cm}
\hrule
\vspace{1.25cm}
Also the so-called constitutive power, that is, the ordering of all things which are not ordered through God’s Word {\scriptsize\textsc{(adiaphora)}} belongs to the congregation itself, not to the pastor, and not to persons outside the local congregation.  The local congregation possess the supreme jurisdiction in its own sphere.\footnote{Pastorale, p. 365}  The jurisdiction which persons outside the local congregation have over it and its pastors is only of human right.\footnote{Rechte Gestalt, etc. p. 30}.  All congregations and pastors have of themselves equal ecclesiastical power, and no congregation is of itself superior or subject to another congregation, nor is any pastor of himself superior or subject to another pastor.\footnote{L.c., p. 212}  An organizational joining together of a number of congregations into a larger church body, e.g., by means of a synod with powers of visitation, a church council, a consistory, a bishop, etc., is not of divine but of human authority, and hence not absolutely necessary.\footnote{Pastorale, p. 393 f.} \par Every congregation may maintain its independence.  That every congregation is of itself independent is pure Lutheran teaching, not separatistic, as it is frequently called today.  Separatistic teaching is that each congregation shall be and remain independent\footnote{Rechte Gestalt, etc. p. 22}.  That a local congregation, in order to possess and exercise all church powers, must be externally connected with other congregations and together with them stand under one church government, and thus be dependent on other congregations, is an error upon which the papacy is founded.\footnote{L.c., p. 19f., Pastorale, p. 393}.  Moreover upon this assumption we should never be sure how large a church body would have to be in order to possess all church power.  But it is not so, for every local congregation possesses with the keys also all church power.  As no one dare impose anything upon an individual Christian against his will, so also the same rule holds with an individual congregation.  Synods, consistories, or church councils can have only advisory power over against the individual congregations. \par Every congregation must also retain the right at any time to withdraw from its connection with a larger church body, and to retrieve the rights delegated to others {\scriptsize\textsc{(e.g., consistories)}}, just as it may otherwise undertake such alterations in matters of indifference as may appear advisable to it.  Those who wish to establish a church government which by divine right stands over the individual congregations, and upon which therefore the individual congregations should be dependent, thereby deny the word: “\textit{One is your Master, even Christ, and all ye are brethren}”, and want to introduce another authority than the authority of God’s Word into the Christian Church.

                \begin{fancyquotes}They rob Christ’s Church of that liberty which He has obtained for her with so costly a price, with His own divine blood, and degrade this free Jerusalem which is above, in which only kings, priests, and prophets exist, this kingdom of God, this heavenly kingdom of truth, to a political institution, in which a man must submit to every human ordinance.  They aspire to the royal crown of Christ, the true and only King, they make themselves kings over His kingdom; they drive Christ, the true and only King, they make themselves kings over His kingdom; they drive Christ, the true and only Master, from His seat and set themselves up for in His Church; they strive to sever Christ, the true and only Head, from His body, the Church, and assume the authority of heads to His spiritual body.  They exalt themselves above the holy Apostles, and arrogate to themselves powers which are clearly denied them in the Word of God, nay, which God has conferred upon no man whatever, no creature, not even angels and archangels.\footnote{Brosamen, p. 523. Translation in June 15, 1949, Okabena Lutheran, pp. 4,5}\end{fancyquotes}  Hence church polity is a matter of indifference only so long as it does not deprive the Christians of their Christian rights bestowed upon them by Christ.\footnote{Brosamen, p. 496. K.u.A., p. 371.}

                Nevertheless, so Walther further declares, every congregation should be ready to unite with other orthodox congregations when there is opportunity for such union and this tends to serve and promote the glory of God and the upbuilding of His kingdom.  Every congregation should on its part endeavor to keep the unity of the Spirit, and provide that the gifts of the Spirit shall be manifest unto the common good, and that in every way the purposes of the Kingdom of God in general should be furthered.\footnote{ (Rechte Gestalt, etc. p. 212 ff. “Walther and the Church”, p. 115} The congregation will achieve these ends if it unites with other congregations into a larger church body, when, for instance, it enters into a synodical fellowship with other congregations \begin{displayquote}“\textit{for mutual fraternal consultation, inspection, and assistance, and for a united cooperation in spreading the kingdom of God}”.\footnote{Brosamen, p. 524. June 15, 1949, Okabena Lutheran, p. 6}\end{displayquote}  In his “\textit{Pastorale}”\footnote{Pastorale p. 69}, therefore, Walther calls the following to the attention of the pastors: \begin{displayquote}“\textit{After his ordination a pastor who has entered into the office should at his first opportunity join an orthodox synod.  If he should fail to do so when opportunity offers, he would thereby betray a sinfully separatistic, schismatic spirit, contrary to {\scriptsize\textsc{Ephesians 4:3; 1 Corinthians 1:10-13; 11:18-19; Proverbs 18:1}}}”.\end{displayquote}  And in another place, after he has first rejected the idea that a church-government organization of a number of congregations into a larger church body is of divine right and therefore absolutely necessary, Walther remarks: \begin{displayquote}“\textit{Nevertheless a preacher who, insisting upon his freedom, would with his congregation remain independent, even though an opportunity were offered him to join an orthodox synod, would thereby act contrary to the purpose of his office, the welfare of his congregation, and his duty to the church at large, and would reveal himself as a separatist}”.\footnote{Pastorale, p. 397.}\end{displayquote}  A Pastor should therefore endeavor to prevail upon his congregation, if it is still without synodical connection, to join a synod.  To be sure, a pastor must do this only by way of patient instruction, pointing out the true character of a synod.  Walther writes with reference to this point: \begin{fancyquotes}The pastor is indeed to endeavor to prevail upon his congregation to join the synod, but great caution is to be exercised in this endeavor; the congregation is first to be instructed concerning the significance of a synod, and is to be given time, in order that it may not form the opinion that it merely a matter of leading it with burdens diminishing its freedom, tricking it out of its church property, an subjecting it to the yoke of a so-called spiritual government.  Rather it is to be shown that this is purely a matter of its own welfare and its duty to care for its children and posterity and for the Kingdom of God in general, and finally, that a synod desires to be merely an advisory, auxiliary body, not a body which exercises dominion over the individual congregations.\footnote{Pastorale, p. 400f.}\end{fancyquotes}

                The proof that an ecclesiastical fellowship can very well exist, do the work of the church, and gloriously flourish on the basis of these principles is offered by the example of the Missouri Synod itself.\footnote{Editor would remark-Old Missouri itself, as long as she held to the basic principles, from which she has definitely and violently departed as of this date.}  Walther says in his synodical address of the year 1848: \begin{fancyquotes}One thought, however perhaps agitates the minds of us all, with some in a greater, with others in a lesser degree, and induces the anxiety that our transactions might easily remain fruitless; I mean the thought that our constitution, which forms the basis of our synodical connection, invests us with no other power besides that of deliberation, it confers upon us no authority but that of the Word and persuasion.  Pursuant to our constitution we have no right to issue decrees, to enact laws or orders, or in any way to deliver a judicial decision in matters imposing any duty upon the congregation, so that they should be forced absolutely to submit.  Our constitution by no means constitutes us a kind of consistory, or highest tribunal of our congregations. \par  On the contrary, it assures to them the most perfect liberty, nothing excepted save the Word of God, faith, and love.  According to our constitution we do not occupy a position above our congregations, but we stand among them and by their side.  Does it not seem, then, as though it were made utterly impossible for us to exercise a thorough-going, salutary influence upon our congregations?  Do we not, by reason of the relations entered into, run the hazard of wearying ourselves with labors that might but too easily prove utterly fruitless, as none are compelled to comply with our resolutions?\end{fancyquotes}  Walther answers this question with \textbf{No!} and then treats the question: \begin{displayquote}``\textit{Why shall and can we pursue our work joyfully, although we are possessed of no power but that of the Word}?''\end{displayquote}  He shows that Christ has given His servants no other power than the power of the Word, but that his power is also fully sufficient for the building of the Church.  Says Walther: \begin{fancyquotes}When a preacher is invested only with the power of the Word, but with its full power, and where the congregation receives his word as God’s Word, whenever he delivers to them the Word of Christ, just there the minister stands in the proper relation to his congregation; he performs his duties not as a hired mercenary, but as an ambassador of the Most High; not as a servant of men, but of Christ, teaching, exhorting, and rebuking in the place of Christ.  Just there the apostolic exhortation is obeyed: \begin{displayquote}‘\textit{Obey them that have the rule over you and submit yourselves}’\end{displayquote}  But the more a congregation perceives that the person who is over them in the Lord desires nothing but that the congregation exercise subjection to Christ and His Word; the more it sees that he does not desire to lord it over them, nay, that he even watches over the liberty of the congregation with a jealous eye, the more willing will it become to listen to his salutary proposals, even in matters left free by God... \par This very same expectation of salutary influence our synodical body may entertain, if it seeks to effect its objects by means of no authority save that of the Word.  Of course we shall have to encounter struggles and contests, but it will not be those little discouraging struggles for obedience to human ordinances, but high and holy struggles for the Word of God, and hence for the honor and kingdom of God.  And the more our congregations learn to know that we desire to exercise no authority over them but the divine power of the Word which saves all who believe in it, the more readily will our advice and counsel find an open door with them.  It is true, all that do not like the Word will separate themselves from us; but to those who love it our communion will be comfortable refuge; and in sanctioning resolutions they will not bear them as a strange burden imposed upon them by outward force, but will regard them as a blessing and a gift of brotherly love; they will adopt, defend, and keep them as their own proper possession.\footnote{Brosamen, p. 518-527.  Translated in June 15, 1949, Okabena Lutheran, pp. 2,7. A history of nearly fifty years confirms these words of Walther (1847-1890.}\end{fancyquotes}
\chapter{Church and State}

\hrule
\vspace{.30cm}
With regard to the relation of the Church to the State Walther teaches that the Church should be independent of the State, that is, that it should govern itself in all respects. ``\textit{As important as it is}”, he writes, in \textit{Die Rechte Gestalt}\footnote{Die Rechte Gestalt, p. 5 f.}, \begin{displayquote}“\textit{that the government of a land in which the orthodox Church has its dwelling should also belong to the Church, as great a blessing as this can be for the Church, nevertheless the separation of the Church from the State is not a defect or an irregularity, but the correct and normal relation which the Church should always bear toward the State}.”\end{displayquote}
\vspace{.30cm}
\hrule
\vspace{1.25cm}
                As proof for his position Walther appeals in the first place to the fact that “\textit{according to God’s Word Church and State are entirely separate domains and hence not to be mingled the one with the other}”.\footnote{John 18:36; 2 Corinthians 10:4; Matthew 22:21; Luke 12:13-14}  The most complete exposition by Walther concerning the utter dissimilarity of Church and State and the consequent separation of Church and State is contained in a synodical address on {\scriptsize\textsc{John 18:36-37}}.  Here Walther says: \begin{fancyquotes}Church and State are, according to God’s Word, as different from each other as heaven and earth.  The State is a kingdom of this world, hence an earthly kingdom; the Church, however, is ‘\textit{not from hence}’, not an earthly kingdom, it is as the Lord so often says, the ‘kingdom of heaven’ upon earth.  The State is an external, physical, visible kingdom, the Church an internal, spiritual, invisible kingdom, for, as Christ says with plain words, \begin{displayquote}\textit{‘the kingdom of God cometh not with observation; neither shall they say, Lo here! Or lo there! For, behold, the kingdom of God is within you’.}\end{displayquote}  The State has as its members all who allow themselves to be taken up externally into its association, bad and good, ungodly and pious, unbelievers and believers, non-Christians and Christians; the Church, on the contrary, has only those as members who are Christ’s sheep, who hear His voice and from their hearts believe on Him.  The State has for its purpose only the earthly welfare of men, protection of the body, property, and honor of its citizens, and external quietness, peace, discipline, and order in this world; the Church, on the contrary, has for its purpose the peace of men with God, protection against sin, death, devil, and hell, eternal righteousness, eternal life, and eternal blessedness.  The State has as its norm the light of nature or of human reason; the Church has the light of the immediate divine revelation embodied in the Holy Scripture.  The State has for laws those which it makes itself; the Church gives no laws, but only urges the eternal laws of God.  The State reproves only the outward evil deed; the Church reproves also the ungodly attitude of the heart.  The State allows everything which its earthly purposes demand or at least permit;\footnote{See ``Walther's Note''} the Church allows only what God in His Word declares allowable.  The State commands on its own authority and hence demands obedience to its commands on the basis of its official power; the Church commands nothing on its own authority and demands obedience only to the commands of Christ.

                The State has as its means and weapons, the bodily sword and external power of compulsion; the Church has the sword of the Spirit, namely the Word of God and the power of conviction through this Word.  The State has as its component parts government and subjects, those who command and those who obey; in the Church all are equal and subject one to another by love alone; even as Christ says in plain words to His disciples: \begin{displayquote}\textit{‘One is your Master, even Christ; and ye all are brethren.  It shall not be so among you: but whosoever will be great among you, let him be your minister’.}\footnote{Brosamen, p. 498.}\end{displayquote}\end{fancyquotes}\footnote{Walther's Note --- Here Walther remarks in a note: \begin{displayquote}\textit{Moses in his political laws had to allow divorce even aside from the case of adultery {\scriptsize\textsc{(Deuteronomy 24:1)}} because of the hardness of heart of the Jews {\scriptsize\textsc{(Matthew 19:7-9)}}; but the prophets reproved the use of this license by those who wanted to be members of the Church, according to {\scriptsize\textsc{Malachi 2:14-16}}.\end{displayquote}} We add a weighty utterance of Walther concerning this point from the Report of the {\scriptsize\textsc{Western District, 1885, p. 21 [Essays for the Church, Vol. II, pg 273 ]}}\begin{displayquote}  “\textit{Let it be noted that our Church does not teach that the secular government has no right to allow anything, that is to declare it exempt from punishment, which God has forbidden.  The State has that right, to be sure.  Even Moses, as a political lawgiver, allows much which the prophets forbid.  The government does not have only Christians under it, who are ruled by the Word of God; it is also not to rule the State, which is no institution for the salvation of souls but for the protection of body and property, according to the Word of God, but in accordance with reason.  But a prohibition of God does not lose its binding through the allowance of the government.  When the government, for instance, licenses sinful amusements, divorces on invalid grounds, the conducting of saloons, a Christian can make no use of this allowance.  The government must allow such things because of the ‘hardness of heart’ of its subjects in order to prevent rebellion, murder, and manslaughter.  Hence when the Pharisees, to adorn their false doctrine of divorce, submitted to Christ the question:\begin{displayquote}‘Why did Moses then command to give a writing of divorcement, and to put her away?’\end{displayquote} Christ answered:\begin{displayquote} ‘Moses because of the hardness of your hearts suffered you to put away your wives: but from the beginning it was not so}’.  {\scriptsize\textsc{(Matthew 19:7-8)}}\end{displayquote}\end{displayquote}}}Now, because Church and State according to God’s Word are so fundamentally different-- ``\textit{their entire character and nature are different, different are the requirements of their members, different their aim, norm, rule, their commands and prohibitions, their freedoms, their power, their means, the mutual relationships of those who belong to them, in short, their entire quality}” -– therefore the Church can neither be governed according to civil principles nor the State by ecclesiastical principles, that is to say, State and Church must remain unmixed, or, the Church shall be independent of the State.

                Walther further sets up the following statements concerning the relation of Church and State: \begin{displayquote}``\textit{Government officials, if they are believers, are also in the Church, but not as officials with their laws and their external authority, but as Christians and brothers, and hence equal in power and privilege with all other church members, even if they are princes, kings, or emperors}.''\footnote{Matthew 23:8; Luke 22:25-26; Galatians 3:28}\footnote{Die Rechte Gestalt, p. 8; Brosamen, p. 500. Walther remarks on the last passage in a note: “\textit{Even in the middle of the Fourth Century the ancient teacher of the church Optatus of Mileve wrote: ‘The State is not in the Church, but the Church is in the State’}.”}\end{displayquote}

                The civil government has indeed the duty over against the Church to guard it in its freedoms and rights against all outward force, to afford the Church as a society in the State the same protection which all other societies in the State enjoy.  In this way the civil government in our land fulfills its duty toward the Church.  “\textit{Our civil government here}” – says Walther\footnote{Brosamen, p. 507} – \begin{displayquote}“\textit{is indeed, as Isaiah prophesied, a nursing father and nursing mother also of our Church, for it powerfully protects us here in accordance with its office against all outward force, against the bloodthirstiness of the Antichrist and his minions, as against the murderous desires of the atheists of this last age of apostasy}”.\end{displayquote}  And persons in civil authority have this obligation in double measure when they are themselves members of the Church, as indeed every Christian should place his gifts into the service of Christ and His kingdom.\footnote{Report of the Western District, 1885, p. 28.} For as the rich man serves the Church with his riches, and the artist with his art, so should also persons in civil authority, if they are Christians, serve the Church with their power and reputation.\footnote{Pastorale, p. 368. Western District, p. 27.}  This is also the meaning of the Smalcald Articles when they say: \begin{displayquote}“\textit{Especially the chief members of the Church, kings and princes, ought to guard the interests of the Church, and to see to it that errors be removed and consciences be rightly instructed}”.\footnote{Mueller, p. 339; Triglotta, p. 519}\end{displayquote} --which words, as the word “\textit{especially}” already shows, speaks of a general duty of Christians\footnote{Western District, p. 29}, and ascribe to the princes not so much rights and powers over the Church, but rather duties toward it, and so much the greater as their station in life has more opportunity than others to lend a helping hand to the Church.\footnote{Rechte Gestalt, p. 8.}  For the protection which the government has to afford also to the Church is not to be extended or rather perverted as though the secular government had also the right to rule the Church.  Walther says: \begin{fancyquotes}The secular government has neither the right nor power to usurp rulership over the Church nor to attempt by compulsion to force upon men the true faith, or what it holds to be the true faith.

                Christ not only declares Himself to be the one who alone has authority in His Church and exercises it through His Word, but He also denies to all others any authority whatever in His Church.\footnote{Matthew 23:8. \& Brosamen, p. 520.}\end{fancyquotes} \begin{displayquote}“\textit{The dogmaticians of the Seventeenth Century have here departed from Scripture and the Confessions in favor of the State Church and call it \textbf{Gallionism} when one denies to the secular government as such the right to judge \textit{ex officio} concerning true and false doctrine}”,\end{displayquote} whereas -- \begin{displayquote}“\textit{the Holy Spirit has undoubtedly had this history {\scriptsize\textsc{(of Gallio, Acts 18:12-16)}} recorded, among other reasons, for the very purpose of letting us know that the secular government as such can pronounce no judgment in matter of doctrine}.\end{displayquote}  After Walther has expounded Baier’s doctrine of the power of secular government in the Church, he continues: \begin{fancyquotes}Secular and church government can hardly be worse confounded and confused with each other than our dear Baier dies here, contrary to the clear testimony of our Church in its basic Confession.  What applies only to the Church of the Old Testament which according to God’s will was to be bound up with the State until Christ’s coming, is here transferred to the Church of the New Testament, and what belongs to a David, a Josiah, etc., is here without distinction attributed to all princes and supreme secular authorities, and so a manifest \textbf{Caesaropapism} is established!  May God have mercy!\footnote{Western District, pp. 30-37.}\end{fancyquotes}

                In particular shall the secular government not attempt with outward force to compel men unto the right faith.  This is contrary to God’s will {\scriptsize\textsc{(John 18:36-37)}}; even the Jews in the Old Testament were to compel no one to adopt their religion; a war which is waged for the propagation of religion cannot please God.  \begin{displayquote}“\textit{Only when waged for the protection of the persons, of the confessors of a religion against its persecutors, can a religious war under certain circumstances be pleasing to God}”.\end{displayquote}  And as the application of outward forces is against God’s will, so also it works harm to the Church.  The Church in this way either wins hypocrites, since outward force cannot change the soul and make it obedient to the faith, or else it entirely repels the unbelievers.  \begin{displayquote}“\textit{Unbelievers indeed seek to justify their rejection of the Christian religion by pointing to the blood {\scriptsize\textsc{(supposedly)}} shed by the Church.  And they rightly assert that a church which makes use of such methods for its extension and preservation cannot possibly be the true Church}”.\footnote{L.c., p. 31-37.}\end{displayquote}  May then the secular government never employ force against ecclesiastical organizations?  It may do so in only one case.  And that is, when erring ecclesiastical organizations adopt, or at least practice, principles which are dangerous to the State.  So, for instance, the State had plenty of reason to proceed against the Pope as an errorist with principles dangerous to the State.  But apart from this case the secular authority has neither the right nor the authority to put its power of coercion into execution against false faith or false worship, pr what it holds to be such.\footnote{L.c., p. 42 ff.}

                Besides the principle that persons belonging to the government are not as such in the Church\footnote{What is true of the secular government is true of secular estates in general.  Walther expounds the matter as follows: \begin{displayquote}The Church indeed consists of men of various stations in life, but the domestic and civil estates do not as such belong in the Church, but are ordained by God alongside the Church.  The estates are not as such in the Church nor do they possess special rights in the Church.  When we say that the Church consists of people of all stations in life, this must be understood to mean that no station, however secular it may appear to be, can deprive Christians of their spiritual and priestly character and their share in the rights and privileges of the Church.{\scriptsize\textsc{(Rechte Gestalt, p. 11)}}\end{displayquote}}  Dr. Walther places the other principle “\textit{that the members of the Church are obliged to render obedience to the State not as Church but as citizens and subjects}''.\footnote{Die Rechte Gestalte, p. 7.10; Brosamen, p. 500.}  This latter, indeed, Walther very strongly emphasizes.  He says that a subject must obey the civil government, no matter what it may command, if only he is not thereby compelled to act against his conscience.  But our government is that which actually has power over us.  Whether it has come into office legitimately, whether it is godly, whether it is of our faith, are questions which here do not come into consideration.  He who is not subject to the government which has power over him not merely against man but against God Himself, whose ordinance the government is.  It is not just a disturbance of the public peace when one rebels against the secular government, but it is, properly speaking, a warring against the divine Majesty.  We must be subject for God’s sake and conscience’s sake.\par  Hence we must honor the powers that be not only with outward demonstrations of respect but in our hearts.  And everyone, specifically also every Christian and every preacher, is bound as a citizen to be subject to the secular government; it is anti-christian when popes and priests do not want to be subject to secular jurisdiction.\footnote{Western District, p. 15, 16.}
\divider
                On the other hand, Walther emphasizes just as strongly that Christians as Christians or as members of the Church are subject to no secular authority, but solely to Christ as their only Master who has made His will known to them in Holy Scripture.  If, therefore, the government commands something which God has forbidden or forbids something which God has commanded, the Christians must be disobedient to the government, which in this case has become guilty of a shameful usurpation, in order to remain obedient to God and keep their conscience undefiled.\footnote{Western Dist. P. 21.}  In all spiritual matters a Christian may not submit to be commanded by any man, also not by the secular government, because in the conscience of a Christian God alone rules through His Word. \begin{displayquote}``\textit{Also {\scriptsize\textsc{(the)}} we Lutherans annually celebrate the so-called National Day of Thanksgiving which our governors and presidents recommend should be celebrated; but we would not do this if they should ever by virtue of their office command it}”.\footnote{L.c. p. 32.}\end{displayquote} In the previously mentioned synodical address Walther says: \begin{fancyquotes}The secular authorities indeed rule over the members of the Church, but not in so far as they, being Christians, belong to the Church, but in so far as they, being men, belong to the State; hence also the State does not rule over the Church itself and over the conscience, faith, and worship of the Christians, but only over their mortal body and earthly goods.  Hence Christ declares: \begin{displayquote}‘ \textit{Render unto Caesar the things which are Caesar’s, and unto God the things that are God’s}',\end{displayquote} -- and thereby draws for all times a strict boundary and dividing-line between the kingdom of God and that of Caesar, between Church and State.\end{fancyquotes}

                This doctrine may indeed seem to contradict more than a thousand years of the Church’s history.  Not from history, however, but from God’s clear Word is to be learned that which is right with regard to the Church.  Even the Lutheran Church itself has been from the beginning till the present day, especially in the land of its origin, connected with the State, or a State-Church.  But this was only the consequence, in part of deplorable circumstances at the beginning, in part of the heedlessness of the appointed watchmen, but in no way a fruit of the doctrine of Luther and the Church which bears his name, the Evangelical Lutheran Church.\footnote{Brosamen, p. 500-503.} {\color{black}\par Moreover even history itself raises its voice loudly against the coupling of the Church with the State.  For great as was the blessing which true Lutheran princes brought to the Church by conducting the office of the territorial episcopate which had devolved upon them purely for the best interests of the Church, at peril of the loss of land and people, yea, to the endangering of their liberty and live, yet far greater has been the calamity which has come upon the Church through the unhappy mixing of Church and State.

                Walther shows in a portrayal as lively as it is historically true how the Church has been pressed almost to death in the arms of the State.  He says:} \begin{fancyquotes}The first consequence {\scriptsize\textsc{(of the unhappy mixing of Church and State)}} was that the Christian congregations lost all the rights and privileges so dearly won for them by Christ, so that hardly any of them remained.  Their right to call, install, and depose their own teachers and preachers, their right to determine ecclesiastical ceremonies and ordinances and to decide all matters of indifference in the Church, and again to abolish, alter, increase or diminish them, their right to exercise Church discipline upon all members in matters of doctrine and life, -- all these rights were almost wholly lost in the State-Church. \par If the territorial lord was worldly-minded he hindered through his like-minded officials all wholesome church discipline, forced the ministers of the Church to give that which was holy to the dogs and cast their pearls before swine, to solemnize marriages in conflict with God’s Word, to accept ungodly persons as sponsors in Baptism, to bury with Christian honors those who had lived as despisers of Word and Sacraments, and the like.  But if the territorial lord fell away from the true religion also outwardly, then he used his power as territorial bishop and prince to draw his people after him in his apostasy; for now he would depose and banish the faithful teachers in church and school and forced upon the congregations in their place belly-serving and fanatical errorists, eliminated pure books for church and school and introduced corrupted books instead.  The farther they traveled along this road the more fully they lost not only the correct practice but with it also the correct doctrine and knowledge, namely, that whatever power the territorial lord might have in the church, it flowed not from divine right either ecclesiastical or civil, but, if at all, only from human and hence at any time retrievable right.\par  Finally it went so far that the principle was enunciated: \begin{displayquote}‘\textit{To whomsoever belongs dominion over a land, his is also the religion of the land}’\end{displayquote} --so that now people began to regard the Church as properly a State institution, its ministers as State officials, and all subjects of the State as members of the State Church--.  What corruption in doctrine and life in this way intruded into the church and what distress of conscience was thereby inflicted upon sincere ministers of the Church and godly laymen cannot be expressed in words.  Here and there even the right to escape the tyranny over conscience by emigration was denied to the oppressed.  What has finally then become of the State-Churches? -- Citadels they are in which the enemies of the Church hold sway, from whose bastions the snow-white banner of the pure confession has been torn down and in its place the varicolored banners of heterodoxy, syncretism, and manifest unbelief now flutter in the breeze.\footnote{Brosamen, p. 503, 504.}\end{fancyquotes}

                Hence Walther summons us to acknowledge it as a great blessing of God that the Lutheran Church here in America is completely independent of the State and enjoys the freedom given her by Christ.
\include{parts/part10a}
\chapter{Justification -- Universal}

\hrule
\vspace{.30cm}
We now indicate the points upon which, according to Walther, everything depends if we are to keep the doctrine of justification pure, also in our times.  Walther says: “\textit{In connection with the pure doctrine of justification, as our Lutheran Church has again expounded and upheld it on the basis of God’s Word, there are chiefly three points at issue:}

\begin{itemize}
\item  \textit{the doctrine of the universal complete redemption of the world through Christ;}
\item  \textit{the doctrine of the power and effectiveness of the means of grace; and}
\item  \textit{the doctrine of faith}”\footnote{First Report of the Synodical Conference, p. 20}
\end{itemize}
\vspace{.30cm}
\hrule
\vspace{1.25cm}
If the people are agreed in these points then they are truly agreed in the doctrine of justification and in general in the entire Christian doctrine.  If there is a deficiency in one or more of these points, as there is in the Protestant sects and among the modern rationalistic-synergistic Lutherans, then the doctrine of justification is defective, even though there may still be outward agreement in phraseology with the orthodox church, i.e., even though one still says that man is justified before God alone by grace, through faith, for Christ’s sake, and not through the works of the Law.\footnote{Die lutherische Lehre von der Rechtfertigung, p. 35. Western District, 1875, pp. 32-40.}

                We give here first of all a summary of Walther’s expositions concerning these points.  If anyone denies the universality of the atonement, if he denies with Calvin that Christ has redeemed all and that God in the Gospel earnestly offers grace to all without distinction, then he certainly overthrows the doctrine of justification.  Furthermore, if anyone teaches indeed that Christ has redeemed all men, but has not fully redeemed them, i.e., if he teaches that Christ has indeed made the forgiveness of sins possible, but that the forgiveness of sins of justification is not actually already at hand for every sinner, then faith and conversion is made a meritorious cause of the forgiveness of sins and the doctrine of justification by grace for Christ’s sake is overthrown.\par  If anyone teaches falsely concerning the means of grace, i.e., if he does not teach that God offers grace to the sinner in the Word and Sacrament and that the sinner is to seek and find grace in Word and Sacrament, then he directs the sinner to seek grace in his subjective condition in conversion and renewal, i.e., in human works. \par  If anyone teaches falsely concerning faith, if he does not teach that faith is reliance upon the grace offered in the Word, but rather identifies faith with feeling, then again the condition of the human heart is made the basis of righteousness and salvation instead of grace of God.  If anyone teaches falsely concerning faith in this manner, that he ascribes to human cooperation or the good conduct of man, then again, even with retention of the phraseology “\textit{by faith alone}”, the “\textit{by grace for Christ’s sake}” and therewith the pure doctrine of justification is abandoned.

                This subject, however, seems so important to us that we wish to expound each of the three points somewhat more fully in accordance with the utterances of Walther which are here so abundantly available.

                \section{Redemption through Christ}
                \hrule
\vspace{.30cm}
To the correct doctrine of justification belongs then, in the first place, the correct Biblical doctrine of the complete redemption of all men through Christ.
\vspace{.01cm}
\hrule
\vspace{1.25cm}
                In order to place the complete redemption through Christ in the right light Walther is concerned with impressing the fact that even before faith grace, righteousness, and salvation is at hand for every man, that even before faith God is in Christ fully reconciled to all sinners, that even before faith every sinner is righteous before God with respect to the attainment and the divine intention\footnote{First Report of the Synodical Conference, p. 68}, or in accordance with the judgment which God by raising Christ from the dead has already pronounced upon all men.\footnote{L.c., p. 31.}  \begin{displayquote}“\textit{A justification has not only been made possible, but is has been obtained and has taken place}”.\footnote{L.c., p. 61.}\end{displayquote}  Walther is above all concerned to reject the idea that man through his faith and through his conversion renders God fully favorable to him or completes his redemption and righteousness.  The man who is to be saved must indeed be converted, but this conversion is not that for the sake of which God saves him, but the way upon which a man comes to faith, who himself does nothing but receive the complete and already bestowed redemption.\footnote{L.c. p. 34.}  \par The fanatics usually think of the matter as though Christ has brought to pass that which the Scripture calls atonement, so that God can now receive a man into heaven merely for the sake of his conversion.  They do not believe that through Christ all without exception has taken place which had to take place in order that God may save us and give us eternal life.  Something, they suppose, must yet remain for man to do, and this something is conversion.\par  But Scripture teaches that Christ has done all and has already obtained reconciliation with God, righteousness, etc., that it is already there and is distributed in the holy Christian Church through the Gospel,  Now no one has anything further to do than to take salvation.  That is what we wish to say when we speak of a complete redemption.  Not that man already has something and God supplies the rest; also not that God has done something and man must add that which is lacking; but that God has already done everything entirely alone.\footnote{L.c., p.34.}

                This doctrine – as Walther urges again and again – is the characteristic doctrine of Christianity, that whereby the Christian doctrine is distinguished from heathenism.  He who denies this doctrine denies all of Christianity. \par “\textit{That man could procure grace or the forgiveness of sins for himself}”, says Walther, “\textit{is what the heathen believed; but that the forgiveness of sins, gained by Another, is already at hand, is a truth of which the heathen knew nothing}”.  And in another place: \begin{fancyquotes}While all religions, except the Christian, have showed man how he must himself do that thereby he is rescued and saved, the Christian religion, on the other hand, teaches not only how men may yet be eternally saved, but how they have already been saved.  According to the teaching of the Christian religion man is already redeemed, is already freed from sin and all ill, and God is already reconciled to him.  The Christian religion says to man: \begin{displayquote}{\footnotesize You need not redeem yourself and reconcile God.  Christ has already done all for you.  Nothing is left for you but to believe this, that is, to receive it.  It is just this which distinguishes the Christian religion from all other religions.}\end{displayquote}  The Jew says: \begin{displayquote}{\footnotesize If you want to be righteous you must keep the Law of Moses;}\end{displayquote} --the Turk says:\begin{displayquote}{\footnotesize If you want to be saved you must conduct yourself in accordance with the Koran;}\end{displayquote} --the Papists say: \begin{displayquote}{\footnotesize  If you want to get to heaven you must do good works, be sorry for your sins and make satisfaction for them yourself, and if you want to be perfect enter the cloister; and all the sects which pervert Christianity without exception lay something upon man which he must do in order to become righteous before God and be saved.}\end{displayquote}  The Lutheran Church, on the other hand, says to man in accordance with God’s Word: \begin{displayquote}{\footnotesize It is all done already: you are already redeemed, you have already been made righteous before God, you have already been saved; hence you have nothing to do in order to redeem yourself, and you do not have to reconcile God and earn your salvation.  You shall only believe that Christ, the Son of God, has already done all this for you, and through this faith you become partaker of it and are saved.}\footnote{L.c., p.34.}\end{displayquote}\end{fancyquotes}

                That grace, righteousness, salvation reconciliation, etc., are already at hand before faith, as Walther further explains, is already demanded by the very concept “\textit{faith}”, and he who denies the former must also deny that we are righteous and saved through faith.  If I am to be saved, says Walther, by believing that I am redeemed, reconciled to God, and my sins forgiven, then all this must already be on hand in advance.  As surely as God’s Word promises us that we shall be righteous, reconciled to God, and saved through faith, so surely must all these things be present before my faith, and waiting only for me to receive them.  That a man should be justified by faith alone is possible only because that which is necessary to salvation is already at hand and accomplished, so that on my part only acceptance is necessary. \par But this acceptance is just what Scripture calls believing.  Since God takes into heaven all who believe, righteousness and atonement must already be present and have taken place.---\par All who will not admit that reconciliation and righteousness are already complete before faith do not regard faith as a mere hand which accepts that which has been gained by Christ, but as a work through which man cooperates toward his redemption and righteousness, as a condition which man fulfills and for the sake of which God receives man into heaven.

                Only when the complete redemption is thus held fast will the concept of the Gospel also be held fast.  \begin{displayquote}{\footnotesize Why is Christ’s doctrine called Gospel or good news?}\end{displayquote}  Simply because when I preach the Gospel I preach nothing else than what has already been obtained and bestowed upon men and what they should therefore receive and in which they should heartily rejoice.  The Gospel is the joyful tidings that Christ has done the work which we should have done and yet could not do, and that the heavenly Father by raising our Redeemer from the dead has given a sign from heaven that He is fully satisfied.\footnote{L.c., p.39.} In the Gospel the peace which God has made with men is proclaimed.\footnote{Western district 1868, p. 31}  It must be stressed earnestness that God’s wrath is turned away from all men by the work of Christ and that through the Gospel everyone is invited to partake of grace.  If a preacher had to come before his hearers with the thought: the wrath of God is still resting upon them and they must be induced to appease Him---that would be terrible; but because he knows that the atonement has already been rendered for all and God’s wrath against all has been quenched, therefore he can say confidently:\begin{displayquote} Be ye reconciled to God, do but receive His hand of grace!\footnote{1st Report of Synodical Conference, p. 36.}\end{displayquote}  He who will not preach the Gospel thus might as well preach the Koran or the Talmud or the papal decretals or what he will;  but if he wishes to {\scriptsize\textsc{(preach the Gospel and)}} make happy Christians, then let him preach this good news.\footnote{L.c., p. 39.} And again: \begin{displayquote}“\textit{Since all men are reconciled to God, and the Gospel is the tidings of this reconciliation, therefore it is such an unutterable grace to live under the sound of the Gospel}”.\end{displayquote}  The fanatics indeed have such thoughts concerning Christ’s work that they regard Him as having only made it possible for man to attain grace by his own efforts.  It is likewise the papistical teaching that men through contrition, penance, and other good works can secure for himself the salvation which Christ has made possible.  But thereby the Gospel, the preaching of which Christ has committed His Church, is denied.

                To the Scriptural presentation of the complete redemption as a premise for the correct doctrine of justification belongs, according to Walther, also the doctrine that in Christ’s death and resurrection a justification of the entire world of sinners is already implied.  “\textit{As by the vicarious death of Christ}”, says Walther, \begin{displayquote}“\textit{the guilt of the entire world was cancelled and its punishment suffered, so also by the resurrection of Christ righteousness, life, and salvation is restored for the entire world and in Christ, as the Substitute of all mankind, has come upon all men}”.\end{displayquote}  \begin{displayquote}“\textit{Christ’s glorious resurrection from the dead is the actual absolution of the entire world of sinners}”,\end{displayquote} and \begin{displayquote}“\textit{The resurrection of Christ the plenary justification of all men}”\end{displayquote} --such are the themes of Easter sermons delivered by Walther.\footnote{Brosamen, p. 138; Epistelpostille, p. 211} \divider Many, even among preachers, do not rightly know what to do with the resurrection of Christ.  They read that Christ raised Himself and then again that the Father raised Him, and they do not know how to harmonize this.  They suppose at one time that Christ arose in order to prove His deity and at another that He was raised in order that the possibility and certainty of our resurrection might be established.  True as both these assertions are, yet neither one is the chief matter.  Christ would not have died and risen again only to prove His deity; and the possibility of our resurrection had indeed already been proven by the resurrection of others before Christ; the chief matter remains that God through Christ declared: Christ has now paid for the sins of the whole world, it is therefore free from its guilt; now the entire world can raise the shout of victory, for its freedom from sin and its righteousness is won.  Furthermore: when God raised His son from the dead He did not forgive Him His own sin but that of all mankind which He had taken upon Him; He did not justify Christ from His own guilt but from our guilt which He had allowed to be imputed unto Him.  Thus the whole world has been justified through the resurrection of Christ.\footnote{Western District, 1875, p. 33.}\par  With this the fact that man is justified by faith in no way stand in contradiction, for when we speak of faith the personal appropriation on the part of man and the imputation of the righteousness which has been won on the part of God is emphasized.  But this would not be possible if the world had not been first justified through Christ’s death and resurrection, if the condemnation in death had not been followed by the acquittal in the resurrection.\footnote{1st Report of the Synodical Conference, p. 41f}  And this justification applies not only to men in general but to all individual men.  \begin{displayquote}“\textit{If it be asked whether one could say that man collectively has indeed been absolved but not the individuals, our answer is: God through Christ is reconciled to each individual}.''\footnote{L.c., p. 32} \end{displayquote}

                This doctrine of a universal justification of all men before faith is not a theological construction, but a Biblical doctrine.  Biblical not only in its content- which in itself would be fully sufficient, -- but even in its phraseology.  “\textit{It is this doctrine,}” says Walther, \begin{fancyquotes}which is expressly declared in the passage, {\scriptsize \textsc{Romans 5:18}}\begin{displayquote} ‘\textit{As the offense of one judgment came upon all men to condemnation; even so by the righteousness of One the free gift came upon all men unto justification of life}’,\end{displayquote} and it is therefore not merely a Biblical doctrine but also a Biblical expression that justification of life has come upon all men.  Only a Calvinistic exegesis could explain this passage to the effect that only the elect are justified.\end{fancyquotes}  Although Scripture is most places speaks of that justification which takes place in the moment when a man comes to faith, and accordingly in ecclesiastical usage the justification by faith is simply called the justification of a poor sinner\footnote{L.c., p. 68.}, nevertheless the doctrine of the universal justification of all men before faith, which is clearly attested by Scripture in several places, is of the very greatest importance.  Let no one think in this matter a mere strife of words is involved.  Rather is the most highly important matter here to be maintained against attacks and error.  Especially in this land of sects and fanatics we must earnestly urge the doctrine of universal justification, for they indeed also teach that man is justified by faith, but they speak of faith in such a manner that one soon notices that they make faith itself the effective cause of justification, whereby they rob the Lord Christ of His honor.\foootnote{L.c., p. 46.}\par  Without the universal justification before faith there is no justification by faith.  ``\textit{We could not then,}'' continues Walther,  \begin{displayquote} ``\textit{speak of the justification of the sinner by faith, for to believe means to receive what is there.  If the world were not already justified, then believing would mean accomplishing a work unto justification.  The entire preaching of the Gospel is a message of God concerning a righteousness which has already been gained by Him and is there for all}.''\footnote{Cf. On this subject especially Brosamen, pp. 142, 143.}\end{displayquote}

                Those who say that God has made the whole world righteous, but has not declared it righteous thereby really deny the whole of justification.  Yea, if God had not {\scriptsize\textsc{(already)}} written and sealed the document of pardon, we preachers would be liars and deceivers of the people when we tell them: \begin{displayquote}{\footnotesize Only believe and your are righteous;}\end{displayquote}  but now that God through raising His Son has subscribed the document of pardon for the sinners and provided it with His divine seal, we can confidently preach:\begin{displayquote}{\footnotesize the world is justified, the world is reconciled to God, which latter expression we could not use if the former were not true.}\end{displayquote} –When the Lutheran Confession repeatedly says that justification is grasped by faith these passages express the truth that a justification must first be at hand which faith can receive, and that faith must not first effect it, but that it grasps it as already at hand.  If anyone would say: \begin{displayquote}{\footnotesize the forgiveness of sins is indeed already there, but not justification,}\end{displayquote} --he must indeed be ignorant of our Confessions, which expressly teach that justification and forgiveness of sins are the same.  \begin{displayquote}“\textit{We believe, teach, and confess that according to the usage of Holy Scripture the word justify means in this article, to absolve, that is, to declare free from sins}”.\footnote{Formula of Concord, Article 3; Mueller, p. 528; Triglotta p. 793. L.c., p.46.}\end{displayquote}

                Particularly in connection with Walther’s discussions on absolution, that is, the “\textit{preaching of the Gospel to one or more particular persons who desire the comfort of the Gospel}”, the manner in which the complete redemption of all men through Christ lived in Walther’s heart came to expression.  Absolution, says Walther, is based upon the perfect redemption or universal justification.  “\textit{When the pastor absolves he distributes a treasure which is already at hand, namely the forgiveness of sins which has already been gained}”.\footnote{L.c., p. 43.}  Walther holds only that man to be a true Lutheran preacher who holds that he by speaking the absolution has absolved all the penitents and only that man to be a true Lutheran Christian who believes that through the absolution of the pastor he has truly been absolved by God.  He adds: \begin{displayquote}“\textit{Only he indeed can believe thus who believes that the world is redeemed; for if I believe that, then the absolution is only the communication to the penitents of the fact that they were redeemed 1800 years ago, and the plea: Only believe that and you are saved}”.\end{displayquote}  That so many take offense at the absolution which is customary in the Lutheran Church comes from the fact that they do not believe in the complete redemption of all men through Christ and hence suppose that we ascribe to the preachers as “\textit{ordained persons}” a special authority and mysterious power.  \begin{displayquote}“\textit{But we say:\begin{displayquote} It is no art to absolve someone; that any ordinary Christian man, any woman, any child can do, if it can only tell that the Lord Jesus died for all, and that whoever believes in Him receives the forgiveness of sins.\end{displayquote}  For the absolution depends not upon the quality of the speaker but upon the word of the Gospel concerning the accomplished redemption}”.\end{displayquote}

                In this connection Walther insists again and again that one must not make the essence of the Gospel dependent upon faith, but is to regard it as an offer of God’s grace which is valid of itself.  \begin{displayquote}“\textit{The glorious benefits of Christ have been given us; mark well! They have already been given us (in the Gospel) and indeed they are always at hand for us, even if we do not believe}”.\footnote{Western District, 1874, p. 47.}\end{displayquote}  If one makes the Gospel essentially dependent upon a man’s believing, or, which is the same, if one talks as though faith must first be there before the Gospel is in itself valid and effective or before the benefit of the forgiveness of sins is at hand for the sinner, he thereby both denies Christ’s all-sufficient merit, the redemption and reconciliation of the world, and then also faith is thereby made something quite different from what it properly is; it is then no more a grasping and receiving of the present forgiveness, but a work which must be furnished in order that there may be forgiveness in the Gospel; finally, faith has then simply nothing on which it can take hold.  “\textit{If the Gospel is not valid unless a man first believe it, what then shall he believer?}”  Faith thus comes to be founded on itself instead of on the Gospel.  “\textit{That amounts to increasing the distress of people who are in anxiety and doubt concerning their salvation}”.\footnote{L.c., pp. 57-64.} Walther reminds us again and again that, with a doctrine or practice according to which faith is first demanded in order that forgiveness of sins may be there, no tempted person can be comforted.  \begin{displayquote}``\textit{The tempted supposes that he cannot believe.  Such a person must despair with this doctrine, whereas one should seek to convince him that the Savior is already there for him, has already forgiven him and will receive him}”.\footnote{Western District, 1875, p. 38.}\end{displayquote}

                Walther here examines an objection.  The objection asks how this argument concerning complete forgiveness, universal justification, the Gospel as an absolution of the whole world of sinners, harmonizes with those Scripture passages which speak of God’s wrath upon the world lying in wickedness, in particular upon the unbelievers.  Walther answers by means of the distinction between Law and Gospel.  In so far as God views the world in Christ “\textit{pure love, pure favor, pure grace}” toward the world is in His heart.  In so far as He contemplates the world outside of Christ as lying in wickedness, and particularly as rejecting the Gospel, it lies under His wrath.  Although there is indeed no real contradiction here, since grace and wrath are predicated of God’s relation to the world in different respects, yet “\textit{an unutterable and unfathomable mystery}” is to be acknowledged here.  Since Scripture teaches both facts we let them stand side by side.  \begin{fancyquotes}It is the Lutheran way that when we find in God’s Word two things which we are not able to harmonize we let both stand and believe both as they read.\footnote{Synodical Conference, 1st Report, pp. 31f, 36 f.}\end{fancyquotes}
\include{parts/part11}
\chapter{Justification -- Faith}
\hrule
\vspace{.30cm}
If the doctrine of justification is to remain pure, then finally also that concerning faith must be rightly taught.  This point had to be touched upon also in the preceding treatment.  So we can here confine ourselves to emphasizing a few principal thoughts.
\vspace{.30cm}
\hrule
\vspace{1.25cm}
                Walther mentions that the ignorance concerning the nature of justifying faith or the manner in which it justifies is very widespread in external Christendom.  He says: “\textit{However much all Christian parties speak of faith, yet very few have a correct conception of faith and how it justifies}”.\footnote{Western District Report, 1875, p. 35.}  Yes, there reigns with regard to this point a “\textit{truly Babylonian confusion}”.\footnote{Report of the First Convention of the Synodical Conference, p. 29.} \begin{displayquote} “\textit{People talk so much about the fact that faith alone justifies and saves but when it comes to the real issue they desire to know nothing of it}”.\end{displayquote}  If we investigate more closely it comes to light that they ever and again mix works into the article of justification.  Even when they verbally exclude works from justification and take the \textit{sola fide} phrase into their mouths, actually all this is again retracted and the fundamental article of the Christian religion is entirely falsified by making faith itself into a work.  They still want to find some place for an activity of man whereby he distinguishes himself before others.  This activity is placed sometimes in his repentance, sometimes in his conversion, sometimes in his sanctification, and sometimes in faith itself.

                Hence Walther’s efforts are directed toward warding off the tendency in some to change faith itself whereby justification takes place into a work or to mix one’s own works, one’s own worthiness, one’s own doings, etc. into faith.  Walther brings out again and again: \begin{displayquote}“\textit{If God demands of us faith He does not thereby say: My son has indeed made satisfaction for you and redeemed the world, but now you must also do something; on the contrary the situation is this: just because we have nothing whatever yet to do toward our salvation, therefore faith is necessary}”.\\\\  “\textit{The reason why faith justifies, and not something else, is this, that on the part of man there is nothing left to do, but righteousness and salvation has already been completely won for all men by Christ and is offered as a gift in Word and Sacrament}”.\end{displayquote}  Or: faith justifies and saves because man can in no way be justified and saved by his own deeds, but freely, by grace.  Faith comes into consideration in justification as the antithesis to all works and every sort of merit.  \begin{displayquote}“\textit{If indeed righteousness were not by grace, then something else would have to be required for it’s attainment, but since it is by grace faith is enough, for it is only a receiving}”.\end{displayquote}  Yes, faith justifies inasmuch as it is the receiving of the righteousness and salvation which is already at hand through Christ’s merit and offered in the promise of the Gospel.  Faith comprises knowledge, assent, and confidence; but not “\textit{because it is knowledge, assent, and confidence {\scriptsize\textsc{(and hence a certain quality in man)}} does it justify, but inasmuch as it is the means whereby the righteousness which is at hand is received}.” \par Man shall not first in some way make himself worthy of righteousness and salvation through faith.  Faith does not come into consideration in justification in so far as it is itself a deed or an obedientia, nor in so far as it is effects an internal change in man and has blessed feelings, good works, etc., as it consequences.  In order quite sharply to express the thought that faith is not to be conceived of as a supplement to the grace of God and the merit of Christ, Walther says: \begin{displayquote}“\textit{If the word ‘faith’ never occurred in the Scripture it would still teach salvation by faith, through the very fact that it teaches salvation by grace for Christ’s sake}”.\end{displayquote}  And in another place Walther declares: \begin{displayquote}``\textit{If I had nothing else than faith, and not Christ {\scriptsize\textsc{(which is indeed not possible)}}, then I should be damned together with my faith, for it is not the act of faith which makes me acceptable to God, but it is Christ and His righteousness, which I grasp with the hand of faith.}''\footnote{Report of First Convention of the Synodical Conference, p. 35.}\end{displayquote} Walther liked to adduce in this connection the statement of Calov, that also faith itself, in so far as it is an instrument, is rightly set over against not only all works of obedience and godliness, but also over against faith itself, in so far as it is our work and act.\footnote{Baieri Compendium, ed., Walther, III, p. 270.}\footnote{That faith comes into consideration in justification “\textit{not as a work, but as an instrument}”{\scriptsize\textsc{(nicht als Werk, sondern als Werkzeug)}}, is a theme which Walther again and again expounded in his well-known “\textit{Lutherstunden}”.  The following may find place as an annotation here. \par Walther declares that faith does not justify in so far as it in a general way believes something or other, but in so far as it believes the Gospel that God through Christ is gracious to men.  Walther said on September 14, 1877: \begin{displayquote}“When the unbelievers hear that in the Christian religion the grace and favor of God and eternal salvation is ascribed to faith, they usually think that this is just the way of all religions which claim to be revealed by God in a supernatural manner, that they require of their adherents above all else faith in their mysteries which are contrary to reason, and for this promise heaven to those who believe them.  Above all else faith was demanded of his adherents by Mohammed, above all else faith was demanded by the founders of the Mormon sect, above all else faith was demanded by Moses, and so also by Christ. \par But, think the unbelievers, what can God care {\scriptsize\textsc{(if indeed there is a God)}} whether one believes something contrary to reason or not?  What better is he who treads his reason under foot, or why should he be more worthy of heaven {\scriptsize\textsc{(if indeed there is a heaven)}}, than he who uses his reason? \par --From all these judgments one can perceive that the unbelievers have no idea what faith really is to which in the Christian religion God’s grace and eternal salvation is ascribed.  The mere regarding what is written in the Bible as true is according to our holy Christian religion so far from being the faith which justifies and saves that the Bible itself says: \begin{displayquote}‘\textit{Thou believest that there is one God; thou doest well: the devils also believe, and tremble.}’{\scriptsize\textsc{(James 2:19)}}.\end{displayquote} The mere regarding as true what Holy Scripture says is, therefore, according to Scripture itself, something which also the devil can do and which hence does not save.  The faith to which the Christian religion promises salvation is accordingly something entirely different.  It is, in a word, as Holy Scripture itself says, ‘\textit{a sure confidence}’ {\scriptsize\textsc{(K.J.V. marginal reading; German: ‘ein gewisse Zuversicht’)}}, a ‘\textit{receiving}’ {\scriptsize\textsc{(German: ein ‘Auf – und Annehmen’)}}. {\scriptsize\textsc{(Hebrews 11:1; John 1:12.)}} \par God performed the great wonder of His eternal love in sending His only begotten Son into the world, in having Him become a man, that through Him He might Himself pay all men’s debt of sin, and thus win again for all men the heaven forfeited through sin and the salvation lost through the same, and finally offer and bestow all this through Word and Sacrament.  What is there then to do on the part of man?  Nothing, nothing whatsoever; but to give the glory to God and to receive the gift; and this and nothing else is faith.  In the merely ‘\textit{regarding as true}’ the ‘\textit{receiving}’ is lacking, and therefore true faith is lacking.  But if a man really receives the grace and salvation offered in Word and Sacrament to all men and hence also to him whoever is offended at this doctrine of faith is really offended only at the greatness of divine grace, at the blessed counsel of redemption, at Christ ‘\textit{the Savior of the world}.  Would God that it were only the unbelievers who reject the correct doctrine of faith!  but alas entire great church-parties do just this”\end{displayquote}. --Upon the same subject Walther said in another “\textit{Lutherstunde}”: \begin{displayquote}“The Christian religion demands also faith in its divinity and truth.  But this faith is by no means that to which the Christian religion promises salvation.  When Christ says: \begin{displayquote}‘{\color{red}\textit{He that believeth shall be saved}}’,\end{displayquote}  --that does not mean merely: he that regards what I teach as true shall be saved.  Rather does it mean this: You human beings by sin have fallen away from God and into an eternal debt which you are unable to pay.  But be of good cheer; I the Son of God, have paid your debt and thereby won back for you God’s grace and eternal salvation, and all this I offer you as a free gift.  Come then, receive this gift, and thus you are helped.  And this receiving is exactly the faith of which the Christian religion speaks.”\end{displayquote}}

                And this careful separation of faith from everything which is a deed or quality of men is altogether necessary.  First, in order that His glory as Savior may remain Christ’s own.  And then also because consciences are confused through all false thoughts concerning faith.  “\textit{How many}”, says Walther, “\textit{do not dare to believer because faith has been falsely described to them {\scriptsize\textsc{(namely, as one’s own deed, as a good quality, as fides formata, as feeling)}}}.”

                In the following we shall just briefly indicate what Walther taught with regard to this point.
\divider
                The Papists err quite grossly in the doctrine of faith, since they expressly say that faith justifies in so far as it is a good quality, a virtue in the heart of man which includes love and all good works.

                Here also those err who, with the fanatics, conceive of justifying faith as a change in the heart of man.  It is indeed faith, and faith alone, which does produce a change in the heart of man.  But this changing, sanctifying power is not the reason why faith justifies.  If one ascribes justification to faith in this respect, then justification is again based not upon Christ but upon man himself, namely upon the new life begun in man.

                It is likewise to be rejected when justifying faith is, with the fanatics, conceived of as wrestling and striving for grace.  It is indeed true that faith wrestles and strives.  “\textit{They are greatly in error}”, says Walther, \begin{displayquote}“\textit{who suppose that we are against an earnest godliness, that we reject striving, praying and wrestling, sighing and weeping; O no! many a one of us perhaps spends more time upon his knees than those who want to earn grace thereby; only we are against the idea that grace must be obtained by praying, sighing, and wrestling}”.\end{displayquote}  Faith comes into consideration in justification not in so far as it wrestles and strives, but in so far as it rests in the promise of the Gospel, in so far as it is the sure confidence which appropriates to itself the promise of grace contained in the audible {\scriptsize\textsc{(Word of God)}} and visible {\scriptsize\textsc{(Sacrament)}} Gospel.\footnote{Western District Report, 1875, p. 22.}  If anyone says that faith justifies in so far as it wrestles and strives for grace, he would thereby “\textit{take away from God the honor and set up a heathenish justification adorned with a few Christian patches}”.\footnote{L.c.}

                Faith is also not to be conceived of, with the fanatics and the modern theologians, as a condition of justification, if one takes the word condition in its proper and primary meaning.  Walther often insists upon this point very emphatically.  He indeed acknowledges that one can well make use of this expression in speaking of the necessity of faith, or when one wishes to stress the point that justification cannot take place without faith.  But one must then carefully avoid all misunderstanding, for the word condition, as it is ordinarily used, includes a performance {\scriptsize\textsc{(Leistung)}} on the part of him who is to receive something.  But faith comes into consideration in justification not as a performance, but as the antithesis of all human performance.  Faith is therefore not a condition under which we become righteous, but the way and manner in which we become partakers of the righteousness with which God has long ago {\scriptsize\textsc{(in Christ’s resurrection)}} endowed us and offers us in the Word of the Gospel.  We do indeed read in the Scripture: “\textit{With the heart man believeth unto righteousness}”.  But the particle “\textit{if}” has a double sense.  It is used either to give the basic cause \marginpar{\scriptsize\textit{Etiologically} \\ --assigning or seeking to assign a cause}{\scriptsize\textsc{(etiologically)}}, or to designate the way and manner \marginpar{\scriptsize\textit{Syllogistically}\\ --deductive reasoning}{\scriptsize\textsc{(syllogistically)}}.\par  In the preaching of the Law: “\textit{If you do this you shall live}”, the “\textit{if}” gives the basic cause, since obedience is the cause on account of which eternal life is given to those who keep the Law; but in the Gospel promises: “\textit{If you believe you shall be saved}”, the “\textit{if}” is syllogistic, for it designates the divinely ordained way and manner of appropriation.

                To be sure, modern Lutheran theologians favor the expression that man is justified under the condition of faith.  But this is due to the fact that the modern Lutheran theology is synergistic through and through.  It calls faith “\textit{a deed of our age}”, an exalted “\textit{moral deed of the will appropriating salvation}”  Therewith the doctrine of justification by faith is actually given up, even though they may still say that man is justified by faith alone.  All  justification by faith is actually given up, even though they may still say that man is justified by faith alone.  The word “\textit{faith}” and so also the expressions “\textit{by faith}”, “\textit{by faith alone}” have obtained an entirely different sense.  All synergists must also falsify the doctrine of justification, because they make out of faith something which is likewise a deed of man.  Therefore Walther says with regard to the most recent controversy: \begin{displayquote}“\textit{Also at the present time {\scriptsize\textsc{(in the controversy concerning the doctrine of conversion of election)}} the issue is no other article than that of justification.}\end{displayquote}  The present question is: \begin{displayquote}{\footnotesize Is man really justified and saved alone by grace?  Does Christ do this alone or does the reason why a man is saved lie in man?  Does faith justify because Christ has already done all, so that we have only to appropriate it to ourselves?  Or is faith that which man must do on his part, is faith necessary because also on the part of man something must be done?}\end{displayquote}  Hence Walther also repeatedly designated retaining the purity of the doctrine of justification for our fellowship as the foremost fruit of the most recent doctrinal controversy.
\chapter{Conversion and Election}
\hrule
\vspace{.30cm}
As we have endeavored up to this point to give a presentation of Walther’s battle for the doctrine of the Church and of justification, so we will now turn our attention to a still more comprehensive discussion of the doctrine of conversion and election which Walther maintained over against the errors which have emerged in our time.
\vspace{.30cm}
\hrule
\vspace{1.25cm}
                While the battle for the Scriptural doctrine of the Church occupied Walther during the first period of his activity, the contention for the pure doctrine of election and conversion stood in the foreground during the last fifteen years of his life.  And we must say: as the correct doctrine of the Church, almost forgotten even within the Lutheran Church, was made known again chiefly through Walther, so also it is principally to be ascribed to his testimony that the Biblical doctrine of conversion and election has not been completely swept away by the stream of modern error.

                With a view toward a more lively perception of the grace which God also in this connection has showed His Church through the service of Walther we shall permit our attention first to be directed to the position of modern Lutheran theology in these doctrines.

                In modern Lutheran theology synergism is predominant, in part the grosser Melanchthonian type, in part the finer Latermannian.  Kahnis,  expressly professes the synergism of Melanchthon when he writes: \begin{displayquote}“\textit{Melanchthon through his doctrine of the cooperation of the human will in the appropriation of salvation {\scriptsize\textsc{(synergism)}} took the correct evangelical and at the same time traditional way to hold fast the substance of the Augustinian doctrine without its excesses}”.\footnote{Baieri Compendium, ed. Walther, II, pg 302.}\end{displayquote} Yes, Kahnis calls it an exaggeration when one teaches that “\textit{the natural man is totally dead to good}”.\footnote{L.c., p. 301.}  The majority of more recent Lutheran theologians, however, advocate synergism in the Latermannian form: \begin{displayquote}{\footnotesize The will of man is made free through grace to such an extent that man can now decide for himself for or against grace.}\end{displayquote}  According to this doctrine the Holy Ghost works so much, that man can convert himself, but the actual conversion man himself must perform.  Or: \begin{displayquote}{\footnotesize The Holy Spirit confers the power to believer, the ability to believer; the act of faith, the actual faith itself, man himself must produce on the basis of that competence conferred by the Holy Ghost.}\end{displayquote}  Hence they call faith a “\textit{performance}” {\scriptsize\textsc{(“Leistung”)}} of man, “\textit{man’s own deed}”, a “\textit{moral self-activity of man}”, etc.  \par Hence they posit a cooperation of man not only after conversion but in conversion for the purpose of bringing it to pass.  The statement of the Lutheran Confessions, that man in his conversion is not active but only passive, mere passive-, in need- according to the teaching of the newer Lutheran theology – of a “\textit{qualification}” {\scriptsize\textsc{(“Einchränkung)}}, namely, of the qualification that man is not mere passive, but conducts himself actively, cooperatively, and this “\textit{qualification}” of the Lutheran doctrine “\textit{now enjoys a virtually universal acceptance}”, as Dr. Luthardt remarks.

                The motivation for the setting up of such a doctrine divergent from the Lutheran Confession on the part of modern Lutheran theologians is the circumstance that they assign to theology a very peculiar task, namely the task of convincing human reason of the correctness of Christian doctrine.  While the old Lutheran theologians regarded the task of theology as the compilation and orderly presentation of the articles of faith revealed in Scripture, the newer Lutheran theology has set as its goal the mediation of the articles of Christian doctrine to human reason, specifically, the supplying of a rational consistency between the individual Christian doctrines.  And from this standpoint the modern Lutheran theology has arrived at its synergism.  For human reason concludes in this manner:\begin{displayquote}{\footnotesize if those who are saved were converted by the gracious working of God, without cooperation on their part or without a better conduct on their part exerting any influence upon their conversion, one should have to conclude that God passes by the rest with His grace or that God’s grace is not universal.}\end{displayquote}  Now if this latter supposition is inadmissible, then one must take refuge in the former, namely, that conversion is dependent upon the good conduct of man.  To acknowledge at this point, with the Formula of Concord, a mystery insoluble in this life, and to allow both facts, that those who are saved are converted alone by the gracious working of God and that those who are lost remain in unbelief by their own fault alone, to stand unharmonized {\scriptsize\textsc{(“unvermittelt”, “unmediated”)}} would be diametrically opposed to the purpose which modern theology desires to serve.  These two truths must be harmonized.  The \textbf{Formula of Concord}, which not only does not do this but even warns against doing it, and which holds fast both universal grace and also the doctrine that in conversion God alone does everything and that no cause of conversion and election is to be acknowledged in man, is subjected to the criticism that it does not with sufficient circumspection keep within the bounds of necessary moderation, that it is “\textit{not entirely free from predestinarian}”, that is to say Calvinistic, “\textit{tendencies}”, that it “\textit{contains unassimilated elements of the doctrine of absolute predestination}”.  Others, who do not venture to speak of Calvinistic tendencies in the Confession of the Church to which they wish to belong, allow themselves to interpret the Confession of the Church to which they wish to belong, allow themselves to interpret the Confession in such a way that the recent synergistic doctrine eventuates.

                In this way the recent Lutheran theology has turned everything upside down in the articles of conversion and election.  It has corrupted the entire \marginpar{\scriptsize\textit{usus loquendi}\\ --Manner of speech.  Words are assumed to be defined by common parlance; unless special context dictates otherwise}\textit{usus loquendi}.  If anyone teaches that the mercy of God and the merit of Christ and nothing in us – no cooperation, no better conduct – is the cause of conversion and salvation, they declare this to be Calvinizing.  According to the usage which previously obtained in the Lutheran Church absolute election was understood to be the doctrine of the Calvinists according to which election was not based upon Christ’s merit but embraced the merit of Christ only as a means of execution.  But the modern Lutheran theology speaks of absolute election when one will not have election based upon man's self determination, good conduct, etc.  According to the Lutheran Confession it is Scriptural and alone correct when one refuses to answer the question why some are saved rather than others, but acknowledges a mystery at this point.  In our time “\textit{Lutheran}” theologians declare such silence to be a mark of Calvinism.  The Lutherans are called Calvinists today and the synergists are called Lutherans.  According to the Lutheran Confessions it is the highest comfort for a Christian to know that his salvation rests not in the least in his hand but alone in God’s hand.  According to recent Lutherans the comfort of the Gospel remains undiminished only when one holds that conversion and salvation depend in the last analysis upon man’s free self-decision, or, which is the same thing, upon man’s conduct.  From a desire to rescue universal grace by the use of rational arguments they have lost the whole concept of grace entirely.

                This is the situation which today’s synergistic-rationalistic “\textit{Lutheran}” theology has brought about.  There appears here a depth of Satan which fills everyone, who by the grace of God has a seeing eye, with consternation.

                The doctrine of the modern Lutheran theologians as it has just been described has now in all essentials become current also in America, and this has been brought about in the first place by the Iowa Synod.

                So far the doctrine championed by Iowa is concerned, this becomes clear from the following quotations.  We here offer a series of utterances of the one time leader, of the Iowa Synod, in order that it may be the more clearly perceived of what sort the doctrine was which sought entrance into the American Lutheran Church and was limited to more confined circles chiefly through Walther’s opposing testimony.  Walther brings the following utterances of \marginpar{\scriptsize\textit{Gottfried William\\ Leonhard Fritschel}\\ (Dec, 19 1836-1889)\\--was a German-born Lutheran who emigrated to Iowa}Prof. G. \textbf{Fritschel} in “\textit{Lehre und Wehre}”\footnote{Lehre und Wehre 1872, p. 204f.}: \begin{fancyquotes}Two statements must be placed side by side and both together be held fast.  \par The first:\begin{displayquote}{\footnotesize Man can in no way prepare himself for divine grace, but he owes all his salvation entirely and alone to grace; grace must itself bring it about that he accepts grace.}\end{displayquote} \par The other: \begin{displayquote}{\footnotesize Whether man is saved or is lost depends in the last analysis upon man’s own free decision for or against grace}.\end{displayquote}  That of two men who hear the Gospel the resistance and death of the one is taken away while that of the other is not..., this has its basis in the free self-decision of man, although this decision is first made possible by grace”.\par  --That of two men to whom the Gospel is preached the one comes to faith, the other does not: for this according to God's Word the basis lies only and alone in the decision of man.\par --Therein lies the real inner difference between the Biblical and the predestinarian doctrine, that according to the former man’s eternal fate is rooted in the personal free decision of man for or against the grace offered him in Christ... He {\scriptsize\textsc{(God)}} lets it depend upon the decision of man to whom He will show mercy and whom He will harden.\par --When the Gospel comes to a man there is bestowed upon him by grace the poser to accept it, while he indeed can also willfully reject it by the determination of his will against God.  He receives in consequence of the working of grace \textit{arbitrium liberatum}{\scriptsize\textsc{(a freed will)}}.  His will enslaved by sin is so far set free by the call of grace that he now with his own free will can freely decide for or against God, which decision indeed need not take place like a flash in a moment.\end{fancyquotes}  Finally Prof. F. said in the words of Philippi: \begin{displayquote}“\textit{The statement ita spiritu sancto agimur, ut ipsi quoque agamus}, i.e., \textit{we are moved by the Holy Spirit in such a way that we also do something, is true not only of the converted but of those who are being converted...  As, accordingly, a certain synergism of man in the use of the means of grace even before the beginning of the internal divine work of grace is not to be excluded: so there takes place also a synergism of the human will with the divine grace not only after the completion of conversion but also during the act of conversion itself, except that indeed there is no synergism of the naturally free but only a synergism of the will freed by grace}”.\footnote{before the occurrence of conversion}\end{displayquote}

                That was the doctrine proposed by Iowa.  It is the synergistic teaching of the intermediate state in which the not yet converted man is supposed to be placed through grace into such a position that he can now convert himself.  On the one hand the phrase that man owes all his salvation wholly and alone to grace, but on the other hand the definite assertion that conversion and salvation “\textit{in the last analysis}” depend “\textit{only and alone}” upon man himself, upon his own free decision.  “\textit{Everything}” is to be ascribed to divine grace, only not that which is finally decisive, that a particular person is converted and saved.  As long as they are speaking of being saved in itself, they are willing to ascribe all to grace.  But as soon as those who are being saved are placed in comparison with those who are being lost, then the “self-determination”, “conduct”, of the former must be the ultimate ground of salvation.  And scarcely ten years later this Iowan doctrinal position was adopted by Prof. \textbf{Schmidt} and the Ohio Synod.  They refrained from the expression that the “\textit{self-determination}” is the deciding factor on the part of those who are being saved, but they substituted for it the equivalent expression “\textit{conduct}”, and expressly asserted, herein going beyond Iowa in their phraseology, that man does not owe his salvation alone to grace.  They proposed statements like the following: “\textit{Salvation in a certain sense does not depend upon God}”.  In a certain respect “\textit{conversion and salvation depends also upon man and alone upon God}”.

                And this doctrine Iowa and Ohio sought to support with the same rationalistic argumentation as the German Lutheran theologians employed.  Iowa expressed herself as follows: \begin{displayquote}“\textit{It remains true that, if God has predestined only a certain number of men unto eternal life, {\scriptsize\textsc{(which Iowa acknowledged as correct)}} the basis for this lies either in the absolute election of God who simply wills to bestow faith upon all these men, or else it lies in the decision of man which God has foreseen}”.\footnote{L.u.W., 1872, p. 243.}\end{displayquote}  On the part of Ohio it was expressed thus: “\textit{It would be clear that if God would decide the matter no one would be lost}”.\footnote{L.u.W., 1886, p. 25.}

                That there should be a mystery here, that one must accept both truths: in man no cause of conversion and salvation, in God no cause of unbelief and damnation, without harmonizing, this was ridiculed by Iowa and Ohio and declared to be a sign of Calvinism.  On the part of Iowa it was said: “\textit{Perhaps someone would offer as such a third possible explanation}”\footnote{(namely, in addition to the two rejected by Missouri, that either the Calvinistic absolute election or else human conduct explains the election of certain persons)} --“this:\begin{displayquote}\textit{ Why God has chosen some and left the others we cannot understand, as this belongs to the secret will of God which we should not investigate; which might be the one intended by the Missouri Synod in the Synodical Report in question.  But that is not a third explanation in addition to the other two mentioned above” {\scriptsize\textsc{(absolute election or human conduct),}} “but merely a non-explanation.  It is a mere forced suppression of the question, by which no help is offered}”.\footnote{L.u.W., 1872, p. 243.} \end{displayquote} On the part of Ohio it was untiringly asserted that it was only Calvinistic evasions when the “\textit{Missourians}” spoke of a mystery and would answer the question, why some are converted and saved rather than others, neither by Calvin’s particular grace nor by the assumption of a better conduct on the part of those who are saved.

                That was the position of Iowa and Ohio and the argumentation by which it was supported.

                How did Walther attack this position?  He polemicized above all things against the assumption on which the entire proposition of the adversaries rested, against the assumption, namely, that an explanation must be found and given for the fact that of two men who hear the Gospel the one is converted rather than the other.  Rather does he demonstrate that the Scripture and after the Scripture also the Lutheran Confession demands that a mystery be acknowledged at this point.

                In the somewhat lengthy article in which Walther for the first time polemicized in a very comprehensive way against the synergism which has emerged in the American Lutheran Church Walther attacks its position immediately at the very center.  He writes in this article: “\textit{Is it really Lutheran doctrine that the salvation of man depends in the last analysis upon man’s own free decision?}” as follows: \begin{fancyquotes}the first reason why this is not Lutheran doctrine but a teaching which has always been most decidedly repudiated by the Lutheran Church is that hereby the inexplicable mystery, why certain men come to faith and are saved, while other men do not come to faith and are lost, although both lie in the same impotence and guilt, is entirely destroyed by explaining this mystery according to one’s own thoughts.\footnote{L.u.W., 1872, p. 240.}\end{fancyquotes} 
\include{parts/part13b}
\chapter{Conversion}
\hrule
\vspace{.30cm}
Having examined the point which according to Walther constitutes the main point of difference between the modern and the old sound Lutheran theology of conversion and election, we now undertake to present the main points of these doctrines themselves.  First, the doctrine of Conversion.
\vspace{.30cm}
\hrule
\vspace{1.25cm}
     \textbf{Luthardt} finds fault with the Formula of Concord as well as the orthodox Lutheran dogmaticians that they do not begin the cooperation of man already in conversion, but let it enter only after conversion.  Walther’s position, on the contrary, is the following: \begin{displayquote}{\footnotesize no kind of cooperation of man in conversion is to be admitted, neither from natural nor from so-called spiritual powers, but it must be maintained that God is alone active in conversion, but man is purely passive {\scriptsize\textsc{(mere passive)}}, merely \textit{subjectum convertendum}.}\end{displayquote}  If this is not maintained, if man is allowed to cooperate or contribute toward his conversion, then the characteristic feature of the Christian doctrine whereby it distinguishes itself from heathenism is  abandoned, then man is no longer saved by grace, then the doctrine of justification is subverted, then the assurance of the state of grace ceases.  Walther’s presentations with regard to the doctrine of conversion have the purpose of excluding synergism in every form, even the most subtle.

     Walther first rejects the teaching which openly accepts a cooperation of man toward his conversion or a preparation for it from natural powers.  Walther preserves the boundary between the kingdom of grace.  The distinction between these two spheres, he points out, “\textit{is not merely of degree but of kind}”.  Between nature and grace there is a gulf which only God’s almighty work of grace can bridge.  Hence there is no preparation for conversion on the part of natural man, as, for example, by a decent life, by “\textit{normal use of reason}”, by education and culture, etc.  In opposition to Kahnis, who attributes the rapid propagation of Christianity in the environment of the classical world in part to this environment itself, Walther says: \begin{displayquote}``\textit{In classical Athens Paul did not at all experience that people there were prepared for Christ more than others, and we are convinced that Dr. Kahnis experience nothing of the kind in classical Leipzig, but rather the reverse; as far, that is, as the Gospel is still preached there}.”\footnote{L.u.W., 1878, p.261—264.}\end{displayquote}  Yes, not only Scripture\footnote{Synodical report of the Northern District, 1873, p.47}, but even experience teaches that external worldly decency is no basis for conversion; “\textit{for often just the most vicious heathen have accepted the Gospel first of all}”.  Yes, \begin{displayquote}“\textit{outward worldly decency is often the most powerful hindrance to conversion.  It is for this reason no doubt that God withdraws his hand from many a man and allows him to fall into sin and shame in order that He may bring him to conversion}”.\footnote{L.c..45.}\end{displayquote}

     But there is also no cooperation toward conversion from so-called spiritual powers, or from powers conferred by grace.  This was indeed the position of the \textbf{Latermannian} synergists in the Seventeenth Century, and this is also the position of most modern Lutherans.  They say:  \begin{displayquote}{\footnotesize Indeed man can not cooperate toward his conversation from natural powers, but the man who lives under the sound of the Word, and stands under the influence of converting grace, can be active toward his conversion through the powers.}\end{displayquote} Iowa says: \begin{displayquote}{\footnotesize The man who is not yet converted but stands under the influence of converting grace can freely decide for conversion through the powers conferred upon him by grace.  Upon this “\textit{free decision}” depends his conversion and salvation.}\end{displayquote} Ohio says:  \begin{displayquote}{\footnotesize The man who is not yet converted can by God’s grace so conduct himself that upon him before others salvation is conferred.  Upon this “\textit{conduct}” depends his conversion and salvation.}\end{displayquote}  Thus it is asserted that the man who is not yet converted can through powers of grace be active to bring about his conversion.  It is also principally against these forms of synergism that Walther directed to his fight.

                Walther points out ever and again that this position involves a self-contradiction.  The right use if powers of grace implies a spiritual life-principle in man, or a man who can make the right use of powers of grace must already be converted.  He says: \begin{fancyquotes}If anyone says \begin{displayquote}“\textit{he ascribes to man a synergism toward his conversion not through his natural powers but only in the sense that he cooperates through powers conferred upon him by grace for this purpose}”,\end{displayquote} that is merely a theological sleight-of-hand.  For he who is himself able to effect something through powers of grace must either possess by nature the ability to put these powers of grace to use, or else he is already converted.\footnote{L.u.W., 1885, p. 109.}\end{fancyquotes}  In more detail Walther says on the same point: \begin{fancyquotes}Only after we are converted do we ourselves begin to work; the new man must first be born, then he begins to bestir himself, to speak, to do something; previously he does nothing at all, just as a child does nothing to bring itself to birth.  Hence man also cannot decide for himself {\scriptsize\textsc{(in conversion)}}. \par Many imagine conversion in such a way as though man found himself confronted with a cross-roads where the ways to heaven and to hell diverge; now man is given his choice which way he will go; if he goes the right way he will be converted, if he goes the wrong way he will be lost.  Thereby all honor is likewise taken from God; for if man can himself decide for the good, then there must be some good in him, and the decision itself would be a good work which he does before his is yet converted. --Those who hold this false doctrine of decision say indeed: \begin{displayquote}{\footnotesize Our doctrine takes no honor from God, for we do not say that man decides by his own natural powers, but we say that he does this with the powers of grace which are given him, and so nothing at all is ascribed to man;}\end{displayquote} --but they do not consider that only he can possess and use powers who is already alive.  Take a stock or stone and suppose that powers are blown into it – the stone would not trouble itself at all about the powers but would remain as before.  Powers presupposes a subject which uses the powers; and so man would have to be converted already in order to be able to convert himself; he would have to be already awakened in order to be able to convert himself; he would have to be already awakened in order to be able to awaken himself; he would have to be already renewed in order to be able to renew himself. \par No, as soon as a man is so far along that he can use the divine powers of grace he is also converted, then God has already decided and determined him, then He has already given him a new heart, then He has already regenerated Him through His Holy Spirit.\footnote{Report of the Western District, 1876, p. 67, 68.}\end{fancyquotes}  Walther says, with the old theologians who opposed the Latermannian synergism: \begin{displayquote}``\textit{Spiritual powers are not first given, that man may afterwards convert himself by means of them, but the bestowal of spiritual powers is in fact the conversion itself.}\footnote{L.u.W., 1872, p. 268.}\end{displayquote} If one says: \begin{displayquote}{\footnotesize The Holy Spirit so operates in liberating a man that a man can thereafter convert himself,}\end{displayquote} -- then Walther asks: \begin{displayquote}“\textit{Can a man be liberated and yet not be converted or regenerated?  The liberation of the man is itself the conversion or regeneration}”.\end{displayquote}

                As this doctrine is self-contradictory, so also it contradicts Scripture, the Lutheran Confession, and also experience.  According to Scripture conversion is “\textit{a great miracle which God performs}”, which God brings about by His good pleasure, and in which every cooperation of man is excluded.\footnote{Report of the Western District, 1876, p. 63, 65. Jeremiah 31:18; Philippians 2:13; Psalms 51:10; Isaiah 65:1; 2 Corinthians 4:6.}  Conversion according to Scripture is worked by God the Holy Ghost by grace alone for Christ’s sake.\footnote{Report of the Northern District, 1873, p. 43.56;  Romans 3:23-24; Ephesians 2:1 ff.; 2 Timothy 1:9; Colossians 2:12.}  In particular Walther refers to those passages of Scripture in which conversion is described as a new creation, an awakening from death, a new birth.  He says, for instance: \begin{fancyquotes}Holy Scripture compares conversion with creation, for we are called {\scriptsize\textsc{(after the change which has taken place in us through conversion)}} new creatures.  But what can the thing created do toward its own creation?  What did the world do toward its creation? For it was not yet there at all; so it could also do nothing.  What did Lazarus do toward his re-vivification? – for conversion is called a quickening in Holy Scripture – for he was dead; therefore he could also do nothing. \par Christ did it; He said: ‘{\color{red}\textit{Lazarus, com forth!}}' and Lazarus came forth.  Or what have we done toward our own birth?  Nothing, for all took place without us.  Only after we have been created, born, and quickened, our cooperation begins, not sooner.  Hence all who ascribe to man a cooperation toward his conversion thereby overthrow the entire Scriptural doctrine of conversion.  For, in the first place, we are entirely dead in sins, so that we can in no way cooperate toward our conversion, and, in the second place, the apostle says: \begin{displayquote}‘\textit{It is God which worketh in you both to will and to do of His good pleasure}’.\footnote{Report of the Western District 1876, p. 69.}\end{displayquote}\end{fancyquotes} – Walther offers the proof that according to the teaching of the Lutheran Confession the cooperation enters only after conversion, that in conversion man is merely subjectum convertendum, i.e., that he is purely passive {\scriptsize\textsc{(mere passive)}}, not active, e.g., in “\textit{L.u.W.}”.\footnote{L.u.W, 1872, p. 259 f., 290f.}  The champions of the teaching that man by virtue of grace is active toward his own conversion have indeed claimed that they were able to hold fast the pure passive of the Confession.  But Walther replies: \begin{displayquote}“\textit{To assume a synergism {\scriptsize\textsc{(cooperation)}} of the human will with divine grace not only after completed conversion but also during the act of conversion and still to be in agreement with the Confession of our Church is obviously a contradictio in adjecto. – For cooperation {\scriptsize\textsc{(which is activity)}} and passivity so completely exclude each other that it seems foolish even to waste a word on the matter}”.\footnote{L.u.W., 1872, - 289 f.}\end{displayquote} – Walther also appeals to the experience of Christians.  He writes, for instance: \begin{fancyquotes}We on our part can not only not understand how Prof. F. can regard this as Lutheran doctrine, but also not how any Christian who has come to true faith can so judge.\footnote{namely, that the conversion and salvation of definite individuals should depend upon their own free decision}  If we should say that we came to faith, while so many of our contemporaries, who, let us merely say, were not more depraved than we, remained in unbelief, for the reason that we freely decided for God with our own will: we would thereby have to deny our innermost Christian consciousness. \par Also all those who bear the unmistakable tokens of being truly believing Christians and who have communicated their experiences to us have always hitherto confessed that their having become believers truly did not have its basis in their own free decision but in nothing else than an incomprehensible eternal mercy of God in Christ.  All who with the poet could triumphantly exclaim: \begin{displayquote}{\footnotesize ‘Now I have found the firm foundation’}\end{displayquote} we have always heard confess with the same poet:

                  \begin{displayquote}
                    {\footnotesize It is that mercy never ending,

                    Which human wisdom far transcends,

                    Of Him who, loving arms extending,

                    To wretched sinners condescends;

                    Whose heart with pity still doth break

                    Whether we seek him or forsake.}\footnote{L.u.W., 1872, p. 243-244; Western Dist. 1876, p. 64-65}\end{displayquote}\end{fancyquotes}

                This teaching of a self-determination for grace underlies the assumption of a neutral state {\scriptsize\textsc{(status medius)}}, a state which is supposed to be intermediate between being unconverted and being converted.  There is supposed to be a state in which a man is indeed not yet converted but yet has been so far liberated by calling grace that he is able to be active toward his conversion, to decided for grace.  Walther calls this \textit{status medius} a fiction, while he at the same time carefully distinguishes between truth and error in the claims which are brought forward for the support of this neutral state.  Walther does not deny that impulses {\scriptsize\textsc{(Bewegungen)}}, and indeed powerful impulses precede conversion in most cases.  In this connection he often used the figure of a fortress which is to be stormed, whereby a great stir is called fort within the fortress.  So also in unconverted men powerful motions may take place during the preaching of God’s Word. \par Walther was accustomed to adduce the examples of \textbf{Felix}, \textbf{Agrippa}, etc.  The former trembled as Paul reasoned of righteousness, temperance, judgment to come {\scriptsize\textsc{(Acts 24:25)}}.  The latter was so moved by the preaching of the apostle that he said: “\textit{Almost thou persuadest me to be a Christian}”. \par But these motions in the still unconverted prove nothing for a \textit{status medius} or for a cooperation from spiritual powers before conversion.  There is still no life in man in connection with these motions.  \begin{displayquote}“\textit{The Holy Ghost is only working from without into man.  The soul of the man, although moved by the Holy Ghost, has not yet become the dwelling-place of the Holy Ghost}”.\end{displayquote}  No spark of spiritual life has yet been kindled in the man himself.  The impulses have not yet become the man’s own, that is to say, they do not come from a life-center {\scriptsize\textsc{(principium vitale)}} already existing in the man.  As soon therefore as the influence from without ceases the impulses also cease.  Walther was accustomed to use in this connection the figure of pressure upon gutta-percha.  \begin{fancyquotes}A \textit{gutta-percha} yields to the pressure of the finger, but as soon as the finger is removed immediately reoccupies its former space, so also a holy longing and yearning often arises in an unconverted man through the operation of the Holy Spirit without his being in the least active in it; but as soon as the Holy Ghost withdraws His hand this longing also vanishes.  Only when man has given in to the operations of God, when grace is no longer merely an influence working from without {\scriptsize\textsc{(gratia assistens)}} but has become indwelling in him {\scriptsize\textsc{(gratia inhabitans)}} can he cooperate.  He who teaches otherwise can only do it upon Pelagian premises.\footnote{Report of the Northern Dist. 1873, p. 51, 52}\end{fancyquotes}  Walther declares it to be very important that “\textit{the external and the internal working of the Holy Spirit}” be not confused the one with the other.  As long as in man great motions indeed occur, but are only the consequence of the external operation of the Holy Ghost, the man is still unconverted, still in a state of wrath, and no kind of cooperation, no ability to decide for grace, no good conduct by virtue of grace is to be ascribed to him.  But so soon as spiritual power has become man’s own, so soon as a spark of spiritual life has been kindled in man, and man can now make a decision, he is already converted.  We shall now cite a few more utterances of Walther relevant to this point.  He says: \begin{fancyquotes}The synergists after Luther’s death did not present their error in such a refined and subtle manner as did the \textbf{Helmstädt} synergists in the Seventeenth Century.  The course of synergism was the same as the course of error always is.  First came gross \textbf{Arianism}, then the finer semi-arianism; first gross Pelagianism, then the fine semi-pelagianism; first gross synergism, then the fine, so to speak, semi-synergism.  It sound quite fine when recent theologians say: \begin{displayquote}{\footnotesize When God gives unconverted man the power he can cooperate toward his own conversion.}\end{displayquote}  But it is not correct; for a dead man cannot use the powers conferred upon him as long as he does not have that power which is necessary to the use of such powers, as long, that is, as he does not have life in himself.  One can roll a dead body about and operate upon it electrically so that it opens its eyes, opens its mouth, and the like, but all this is only the consequence of powers operating upon it from without; only that one can move himself who has subjectively come into possession of the power.\footnote{Report of the Northern Dist., 1873, p. 52, 53.}\end{fancyquotes}  Furthermore: \begin{displayquote} “\textit{When the father say that one must not think of conversion in such a way as though a man could simply take it lying down, as though it took place as in a sleep, but much must take place in the understanding, will, and affections, this is falsely applied by recent theologians to the cooperation of man toward his conversion.  As the garrison of a fortress does not do anything toward shooting breaches in the walls and bulwarks and towards a setting the defenses on fire at various points, but will rather only close up the breaches and quench the flames, such is the situation also in conversion; in however lively a manner things may take place, yet it is only a life which is suffered, and man in all this is only a passive, not an acting participant.  But though he remains pure passive, he is not in this case like the sealing wax which neither knows nor feels anything of the impression of the seal, but man knows and perceives the work of the Holy Spirit upon him}.\footnote{l.c., p. 51.}\footnote{Walther’s frequently used picture of the fortress to be stormed which he carries out particularly in the {\scriptsize\textsc{Report of the Western Dist., 1876, p. 68, 69}}, has been used, especially on the part of the Iowa Synod, to charge Walther with teaching a most terrible conversion by force.  They have not tired of spreading about the world the report that the Missouri Synod follows Walther in teaching a “\textit{bomb and canon conversion}”.  From the connection it is entirely clear what the \textit{tertium comparationis} is in this figure, namely this, that man in no way comes to meet the activity of the Holy Ghost, but only resists it, and indeed resists until the heart of man is changed by the Holy Ghost, that is, converted.  But this is also the clear teaching of the Lutheran Confession. {\scriptsize\textsc{Formula of Concord, Art. II, par. 21 (Triglot, p. 889)}}: \begin{displayquote}“For man neither sees nor perceives the terrible and fierce wrath of God on account of sin and death, but ever continues in his security, even knowingly and willingly, and thereby falls into a thousand dangers, and finally into eternal death and damnation; and now prayers, no supplications, no admonitions, yea, also no threats, no chidings, are on any avail, yea, all teaching and preaching is lost upon him, until he is enlightened, converted, and regenerated by the Holy Ghost”.\end{displayquote}  Shortly before {\scriptsize\textsc{(par. 18)}} a “\textit{hostiliter repugnare}” is ascribed to man, and shortly after {\scriptsize\textsc{(par. 22)}} an “\textit{obstinate enmity against God}”.  That unconverted man only resists the Holy Ghost and indeed “\textit{hostilely resists}” can be no matter of wonder to anyone who maintains that in natural man there is nothing good by which he would in any way be ready to come to meet the Gospel.  But this is the very point in which modern theology, also in the Iowa and Ohio Synods falls short, as Walther has likewise pointed out.  The Iowan-Ohioan doctrine is based on the assumption that before the spiritual powers there is still something good in man.  They say indeed: \begin{displayquote}“\textit{Through powers of grace}” the still unconverted man can decide for or against grace.\end{displayquote}  But the “\textit{powers of grace}” are certainly not neutral, equally effective in either direction {\scriptsize\textsc{(indifferentes)}} toward conversion or turning away.  \begin{displayquote}“And so there must be a power in man before the powers conferred by the Holy Ghost, by which, with the help of assisting grace and the powers bestowed by the Holy Ghost, that which is necessary unto conversion is performed, and by which also the unwillingness to be converted is effected.  But this is Pelagianism and synergism itself''\end{displayquote} -- and so it comes to light, as soon as one looks into the matter more carefully, that also in connection with the phraseology that man freely decides by virtue of grace or that man conducts himself rightly by virtue of grace conversion is placed with regard to the decisive factor in the natural powers of man, or natural powers are attributed to man whereby he deals rightly with the “\textit{posers of grace}”.  Thus also this subtle form of synergism, that man converts himself by powers of grace, exposes itself as Pelagianism.  Walther says: \begin{displayquote}“Synergism is at bottom nothing else than papistical leaven; for the Papacy is nothing else than hierarchism on the one side and Pelagianism on the other.  Synergism or semipelagianism is only a more euphemistic expression, but in fact the same as Pelagianism. \par When the devil finds himself exposed he dons another garb and seeks through the subtle false doctrine to plunge people into gross heresy to the forfeiture of their salvation – but the final decisive question is just this: Who is the one who is to manage the powers conferred upon him from elsewhere? A dead person can do nothing with vital powers laid into the coffin unless he is first awakened to life.  Christ did not say to Lazarus, the young man at Nain, or the daughter of Jairus, before they were quickened: Here you have vital powers; now make use of them that you may live!  But He made them alive with His Word. – So the Iowans may talk as they will; they let it be known that they ascribe to the unconverted man power to make use of powers bestowed upon him”. {\scriptsize\textsc{(Report of the Northern Dist, 1873, p. 56, 57.)}}\end{displayquote}  That a cooperation toward conversion is ascribed to natural powers also comes to light at times in undisguised form.  So, e.g., when Ohio says that conversion and salvation does not depend upon grace alone, but in a certain respect also upon the conduct of man.  Now what does not depend upon grace depends upon the natural powers of man.  \textit{Tertium non datur!}  Hence if conversion and salvation should not depend only upon grace but besides and in addition also upon conduct, then this conduct must be based upon natural powers. \par Furthermore: that, in spite of all the talk of a conduct by virtue of grace and of a self-decision “\textit{by virtue of grace}”, nevertheless they have in mind a conduct and a self-decision by virtue of natural powers, is evident from the fact that by means of the ``\textit{conduct}” and the “\textit{self-decision}” they wish to explain to human reason why one man is saved rather than others.  Such a “\textit{basis of explanation}” {\scriptsize\textsc{(Erklärungsgrund)}} is obtained only if one lets the decisive conduct be effected purely by natural powers.}\end{displayquote}

 

 

     Thus Walther is determined to hold fast that neither before nor in conversion dies any cooperation of man take place.  During conversion powerful motions take place in man, but in connection with them man is not active, cooperative, but passive.  Hence for Walther transitive and intransitive conversion are not two different stages in the process of conversion, so that God should first convert man or give him powers unto conversion, in order that thereafter man might convert himself, but for him transitive and intransitive conversion coincide in fact.  He says: \begin{displayquote}“\textit{Transitive and intransitive conversion are merely different ways of looking at the same thing.  Man is converted when God converts him.  -- The ship turns when the steersman turns it.''}\end{displayquote}

     Walther also repeatedly dealt with the common objections, that man, if he in no wise cooperates toward his conversion, is in no wise active, does not decide, etc., but only suffers what God works in him, would be degraded to the level of a machine, that conversion would be a conversion by force, that the “\textit{moral element}” in conversion would be lost, etc.  Walther answers the objection, that man would sink to the level of a machine if he could not decide freely for or against grace, by reducing the opponents \textit{ad absurdum} and says: \begin{displayquote}``\textit{If a man is not degraded to the level of a machine when the so-called prevenient grace calls for the motions in man {\scriptsize\textsc{(motus inevitables)}} without man’s own decision or activity, which is admitted by the opponents, then this would also not be the case when converting grace works faith without the free decision or activity of man}.''\footnote{L.u.W., 1872, p. 296.}\end{displayquote}  The “\textit{conversion by force}” Walther repudiates as an insinuation of which synergists have always been guilty against confessionally loyal Lutherans.  Only then could one speak of a “\textit{conversion by force}” if the Lutherans taught a conversion in which no inner change took place in the understanding, will, and heart of man.  But the Lutheran doctrine is this: \begin{displayquote}{\footnotesize Although the human will is corrupt in the extreme and in no wise cooperates toward conversion, yet in it a total change takes place in and through conversion: in and through conversion it is changed from unwilling to willing.}\end{displayquote}  In this conversion consists.  “\textit{God creates the willingness and thereby and therewith God converts man}”.  The will of man is the subject in which conversion takes place.  Through conversion not the Holy Ghost but man becomes a believer.  In the charge of “\textit{conversion by force}” on the part of the champions of self-decision, etc., there lies an artifice.  They pretend that they want to insist upon absence of coercion in conversion, whereas in reality they want to secure in this way a cooperation toward conversion.  After Walther has granted over against Iowa that “\textit{man’s own free decision}” may be accepted \textbf{if} all that is meant thereby is “\textit{that man is not converted by coercion, but that in conversion also the will of man is moved to will and that it is man himself who believes}”, he continues: \begin{fancyquotes}But that Prof. F. with his ‘\textit{free decision}’ does not wish to assert only a freedom which is identical with the absence of coercion is unfortunately only too evident when he expressly writes: \begin{displayquote} ‘He, the natural man, receives in consequence of the operation of grace \textit{arbitrium liberatum}.  His will, enslaved by sin, is so far liberated that he can by his own will decide freely for or against God’.\end{displayquote}  Yea, in order that he may be correctly understood he makes Dr. Philippi’s words his own: \begin{displayquote}‘As, accordingly, a certain synergism of man in the use of the means of grace even before the beginning of the inner working of divine grace not only after completed conversion, but also during the act of conversion itself, only indeed no synergism of the natural free will but only a synergism of the will by grace.'\footnote{L.u.W., 1872, p. 258.}\end{displayquote}\end{fancyquotes}

      With regard to the saying of recent theologians: “\textit{Faith is free obedience}”, Walther remarks: \begin{displayquote}“\textit{Faith is indeed ‘free’, that is, uncoerced, but not a matter of ‘free choice and free determination’, as the moderns want to make it}”.\end{displayquote}  And, as regards the concern of the moderns that “\textit{ethics}” might suffer if man would not “\textit{freely decide}” for faith and faith accordingly would not be a “\textit{personal act}” {\scriptsize\textsc{(Selbstthat)}} of man, Walther again refers to the fact that also most of the moderns let “\textit{the first influence}” of grace come about without man’s cooperation or personal activity.  Now if through this occurrence “\textit{ethics}” is not overthrown, then it is also not overthrown through the occurrence of the conversion itself, even though God alone is active therein and man does not conduct himself actively but only suffers what God works in him.  Walther refers in this connection to the creation.  \begin{displayquote}“\textit{The will to good was created in Adam without his {\scriptsize\textsc{(Adam’s)}} cooperation, and yet this was not contrary to ethics}”.\end{displayquote}  Walther here breaks out in the words \begin{displayquote}“\textit{It is offense at Christ crucified, at the religion of grace, which makes men unwilling to let conversion take place without man’s cooperation}”.\end{displayquote}  Elsewhere Walther demonstrates that men have thought up the whole \textit{status medius}, in which man is supposed to be indeed not yet converted, but still through calling grace already enabled his conversion by good conduct, only for the purpose of solving the mystery that man is saved alone by grace and yet damned by his own fault.\footnote{L.u.W., 1872, p. 293, 294. Note.}
      When Walther repudiates the \textit{status medius} in this manner his answer to the question whether conversion takes place successively or in a moment is already evident.

 
%%% Local Variables:
%%% mode: latex
%%% TeX-master: "../main"
%%% End:

\include{parts/part15}
\include{parts/part16}
\chapter{Election -- Wide or Narrow}
\hrule
\vspace{.30cm}
We desired to conclude our presentation of Walther's doctrine concerning the election of grace by a more careful investigation of certain particularly important points. We have already offered a detailed treatment of two such points, namely, of the real center of Walther's doctrine and of the relation of faith to election.
\vspace{.30cm}
\hrule
\vspace{1.25cm}
A third point which deserves special consideration is the question whether there is such a thing as an election in a wider and in a narrower sense. In particular was the question debated in the recent doctrinal controversy whether the Lutheran Confessions in the Eleventh Article of the Formula of Concord treats of an election in a wider sense.

Already the later Lutheran dogmaticians assert that the Formula of Concord treats of an election in a wider sense. It is easy to understand how they came to make this assertion. The Formula of Concord certainly teaches that election is not merely an ordination to salvation but also to all which belongs to the attainment of salvation on the basis of Christ's merit, to the call, to conversion, to faith, to justification, to sanctification, to preservation in faith. All these particulars are expressly mentioned by the Formula of Concord as a consequence and effect of election\footnote{Solid. Decl. XI, par. 8, 44, 45ff.}.  This does not fit in with the doctrine adopted by the later dogmaticians that election took place "\textit{in view of persevering faith {\scriptsize\textsc{(intuitu fidei finalis)}}};" for according to this doctrine of the dogmaticians the objects of election are such persons as have already, according to God's foreknowledge on which election is supposed to be based, the entire way of salvation from conversion till the last breath of their earthly life behind them as a happily accomplished fact. In order not to place themselves in open opposition to the Confession, they usually say that the Confession uses the word election in a wider sense, in which connection they indeed overlook the fact that the Formula of Concord very emphatically protests against such an idea by explaining from the start that it is speaking of an election which "\textit{does not extend at once over the godly and the wicked, but only ever the children of God, who were elected and ordained to eternal life before the foundation of the world was laid}"\footnote{XI, par. 5}. With the closer investigation of the question whether the alleged "\textit{wider sense}" of the Formula of Concord is founded in Scripture the dogmaticians under discussion concern themselves but little; only in isolated cases do we find the direct charge that the Formula of Concord has an unbiblical concept of election\footnote{(So Caspar Loescher, whose statement is cited by Walther, Berichtigung, etc., p. 77)}. Quenstedt, on the other hand, is satisfied with the declaration that his concept of the election of grace, differing as it does from that of the Formula of Concord, is the only correct one\footnote{Theol.-did.-pol. III, 89}. Walther speaks in more detail concerning the exposition of the Formula of Concord on the part of the later dogmaticians in his \textit{Berichtigung, p. 76ff}.\footnote{Cf. also L.u.W., 26, 68, 167}

Walther himself teaches: \begin{displayquote}``\textit{There is an election of grace in only one sense, and that is the sense which is presented by the Formula of Concord on the basis of Scripture. That is the election of grace which extends not over all men but only over the children of God who are being saved, and which is not merely an ordination to the termination of the way of salvation, to blessedness, but also a cause of the entire Christian status through which God leads the elect unto eternal life.}\footnote{L.u.W., 26, 292f., 26, 72, 135f., 161f., 165, 166, 355}\end{displayquote}

In reply to the appeal to the so-called "\textit{eight points}"\footnote{Solid.Decl. XI, par. 15-22} as proof that the Formula of Concord teaches an election in a wider sense, Walther says: \begin{displayquote}"\textit{The eight points are adduced, inasmuch as God leads the elect to salvation on no other path and in no other order than He is willing to lead all men to salvation}."\end{displayquote} Or: \begin{displayquote}"\textit{In the eight points 'the manner is declared' in which God wants to bring the elect to salvation, aid, promote, strengthen, and preserve them}."\end{displayquote} That the eight points arc not to be understood in any other way Walther finds expressly testified in the Formula of Concord itself, namely, in the words preceding and following the eight points. He points out that only in this way is one guarded against the supposition, so discreditable to our Confession of our Church, that it right at the start defines election as something which has reference only to those who are being saved, and soon thereafter, without giving any indication to this effect, gives a wider sense to the word election.\footnote{Beleuchtung, p. 64ff., L.u.W., 26, p. 298ff}

Against this doctrine, that election extends only over the elect children of God and is the cause of their faith, the charge has been raised that it overthrows the universal will of grace, or – which amounts to the same thing, – that through this doctrine of election a special way of salvation for the elect is posited outside and apart from the universal way of salvation. We here come to a fourth controverted point, namely, how the doctrine of the election of grace is related to the universal way of salvation or to the universal will of grace.

In almost innumerable variations during the last ten to twelve years it has been objected against the doctrine – that the election of grace which extends only over the children of God is a cause of their {\scriptsize\textsc{(the elect's)}} conversion and perseverance in faith – that then there would be two ways of salvation, one for the elect, who obtain faith and salvation in consequence of their eternal election, and another for the rest of mankind which lacks the power to effect and preserve faith. According to the doctrine of election propounded by Walther, God is supposed to have "\textit{so arranged it}" by His election that the majority of men could not come to faith or at least could not remain in faith. – There is no doubt that through this objection many simple souls have been and are still being predisposed against the Scriptural and confessional doctrine of election. It did not help Dr. Walther at all that he ever and again unceasingly declared: \begin{fancyquotes}We believe, teach, and confess that no man is lost because God would not save him, or because God with His grace passed him by, or because He did not offer the grace of perseverance to him also and would not bestow it upon him; but that all men who are lost perish by their own fault, namely on account of their unbelief, and because they have obstinately resisted the Word and grace of God to the end, of which 'contempt for the Word of God the cause is not God's foreknowledge\footnote{vel praescientia vel praedestinatio}, but the perverse will of man, which rejects or perverts the means and instrument of the Holy Ghost, which God offers him through the call, and resists the Holy Ghost, who wishes to be efficacious, and works through the Word, as Christ says: \begin{displayquote}`\textit{How often would I have gathered you together, and ye would not!}'\footnote{Matthew 23:37}\end{displayquote} Hence we heartily reject and condemn the contrary Calvinistic doctrine.\footnote{The fourth of the 13 Theses, Lutheraner, 1880}\end{fancyquotes}  This protestation, as we have said, did not help Dr. Walther at all. In spite of it, many stuck by the assertion that Walther, resp. the Missourians, taught a double way of salvation. The universal will of grace and the doctrine that election is a cause of faith and of the entire Christian status of the elect cannot, said they, stand side by side. According to them the analogy of faith demands the surrender of this doctrine of the election of grace.

Over against this argumentation Walther first of all guards the correct principle. He calls attention to the fact that also the Calvinists appealed to the analogy of faith against the Lutheran doctrine of the Lord's Supper and claimed that the essential presence of the body and blood of Christ in the Lord's Supper, as taught by the Lutherans, conflicts with the truth clearly attested in Scripture that Christ's body is a true human body. But against this the Lutherans always asserted: \begin{displayquote}"\textit{The Scripture teaches both: that Christ's body is a true human body and that it is nevertheless truly distributed in the Lord's Supper; hence both must be believed and the one must not be placed in opposition to the other}."\end{displayquote} And so Walther demands that also in connection with the question whether the doctrine of universal grace, according to which God earnestly desires to save all men, and the doctrine of particular election, which is a cause of the faith and the entire state of grace of the elect, harmonize with one another, – that in this question the Scripture principle be held fast. The only question is whether the Scripture does not teach, just as clearly as it teaches universal grace, also this doctrine, that election pertains only to those who are being saved and is the cause of their faith and of their entire Christian status. This Walther teaches and most emphatically rejects the assertion that the passages of Scripture which treat of the election of the saved are obscure and hard to understand. Accordingly he demands: \begin{displayquote}``\textit{Both must be believed by one who wants to be a Christian and even an orthodox Lutheran. To correct one Scripture doctrine by another for the sake of one's reason, because the former seems obscure and contradictory, yea, entirely to cancel it on the pretext that obscure passages must be interpreted according to the clear, – this is a terrible abomination}.''\footnote{Beleuchtung, p.25ff., L.u.W., 29, 12ff.; 26, 264-270}\end{displayquote}

But after Walther has guarded the correct principle, he also demonstrates, through a closer investigation of the matter itself, that two different ways of salvation simply do not result from the doctrine that election is a cause of the faith and salvation of the elect. He shows: God leads the elect upon no other way of salvation than that upon which He earnestly wills to lead all men.\footnote{L.u.W., 26:296} The elect are by grace alone, for Christ's sake, through the Gospel, called, enlightened, sanctified, and preserved in time, and God has from eternity determined to deal with the elect on this basis and in this manner\footnote{L.u.W., 26:367}. Both the eternal counsel of election and also the execution of it in time correspond exactly to the universal way of salvation. Walther rejects as false the teaching of the Calvinists, that God has first elected to salvation in an absolute manner and then subsequently determined to redeem the elect through Christ and to endow them with faith. He writes: \begin{fancyquotes}We believe, teach, and confess that God did not first unconditionally and absolutely choose the elect unto salvation, as the Calvinists say, and then subsequently determine to give them faith as the means to obtaining salvation, but that God has at the same time elected them to all '\textit{which},' as our Confession says, '\textit{procures, works, helps, and promotes our salvation and what pertains thereto}' and so also indeed, and before all, to faith; as the Formula of Concord expressly says when it cites as proof from words just quoted, the text {\scriptsize\textsc{Acts 13:48}}, \begin{displayquote}`\textit{And as many as were ordained to eternal life believed}.'\end{displayquote} Hence we also believe, teach, and confess that according to God's Word the just God could not elect any man to salvation in an absolute manner, i.e. to say, if God had not first provided for his redemption and if He had not at the same time elected him to faith, i.e., if He had not at the same time determined to give him faith, for aside from Christ there is no salvation \footnote{Acts 4:12} and `\textit{without faith it is impossible to please God}'\footnote{Hebrews 11:6}. Hence when the Calvinists want nothing to do with an election '\textit{in view of faith},' that means something entirety different than when we reject this teaching. The Calvinists do this, as we said, because according to their teaching God has first elected to salvation in an absolute manner without regard to Christ or faith; we do this because God's Word teaches that God has decided to give us by grace not only salvation, but also faith, since the election unto salvation and unto faith coincides.\end{fancyquotes}  Walther therefore declares it to be a gross perversion of his doctrine when any one asserts that by it faith is excluded from the election of grace, and proceeds: \begin{displayquote}"\textit{We on our part rather regard faith as so necessary to salvation that we believe, teach, and confess that God, according to {\scriptsize\textsc{Romans 8:29-30}}, chose the elect first unto the Gospel call and thereby unto faith {\scriptsize\textsc{(not according to temporal sequence but according to the nature of the matter)}} and unto justification, and then unto salvation}."\footnote{Beleuchtung, p.19, 20}\end{displayquote}

In order further to evince that through the doctrine of eternal election as a cause of the faith and salvation of the elect no special way of salvation for the elect is posited, Walther ever and again points out that in connection with this doctrine of election we are confronted with no other mystery and no other difficulty than that which meets us in the doctrine of conversion and in general in the contemplation of the universal way of salvation in itself. If, e.g., human reason is allowed to draw its so-called necessary consequences, it concludes: \begin{displayquote}{\footnotesize If grace alone is the cause of faith and of preservation in faith, as Scripture testifies, and if nevertheless only a part of the human race lying in the same total depravity is converted and preserved in faith, then it is evident that in the case of the rest of mankind this grace has been either not at all or not sufficiently efficacious; there is, in spite of all the assurances of Scripture that God would have all men to be saved, no such thing as universal grace.}\end{displayquote}  Thus rationalizing human reason arrives at a double salvation from the premise of the simple concept of grace. Hence also the assertion of the modern rationalistic, synergistic theologians that the Formula of Concord would indeed fall into Calvinism if it should let actual faith be worked by the Holy Ghost\footnote{Cf. here the discussions in L.u.W., 1890, 349ff}. The Lutheran Church, on the other hand, in the clear knowledge that such conclusions are irrelevant deductions of reason, holds fast to the one revealed way of salvation, and says:\begin{displayquote}{\footnotesize  It is only and alone the work of God's grace that men are converted and saved, and it lies only and alone in the wicked obstinate resistance of man, and not in any lack of the gracious working of God in His Word, that men are not converted and saved {\scriptsize\textsc{(Hosea 13:9)}}.}\end{displayquote} The former, namely, the fact that those who are saved come to faith and are preserved in faith by grace alone, the Scripture traces back into eternity. Scripture says that God not only in time gives faith to those who are being saved and preserves it, but that He has already from eternity determined to do this for them. That is the election of grace. Therefore as little as one can raise the objection against the doctrine that God brings the saved to faith and preserves them in faith by grace alone, that thereby a double way of salvation is posited, so little can one raise this objection when the same effect is attributed to the election of grace, for the election of grace is nothing else than eternal grace viewed in relation to those who are saved. Here belong such utterances of Walther as the following: \begin{fancyquotes}If you, dear reader, are already by the grace of God standing in living faith, then let me further ask you; Did you perhaps give yourself faith?  –You will say: \begin{displayquote}{\footnotesize Ah, no; I could not do even the least thing toward my receiving through the Word of the Gospel a living faith, and I did not come to the Word, but the Word came to me.}\end{displayquote} -- Well!  Do you suppose then that you have just accidentally come to faith? –  You will doubtless answer: \begin{displayquote}{\footnotesize Ah, No; if I thought that I would indeed have to be a mere heathen; nothing takes place by chance.}\end{displayquote} – Well then; let me further ask you: To whom then do you owe it that you have through the Word of God come to faith? – You say: \begin{displayquote}{\footnotesize That I owe alone to the mercy of God and the most holy merit of Christ. It was God who opened my fast closed heart, as He did once for Lydia, that I gave attention to what I read and heard out of God's Word. I certainly did not deserve that in any way! On account of my many sins I rather deserved that God should neither have called me nor brought me to faith, but rather that He should have let me die and perish in my sins. My conversion is a mystery to me. Only so much I know, that I did nothing toward it.}\end{displayquote} – Do you suppose then that God first in time thought of bringing you to faith? then first, when your eyes were opened, and you recognized your wretchedness in sin and God's grace in Christ, came to faith, and became a different man? – you will say: \begin{displayquote}{\footnotesize How could I think that? For I know from God's Word that God has not only foreknown from eternity all the good which He does in time, but has also from eternity predetermined it.}\end{displayquote} – Let me then ask you just one more question: Do you also hope to be saved? – You will answer: \begin{displayquote}{\footnotesize Yes, such is my hope. If I did not hope that I would have to reject Luther's '\textit{Christian Questions};' indeed then I could not even with the entire holy Christian Church recite the Third Article in firm faith, in which it says: '\textit{I believe .... the life everlasting},' and I could not say with our Catechism: '\textit{I believe... that God will give unto me and all believers in Christ eternal life. This is most certainly true}.' And my dear Lord Jesus Christ says: \begin{displayquote} `\textit{My sheep hear My voice, and I know them, and they follow Me: and I give unto them eternal life; and they shall never perish, neither shall any man pluck them out of My hand}'.\footnote{John 10:27-28}\end{displayquote} So how could I doubt my salvation?}\end{displayquote} Just so, dear reader, – Behold, there have in brief words the entire doctrine of election that, and nothing else, is what the Formula of Concord teaches concerning election and what we teach with it.\footnote{Lehre von der Gnadenwahl, p. 58f.}\end{fancyquotes}

Permit us to add: \begin{displayquote}{\footnotesize In case anyone really holds with the Lutheran Church to both propositions, that the unbelief and damnation of those who are lost is to be traced alone to the obstinate resistance of man, while the faith and salvation of those who are saved is to be traced alone to the working of God's grace, then it can only be the result of intellectual confusion if such an one still claims that a double way of salvation is posited when it is said of the election of grace or of eternal grace that it stands in a causal relation to the faith and the entire Christian status of the saved.}\end{displayquote} To be sure, he who teaches that conversion and salvation do not depend alone upon the grace of God, but in a certain respect also upon the conduct of man, cannot do otherwise than regard the doctrine that the election of grace is the cause of the faith and preservation of the elect as a falsification of the universal way of salvation. For by this doctrine of election – to speak with the Formula of Concord – \begin{displayquote}``\textit{all opinions and erroneous doctrines concerning the powers of our natural will are overthrown, because God in His counsel, before the time of the world, decided and ordained that He Himself, by the power of His Holy Ghost, would produce and work in us, through the Word, everything that pertains to our conversion}".\footnote{Sol. Decl., XI, par. 44. Mueller, p. 714; Trigl. p. 1077}\end{displayquote}  The fact that our opponents see in this a falsification of the universal way of salvation or a special way of salvation for the elect aside and apart from the universal way of salvation is due to their holding in general a false doctrine of the universal way of salvation, specifically to their harboring the gross delusion that according to the universal way of salvation conversion and salvation does not depend upon the grace of God alone, but also upon the conduct of man, and that therefore a special way of salvation is posited for the elect when their conversion and election is made to depend not upon their conduct but alone upon the grace – the eternal grace – of God. As a matter of fact, the real situation is this: according to the universal way of salvation conversion and salvation depends upon the grace of God alone, and not – even in the thousandth part – upon the conduct of man, and according to the election of grace the case is not otherwise but exactly the same. By the election of grace according to which God has from eternity "\textit{not only before we had done anything good, but also before we were born}"\footnote{F.C., par. 88; Mueller, p. 723; Trigl. p. 1093}, endowed us with conversion, righteousness, and salvation\footnote{F.C. par. 45; Mueller, p. 714; Trigl. p. 1079}, the pure grace of God is only brought more clearly to light: "\textit{it establishes}," as the Formula of Concord says - \begin{displayquote}"\textit{very effectually the article that we are justified and saved without all works and merits of ours, purely out of grace alone, for Christ's sake}".\footnote{Formula of Concord XI, par. 43; Mueller, p. 713; Trigl. p. 1077}\end{displayquote}

The charge against Dr. Walther and the Missouri Synod, that with their doctrine of election, specifically with the doctrine that election is a cause of conversion and salvation, they assure a double way of salvation, will be silenced on the part of that sector of our opponents which knows what it wants only when {\scriptsize\textsc{(the opponents)}} have given up their false doctrine of the universal way of salvation.
\section{A Letter of Note} The confusion and delusion into which the leaders of the Iowa and Ohio Synods have driven the ignorant among their pastors, especially by the charge that Missouri teaches a double way of salvation, is evident from a document which has come into our hands. A pastor of the Iowa Synod, who in fighting "\textit{the Missourians}" in the State of Wisconsin, sent to a member of a congregation in Waushara County a writing in which we read: \begin{displayquote}``Dear Friend! On the third of July a man came to us who said that you desire from me proofs that the Missouri Synod in its writings teaches that God through His election is Himself to blame for the loss of part of mankind.\end{displayquote} \begin{displayquote}``Dear Friend, I assure you that this is undeniably evident from their writings, for in the fashion in which the Missourians teach the election of grace, there is nothing left for one portion of mankind than that they must through God's election go to hell; if they, the Missourians, do not directly say this, yet indirectly, that is, that it indisputably follows from their doctrine.-- But let us once again look at their statements concerning election is they stand in their own writings. -- I have in my heads a booklet by Pastor W. concerning the Missourian doctrine of election. I suppose that you have such a book also.\footnote{What he means is the tract of Dr. Walther:  The Doctrine Concerning Election Presented in Questions and Answers from the Eleventh Article of the Formula of Concord of the Evangelical Lutheran Church.} Now let us look first at Question No. 10, and especially at its answer. This reads thus: \begin{displayquote}`The eternal election of God not only foresees and foreknows the salvation of the elect, but is also, from the gracious will and pleasure of God in Christ Jesus, a cause which procures, works, helps, and promotes our salvation and what pertains thereto.'\end{displayquote} To this I tell you, my dear friend, if already the election of grace should now be the cause of my salvation, and that it procures and works everything, then I say that is false. Christ's merit and faith in it is the cause and all ground of my salvation, that is the right doctrine, no other. -- Let us look at Question and answer No. 11. Question: \begin{displayquote}`Is it then so important that the eternal election of God is a cause of our salvation and that it procures, works, helps, and promotes all that pertains thereto?' \end{displayquote} Answer: \begin{displayquote}`Yes, indeed, for upon this our salvation is founded that the gates of hell cannot prevail against it.'\end{displayquote} To this we say, it is important that we hear God's Word, believe on Christ, do not grieve the Holy Ghost, pray and work, then the Lord will by grace take us to Himself in heaven. That is important, very important. If the Missourians say the gates of hell cannot overthrow the election of grace and if my salvation is dependent upon it, then salvation can no more be lost to me. This is again false.''\end{displayquote} So far the writing. The pastor had apparently no idea that he had judged so severely the \textit{ipsissima verba} of the Formula of Concord which the leaders of our opposition might well take as a pattern for themselves.} \footnote{Translator's Note: The lame sentence structure and poor choice of words in the above quoted German letter is deliberately imitated by the translator. W.H.M.}
%%% Local Variables:
%%% mode: latex
%%% TeX-master: "../main"
%%% End:

\chapter{Election -- Assurance of Faith}
\hrule
\vspace{.30cm}   One of the questions which was thoroughly discussed in the controversy on election was whether or not a Christian can and should be assured in faith of his election to salvation.  \par For Walther this point is “\textit{one of the most important}”.\footnote{Lutheraner, 1880, p. 25}
\vspace{.30cm}
\hrule
\vspace{1.25cm}
                The modern representatives of the intuitu fidei in decided disagreement with most of the later Lutheran theologians denied that a believing Christian can and should be sure of his election.  They gave voice to such utterances as the following: \begin{displayquote} “\textit{Whether I am indeed elected even in the stricter sense I do not know.  I should believe and hope that I am.}\end{displayquote}  The Christians find themselves “\textit{from day to day between fear and hope as on trial between two millstones}”.\footnote{Berichtigung, p. 120, 121.}

                Walther points out ever and again that his opponents in accordance with the nature of their doctrine cannot do otherwise than deny Christians the assurance concerning their election.  Uncertainty concerning is a necessary consequence of the synergism harbored by the opponents.  If election does not depend on the grace of God in Christ alone, but also upon the conduct of man, then the Christian will have to doubt until his death whether he is elect, since no man can know whether he will conduct himself rightly in the future.  We adduce a few statements of Walther concerning this point.  He writes: \begin{displayquote}“\textit{The sect of the Armenians teaches that man is converted through his own cooperation with grace by means of his own decision, and hence naturally teaches also that man must harbor doubt as to his election until his death, since he cannot know how he will conduct himself in the future}”.\footnote{Lutheraner, 1880, p. 27, 28.}\end{displayquote}  Applying this to his opponents, Walther says: \begin{fancyquotes} Since they {\scriptsize\textsc{(the opponents)}} teach that God in connection with election has followed the ‘\textit{rule}’, to elect those of whom He foresaw that they would conduct themselves rightly and remain faithful unto death, and since they naturally cannot know by their own reason and strength whether they will conduct themselves rightly in the future also and remain faithful unto death, they find themselves, as “\textit{Altes und Neues}”\footnote{Vol I. p. 10, so clearly says, ‘\textit{from day to day between fear and hope, as on trial between two millstones}’!} \footnote{Berichtigung, p. 120 f.}\end{fancyquotes}  Walther writes even more fully on the same point: \begin{fancyquotes}The doctrine of uncertainty of election and salvation is something entirely unheard of in the Lutheran Church.  But our opponents, with their doctrine of election, cannot do otherwise than deny all certainty of these on the part of the elect, just as the papists do.  The Lutheran Confession says that election is a cause of salvation and of all that pertains thereto; our opponents say that the attainment of salvation is on the contrary a sort of cause of election.  The Lutheran Confession teaches that faith which perseveres to the end depends upon election; our opponents say that election depends upon faith which perseveres to the end.  The Lutheran Confession most earnestly rejects the teaching that there is also in man a cause of election; our opponents say that God’s rule in connection with election is the conduct of man; that election took place in consequence of faith, the nonresistance of man, his permitting himself to be converted. \par According to the doctrine of our opponents, therefore, it is entirely impossible that a man should without a special divine revelation be sure of his salvation and election; for since according to this teaching his salvation rests in his own hands, in his perseverance, and he must admit that he could easily stumble, fall, and forever fall away, he has nothing which could make him sure of his salvation, and must therefore necessarily be in doubt.\footnote{Abendvorlesungen [Evening lecture], June 10, 1881}\end{fancyquotes}

                “\textit{Our opponents do indeed declare}”, says Walther in the same connection, \begin{displayquote}“\textit{that a believing Christian can and should indeed have conditional certainty.  But a conditional certainty is simply no certainty.  Or let them say themselves whether that is certainty when a general is sure of victory on the condition that he will defeat his enemies?….. It is evident that to hold something of this sort to be certainty is simply ridiculous}”\footnote{Abendvorlesungen [Evening lecture], June 10, 1881.}\end{displayquote}

                Now Walther himself teaches that a Christian can and should be sure of his election and salvation in faith.  Since election depends only upon God’s grace in Christ, therefore the Christians can and should in faith recognize his election from the Gospel, which reveals and assures to him the grace of God in Christ.  The teaching of the uncertainty of election and salvation is to Walther in itself a distinctive mark of false doctrine.  He calls this teaching, in the words already quoted above: “\textit{something entirely unheard of in the Lutheran Church}”.  He writes: \begin{fancyquotes}By this denial of the certainty of salvation the teaching of our adversaries is already condemned, even if there were nothing else against it.  For that a Christian should and may be sure of his election and salvation Holy Scripture teaches in innumerable places.  As often as the believers are called blessed in Holy Scripture, so often does Scripture preach this certainty and summon the believers not to doubt their coming salvation, however sad at heart they may be; hence Paul says: \begin{displayquote}‘\textit{we are saved by hope}’,\end{displayquote} whereby he testifies that the blessedness of the believers in this life does not consist in their already enjoying, feeling, and perceiving it, but in their hoping for it, that is in awaiting it with assurance.  For Christian hope is nothing else than assured faith directed toward that which is to come.\end{fancyquotes}  In a more thorough discussion of this point Walther says: \begin{fancyquotes}Prof. S. indeed says that the assurance of his election, which he in the second thesis quite correctly calls an ‘\textit{assurance of faith}’ has no foundation in Holy Scripture; but thereby he contradicts a good many Scripture passages which are as clear as daylight.  Christ says to the seventy disciples: \begin{displayquote}{\footnotesize ‘In this rejoice not, that the spirits are subject unto you; but rather rejoice, because your names are written in heaven’.}\footnote{Luke 10:20}\end{displayquote} To the apostles the Lord says: \begin{displayquote}{\footnotesize ‘Ye have not chosen Me..., but I have chosen you’}.\footnote{John 15:16}\end{displayquote} -- and soon after: \begin{displayquote}{\footnotesize ‘Because ye are not of the world, but I have chosen you out of the world, therefore the world hateth you’.}\footnote{John 15:19}\end{displayquote}  Hence, following Christ in this, also the holy Apostles comfort the believers in their congregations with the fact that they are elect.  After St. Paul among other things has treated the doctrine of election, he continues: \begin{displayquote}{\footnotesize ‘Who shall lay anything to the charge of God’s elect? – Who shall separate us from the love of Christ? Shall tribulation, or distress, or persecution, or famine, or nakedness, or peril, or sword?  As it is written, For Thy sake we are killed all the day long; we are accounted as sheep for the slaughter.  Nay, in all these things we are more than conquerors through Him that loved us.  For I am persuaded, that neither death, nor life, nor angels, nor principalities, nor powers, nor things present, nor things to come, nor height, nor depth, nor any other creature, shall be able to separate us from the love of God, which is in Christ Jesus our Lord’.}\footnote{Romans 8:33, 35-39}\end{displayquote}  So, moreover, the same Apostle assures the believers at Ephesus: \begin{displayquote}{\footnotesize ‘According as He hath chosen us in Him before the foundation of the world’.}\footnote{Ephesians 1:4}\end{displayquote}  Further to the believing Thessalonians:\begin{displayquote}{\footnotesize  ‘Knowing, brethren beloved, your election of God’.}\footnote{1 Thessalonians 1:4}\end{displayquote}  He further declares to them: \begin{displayquote}{\footnotesize ‘But we are bound to give thanks always to God for you, brethren, beloved of the Lord, because God hath from the beginning chosen you to salvation’.}\footnote{2 Thessalonians 2:13}\end{displayquote}  To the believing Colossians he writes: \begin{displayquote}{\footnotesize ‘Put on therefore, as the elect of God, holy and beloved, bowels of mercies’, etc.}\footnote{Colossians 3:12}\end{displayquote}  Moreover Peter in the first chapter of his First Epistle greets the believers to whom he writes with the words: \begin{displayquote}{\footnotesize ‘Peter, an apostle of Jesus Christ, to the elect strangers’.}\footnote{1 Peter 1:1,2}\end{displayquote} -- and testifies to them in the second chapter: ‘\textit{Ye are a chosen generation}’.  Who now dares to assert that these are all mere empty assurances, in which the believers could and should not take comfort in faith?  And we are here passing by all the passages in which salvation is promised to the believers and they are assured of its certainty; when, e.g., to cite only this one, the Lord says: \begin{displayquote}{\footnotesize ‘My sheep hear My voice, and I know them, and they follow Me; and I give unto them eternal life; and they shall never perish, neither shall any man pluck them out of My hand.'}\footnote{John 10: 27-28}\end{displayquote}\end{fancyquotes}
\divider \begin{fancyquotes}
                And the teaching that ‘\textit{no believer can be sure of his salvation}’ is as contrary to the Confession as it is contrary to the Bible.  For thus we read there: \begin{displayquote}{\footnotesize ‘This also belongs to the further explanation and salutary use of the doctrine concerning God’s foreknowledge to salvation: Since only the elect, whose names are written in the book of life, are saved, how can we know, whence and whereby we can perceive who are the elect that can and should receive this doctrine for comfort’.}\footnote{Art. XI, par. 25, Mueller, p. 709, Triglotta p. 1071.}\end{displayquote} -- Hence also against this error of Dr. S., that no believer should or can be sure of his election, and thus also of his salvation, we must herewith publicly and solemnly protest.  Otherwise we could no longer say with the whole Christian Church of all ages: ‘\textit{I believe the everlasting life}’, and could no longer teach our dear Christians, and even our children, to confess with the entire orthodox Evangelical Church: \begin{displayquote}{\footnotesize ‘In which Christian Church He daily and richly forgives all sins to me and all believers, and will at the Last Day raise up me and all the dead, and give unto me and all believers in Christ eternal life.  This is most certainly true’.  Yes, to the fifth of our Christian Questions: ‘Do you also hope to be saved?’}\end{displayquote} instead of answering with our Church: ‘\textit{Yes, such is my hope}’, we should then have to answer: ‘\textit{No, such is not my hope!}’\footnote{that is to say, not with the assurance of faith. Cf. ‘Altes und Neues.’, I, p. 235, Antithesis 2.}\end{fancyquotes}

                In what way can and should a Christian come to the certainty of his eternal election?  How Walther answers this question has already been suggested in the preceding discussion by the fact that Walther wishes the certainty which a Christian has concerning his election to be called a “\textit{certainty of faith}”.  Faith has to do with God’s revealed Word, with the Gospel of Christ.  The important thing is to look in faith to the Gospel of Christ.  He who believes the Gospel of the grace of God in Christ thereby recognizes also his election which has taken place by grace for Christ’s sake.  Whatever assures a Christian of the grace of God in general assures him also of his eternal gracious election.  In the further development of this thought Walther directs attention to the Eleventh Article of the Formula of Concord\footnote{Mueller, pp. 709-715; Triglotta pp. 1071-1079} and continues: Here it is taught \begin{displayquote} “\textit{That a believing Christian can and should be assured of his election neither from reason, nor from the Law, nor by any sort of appearance, much less through searching out the secret hidden abyss of divine foreknowledge, but above all from his call through the Word which announces the universal grace, then from his baptism, from the Lord’s Supper, from private absolution, and from the testimony of the Holy Spirit}”.\footnote{Berichtigung, p. 121}\end{displayquote} He who wishes to be sure of his election must in all earnestness walk the way upon which God will save the elect.  “\textit{That}” says Walther, \begin{displayquote} “\textit{is not the right counsel which tells you that you must just firmly convince yourself that you are elect... No! we must also walk the way of salvation.  The doctrine of election is no pillow for the flesh}”.\end{displayquote}  Walther again directs attention to the Formula of Concord.  He writes: \begin{fancyquotes}The Formula of Concord governs itself strictly according to {\scriptsize\textsc{Romans 8:28-39}}, where the way is described upon which God leads His elect to glory, from which the Formula concludes that he who sees that he is in this way should not doubt that he is elect, and hence may confidently join in the jubilation of the Apostle: \begin{displayquote}{\footnotesize ‘I am persuaded that neither death, nor life, etc., shall be able to separate us from the love of God, which is in Christ Jesus our Lord’.}\footnote{l.c., p. 121, 122}\end{displayquote}\end{fancyquotes}

                The exhortations addressed to the believers in Holy Scripture to work out their salvation with fear and trembling have been introduced against the assurance of salvation.  But these exhortations do not contradict the assurance so clearly taught in Scripture.  They are not directed against firm faith in the Gospel promises, but against carnal security; they have not the purpose “\textit{of making us uncertain, but of preserving us in our certainty}”.\footnote{Lutheraner, 1880 p. 27}

                The further objection has been raised that just this doctrine of the certainty of election misleads people into carnal security.  With reference to this objection Walther says: \begin{fancyquotes}Our adversaries say indeed that this doctrine only makes people secure.  But if that were so, then no man could be sure that he is standing in God’s grace.  Christ calls to the seventy disciples: \begin{displayquote}{\footnotesize ‘In this rejoice not, that the spirits are subject unto you; but rather rejoice, because your names are written in heaven’.}\end{displayquote}  Yes, after Christ had forewarned Peter of his deep fall, he added for his comfort: \begin{displayquote}{\footnotesize ‘But I have prayed for thee, that thy faith fail not’.}\end{displayquote}  Did Christ thereby make His disciples secure? No, it was just the assured faith in their election which roused them to contend most faithfully, even to the bloody death of martyrdom.  And as Christ dealt with His disciples in this respect, so the disciples afterwards dealt with the believers converted through them.  We find, for instance, that Paul expressly assures the believers at Ephesus, at Thessalonica, at Colosse, and Peter the believers living in the Diaspora, that they are elect; yea, Peter directly calls the totality of all believers ‘\textit{a chosen generation}’.  Paul even puts on the lips of all believers the triumphant song \begin{displayquote}{\footnotesize ‘Who shall lay anything to the charge of God’s elect?  It is God that justifieth.  I am persuaded, that neither death, nor life, etc., shall be able to separate us from the love of God, which is Christ Jesus our Lord’.}\end{displayquote}  Did the Apostles thereby make the believers secure?  No, thereby they rather placed the helmet of salvation upon their head and the shield of faith in their hands, to quench all the fiery darts {\scriptsize\textsc{(doubts)}} of the evil one.  So then it is beyond all doubt that he who teaches concerning the election of grace in such a way that the believers cannot become certain of it is teaching a false doctrine, for it is unbiblical.\footnote{Abendvorlesungen [Evening lecture], Oct. 28, 1881.}\end{fancyquotes}   Entering even further into the discussion of the state of heart of him who is assured in faith of his election, Walther writes: \begin{fancyquotes}He who knows upon what way alone God leads His elect unto salvation, namely by the way of repentance and conversion, of faith and sanctification, of cross and perseverance, – he has in this knowledge, we think, enough warning and admonition; for as soon as he willfully forsakes that way his assurance of salvation is lost, according to our pure doctrine.\end{fancyquotes}  Only the “\textit{godless}” think and say that they need not trouble themselves about their salvation if it does not rest in their hand but alone in God’s hand.  In the case of truly believing Christians the situation is altogether different; “\textit{the more sure they are in faith that they are elect and so will be saved, the more zealous they are in all good}”.  \begin{displayquote}“\textit{The situation is just the same with the doctrine of election as it is with the doctrine of justification.  When a godless man hears that we are justified before God and saved without the works of the law by faith alone then he immediately thinks: that is a shameful doctrine, destructive of all morality, for he who believes that will think: if God according to this doctrine does not take any account of good works, why then do I need to do good works?  Even in the time of the Apostles there were people who actually drew such conclusions from the doctrine of justification...  But St. Paul pronounces upon those who draw such conclusions the dreadful sentence: ‘Whose damnation is just’.{\scriptsize\textsc{(Romans 3:8)}}”}\end{displayquote}

                Those who claim that the believers cannot and should not be sure of their eternal election have also sought to support this assertion of theirs with the authority of Luther.  But in proving their point from Luther they have led their readers astray by quoting such passages from Luther in which the reformer warns against searching out the secret will of God, while the same Luther tells the believers to be fully assured in faith of their eternal election in so far as this is revealed in the Gospel of Christ.  Walther writes with regard to this point: \begin{fancyquotes}When pure and godly theologians, -- when, e.g., our Luther at times seems to speak against men’s assurance of their eternal salvation, this is above all directed against those who sought to become sure of their salvation by searching out the secret counsel of God, in order then to be rid of all further care for their salvation and of all earnest seeking after salvation; for in Luther’s time there were fanatics who believed that men could and should seek to become sure of their salvation through a special divine revelation, and that then they could and should be unconcerned about their salvation, because then they could not fall away.  Against such abominable fanatics Luther indeed had to cry out: \textit{Away with your assurance!  It is produced in you by the devil!}  The more uncertain of your salvation you become, the sooner you may yet be saved. \par And since Luther had himself for a long time been plunged, as it were, into hell, because he had wanted to search out God’s secret counsel concerning himself, and still been unable to search it out, he held it to be his duty to warn those who had come into severe temptation concerning predestination against sinking in this abyss and against scaling this height.  But thereby Luther not only did not retract his doctrine of eternal, sure, and unalterable predestination, but he also still less desired to propose the terrible teaching that a Christian must doubt his salvation and until death be suspended in uncertainty, as between heaven and hell.  Rather did Luther in one of his last writings, namely in his Exposition of Genesis, in the exegesis of the 26th chapter, so gloriously show how a man can become entirely sure of his salvation in the right way, that the heart of a godly Lutheran Christian leaps for joy when he reads it\footnote{Lutheraner 1880, p. 22.}\end{fancyquotes}  Walther offers in evidence several passages from \textbf{Luther}, -- among other these: \begin{fancyquotes}Why should you want to listen to the Gospel, say the Epicureans, since after all everything depends on predestination?  In this way Satan forcibly takes away the predestination, of which we were firmly assured by the Son of God and through the holy Sacraments, and makes us uncertain, whereas we were before entirely sure.  And when he assails the poor terrified consciences with this temptation, we sink in death, just as it almost happened to me, if Staupitz had not rescued me, when I suffered this very same temptation... Dr. Staupitz used to comfort me with these words, speaking to me thus: \begin{displayquote}{\footnotesize My dear man, why do you plague yourself so with these speculations and high thoughts?  Gaze upon the wounds of Christ and the blood shed for you; there predestination will shine forth.  Therefore one should hear the Son of God, who was sent into the flesh, become man, and was manifested for this purpose, that He might destroy the works of the devil {\scriptsize\textsc{(1 John 3:8)}} and make you sure of predestination.}\end{displayquote}\end{fancyquotes}

                We close these discussions concerning Dr. Walther’s doctrine of election with a few words of Walther which he penned in the midst of the controversy concerning this doctrine.  In his tract “\textit{Die Lehre von der Gnadenwahl in Frage und Antwort}”\footnote{c.f. p. 11} we read: \begin{fancyquotes}God has given to our Lutheran Church in America through the election controversy which has broken out the great task of contending for one of the most mysterious doctrines of His Word, to judge of which not rationalists, not idle, curious, ambitious spirits, not frivolous false Christians, but only true, enlightened Christians concerned for their salvation, humble, and trembling at God’s Word, are fit and competent.  This election controversy deals with the great and highly important questions: \begin{displayquote}{\footnotesize ‘To whom do those who come to faith, remain in faith, and are saved, owe this grace?  Do they owe this to themselves?  Or do they owe it at least in part to themselves?  Or do they owe this alone to the grace of God and the most holy merit of Christ?  Does the glory for our salvation belong to God alone?  Or is there also a cause thereof in man?  Does man by nature possess powers to cooperate to some extent in the work of his salvation, to decide for salvation, or at least to assent thereto, though feebly?  Or is every man by nature spiritually dead, and must God therefore do all by His grace?’}\end{displayquote} – Yes, the present controversy is concerned with these great truths, with the doctrine of salvation by grace alone, for Christ’s sake, alone, through God-given faith alone, not with theological hairsplitting but with the most important points of practical Christianity.  May God therefore have mercy upon our American Lutheran Zion, and help that no upright soul may go astray in this battle for the truth, but that all true children of God within our Church may finally gather under the good banner of our Confessions also in regard to this doctrine, and so become a light for many in the midnight hour of this last time of sore distress.  May God grant it for the sake of Jesus Christ , the universal Savior of all sinners and the eternal King of truth. \textbf{Amen}.\end{fancyquotes}
%%% Local Variables:
%%% mode: latex
%%% TeX-master: "../main"
%%% End:

\blankpage
\end{document}

%%% Local Variables:
%%% mode: latex
%%% TeX-master: t
%%% End:
