\pagenumbering{arabic}
\chapter{Theology}
\vspace{.15cm}
\hrule
\vspace{.30cm}
In the following we do not intend to write a biography of the sainted Dr. Walther or even a part of his biography.  A biography of Walther for our Christian people will begin to be published in the current volume of the “\textit{Lutheraner}”.  And it is hoped that at a later date, when the literary remains, especially the extensive correspondence of the departed, shall have been assembled and made available, a comprehensive book may be written which will describe the life and work of this teacher of the Lutheran Church in America for the use and benefit of the entire Lutheran Church.  In the mean time the following dissertations may find their place in our monthly theological journal in which the main traits of Walther as a theologian will be delineated.
\vspace{.10cm}
\hrule
\vspace{1.25cm}
We cannot describe Dr. Walther as a theologian without first showing in a general way what he understood by theology.  In this matter he took decided issue with recent theology.\footnote{Of the antitheses in “Lehre und Wehre”, 21, 162ff}  Recent theology defines theology as the “\textit{ecclesiastical science of Christianity}” or as the “\textit{scientific knowledge of faith}” or even as the “\textit{scientific self-consciousness of the church}”.  Recent theology says of the definition of the old Lutheran theologians, who conceived of theology in its proper sense and primary sense as a personal habitus of the theologian, namely as the sufficiency to lead the sinner to salvation by means of the Word of God, that it was indeed well meant but “\textit{scientifically}” untenable.

Recent theology distinguishes between theology and the Church’s proclamation of salvation.  The latter is supposed to present the Christian doctrines in so far as they are to be received by the Christian congregation through faith; theology on the other hand is said to have the function of “\textit{scientifically mediating}” the congregation’s faith to the thinking intellect.  For this reason also recent theology abandons its “\textit{direct relation to salvation}”.  The old Lutheran definition which consistently held to this relation is said to rest upon a confusion of “\textit{theology}” with “\textit{the Church’s proclamation of salvation}”.

Over against this Walther held with the old Lutheran theologians that theology is a \textit{habitus practicus}.  In “\textit{Lehre und Wehre}”\footnote{Lehre und Wehre, Vol.14, p.4ff}, he published a lengthy article entitled: “\textit{What is Theology?  A contribution to the Prolegomena of  Dogmatics}”, in which he begins with the following thesis: \begin{displayquote}“\textit{Theology is the practical habitude, wrought by the Holy Ghost and drawn from the Word of God by means of prayer, study, and trial, vitally to know and to impart the truth revealed in the written Word of God unto salvation, to establish it therefrom, to expound, apply, and defend it, in order to lead sinful man through faith in Christ unto eternal salvation}.”\end{displayquote}

Of this definition Walther then proves that it is both Scriptural and also that given by most Lutheran teachers.  On the objective and subjective concepts of theology, or of theology conceived as teaching and as habitus of the theologian, Walther prefaces the following:

\begin{fancyquotes}Christian theology can be regarded in several ways, either subjectively, as something inhering in the soul of a man, or objectively, as teaching in which this is presented orally or in writing.  In the first case it is regarded absolutely, as it is in itself, apart from what may be done with it; in the other case it is regarded relatively, as it is in a certain respect, in accordance with a certain accidental characteristic, with respect to a use which may be made of it. \par In the first case Christian theology is taken in its primary and proper, in the second case in its secondary and improper significance.  Since theology must first be in the soul of a man before it can be taught by him, or presented either orally or in writing, and since everything connected with theology must be judged in accordance with what it is in itself and in its essence, therefore in the thesis, according to the example of most dogmaticians in our church, the definition of theology regarded subjectively of concretely, i.e. as it inheres in a concretum or in a person, is given precedence.\footnote{ Lehre und Wehre, 14, 8 f.}\end{fancyquotes}

Theology, subjectively regarded, is to Walther “\textit{not the sum total of certain intellectual acquisitions}”, but a habitude, a sufficiency or skill to perform certain functions.  “\textit{The Holy Scripture}”, says he\footnote{l.c., p.10},\begin{fancyquotes}although the word theology does not occur in it, itself specifies this as the category to which theology belongs.  For since theology, subjectively considered, is what should be in those who are to administer the office of teachers in the church, we may therefore seek and recognize in the Biblical description of a teacher also a description of a true theologian.\end{fancyquotes}

With regard to {\scriptsize\textsc{2 Corinthians 3:5}} he remarks:\marginpar{{\scriptsize Walther also refers to Hebrews 5:12-14 \& 2 Timothy 3:17.}}  \begin{fancyquotes}In this passage the Apostle, after he has exclaimed in 2:16\footnote{2 Corinthians 2:16} with regard to his teaching office:  ‘\textit{Who is sufficient for these things?}’  writes as follows: \begin{displayquote} ‘\textit{Not that we are sufficient of ourselves to think anything as of ourselves; but our sufficiency is of God}.’\end{displayquote}  So that which in {\scriptsize\textsc{Hebrews 5:14}} is called a skill  is here called sufficiency.  Now sufficiency implies not only a certain competence and skill by the observance of certain rules to produce a certain effect, but also at the same time a disposition of the soul, thus a habitude.\end{fancyquotes}
 Walther lays special emphasis on the fact that theology is altogether practical, that it is not concerned with satisfying the thirst for knowledge but with leading sinners to salvation.  Theology is for him not a “\textit{theoretical habitude}”, “\textit{which has knowledge itself for its goal and therewith rests content}” but a “\textit{practical habitude}”.


“\textit{It is the latter}”, he writes\footnote{l.c., p.72} for the reason that its purpose is a purely practical one.  St. Paul indicates wherein the purpose of theology consist when he writes, {\scriptsize\textsc{Titus 1:1-2}}: \begin{displayquote}‘\textit{Paul, a servant of God, and an apostle of Jesus Christ, according to the faith of God’s elect and the acknowledging of the truth which is after godliness a hope of eternal life.}’ \end{displayquote} Herewith the apostle obviously indicates the purpose of his office, namely that he has received it in view of the faith of the elect and the acknowledging of the truth unto godliness, and all of this in hope of eternal life.  But the purpose of the office is also the purpose of the office is also the purpose of theology.  This purpose therefore is the true faith, the knowledge of the truth unto godliness, and finally eternal life.\footnote{See Romans 1:5 in connection with 1 Timothy 4:16.}


No one will attempt to assail the Scripturalness of this statement.  Scripture refers all offices and gifts, which God gives in the Church to practice; through them the body of Christ shall be edified unto spiritual and eternal life.\footnote{Ephesians. 4:11 ff}  If then modern theology finds that this description does not fit it, that merely demonstrates that Scripture knows nothing of this theology, that it has no right to existence, at least not in the Church of God. Walther further proves that theology is altogether practical from the fact that true theology is completely bound to Holy Scripture, has no more and no less to present than what is written in the Scripture.  But Holy Scripture has according to its own testimony no other purpose than to bring man to salvation through faith in Christ.\footnote{2 Timothy 3:15-16; John 5:39; John 20:30-31}  So also theology has no other purpose.  Walther writes:  \begin{displayquote}`\textit{That the … purpose of theology is to lead sinful man through faith in Jesus Christ unto eternal salvation is…indisputable.  For since theology has no other subject than the truth revealed in God’s word unto salvation in Christ, so also it can have no other purpose than this purpose of the Word of God}.'\end{displayquote}  Only He can deny this purpose of theology who permits his theology to be drawn, instead of from the pure clear fountain of Israel, from the muddy waters of human speculation.  Walther is determined to hold fast that whatever is not revealed in God’s Word and is not directed to the furtherance of man’s salvation does not belong to theology at all.  He writes: \begin{fancyquotes} Not only does the discussion of philosophical questions on the basis of the light of nature and the principles of reason from no part of theological study, but even such researches as concern themselves with matters contained in Holy Scripture are only to that extent pertain to the subjects of theological study in the proper sense, as they serve the purpose of deed hardly an art or science which could not and should not in some way subserve theology, but wherever a truth contained in God’s Word, and indeed in so far as it is revealed unto salvation, is not concerned, there theological study in the proper sense has not yet begun.\end{fancyquotes}  Walther says with \textbf{Meisner}\footnote{l.c., p.76}:  \begin{displayquote}“\textit{He who does not always regard this purpose, and does not in all his theory keep it in sight, does not deserve the name of a true theologian}.”\end{displayquote}


Also that which is apparently theoretical in theology is nevertheless, when more accurately considered, thoroughly practical.  Walther appropriates from \textbf{Calov}\marginpar{{\scriptsize \textit{Abraham Calovius}\\(16 Apr 1612 – 25 Feb 1686) was a Lutheran theologian, and was one of the champions of Lutheran orthodoxy in the 17th century. }}\footnote{Lehre und Wehre 14, 374} the following:  “\textit{Toward this goal}” -- namely toward the furtherance of the enjoyment of God and of eternal salvation -- \begin{fancyquotes} everything which is taught in theology is directed.  Although, indeed, some parts thereof may seem to be theoretical, yet it is not presented as theory and thus as an object of mere intellectual investigation {\scriptsize\textsc{(contemplationis)}} in theology, but rather for the sake of practice.  When, for example the nature of God, or of an angel or of man, becomes an object of cognition, this is not done to the end that we may rest in such knowledge; this knowledge is rather directed toward practice, that we should enjoy God, become like unto the angels, and attain to the blessedness appointed for man. \par All which is not directed toward this end or does not serve it, either directly or indirectly, either immediately or mediately, that\footnote{says Walther with Gerhard (l.c., p. 376)} does not belong to theological knowledge.\end{fancyquotes}


And in this end and purpose of theology, to lead sinners through faith in Christ unto salvation, Walther saw the most precious thing about the vocation of a theologian.  On this subject he often spoke to the students with fervent eloquence, the he might endear to them that service in the Church, which is so despised by the world, as the most important and blessed service in which a man can be engaged.


Walther was also accustomed to speak of the fact that theology contains a most powerful admonition for every theologian, just for this reason, that, in theology, everything is directed toward the salvation of men.  Without doubt a great contributory cause of the retrogression of theology in our time is that men have either left the purpose of theology entirely out of view or relegated it far to the background, that men no longer want to consider theology as a \textit{habitus practicus}.  If modern theologians, who after all want to be teachers of the church, would but hold fast to the truth that all their teaching and writing must have only the one purpose, namely, of leading sinners through faith in Christ unto salvation, they would spare to inflict upon the church their theological speculations which can neither produce nor support faith in Christ.

 

%%% Local Variables:
%%% mode: latex
%%% TeX-master: "../main"
%%% End:
