\chapter{Justification -- General}

\hrule
\vspace{.30cm}
We have until now pointed out what Walther taught concerning the Church and the subject immediately connected with this doctrine.  This doctrine it was indeed for which the Saxon immigrants had in the first place to contend.  But when we endeavor to characterize Walther as a theologian we must above all else discuss his position on the doctrine of justification.
\vspace{.30cm}
\hrule
\vspace{1.25cm}
                Walther’s position on this doctrine gives us the key to his entire conduct in a life filled with controversies.  Walther places the doctrine of justification, or the doctrine that a man is justified before God and saved by grace through faith in Christ, in the center of all Christian doctrines.  All other doctrines serve as antecedents to this doctrine or flow from it as consequences.  And because Walther always saw that also this doctrine was placed in jeopardy by various individuals, therefore he sought so determinedly and uncompromisingly against all errors.  \marginpar{{\scriptsize \textit{Synergism} is the position of those who hold that salvation involves some form of cooperation between divine grace and human freedom.}}This doctrine was for him the pivotal point as in the contention for the right doctrine of the Church.\footnote{Die lutherische Lehre von der Rechtfertigung.  Ein Referat, p. 93. (Report to Western District, 1859)} Walther demonstrated, how, in the teaching of a visible church outside of which there should be no salvation, and in the claim that the validity of absolution should be dependent upon the ordination of the one who pronounces it, the doctrine of justification was overthrown.  He offered the same demonstration with reference to the other false doctrines against which he contended, with reference to \marginpar{{\scriptsize \textit{Chiliasm}, is a belief advanced by some Christian denominations that a Golden Age or Paradise will occur on Earth in which Christ will reign for 1000 years prior to the final judgment and future eternal state of the New Heavens and New Earth.}}\textbf{chiliasm}, \textbf{a physical effect of the Sacraments}, \textbf{synergism}, etc.  \par “\textit{The contention against false doctrine}”, he says, “\textit{gains practical significance for the individual Christian only when he realizes how through the falsification of other articles also this doctrine cannot remain pure}”.\footnote{Report of the first convention of the Synodical Conference, p. 23.} \par In this doctrine Walther lived, both as a Christian and as a theologian.  Even his opponents have confessed that he understood how to speak powerfully of this doctrine.  Concerning this doctrine Walther held the most lectures in his so-called Luther-hours.  It was before all else on the subject of how this doctrine should be rightly preached that he gave effective guidance in the theological seminary, both by showing the right way and also by a lively characterization of the most common aberrations.  We believe we are not asserting too much when we say that after Luther and \textbf{Chemnitz}\marginpar{{\scriptsize \textit{Martin Chemnitz}\\ (9 Nov 1522 –- 8 Apr 1586) was an eminent second-generation German, Evangelical Lutheran, Christian theologian, and a Protestant reformer, churchman, and confessor.}} probably no teacher of our Church has given more vital witness of the doctrine of justification than Walther.  Walther had Luther as his teacher especially also in this doctrine, and gathered the luminous rays which the later teachers shed upon this doctrine into one beam of brilliant light.

                As we prepare to expound Walther’s position on the doctrine of justification we wish, first to direct attention to the general characterization which Walther gives of the doctrine of justification with respect to its importance, etc.  In the second place we intend to bring out the points which Walther emphasizes in connection with this doctrine in order to keep it intact over against the errors of the time.
\divider
                The doctrine of justification is for Walther that whereby the Christian religion distinguishes itself from all other so-called religions; it is the \textbf{distinctive characteristic} of the Christian religion.  When we are speaking of justification, he says, we are speaking of the Christian religion, for the doctrine of the Christian religion is none other than the revelation of God as to how a person is justified before God and saved through the redemption which has been accomplished by Christ Jesus.  All other religions show other ways which are supposed to lead to heaven {\scriptsize\textsc{(namely, the way of works)}}, while the Christian religion alone shows a different way to heaven through its doctrine concerning justification, and this is something unheard of and undreamed of for the whole world, thoughts which were hidden in the heart of God before the foundation of the world.  And in another place: \begin{displayquote}{\footnotesize This doctrine is the heavenly sun of the Christian religion, whereby it distinguishes itself from all other religions as the light from the darkness.}\footnote{Evangelienpostille, p. 278.}\end{displayquote}  Hence he who attacks our doctrine of justification attacks our entire doctrine, the entire Bible, the entire Christian religion.  Where this doctrine is falsified, there another way of salvation, and thus another religion, is taught.  To contend for the doctrine of justification, and for the Bible and the Christian religion, is one and the same thing.  Without the doctrine of justification the entire Christian doctrine is like a watch which lacks the mainspring.  All other doctrines lose their significance when the doctrine of justification is not right.  When the cornerstone falls the entire building caves in.  So also the whole structure of Christianity collapses where the doctrine of justification falls away; the Church is then transformed into a mere correctional institution.  And so far as the understanding of Scripture is concerned, theologians who do not stand right on the doctrine of justification, in spite of all their occupation with Scripture and all their citations of Scripture, take their position not in Scripture, but before its fast closed door.  \begin{displayquote}{\footnotesize For without the doctrine of justification the Bible becomes for a person a mere book of morals with all sorts of strange supplementary teachings.} \end{displayquote}

                Therefore the doctrine of justification is “\textit{the foremost chief article of the Christian faith}”.  \begin{displayquote} “\textit{As long as anyone has gotten no farther than to think that the doctrine of justification is also an important article he has not yet seen the light}”.\end{displayquote} All praise of Christ, of grace, and of the means of grace, without the right doctrine of justification, is nothing.  All teaching in the Church must serve this article.  Not as though one should or could urge only this article.  All revealed doctrines must be taught with the greatest care.  But even when one is treating of hell the goal must be to show the hearers the deliverance from hell.

                The knowledge of the doctrine of justification is unconditionally necessary for the salvation of the individual.  Christians are people who are in possession of the knowledge of the article of justification, i.e., people who believe that God forgives their sins by grace for Christ’s sake.  This knowledge, this faith, makes a man a Christian.  “\textit{Upon this article}”, declares Walther, \begin{displayquote}``\textit{rests all salvation, and therefore it is unconditionally necessary for every Christian.  It would be of no profit if one should have an exact knowledge of all other articles, e.g., those concerning the Holy Trinity, the Person of Christ, etc., if he did not know and believe this article}”.\footnote{Synodical Conference Report, p. 21.}\end{displayquote}

                This article is rightly called the article with which the Church stands and falls.  \begin{fancyquotes}For what is the Church?  \divider It is the totality of believing Christians.  Therefore the Church is where Christ rules and reigns in grace; but He rules inwardly in a man in such a way that He offers and conveys grace to him.  Now where He has conquered a heart there is His Kingdom.  Hence where there are regenerate living Christians there is His Church.  But no man becomes a true regenerate Christian without this doctrine of justification.  Every other doctrine can indeed make great Pharisees, but no Christians.  One becomes a Christian only in this way that through the Holy Ghost it is revealed in his heart that he is truly redeemed by Christ, has forgiveness of sins, a reconciled heavenly Father, righteousness which avails before God, and may therefore lay himself down with confidence even upon his deathbed.\footnote{L.c. p. 24, 25.}\end{fancyquotes}  And in another place: \begin{displayquote}“\textit{When Luther says that without the article of justification the Church could not endure for an hour, that is no exaggeration.  For the Church is not an external institution, but the assembly of the believers.  Hence  where there are no believers there is also no Church}”.\end{displayquote}

                If therefore, the Church is to be built and preserved it is necessary before all else that the doctrine of justification be preached.  Through the preaching of this doctrine the Reformation of the Church was brought about, while all means which had previously been tried for the renovation of the Church had failed.  Also in other lands and at other times it has been this doctrine which has renewed the Church.\footnote{Synodical Conf. Report, p. 25-27}  And if we in our time wish to build the Church it must take placed through the preaching of the doctrine of justification.  Not “\textit{eloquent}” and “\textit{popular}” preachers, nor yet “\textit{reverend clergymen}”; but pastors who preach the doctrine of justification build up the congregations.\footnote{L.c., p. 27f.}  The knowledge and preaching of this doctrine outweighs many a shortcoming in external education and endowment.  If the Church had only the choice between externally inadequately educated preachers, who, however, live in the article of justification and preach it, and externally highly cultivated preachers, who, however, do not understand the article of justification and therefore also do not preach it, it would necessarily choose the former without any hesitation.  \par “\textit{As supremely important as this doctrine is}”, says Walther, \begin{fancyquotes}yet it can be preached in its fullness and power, in its clarity and rich comfort, also by less gifted men... even the weakest , if he has only grasped the doctrine that the grace of God in Christ Jesus has appeared to all men and is apprehended by faith, can preach to the people in such a way that they become assured of their salvation; and that outweighs all wisdom and gifts and treasures of the world.\par  Such preachers will never be lacking in preaching material.  They will always know how to preach of what God by grace has done for us, and that will give them ever new joy.  What is learning, all of it, necessary as it is in its place, compared with the wisdom of God, which is proclaimed when even only the one passage: “\textit{God so loved the world}”, is proclaimed?  Over this the poor sinners rejoice, over this all the holy angels wonder, at this should the whole world fall on its knees and shout Glory and Hallelujah.  \par If our ministerial candidates preach this, what a reformation they can begin also in this land; as indeed a small beginning has also already been made in this direction.  For this makes really live congregations, not such as make a great noise about their life and their accomplishments, but such as, living in this doctrine, offer willingly to God in the beauty of holiness.  In fine: let us learn from Luther that we can initiate no reformation here if we do not firmly believe this doctrine, with divine assurance proclaim, maintain, and hold it fast.\end{fancyquotes}

                Hence a living knowledge of the doctrine of justification belongs to the right preparation for the office of the ministry.  Walther says: \begin{displayquote}“\textit{The most necessary thing which students of theology can take with them from the theological seminary, without which everything else would be worthless, is a clear and thorough insight, grounded upon experience, into the exalted doctrine of the justification of a poor sinner before God}”.\end{displayquote}  And to the right administration of the office belongs before all else the public and private proclamation of the doctrine of justification.  Because he is permitted to proclaim this doctrine a preacher should gladly want to be a preacher.  And as the preacher’s joy in the performance of his office, so also all his hope of effectiveness should come from this doctrine.  This will preserve the pastor from a legalistic tendency.\footnote{Referat, p. 95 f.}  In the doctrine of justification one has also the means of continuing in the true doctrine.  “\textit{As long as this doctrine is entirely pure}”, says Walther, \begin{displayquote}“\textit{no error in other points can find lodgment with us.  It is just as Luther says: ‘This doctrine tolerated no error’.  It is the sun in the heaven of the Church and where it arises all shadows must flee}”.\end{displayquote}  We have in the doctrine of justification a “\textit{standard which makes it impossible for us if we are governed by it to take up an error}”.  \begin{displayquote}“\textit{He who has come to the knowledge of the doctrine of justification laughs at all learned unbelieving and self-believing professors with all their eloquence and learning when they teach falsely; when what they determine and say does not agree with his childhood text: ‘The blood of Jesus Christ, God’s Son, cleanseth us from all sin’, even the simplest believer treads it under foot, however great an appearance of wisdom or holiness it may have}”.\footnote{Synod. Conf. Report p. 27.}\end{displayquote}  He, on the other hand, who is not straight in the doctrine of justification cannot realize and show how dangerous an error is.  He who does not know what is the chief thing in Christianity is like a child who does not know the purpose of a clock and hence regards this little wheel or that shaft as unnecessary.  Without the right knowledge of the doctrine of justification the individual doctrines of the Word of God are an incoherent heap of stones from which one may carry some away without inflicting any essential damage upon the whole.  Without the right knowledge of this doctrine one will always be in doubt where the right Church is, particularly when one looks upon the humble appearance and small numbers of the true church and also the offenses which occur in it.  But if one holds fast to the doctrine of justification he will not allow the multitude, the age, the splendor, the strict order, and the great works of the false churches to impose upon him.  Also not the learning of the apologetic efforts of the modernists.  \par For all of this without the doctrine of justification one can have no profit or effect in the Church.
