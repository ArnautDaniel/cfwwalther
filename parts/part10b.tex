\chapter{Justification -- Universal}

\hrule
\vspace{.30cm}
We now indicate the points upon which, according to Walther, everything depends if we are to keep the doctrine of justification pure, also in our times.  Walther says: “\textit{In connection with the pure doctrine of justification, as our Lutheran Church has again expounded and upheld it on the basis of God’s Word, there are chiefly three points at issue:}

\begin{itemize}
\item  \textit{the doctrine of the universal complete redemption of the world through Christ;}
\item  \textit{the doctrine of the power and effectiveness of the means of grace; and}
\item  \textit{the doctrine of faith}”\footnote{First Report of the Synodical Conference, p. 20}
\end{itemize}
\vspace{.30cm}
\hrule
\vspace{1.25cm}
If the people are agreed in these points then they are truly agreed in the doctrine of justification and in general in the entire Christian doctrine.  If there is a deficiency in one or more of these points, as there is in the Protestant sects and among the modern rationalistic-synergistic Lutherans, then the doctrine of justification is defective, even though there may still be outward agreement in phraseology with the orthodox church, i.e., even though one still says that man is justified before God alone by grace, through faith, for Christ’s sake, and not through the works of the Law.\footnote{Die lutherische Lehre von der Rechtfertigung, p. 35. Western District, 1875, pp. 32-40.}

                We give here first of all a summary of Walther’s expositions concerning these points.  If anyone denies the universality of the atonement, if he denies with Calvin that Christ has redeemed all and that God in the Gospel earnestly offers grace to all without distinction, then he certainly overthrows the doctrine of justification.  Furthermore, if anyone teaches indeed that Christ has redeemed all men, but has not fully redeemed them, i.e., if he teaches that Christ has indeed made the forgiveness of sins possible, but that the forgiveness of sins of justification is not actually already at hand for every sinner, then faith and conversion is made a meritorious cause of the forgiveness of sins and the doctrine of justification by grace for Christ’s sake is overthrown.\par  If anyone teaches falsely concerning the means of grace, i.e., if he does not teach that God offers grace to the sinner in the Word and Sacrament and that the sinner is to seek and find grace in Word and Sacrament, then he directs the sinner to seek grace in his subjective condition in conversion and renewal, i.e., in human works. \par  If anyone teaches falsely concerning faith, if he does not teach that faith is reliance upon the grace offered in the Word, but rather identifies faith with feeling, then again the condition of the human heart is made the basis of righteousness and salvation instead of grace of God.  If anyone teaches falsely concerning faith in this manner, that he ascribes to human cooperation or the good conduct of man, then again, even with retention of the phraseology “\textit{by faith alone}”, the “\textit{by grace for Christ’s sake}” and therewith the pure doctrine of justification is abandoned.

                This subject, however, seems so important to us that we wish to expound each of the three points somewhat more fully in accordance with the utterances of Walther which are here so abundantly available.

                \section{Redemption through Christ}
                \hrule
\vspace{.30cm}
To the correct doctrine of justification belongs then, in the first place, the correct Biblical doctrine of the complete redemption of all men through Christ.
\vspace{.01cm}
\hrule
\vspace{1.25cm}
                In order to place the complete redemption through Christ in the right light Walther is concerned with impressing the fact that even before faith grace, righteousness, and salvation is at hand for every man, that even before faith God is in Christ fully reconciled to all sinners, that even before faith every sinner is righteous before God with respect to the attainment and the divine intention\footnote{First Report of the Synodical Conference, p. 68}, or in accordance with the judgment which God by raising Christ from the dead has already pronounced upon all men.\footnote{L.c., p. 31.}  \begin{displayquote}“\textit{A justification has not only been made possible, but is has been obtained and has taken place}”.\footnote{L.c., p. 61.}\end{displayquote}  Walther is above all concerned to reject the idea that man through his faith and through his conversion renders God fully favorable to him or completes his redemption and righteousness.  The man who is to be saved must indeed be converted, but this conversion is not that for the sake of which God saves him, but the way upon which a man comes to faith, who himself does nothing but receive the complete and already bestowed redemption.\footnote{L.c. p. 34.}  \par The fanatics usually think of the matter as though Christ has brought to pass that which the Scripture calls atonement, so that God can now receive a man into heaven merely for the sake of his conversion.  They do not believe that through Christ all without exception has taken place which had to take place in order that God may save us and give us eternal life.  Something, they suppose, must yet remain for man to do, and this something is conversion.\par  But Scripture teaches that Christ has done all and has already obtained reconciliation with God, righteousness, etc., that it is already there and is distributed in the holy Christian Church through the Gospel,  Now no one has anything further to do than to take salvation.  That is what we wish to say when we speak of a complete redemption.  Not that man already has something and God supplies the rest; also not that God has done something and man must add that which is lacking; but that God has already done everything entirely alone.\footnote{L.c., p.34.}

                This doctrine – as Walther urges again and again – is the characteristic doctrine of Christianity, that whereby the Christian doctrine is distinguished from heathenism.  He who denies this doctrine denies all of Christianity. \par “\textit{That man could procure grace or the forgiveness of sins for himself}”, says Walther, “\textit{is what the heathen believed; but that the forgiveness of sins, gained by Another, is already at hand, is a truth of which the heathen knew nothing}”.  And in another place: \begin{fancyquotes}While all religions, except the Christian, have showed man how he must himself do that thereby he is rescued and saved, the Christian religion, on the other hand, teaches not only how men may yet be eternally saved, but how they have already been saved.  According to the teaching of the Christian religion man is already redeemed, is already freed from sin and all ill, and God is already reconciled to him.  The Christian religion says to man: \begin{displayquote}{\footnotesize You need not redeem yourself and reconcile God.  Christ has already done all for you.  Nothing is left for you but to believe this, that is, to receive it.  It is just this which distinguishes the Christian religion from all other religions.}\end{displayquote}  The Jew says: \begin{displayquote}{\footnotesize If you want to be righteous you must keep the Law of Moses;}\end{displayquote} --the Turk says:\begin{displayquote}{\footnotesize If you want to be saved you must conduct yourself in accordance with the Koran;}\end{displayquote} --the Papists say: \begin{displayquote}{\footnotesize  If you want to get to heaven you must do good works, be sorry for your sins and make satisfaction for them yourself, and if you want to be perfect enter the cloister; and all the sects which pervert Christianity without exception lay something upon man which he must do in order to become righteous before God and be saved.}\end{displayquote}  The Lutheran Church, on the other hand, says to man in accordance with God’s Word: \begin{displayquote}{\footnotesize It is all done already: you are already redeemed, you have already been made righteous before God, you have already been saved; hence you have nothing to do in order to redeem yourself, and you do not have to reconcile God and earn your salvation.  You shall only believe that Christ, the Son of God, has already done all this for you, and through this faith you become partaker of it and are saved.}\footnote{L.c., p.34.}\end{displayquote}\end{fancyquotes}

                That grace, righteousness, salvation reconciliation, etc., are already at hand before faith, as Walther further explains, is already demanded by the very concept “\textit{faith}”, and he who denies the former must also deny that we are righteous and saved through faith.  If I am to be saved, says Walther, by believing that I am redeemed, reconciled to God, and my sins forgiven, then all this must already be on hand in advance.  As surely as God’s Word promises us that we shall be righteous, reconciled to God, and saved through faith, so surely must all these things be present before my faith, and waiting only for me to receive them.  That a man should be justified by faith alone is possible only because that which is necessary to salvation is already at hand and accomplished, so that on my part only acceptance is necessary. \par But this acceptance is just what Scripture calls believing.  Since God takes into heaven all who believe, righteousness and atonement must already be present and have taken place.---\par All who will not admit that reconciliation and righteousness are already complete before faith do not regard faith as a mere hand which accepts that which has been gained by Christ, but as a work through which man cooperates toward his redemption and righteousness, as a condition which man fulfills and for the sake of which God receives man into heaven.

                Only when the complete redemption is thus held fast will the concept of the Gospel also be held fast.  \begin{displayquote}{\footnotesize Why is Christ’s doctrine called Gospel or good news?}\end{displayquote}  Simply because when I preach the Gospel I preach nothing else than what has already been obtained and bestowed upon men and what they should therefore receive and in which they should heartily rejoice.  The Gospel is the joyful tidings that Christ has done the work which we should have done and yet could not do, and that the heavenly Father by raising our Redeemer from the dead has given a sign from heaven that He is fully satisfied.\footnote{L.c., p.39.} In the Gospel the peace which God has made with men is proclaimed.\footnote{Western district 1868, p. 31}  It must be stressed earnestness that God’s wrath is turned away from all men by the work of Christ and that through the Gospel everyone is invited to partake of grace.  If a preacher had to come before his hearers with the thought: the wrath of God is still resting upon them and they must be induced to appease Him---that would be terrible; but because he knows that the atonement has already been rendered for all and God’s wrath against all has been quenched, therefore he can say confidently:\begin{displayquote} Be ye reconciled to God, do but receive His hand of grace!\footnote{1st Report of Synodical Conference, p. 36.}\end{displayquote}  He who will not preach the Gospel thus might as well preach the Koran or the Talmud or the papal decretals or what he will;  but if he wishes to {\scriptsize\textsc{(preach the Gospel and)}} make happy Christians, then let him preach this good news.\footnote{L.c., p. 39.} And again: \begin{displayquote}“\textit{Since all men are reconciled to God, and the Gospel is the tidings of this reconciliation, therefore it is such an unutterable grace to live under the sound of the Gospel}”.\end{displayquote}  The fanatics indeed have such thoughts concerning Christ’s work that they regard Him as having only made it possible for man to attain grace by his own efforts.  It is likewise the papistical teaching that men through contrition, penance, and other good works can secure for himself the salvation which Christ has made possible.  But thereby the Gospel, the preaching of which Christ has committed His Church, is denied.

                To the Scriptural presentation of the complete redemption as a premise for the correct doctrine of justification belongs, according to Walther, also the doctrine that in Christ’s death and resurrection a justification of the entire world of sinners is already implied.  “\textit{As by the vicarious death of Christ}”, says Walther, \begin{displayquote}“\textit{the guilt of the entire world was cancelled and its punishment suffered, so also by the resurrection of Christ righteousness, life, and salvation is restored for the entire world and in Christ, as the Substitute of all mankind, has come upon all men}”.\end{displayquote}  \begin{displayquote}“\textit{Christ’s glorious resurrection from the dead is the actual absolution of the entire world of sinners}”,\end{displayquote} and \begin{displayquote}“\textit{The resurrection of Christ the plenary justification of all men}”\end{displayquote} --such are the themes of Easter sermons delivered by Walther.\footnote{Brosamen, p. 138; Epistelpostille, p. 211} \divider Many, even among preachers, do not rightly know what to do with the resurrection of Christ.  They read that Christ raised Himself and then again that the Father raised Him, and they do not know how to harmonize this.  They suppose at one time that Christ arose in order to prove His deity and at another that He was raised in order that the possibility and certainty of our resurrection might be established.  True as both these assertions are, yet neither one is the chief matter.  Christ would not have died and risen again only to prove His deity; and the possibility of our resurrection had indeed already been proven by the resurrection of others before Christ; the chief matter remains that God through Christ declared: Christ has now paid for the sins of the whole world, it is therefore free from its guilt; now the entire world can raise the shout of victory, for its freedom from sin and its righteousness is won.  Furthermore: when God raised His son from the dead He did not forgive Him His own sin but that of all mankind which He had taken upon Him; He did not justify Christ from His own guilt but from our guilt which He had allowed to be imputed unto Him.  Thus the whole world has been justified through the resurrection of Christ.\footnote{Western District, 1875, p. 33.}\par  With this the fact that man is justified by faith in no way stand in contradiction, for when we speak of faith the personal appropriation on the part of man and the imputation of the righteousness which has been won on the part of God is emphasized.  But this would not be possible if the world had not been first justified through Christ’s death and resurrection, if the condemnation in death had not been followed by the acquittal in the resurrection.\footnote{1st Report of the Synodical Conference, p. 41f}  And this justification applies not only to men in general but to all individual men.  \begin{displayquote}“\textit{If it be asked whether one could say that man collectively has indeed been absolved but not the individuals, our answer is: God through Christ is reconciled to each individual}.''\footnote{L.c., p. 32} \end{displayquote}

                This doctrine of a universal justification of all men before faith is not a theological construction, but a Biblical doctrine.  Biblical not only in its content- which in itself would be fully sufficient, -- but even in its phraseology.  “\textit{It is this doctrine,}” says Walther, \begin{fancyquotes}which is expressly declared in the passage, {\scriptsize \textsc{Romans 5:18}}\begin{displayquote} ‘\textit{As the offense of one judgment came upon all men to condemnation; even so by the righteousness of One the free gift came upon all men unto justification of life}’,\end{displayquote} and it is therefore not merely a Biblical doctrine but also a Biblical expression that justification of life has come upon all men.  Only a Calvinistic exegesis could explain this passage to the effect that only the elect are justified.\end{fancyquotes}  Although Scripture is most places speaks of that justification which takes place in the moment when a man comes to faith, and accordingly in ecclesiastical usage the justification by faith is simply called the justification of a poor sinner\footnote{L.c., p. 68.}, nevertheless the doctrine of the universal justification of all men before faith, which is clearly attested by Scripture in several places, is of the very greatest importance.  Let no one think in this matter a mere strife of words is involved.  Rather is the most highly important matter here to be maintained against attacks and error.  Especially in this land of sects and fanatics we must earnestly urge the doctrine of universal justification, for they indeed also teach that man is justified by faith, but they speak of faith in such a manner that one soon notices that they make faith itself the effective cause of justification, whereby they rob the Lord Christ of His honor.\foootnote{L.c., p. 46.}\par  Without the universal justification before faith there is no justification by faith.  ``\textit{We could not then,}'' continues Walther,  \begin{displayquote} ``\textit{speak of the justification of the sinner by faith, for to believe means to receive what is there.  If the world were not already justified, then believing would mean accomplishing a work unto justification.  The entire preaching of the Gospel is a message of God concerning a righteousness which has already been gained by Him and is there for all}.''\footnote{Cf. On this subject especially Brosamen, pp. 142, 143.}\end{displayquote}

                Those who say that God has made the whole world righteous, but has not declared it righteous thereby really deny the whole of justification.  Yea, if God had not {\scriptsize\textsc{(already)}} written and sealed the document of pardon, we preachers would be liars and deceivers of the people when we tell them: \begin{displayquote}{\footnotesize Only believe and your are righteous;}\end{displayquote}  but now that God through raising His Son has subscribed the document of pardon for the sinners and provided it with His divine seal, we can confidently preach:\begin{displayquote}{\footnotesize the world is justified, the world is reconciled to God, which latter expression we could not use if the former were not true.}\end{displayquote} –When the Lutheran Confession repeatedly says that justification is grasped by faith these passages express the truth that a justification must first be at hand which faith can receive, and that faith must not first effect it, but that it grasps it as already at hand.  If anyone would say: \begin{displayquote}{\footnotesize the forgiveness of sins is indeed already there, but not justification,}\end{displayquote} --he must indeed be ignorant of our Confessions, which expressly teach that justification and forgiveness of sins are the same.  \begin{displayquote}“\textit{We believe, teach, and confess that according to the usage of Holy Scripture the word justify means in this article, to absolve, that is, to declare free from sins}”.\footnote{Formula of Concord, Article 3; Mueller, p. 528; Triglotta p. 793. L.c., p.46.}\end{displayquote}

                Particularly in connection with Walther’s discussions on absolution, that is, the “\textit{preaching of the Gospel to one or more particular persons who desire the comfort of the Gospel}”, the manner in which the complete redemption of all men through Christ lived in Walther’s heart came to expression.  Absolution, says Walther, is based upon the perfect redemption or universal justification.  “\textit{When the pastor absolves he distributes a treasure which is already at hand, namely the forgiveness of sins which has already been gained}”.\footnote{L.c., p. 43.}  Walther holds only that man to be a true Lutheran preacher who holds that he by speaking the absolution has absolved all the penitents and only that man to be a true Lutheran Christian who believes that through the absolution of the pastor he has truly been absolved by God.  He adds: \begin{displayquote}“\textit{Only he indeed can believe thus who believes that the world is redeemed; for if I believe that, then the absolution is only the communication to the penitents of the fact that they were redeemed 1800 years ago, and the plea: Only believe that and you are saved}”.\end{displayquote}  That so many take offense at the absolution which is customary in the Lutheran Church comes from the fact that they do not believe in the complete redemption of all men through Christ and hence suppose that we ascribe to the preachers as “\textit{ordained persons}” a special authority and mysterious power.  \begin{displayquote}“\textit{But we say:\begin{displayquote} It is no art to absolve someone; that any ordinary Christian man, any woman, any child can do, if it can only tell that the Lord Jesus died for all, and that whoever believes in Him receives the forgiveness of sins.\end{displayquote}  For the absolution depends not upon the quality of the speaker but upon the word of the Gospel concerning the accomplished redemption}”.\end{displayquote}

                In this connection Walther insists again and again that one must not make the essence of the Gospel dependent upon faith, but is to regard it as an offer of God’s grace which is valid of itself.  \begin{displayquote}“\textit{The glorious benefits of Christ have been given us; mark well! They have already been given us (in the Gospel) and indeed they are always at hand for us, even if we do not believe}”.\footnote{Western District, 1874, p. 47.}\end{displayquote}  If one makes the Gospel essentially dependent upon a man’s believing, or, which is the same, if one talks as though faith must first be there before the Gospel is in itself valid and effective or before the benefit of the forgiveness of sins is at hand for the sinner, he thereby both denies Christ’s all-sufficient merit, the redemption and reconciliation of the world, and then also faith is thereby made something quite different from what it properly is; it is then no more a grasping and receiving of the present forgiveness, but a work which must be furnished in order that there may be forgiveness in the Gospel; finally, faith has then simply nothing on which it can take hold.  “\textit{If the Gospel is not valid unless a man first believe it, what then shall he believer?}”  Faith thus comes to be founded on itself instead of on the Gospel.  “\textit{That amounts to increasing the distress of people who are in anxiety and doubt concerning their salvation}”.\footnote{L.c., pp. 57-64.} Walther reminds us again and again that, with a doctrine or practice according to which faith is first demanded in order that forgiveness of sins may be there, no tempted person can be comforted.  \begin{displayquote}``\textit{The tempted supposes that he cannot believe.  Such a person must despair with this doctrine, whereas one should seek to convince him that the Savior is already there for him, has already forgiven him and will receive him}”.\footnote{Western District, 1875, p. 38.}\end{displayquote}

                Walther here examines an objection.  The objection asks how this argument concerning complete forgiveness, universal justification, the Gospel as an absolution of the whole world of sinners, harmonizes with those Scripture passages which speak of God’s wrath upon the world lying in wickedness, in particular upon the unbelievers.  Walther answers by means of the distinction between Law and Gospel.  In so far as God views the world in Christ “\textit{pure love, pure favor, pure grace}” toward the world is in His heart.  In so far as He contemplates the world outside of Christ as lying in wickedness, and particularly as rejecting the Gospel, it lies under His wrath.  Although there is indeed no real contradiction here, since grace and wrath are predicated of God’s relation to the world in different respects, yet “\textit{an unutterable and unfathomable mystery}” is to be acknowledged here.  Since Scripture teaches both facts we let them stand side by side.  \begin{fancyquotes}It is the Lutheran way that when we find in God’s Word two things which we are not able to harmonize we let both stand and believe both as they read.\footnote{Synodical Conference, 1st Report, pp. 31f, 36 f.}\end{fancyquotes}