\chapter{Justification -- Means of Grace}
\hrule
\vspace{.30cm}
To the right of doctrine of justification belongs, according to Walther, in the first place, the doctrine of the full redemption of all men through Christ, that is, the doctrine that through Christ the forgiveness of sins is already at hand for all men, yea, that in Christ’s death and resurrection all men have already actually been absolved of their sins or have been justified.  This we endeavored to present in the last article.
\vspace{.30cm}
\hrule
\vspace{1.25cm}
                But there belongs further to the right doctrine of justification also the correct doctrine of the \textbf{means of grace}, that is, it must also be taught, if the doctrine of justification is to remain pure, that God offers and conveys to men the forgiveness of sins which Christ has gained and which is at hand for them in no other way than through the Word of the Gospel and the Sacraments.  The correct doctrine of justification stands and falls with the correct doctrine of the means of grace.

                Hence Walther inculcates, in the first place, the truth that the Word of the Gospel, as it reaches men in the preaching of the Gospel and in the Sacraments, possesses a double power, not only an effective {\scriptsize\textsc{(vis effectiva)}}, \textit{operativa}, according to which it works faith and all which must take place in a man, but also a collective power {\scriptsize\textsc{(vis collativa)}} according to which it also truly offers and gives what the words say and express.  \begin{displayquote}“\textit{Word and Sacrament are the hand of God, whereby that which Christ has merited and brought for us out of His grave is conveyed to us.  When therefore we speak of the power and effectiveness of the means of grace, we mean that Word and Sacrament are not only a notice and announcement, nor only a power that produces faith, but an offer, conveyance, and seal of the very blessings which they report and announce}.''\footnote{First Report of the Synodical Conference, pp. 48-57.}\end{displayquote}

                Hence the situation is that forgiveness of sins or justification is now at hand for men in the means of grace and must there be grasped by faith.  The man smitten by the Law is not to be directed through prayer and struggling to draw down the forgiveness of sins from heaven, as it were, but simply to be incited to faith in the grace which God has brought down from heaven for all sinners and now offers them in the Word and in the Sacraments.  Here again is a parting of the ways between the Lutheran Church and the sects.  Walther was accustomed to expound this in connection with the treatment of the jailer at Philippi {\scriptsize\textsc{(Acts 16:30- 31)}}.  \begin{fancyquotes}The fanatics say to a man whose heart the Law has touched: \begin{displayquote}{\footnotesize You are indeed terrified over your sins, and God’s grace must help you, but do not grasp it too quickly.  Go first into your closet, pray and strive with God until you have won your way through to the feeling of grace; then you may believe that you have grace.}\end{displayquote}  That is {\scriptsize\textsc{(however)}} a godless way to deal with souls.  In that way one can bring souls to despair but not to a true certainty of their salvation.  Therefore one must say to the sinner: \begin{displayquote}{\footnotesize Do you confess that you are a sinner, and are you terrified in your heart that you are lying under the wrath of God?}\end{displayquote}  If it is so with you, then believe on the Lord Jesus Christ, and thou shalt be saved.  That is what the Apostle said to the jailer.  And remember: he says that to a man who had just been about to kill himself with his own hand, but who was now in distress over his sins and asked: ‘\textit{What must I do to be saved}’?  What would a Methodist have answered?  He would no doubt have said: \begin{displayquote}{\footnotesize That cannot be managed so quickly.  Put forth an effort, pray and struggle but it cannot be very long until grace breaks through for you, and then you will notice that God has accepted you.}\end{displayquote}  But Paul was no Methodist; that we may see from his conduct toward that criminal, the jailer.  And why could the apostle speak in this way to the jailer?  Because he knew that the Word {\scriptsize\textsc{(of the Gospel)}} was a means of grace whereby he was at the same time offering life and salvation.\footnote{L.c. p. 52.}\end{fancyquotes}  Walther constantly reminds us indeed that one is not to speak on such a way as though he were simply waging a campaign against the feeling of grace and against prayer for grace.  With reference to the latter he says: \begin{displayquote}“\textit{It would indeed be terrible to say anything against prayer; for we know that God has commanded prayer and promised to hear it; but it would be just as terrible to suppose that prayer is a means of grace.  I can and should indeed call upon God in prayer for grace, but such invocation cannot convey, give, and bring grace}”.\end{displayquote}  The forgiveness of sins takes place through the Word of the Gospel.  “\textit{We pray for the forgiveness of sins, but not so much to obtain it immediately as rather to strengthen our faith in prayer}”.  With regard to the feeling of grace Walther says: \begin{fancyquotes}Far be it from us to deny that the Spirit of grace makes Himself known in the heart of the sinner when man does not willfully close his heart against His operations.  But it would be a terrible confusion if one would regard this feeling which is stirred up in the fanatics through their own praying and struggling as being grace itself.  At best – for very often this feeling is produced by other causes entirely and is not the work of the Holy Ghost – it is only a gracious operation of the Holy Spirit which the fanatics call grace.  \par The grace by which we are justified and saved is something outside of us, not in us...  Therefore when a poor sinner comes to a Lutheran pastor and says: \begin{displayquote}{\footnotesize Where shall I find grace?  I have now come to the knowledge that I am a poor lost and condemned sinner,}\end{displayquote}  --then the Lutheran pastor answers: \begin{displayquote}{\footnotesize Take comfort in the grace of God.  But this grace is in the Gospel and in the holy Sacraments.  Believe what God has said to your there, and take comfort in the grace which is thereby bestowed upon you.  Take comfort in your Baptism and that in it grace has been bestowed upon you.  Make use of absolution and go to the holy Supper, for it is there that God offers, conveys, bestows, and seals to you grace and the forgiveness of all your sins.}\footnote{L.c. p. 49}\end{displayquote}\end{fancyquotes}  What effect does the fanatical denial of the means of grace have upon the doctrine of justification?  The sects, says Walther, hold it to be a great advantage which they have over the Lutheran Church, that they direct people into their own heart instead of to the means od grace.  “\textit{But among all the errors which form the wall of partition between the sects and the Lutheran Church the greatest and most harmful is their false doctrine of the power of the Word}”, their denial of the collative power of the means of grace.\footnote{Lecture on April 20, 1877.}  He who denies the collative power of the Word of God and Sacraments and who directs the sinner seeking grace to prayer, the feeling of grace, the new heart, etc., instead of to the means of grace, thereby falsifies the doctrine of justification in all its parts, he denies that a sinner is justified and saved by grace, for Christ’s sake, through faith. \par First of all, the concept of justifying and saving grace is falsified.  For he who bases forgiveness upon the feelings, upon so-called experiences, instead of upon the Word, calls these experiences, or peculiar motions and feelings in his soul and mind, the grace of God, whereas Holy Scripture {\scriptsize\textsc{(when it speaks of the cause of justification and salvation)}} understands under grace that which {\scriptsize\textsc{(for Christ’s sake)}} is in the heart of God: God’s favor, mercy, and love, which is expressed in the Word, and shall now be believed, whereas that which they call grace is called a gift.\footnote{Die Luth. Lehre von der Rechtfertigung.  Ein Referat, u.s.w., pp. 85, 86.}

                In the second place, also the “\textit{for Christ’s sake}” is falsified in the practice of the fanatics.  For whereas they direct the sinner seeking grace to prayer, that he may thereby attain to grace through prayer, instead of to the means of grace, that is nothing else than a denial of the fact that God for the sake of Christ’s work of atonement is already gracious to all sinners and assures them of this in the Gospel.  When one teaches to seek Christ in one’s self instead of in the Word, and will not comfort the sinner until he feels grace and has become a new man, this is indeed to “\textit{make for one’s self a false Christ and to reject the Christ who hung upon the cross and offers Himself to us in the Gospel}”.\footnote{L.c. pp. 86, 87 f.}  What Christ crucified has already won for us, man still wants to attain for himself.  Man’s achievement is put in the place of the Word of Christ.  \begin{displayquote}“\textit{To direct anyone to the feeling instead of to the Word is therefore not only a perversion, but is also an entirely different religion than the Biblical religion}”.\end{displayquote}

                In the third place, the concept of faith is also falsified.  Faith means to trust God’s promise in the Gospel.  But what does man make of it?  \begin{fancyquotes}They are accustomed only to ask: \begin{displayquote}{\footnotesize Have you Christ in your heart? Do you feel how He is working in your heart?}\end{displayquote}  If the answer is yes, then first may there be comfort and hope, then they will believe, as for instance, a Methodist will certainly comfort no one until he says that he feels a Christ in his heart.  But what they thus regard as faith is not faith, but a mere delusion, or, at best, a fruit of faith.\end{fancyquotes} Yes, Walther says of that Christianity which will not believe God on His mere Word, but will first believe when one feels grace within himself and supposes that he can be sure of it by himself: that “\textit{is indeed nothing else than to suffer shipwreck concerning the faith, and yet is praised as the highest humility and piety}”.\footnote{L.c., pp. 84, 87.}
\divider
                If the doctrine of justification is thus falsified in all its parts by the denial of the means of grace, the harmful consequence of such falsification will not be lacking, namely the uncertainty of grace or justification.  “\textit{That is the greatest grace}”, says Walther, \begin{fancyquotes}that God has attached forgiveness specifically to the Word, since otherwise no one could assuredly know whether God were really speaking to him, -- as indeed the sects cannot assuredly know whence that comes which they experience on the anxious bench.\footnote{L.c., p. 84.} To be sure, the sects are distinguished from the papists by the fact that they desire to be sure of their state of grace.\footnote{As is well known, the papists declare it to be a culpable presumption when the ordinary Christian wishes to be sure of his state of grace.} Yet in this connection they also return again completely to the papistical principle concerning justification, since they base their assurance not upon the eternally reliable Word but upon their own unstable feeling, for which reason also they are either hypocrites or else must often complain that they have lost Christ. \par Hence also their efforts by all sorts of means to arouse their feelings, and the fact that the one or the other often boasts today of his conversion, and yet must so soon again return disconsolate to the anxious bench.\footnote{L.c. pp. 78f.}\end{fancyquotes}  No, God in His mercy has taken quite another manner of care for sinners.  God did not say: Grace is indeed gained, but now see to it yourself that you obtain it, but God has taken care of sinners in this manner, that “\textit{also the greatest sinner, who is already upon the steps to the gallows, can be sure that also he is righteous before God}”.  God has placed His grace into the Word and Sacraments, from which faith can and should take them at all times.\footnote{Western District Report, 1875, p. 21.}

                And so Walther develops from every angle the thesis: if the doctrine of justification is to remain pure and the sinner is to enjoy its comfort, there must be no budging from the truth that God offers and conveys to men the grace, which through Christ is at hand for all men, in no other way than through the means of grace ordained by Him.  But Walther also reminds us that this correct position is to be maintained not only against the fanatics, but also against ourselves.  \begin{displayquote}“\textit{In the eyes of the fanatics every one who still bases the forgiveness of sins on the Word is regarded as unconverted, and they call only him converted who boasts of so-called experience and builds his hope upon them}”.\footnote{Report, p. 85.}\end{displayquote}  But we too are patients in the same hospital.  By nature “\textit{man always wants to place his redemption not in something outside but only in himself}”\footnote{L.c., p. 81.}  Many a temptation which assaults the believers has no other source than this, that they judge God’s mind toward them not according to God’s gracious promise in Word and Sacrament, but according to their own subjective condition.  So perversely does man constantly behave himself, even where the correct doctrine is heard.  \begin{fancyquotes}Our Church teaches indeed in her Confessions and through her faithful witnesses that God faithfully cares for us so that the grace won by Christ is also conveyed to us through Word and Sacrament... But, alas! How effectively that Reformed error has intruded itself with so many also in our Church!  Otherwise why do people here and there in our congregation oppose the Confession and Absolution after the sermon every Sunday, when they, if they really believed in the imparting of the forgiveness of sins in the absolution, should gladly run a hundred thousand miles after it?... \par {\scriptsize\textsc{(Furthermore)}} Many a one says indeed:\begin{displayquote} ‘\textit{If God Himself would tell me that my sins are forgiven me, as He said to the palsied man, then I would believe it, but how can it help me that the preacher says it, who does not know my condition, but may well consider me more penitent than I am}’.\end{displayquote}  Whence comes this attitude, but that man does not believe that Christ has merited all... And that now the forgiveness of sins is comprised in the Word which we preach?\footnote{L.c., pp. 83,84.}\divider  Many permit themselves to be held back for a long time from going to Confession and to the holy Supper because they do not consider themselves fit; finally they resolve to go, in order not to give offense, or because they fear that they will stand before God as despisers of the means of grace.  But then they hear the Absolution with doubt and uncertainty.  Why?  Because they base the absolution upon some quality in them, instead of supporting themselves in faith upon the objective validity of the absolution.\end{fancyquotes}  So, we ourselves are “\textit{oft times papists without the pope}”, since we want first to earn forgiveness with our repentance instead of freely grasping it by faith.

                Hence if the Christian desires to remain firm in the correct doctrine of justification he must, not only against the doctrine and practice of the fanatics but also against that of his own natural heart, ever and again accustom himself to base his faith upon the objective means of grace.  Failing this, he will found his state of grace upon his subjective condition, that is, upon his own worthiness and his own works.  Hence Walther also reminds all Lutheran pastors never to lose sight of this point.  The pastor must, on the one hand, guard against self-deceit, that no one with an unbroken heart may regard mere head-knowledge as saving faith.  But, on the other hand, he must direct those who are really terrified by God’s wrath not to their heart but directly to the promise of the Gospel.  Walther says: \begin{fancyquotes}Christianity is faith in the message of the Gospel which Christ has committed unto His Church.  A preacher must therefore lead his hearers to say, here it is written: \begin{displayquote}‘\textit{He that believeth and is baptized’ --  ‘This is a faithful saying’, -- ‘Christ is the propitiation’ --  ‘Him that cometh unto me’ -- ‘But to him that worketh not, but believeth’, etc.}\end{displayquote} It is bad enough that all Christians are subject to this disease, that they want first to feel and then believe, but it is terrible when that is actually preached. \divider The characteristic feature of our dear Evangelical Lutheran Church is its objectivity, that is, that all its doctrines have the tendency to lead man away from seeking his salvation in himself, namely, in his ability, will, work, and condition, and to bring man to the point where he seeks salvation outside of himself, while the characteristic feature of all other churches is subjectivity, since they all aim a leading man to build his salvation upon himself.\end{fancyquotes}  But this takes place specifically also through denial of the Biblical doctrine of the means of grace.

                How strictly Walther held fast the truth that the forgiveness of sins or justification take place through the Word of the Gospel became clear in connection with a point which was discussed in the controversy on the doctrine of election in Chicago, 1880, and has recently been touched upon again on the part of the Ohio Synod.  A representative of the theory that election took place in view of faith asserted – in order to find in justification an analogy for his doctrine, – subjective justification is a separate forensic act of God after man has grasped by faith the forgiveness of sins announced in the Word.  Dr. Walther answered him, in part, as follows: \begin{displayquote}“\textit{When I believe on Christ I have righteousness and salvation.  It has already been imputed to me.  It is not true that when I have appropriated objective justification by faith a new act is added.  The act has taken place.  Through faith I already have righteousness.  God needs not still to impute justification to me afterwards}”.\end{displayquote}  When the one concerned responded: \begin{displayquote}“\textit{I must confess: then I have not known what subjective justification is.  I have always been of the opinion that subjective justification is a separate forensic act of God}”.\end{displayquote}  Walther thereupon replied that the grasping of justification through faith and its imputation on the part of God coincide. \begin{displayquote}``\textit{As soon as I believe I have what faith grasps.  Why?  Because God has attributed it to me through His Word... As soon as I believe God has forensically forgiven me my sins.  The Word is the hand of God which offers the gift, faith, my hand, receives what God’s hand bestows upon me}”\footnote{Proceedings, etc. pp. 45, 46.}\end{displayquote}

                What Walther desires to emphasize here is that the so-called \textbf{subjective justification}, or the judgment which God passes upon the believer in Christ in subjective justification, is not to be sought outside of the Word.  Some have recently taken occasion from this to charge that we deny subjective justification altogether.  But that is an unjust charge.  As earnestly as Walther teaches, on the one hand, that in Christ’s death and resurrection of the objective justification of all men is an accomplished fact, and what God does in subjective justification is only a repetition of that justification already actually declared, yet, on the other hand, he sharply distinguishes the subjective from the objective justification and describes subjective justification as a transaction which then first takes place when the sinner believes.  Walther says in exposition of the Twelfth Thesis on Justification in the First Report of the Synodical Conference\footnote{First Report of the Synodical Conference,  p. 68}: \begin{fancyquotes}The purpose of this thesis is to demonstrate that, although we teach that forgiveness of sins has been gained for all men and that so far as its attainment is concerned righteousness and salvation is already at hand for all men, and although we also teach, in the second place, that in Word and Sacrament this treasure is also offered and presented to all, we nevertheless do not deny that God holds the individual, when he accepts this treasure, in and through Christ, as one who has this righteousness, and that he is in that self-same hour, so to speak, written into the book of life, and that this is the justification which in ecclesiastical usage is simply called the justification of a poor sinner, because therein each individual stands in the judgment before God and is personally absolved by Him.  This \textit{actus forensis}, that is judicial transaction, goes through a man’s whole life, for God ever anew declares a man free from sin, death , and judgment.\end{fancyquotes}  But this forensic judgment of God, whereby God attributes righteousness to the believing sinner, is not a judgment which aside from the Word of promise is to be sought immediately in God, in addition to the Word of the Gospel, but it is the Word of the Gospel itself.  Justification, as Walther ever and again inculcates, is a transaction which takes place indeed in God’s heart, not in the heart of man, but in God’s heart in so far as it is revealed in the Word of the Gospel.  Hence the faith which grasps the Word of the Gospel thereby grasps God’s judgment or forensic imputation. \par In this sense Walther says: “\textit{As soon as I believe God has forensically forgiven me}”, and he rejects the idea that justification is a separate forensic act which follows upon the believing apprehension of the Word of promise.  One and the same Word of the Gospel offers forgiveness, works faith, and declares the believer righteous.

                To hold this fast is of the greatest importance and especially also of great practical concern for the spiritual life.  Walther, at Chicago, calls attention in passing to the fact that he who waits “\textit{for a new forensic transaction of God}” aside from the revealed Word of the Gospel {\scriptsize\textsc{(e.g. aside from the Word)}}: Whoever believes in Christ shall have forgiveness of sins, -- such an one makes his justification uncertain.\footnote{Proceedings of the General Pastoral Conference at Chicago, 1880, p. 47 \\\\--Dr. Walther: \begin{displayquote}“\textit{How can I know that? I must hold to this: God has said it. I do not wait for a new forensic act of God}”.\end{displayquote}} \par And so it is in fact.  The sinner, seeking righteousness before God is thereby, in the last analysis, remanded to the standpoint of the fanatics.  If it were so that a new forensic judgment should have to be added to the judgment revealed in the Word of the Gospel, and that in this way the judgment should lie outside of the Word, then no one could become certain of the judgment of justification from the Word.  The sinner would then be directed, when he asks whether God justifies him, to a deduction which he should make from his subjective condition of being a believer.  \par When he asks: does God justify me? He would have to look not to the heart of God, as it is revealed in the Gospel, but he would have to look into his own heart.  Thus justification would no longer take place by faith, for “\textit{faith}” can exist only over against a judgment which is expressed in the Word of the Gospel; a new act of God aside from the Word could not be an object of faith.  Faith always requires the Word as its correlate.  Thus it is evident that to the integrity of the doctrine of justification also this belongs, that the forensic judgment of justification be not sought in a new act aside from the Word of the Gospel.