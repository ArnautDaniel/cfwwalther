\chapter{Justification -- Faith}
\hrule
\vspace{.30cm}
If the doctrine of justification is to remain pure, then finally also that concerning faith must be rightly taught.  This point had to be touched upon also in the preceding treatment.  So we can here confine ourselves to emphasizing a few principal thoughts.
\vspace{.30cm}
\hrule
\vspace{1.25cm}
                Walther mentions that the ignorance concerning the nature of justifying faith or the manner in which it justifies is very widespread in external Christendom.  He says: “\textit{However much all Christian parties speak of faith, yet very few have a correct conception of faith and how it justifies}”.\footnote{Western District Report, 1875, p. 35.}  Yes, there reigns with regard to this point a “\textit{truly Babylonian confusion}”.\footnote{Report of the First Convention of the Synodical Conference, p. 29.} \begin{displayquote} “\textit{People talk so much about the fact that faith alone justifies and saves but when it comes to the real issue they desire to know nothing of it}”.\end{displayquote}  If we investigate more closely it comes to light that they ever and again mix works into the article of justification.  Even when they verbally exclude works from justification and take the \textit{sola fide} phrase into their mouths, actually all this is again retracted and the fundamental article of the Christian religion is entirely falsified by making faith itself into a work.  They still want to find some place for an activity of man whereby he distinguishes himself before others.  This activity is placed sometimes in his repentance, sometimes in his conversion, sometimes in his sanctification, and sometimes in faith itself.

                Hence Walther’s efforts are directed toward warding off the tendency in some to change faith itself whereby justification takes place into a work or to mix one’s own works, one’s own worthiness, one’s own doings, etc. into faith.  Walther brings out again and again: \begin{displayquote}“\textit{If God demands of us faith He does not thereby say: My son has indeed made satisfaction for you and redeemed the world, but now you must also do something; on the contrary the situation is this: just because we have nothing whatever yet to do toward our salvation, therefore faith is necessary}”.\\\\  “\textit{The reason why faith justifies, and not something else, is this, that on the part of man there is nothing left to do, but righteousness and salvation has already been completely won for all men by Christ and is offered as a gift in Word and Sacrament}”.\end{displayquote}  Or: faith justifies and saves because man can in no way be justified and saved by his own deeds, but freely, by grace.  Faith comes into consideration in justification as the antithesis to all works and every sort of merit.  \begin{displayquote}“\textit{If indeed righteousness were not by grace, then something else would have to be required for it’s attainment, but since it is by grace faith is enough, for it is only a receiving}”.\end{displayquote}  Yes, faith justifies inasmuch as it is the receiving of the righteousness and salvation which is already at hand through Christ’s merit and offered in the promise of the Gospel.  Faith comprises knowledge, assent, and confidence; but not “\textit{because it is knowledge, assent, and confidence {\scriptsize\textsc{(and hence a certain quality in man)}} does it justify, but inasmuch as it is the means whereby the righteousness which is at hand is received}.” \par Man shall not first in some way make himself worthy of righteousness and salvation through faith.  Faith does not come into consideration in justification in so far as it is itself a deed or an obedientia, nor in so far as it is effects an internal change in man and has blessed feelings, good works, etc., as it consequences.  In order quite sharply to express the thought that faith is not to be conceived of as a supplement to the grace of God and the merit of Christ, Walther says: \begin{displayquote}“\textit{If the word ‘faith’ never occurred in the Scripture it would still teach salvation by faith, through the very fact that it teaches salvation by grace for Christ’s sake}”.\end{displayquote}  And in another place Walther declares: \begin{displayquote}``\textit{If I had nothing else than faith, and not Christ {\scriptsize\textsc{(which is indeed not possible)}}, then I should be damned together with my faith, for it is not the act of faith which makes me acceptable to God, but it is Christ and His righteousness, which I grasp with the hand of faith.}''\footnote{Report of First Convention of the Synodical Conference, p. 35.}\end{displayquote} Walther liked to adduce in this connection the statement of Calov, that also faith itself, in so far as it is an instrument, is rightly set over against not only all works of obedience and godliness, but also over against faith itself, in so far as it is our work and act.\footnote{Baieri Compendium, ed., Walther, III, p. 270.}\footnote{That faith comes into consideration in justification “\textit{not as a work, but as an instrument}”{\scriptsize\textsc{(nicht als Werk, sondern als Werkzeug)}}, is a theme which Walther again and again expounded in his well-known “\textit{Lutherstunden}”.  The following may find place as an annotation here. \par Walther declares that faith does not justify in so far as it in a general way believes something or other, but in so far as it believes the Gospel that God through Christ is gracious to men.  Walther said on September 14, 1877: \begin{displayquote}“When the unbelievers hear that in the Christian religion the grace and favor of God and eternal salvation is ascribed to faith, they usually think that this is just the way of all religions which claim to be revealed by God in a supernatural manner, that they require of their adherents above all else faith in their mysteries which are contrary to reason, and for this promise heaven to those who believe them.  Above all else faith was demanded of his adherents by Mohammed, above all else faith was demanded by the founders of the Mormon sect, above all else faith was demanded by Moses, and so also by Christ. \par But, think the unbelievers, what can God care {\scriptsize\textsc{(if indeed there is a God)}} whether one believes something contrary to reason or not?  What better is he who treads his reason under foot, or why should he be more worthy of heaven {\scriptsize\textsc{(if indeed there is a heaven)}}, than he who uses his reason? \par --From all these judgments one can perceive that the unbelievers have no idea what faith really is to which in the Christian religion God’s grace and eternal salvation is ascribed.  The mere regarding what is written in the Bible as true is according to our holy Christian religion so far from being the faith which justifies and saves that the Bible itself says: \begin{displayquote}‘\textit{Thou believest that there is one God; thou doest well: the devils also believe, and tremble.}’{\scriptsize\textsc{(James 2:19)}}.\end{displayquote} The mere regarding as true what Holy Scripture says is, therefore, according to Scripture itself, something which also the devil can do and which hence does not save.  The faith to which the Christian religion promises salvation is accordingly something entirely different.  It is, in a word, as Holy Scripture itself says, ‘\textit{a sure confidence}’ {\scriptsize\textsc{(K.J.V. marginal reading; German: ‘ein gewisse Zuversicht’)}}, a ‘\textit{receiving}’ {\scriptsize\textsc{(German: ein ‘Auf – und Annehmen’)}}. {\scriptsize\textsc{(Hebrews 11:1; John 1:12.)}} \par God performed the great wonder of His eternal love in sending His only begotten Son into the world, in having Him become a man, that through Him He might Himself pay all men’s debt of sin, and thus win again for all men the heaven forfeited through sin and the salvation lost through the same, and finally offer and bestow all this through Word and Sacrament.  What is there then to do on the part of man?  Nothing, nothing whatsoever; but to give the glory to God and to receive the gift; and this and nothing else is faith.  In the merely ‘\textit{regarding as true}’ the ‘\textit{receiving}’ is lacking, and therefore true faith is lacking.  But if a man really receives the grace and salvation offered in Word and Sacrament to all men and hence also to him whoever is offended at this doctrine of faith is really offended only at the greatness of divine grace, at the blessed counsel of redemption, at Christ ‘\textit{the Savior of the world}.  Would God that it were only the unbelievers who reject the correct doctrine of faith!  but alas entire great church-parties do just this”\end{displayquote}. --Upon the same subject Walther said in another “\textit{Lutherstunde}”: \begin{displayquote}“The Christian religion demands also faith in its divinity and truth.  But this faith is by no means that to which the Christian religion promises salvation.  When Christ says: \begin{displayquote}‘{\color{red}\textit{He that believeth shall be saved}}’,\end{displayquote}  --that does not mean merely: he that regards what I teach as true shall be saved.  Rather does it mean this: You human beings by sin have fallen away from God and into an eternal debt which you are unable to pay.  But be of good cheer; I the Son of God, have paid your debt and thereby won back for you God’s grace and eternal salvation, and all this I offer you as a free gift.  Come then, receive this gift, and thus you are helped.  And this receiving is exactly the faith of which the Christian religion speaks.”\end{displayquote}}

                And this careful separation of faith from everything which is a deed or quality of men is altogether necessary.  First, in order that His glory as Savior may remain Christ’s own.  And then also because consciences are confused through all false thoughts concerning faith.  “\textit{How many}”, says Walther, “\textit{do not dare to believer because faith has been falsely described to them {\scriptsize\textsc{(namely, as one’s own deed, as a good quality, as fides formata, as feeling)}}}.”

                In the following we shall just briefly indicate what Walther taught with regard to this point.
\divider
                The Papists err quite grossly in the doctrine of faith, since they expressly say that faith justifies in so far as it is a good quality, a virtue in the heart of man which includes love and all good works.

                Here also those err who, with the fanatics, conceive of justifying faith as a change in the heart of man.  It is indeed faith, and faith alone, which does produce a change in the heart of man.  But this changing, sanctifying power is not the reason why faith justifies.  If one ascribes justification to faith in this respect, then justification is again based not upon Christ but upon man himself, namely upon the new life begun in man.

                It is likewise to be rejected when justifying faith is, with the fanatics, conceived of as wrestling and striving for grace.  It is indeed true that faith wrestles and strives.  “\textit{They are greatly in error}”, says Walther, \begin{displayquote}“\textit{who suppose that we are against an earnest godliness, that we reject striving, praying and wrestling, sighing and weeping; O no! many a one of us perhaps spends more time upon his knees than those who want to earn grace thereby; only we are against the idea that grace must be obtained by praying, sighing, and wrestling}”.\end{displayquote}  Faith comes into consideration in justification not in so far as it wrestles and strives, but in so far as it rests in the promise of the Gospel, in so far as it is the sure confidence which appropriates to itself the promise of grace contained in the audible {\scriptsize\textsc{(Word of God)}} and visible {\scriptsize\textsc{(Sacrament)}} Gospel.\footnote{Western District Report, 1875, p. 22.}  If anyone says that faith justifies in so far as it wrestles and strives for grace, he would thereby “\textit{take away from God the honor and set up a heathenish justification adorned with a few Christian patches}”.\footnote{L.c.}

                Faith is also not to be conceived of, with the fanatics and the modern theologians, as a condition of justification, if one takes the word condition in its proper and primary meaning.  Walther often insists upon this point very emphatically.  He indeed acknowledges that one can well make use of this expression in speaking of the necessity of faith, or when one wishes to stress the point that justification cannot take place without faith.  But one must then carefully avoid all misunderstanding, for the word condition, as it is ordinarily used, includes a performance {\scriptsize\textsc{(Leistung)}} on the part of him who is to receive something.  But faith comes into consideration in justification not as a performance, but as the antithesis of all human performance.  Faith is therefore not a condition under which we become righteous, but the way and manner in which we become partakers of the righteousness with which God has long ago {\scriptsize\textsc{(in Christ’s resurrection)}} endowed us and offers us in the Word of the Gospel.  We do indeed read in the Scripture: “\textit{With the heart man believeth unto righteousness}”.  But the particle “\textit{if}” has a double sense.  It is used either to give the basic cause \marginpar{\scriptsize\textit{Etiologically} \\ --assigning or seeking to assign a cause}{\scriptsize\textsc{(etiologically)}}, or to designate the way and manner \marginpar{\scriptsize\textit{Syllogistically}\\ --deductive reasoning}{\scriptsize\textsc{(syllogistically)}}.\par  In the preaching of the Law: “\textit{If you do this you shall live}”, the “\textit{if}” gives the basic cause, since obedience is the cause on account of which eternal life is given to those who keep the Law; but in the Gospel promises: “\textit{If you believe you shall be saved}”, the “\textit{if}” is syllogistic, for it designates the divinely ordained way and manner of appropriation.

                To be sure, modern Lutheran theologians favor the expression that man is justified under the condition of faith.  But this is due to the fact that the modern Lutheran theology is synergistic through and through.  It calls faith “\textit{a deed of our age}”, an exalted “\textit{moral deed of the will appropriating salvation}”  Therewith the doctrine of justification by faith is actually given up, even though they may still say that man is justified by faith alone.  All  justification by faith is actually given up, even though they may still say that man is justified by faith alone.  The word “\textit{faith}” and so also the expressions “\textit{by faith}”, “\textit{by faith alone}” have obtained an entirely different sense.  All synergists must also falsify the doctrine of justification, because they make out of faith something which is likewise a deed of man.  Therefore Walther says with regard to the most recent controversy: \begin{displayquote}“\textit{Also at the present time {\scriptsize\textsc{(in the controversy concerning the doctrine of conversion of election)}} the issue is no other article than that of justification.}\end{displayquote}  The present question is: \begin{displayquote}{\footnotesize Is man really justified and saved alone by grace?  Does Christ do this alone or does the reason why a man is saved lie in man?  Does faith justify because Christ has already done all, so that we have only to appropriate it to ourselves?  Or is faith that which man must do on his part, is faith necessary because also on the part of man something must be done?}\end{displayquote}  Hence Walther also repeatedly designated retaining the purity of the doctrine of justification for our fellowship as the foremost fruit of the most recent doctrinal controversy.