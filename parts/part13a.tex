\chapter{Conversion and Election}
\hrule
\vspace{.30cm}
As we have endeavored up to this point to give a presentation of Walther’s battle for the doctrine of the Church and of justification, so we will now turn our attention to a still more comprehensive discussion of the doctrine of conversion and election which Walther maintained over against the errors which have emerged in our time.
\vspace{.30cm}
\hrule
\vspace{1.25cm}
                While the battle for the Scriptural doctrine of the Church occupied Walther during the first period of his activity, the contention for the pure doctrine of election and conversion stood in the foreground during the last fifteen years of his life.  And we must say: as the correct doctrine of the Church, almost forgotten even within the Lutheran Church, was made known again chiefly through Walther, so also it is principally to be ascribed to his testimony that the Biblical doctrine of conversion and election has not been completely swept away by the stream of modern error.

                With a view toward a more lively perception of the grace which God also in this connection has showed His Church through the service of Walther we shall permit our attention first to be directed to the position of modern Lutheran theology in these doctrines.

                In modern Lutheran theology synergism is predominant, in part the grosser Melanchthonian type, in part the finer Latermannian.  Kahnis,  expressly professes the synergism of Melanchthon when he writes: \begin{displayquote}“\textit{Melanchthon through his doctrine of the cooperation of the human will in the appropriation of salvation {\scriptsize\textsc{(synergism)}} took the correct evangelical and at the same time traditional way to hold fast the substance of the Augustinian doctrine without its excesses}”.\footnote{Baieri Compendium, ed. Walther, II, pg 302.}\end{displayquote} Yes, Kahnis calls it an exaggeration when one teaches that “\textit{the natural man is totally dead to good}”.\footnote{L.c., p. 301.}  The majority of more recent Lutheran theologians, however, advocate synergism in the Latermannian form: \begin{displayquote}{\footnotesize The will of man is made free through grace to such an extent that man can now decide for himself for or against grace.}\end{displayquote}  According to this doctrine the Holy Ghost works so much, that man can convert himself, but the actual conversion man himself must perform.  Or: \begin{displayquote}{\footnotesize The Holy Spirit confers the power to believer, the ability to believer; the act of faith, the actual faith itself, man himself must produce on the basis of that competence conferred by the Holy Ghost.}\end{displayquote}  Hence they call faith a “\textit{performance}” {\scriptsize\textsc{(“Leistung”)}} of man, “\textit{man’s own deed}”, a “\textit{moral self-activity of man}”, etc.  \par Hence they posit a cooperation of man not only after conversion but in conversion for the purpose of bringing it to pass.  The statement of the Lutheran Confessions, that man in his conversion is not active but only passive, mere passive-, in need- according to the teaching of the newer Lutheran theology – of a “\textit{qualification}” {\scriptsize\textsc{(“Einchränkung)}}, namely, of the qualification that man is not mere passive, but conducts himself actively, cooperatively, and this “\textit{qualification}” of the Lutheran doctrine “\textit{now enjoys a virtually universal acceptance}”, as Dr. Luthardt remarks.

                The motivation for the setting up of such a doctrine divergent from the Lutheran Confession on the part of modern Lutheran theologians is the circumstance that they assign to theology a very peculiar task, namely the task of convincing human reason of the correctness of Christian doctrine.  While the old Lutheran theologians regarded the task of theology as the compilation and orderly presentation of the articles of faith revealed in Scripture, the newer Lutheran theology has set as its goal the mediation of the articles of Christian doctrine to human reason, specifically, the supplying of a rational consistency between the individual Christian doctrines.  And from this standpoint the modern Lutheran theology has arrived at its synergism.  For human reason concludes in this manner:\begin{displayquote}{\footnotesize if those who are saved were converted by the gracious working of God, without cooperation on their part or without a better conduct on their part exerting any influence upon their conversion, one should have to conclude that God passes by the rest with His grace or that God’s grace is not universal.}\end{displayquote}  Now if this latter supposition is inadmissible, then one must take refuge in the former, namely, that conversion is dependent upon the good conduct of man.  To acknowledge at this point, with the Formula of Concord, a mystery insoluble in this life, and to allow both facts, that those who are saved are converted alone by the gracious working of God and that those who are lost remain in unbelief by their own fault alone, to stand unharmonized {\scriptsize\textsc{(“unvermittelt”, “unmediated”)}} would be diametrically opposed to the purpose which modern theology desires to serve.  These two truths must be harmonized.  The \textbf{Formula of Concord}, which not only does not do this but even warns against doing it, and which holds fast both universal grace and also the doctrine that in conversion God alone does everything and that no cause of conversion and election is to be acknowledged in man, is subjected to the criticism that it does not with sufficient circumspection keep within the bounds of necessary moderation, that it is “\textit{not entirely free from predestinarian}”, that is to say Calvinistic, “\textit{tendencies}”, that it “\textit{contains unassimilated elements of the doctrine of absolute predestination}”.  Others, who do not venture to speak of Calvinistic tendencies in the Confession of the Church to which they wish to belong, allow themselves to interpret the Confession of the Church to which they wish to belong, allow themselves to interpret the Confession in such a way that the recent synergistic doctrine eventuates.

                In this way the recent Lutheran theology has turned everything upside down in the articles of conversion and election.  It has corrupted the entire \marginpar{\scriptsize\textit{usus loquendi}\\ --Manner of speech.  Words are assumed to be defined by common parlance; unless special context dictates otherwise}\textit{usus loquendi}.  If anyone teaches that the mercy of God and the merit of Christ and nothing in us – no cooperation, no better conduct – is the cause of conversion and salvation, they declare this to be Calvinizing.  According to the usage which previously obtained in the Lutheran Church absolute election was understood to be the doctrine of the Calvinists according to which election was not based upon Christ’s merit but embraced the merit of Christ only as a means of execution.  But the modern Lutheran theology speaks of absolute election when one will not have election based upon man's self determination, good conduct, etc.  According to the Lutheran Confession it is Scriptural and alone correct when one refuses to answer the question why some are saved rather than others, but acknowledges a mystery at this point.  In our time “\textit{Lutheran}” theologians declare such silence to be a mark of Calvinism.  The Lutherans are called Calvinists today and the synergists are called Lutherans.  According to the Lutheran Confessions it is the highest comfort for a Christian to know that his salvation rests not in the least in his hand but alone in God’s hand.  According to recent Lutherans the comfort of the Gospel remains undiminished only when one holds that conversion and salvation depend in the last analysis upon man’s free self-decision, or, which is the same thing, upon man’s conduct.  From a desire to rescue universal grace by the use of rational arguments they have lost the whole concept of grace entirely.

                This is the situation which today’s synergistic-rationalistic “\textit{Lutheran}” theology has brought about.  There appears here a depth of Satan which fills everyone, who by the grace of God has a seeing eye, with consternation.

                The doctrine of the modern Lutheran theologians as it has just been described has now in all essentials become current also in America, and this has been brought about in the first place by the Iowa Synod.

                So far the doctrine championed by Iowa is concerned, this becomes clear from the following quotations.  We here offer a series of utterances of the one time leader, of the Iowa Synod, in order that it may be the more clearly perceived of what sort the doctrine was which sought entrance into the American Lutheran Church and was limited to more confined circles chiefly through Walther’s opposing testimony.  Walther brings the following utterances of \marginpar{\scriptsize\textit{Gottfried William\\ Leonhard Fritschel}\\ (Dec, 19 1836-1889)\\--was a German-born Lutheran who emigrated to Iowa}Prof. G. \textbf{Fritschel} in “\textit{Lehre und Wehre}”\footnote{Lehre und Wehre 1872, p. 204f.}: \begin{fancyquotes}Two statements must be placed side by side and both together be held fast.  \par The first:\begin{displayquote}{\footnotesize Man can in no way prepare himself for divine grace, but he owes all his salvation entirely and alone to grace; grace must itself bring it about that he accepts grace.}\end{displayquote} \par The other: \begin{displayquote}{\footnotesize Whether man is saved or is lost depends in the last analysis upon man’s own free decision for or against grace}.\end{displayquote}  That of two men who hear the Gospel the resistance and death of the one is taken away while that of the other is not..., this has its basis in the free self-decision of man, although this decision is first made possible by grace”.\par  --That of two men to whom the Gospel is preached the one comes to faith, the other does not: for this according to God's Word the basis lies only and alone in the decision of man.\par --Therein lies the real inner difference between the Biblical and the predestinarian doctrine, that according to the former man’s eternal fate is rooted in the personal free decision of man for or against the grace offered him in Christ... He {\scriptsize\textsc{(God)}} lets it depend upon the decision of man to whom He will show mercy and whom He will harden.\par --When the Gospel comes to a man there is bestowed upon him by grace the poser to accept it, while he indeed can also willfully reject it by the determination of his will against God.  He receives in consequence of the working of grace \textit{arbitrium liberatum}{\scriptsize\textsc{(a freed will)}}.  His will enslaved by sin is so far set free by the call of grace that he now with his own free will can freely decide for or against God, which decision indeed need not take place like a flash in a moment.\end{fancyquotes}  Finally Prof. F. said in the words of Philippi: \begin{displayquote}“\textit{The statement ita spiritu sancto agimur, ut ipsi quoque agamus}, i.e., \textit{we are moved by the Holy Spirit in such a way that we also do something, is true not only of the converted but of those who are being converted...  As, accordingly, a certain synergism of man in the use of the means of grace even before the beginning of the internal divine work of grace is not to be excluded: so there takes place also a synergism of the human will with the divine grace not only after the completion of conversion but also during the act of conversion itself, except that indeed there is no synergism of the naturally free but only a synergism of the will freed by grace}”.\footnote{before the occurrence of conversion}\end{displayquote}

                That was the doctrine proposed by Iowa.  It is the synergistic teaching of the intermediate state in which the not yet converted man is supposed to be placed through grace into such a position that he can now convert himself.  On the one hand the phrase that man owes all his salvation wholly and alone to grace, but on the other hand the definite assertion that conversion and salvation “\textit{in the last analysis}” depend “\textit{only and alone}” upon man himself, upon his own free decision.  “\textit{Everything}” is to be ascribed to divine grace, only not that which is finally decisive, that a particular person is converted and saved.  As long as they are speaking of being saved in itself, they are willing to ascribe all to grace.  But as soon as those who are being saved are placed in comparison with those who are being lost, then the “self-determination”, “conduct”, of the former must be the ultimate ground of salvation.  And scarcely ten years later this Iowan doctrinal position was adopted by Prof. \textbf{Schmidt} and the Ohio Synod.  They refrained from the expression that the “\textit{self-determination}” is the deciding factor on the part of those who are being saved, but they substituted for it the equivalent expression “\textit{conduct}”, and expressly asserted, herein going beyond Iowa in their phraseology, that man does not owe his salvation alone to grace.  They proposed statements like the following: “\textit{Salvation in a certain sense does not depend upon God}”.  In a certain respect “\textit{conversion and salvation depends also upon man and alone upon God}”.

                And this doctrine Iowa and Ohio sought to support with the same rationalistic argumentation as the German Lutheran theologians employed.  Iowa expressed herself as follows: \begin{displayquote}“\textit{It remains true that, if God has predestined only a certain number of men unto eternal life, {\scriptsize\textsc{(which Iowa acknowledged as correct)}} the basis for this lies either in the absolute election of God who simply wills to bestow faith upon all these men, or else it lies in the decision of man which God has foreseen}”.\footnote{L.u.W., 1872, p. 243.}\end{displayquote}  On the part of Ohio it was expressed thus: “\textit{It would be clear that if God would decide the matter no one would be lost}”.\footnote{L.u.W., 1886, p. 25.}

                That there should be a mystery here, that one must accept both truths: in man no cause of conversion and salvation, in God no cause of unbelief and damnation, without harmonizing, this was ridiculed by Iowa and Ohio and declared to be a sign of Calvinism.  On the part of Iowa it was said: “\textit{Perhaps someone would offer as such a third possible explanation}”\footnote{(namely, in addition to the two rejected by Missouri, that either the Calvinistic absolute election or else human conduct explains the election of certain persons)} --“this:\begin{displayquote}\textit{ Why God has chosen some and left the others we cannot understand, as this belongs to the secret will of God which we should not investigate; which might be the one intended by the Missouri Synod in the Synodical Report in question.  But that is not a third explanation in addition to the other two mentioned above” {\scriptsize\textsc{(absolute election or human conduct),}} “but merely a non-explanation.  It is a mere forced suppression of the question, by which no help is offered}”.\footnote{L.u.W., 1872, p. 243.} \end{displayquote} On the part of Ohio it was untiringly asserted that it was only Calvinistic evasions when the “\textit{Missourians}” spoke of a mystery and would answer the question, why some are converted and saved rather than others, neither by Calvin’s particular grace nor by the assumption of a better conduct on the part of those who are saved.

                That was the position of Iowa and Ohio and the argumentation by which it was supported.

                How did Walther attack this position?  He polemicized above all things against the assumption on which the entire proposition of the adversaries rested, against the assumption, namely, that an explanation must be found and given for the fact that of two men who hear the Gospel the one is converted rather than the other.  Rather does he demonstrate that the Scripture and after the Scripture also the Lutheran Confession demands that a mystery be acknowledged at this point.

                In the somewhat lengthy article in which Walther for the first time polemicized in a very comprehensive way against the synergism which has emerged in the American Lutheran Church Walther attacks its position immediately at the very center.  He writes in this article: “\textit{Is it really Lutheran doctrine that the salvation of man depends in the last analysis upon man’s own free decision?}” as follows: \begin{fancyquotes}the first reason why this is not Lutheran doctrine but a teaching which has always been most decidedly repudiated by the Lutheran Church is that hereby the inexplicable mystery, why certain men come to faith and are saved, while other men do not come to faith and are lost, although both lie in the same impotence and guilt, is entirely destroyed by explaining this mystery according to one’s own thoughts.\footnote{L.u.W., 1872, p. 240.}\end{fancyquotes} 