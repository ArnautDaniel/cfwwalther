\chapter{Conversion -- A Mystery}
\hrule
\vspace{.30cm}
The recognition of a {\scriptsize\textsc{(for this life)}} insoluble mystery in the doctrines of conversion and election is for Walther of decisive importance for the correct apprehension and presentation of these doctrines.
\vspace{.30cm}
\hrule
\vspace{1.25cm}
                Wherein does this mystery consist?  On this point Walther expresses himself both positively and negatively.  He brings out both that wherein this mystery does not consist and also that wherein it does consist.  His numerous utterances on this subject may be briefly summarized as follows: We know exactly the reason, and it is therefore no mystery to us, why those who are lost are not converted and saved.  God’s Word says clearly that the cause of this lies in man himself; not in the unwillingness of God but in the unwillingness and obstinate resistance of man.  We also know exactly the reason why those who are saved are converted and saved and have been chosen from eternity unto salvation.  The reason does not lie in man himself but alone in God’s mercy and Christ’s merit.  The mystery for us men begins when the saved are compared with the lost, and so in the question: \begin{displayquote}“\textit{Why some are converted and saved rather than others?}”\end{displayquote} Here is the point where we must call a halt to our thoughts if we would avoid coming into conflict with incontrovertible truths.  Since all men are by nature equally corrupt, and those who are converted and come to faith and remain in faith till the end have to ascribe this not to themselves but alone to the grace of God in Christ and to the working of the Holy Spirit, who alone, as the Formula of Concord says, “\textit{changes their resisting will into an obedient will}”, therefore no man can discover by his reason why all other men do not come to conversion and faith and persist therein until death.  At this point therefore we must be silent and confess that there is a mystery here which no man in this life can solve, for the reason that there is no revelation of God concerning this in His Word.  God’s revelation confines itself to teaching that the destruction of man comes from himself, but his help comes from God {\scriptsize\textsc{(Hosea 13:9)}}.\footnote{L.u.W., 1883, p. 91f. Berichtigung, etc. p. 23f.}  All solutions of this mystery which men have attempted or may yet attempt result either in Calvinism {\scriptsize\textsc{(denial of universal grace)}} or in synergism {\scriptsize\textsc{(conditioning of conversion and election upon human conduct, human self-decision, etc.)}}.

                Walther proves this position to be demanded by Scriptures, especially by{\scriptsize\textsc{Hosea 13:9 and Romans 11:33-35}}.\footnote{Berichtigung, etc. p. 24f.}  This position is also that of the Lutheran Confessions.  “\textit{The Formula of Concord}”, says Walther, \begin{fancyquotes}counts among the mysteries of the election of grace not only the fact  that ‘\textit{God knows and has determined for everyone the time and hour of his call and conversion}’, but also ‘\textit{that one is hardened, blinded, given over to a reprobate mind, while another, who is indeed in the same guilt, is converted again}”.\footnote{L.c.}\end{fancyquotes}  That this was also the generally acknowledged and confessed doctrine of the Lutheran teachers of the Sixteenth Century Walther proves with citations from Chemnitz, Andreae, Selnecker, Heerbrand, Timotheus Kirchner, etc.\footnote{L.u.W., 1872, p. 244ff.; 1883, p. 94ff.; Berichtigung, etc., p. 24 ff.}  Walther expressed justifiable astonishment that, in the face of such clear utterances of the Confessions and of the theologians of the Sixteenth Century, anyone could assert that the acknowledgement of a mystery in the point under discussion was not Lutheran but Calvinistic.  Finally he refers to the fact that the acknowledgment of the mystery of the \textit{discretio personarum} had so passed over into the very spirit and blood of the Lutheran Church, that even among those later theologians who to some extent were already following other paths this point still ever and again rang through.  As evidence of this he used to refer to an utterance of John \textbf{Musaeus}.  He writes: \begin{fancyquotes}To the question: \begin{displayquote}‘Whether the Lutherans hold that the \textit{causa discretionis} {\scriptsize\textsc{(the cause of the difference)}} lies only and alone in man?’ \end{displayquote} J. Musaeus replied in his polemics against the Calvinist Wendelin, who had cast this up to the Lutherans, as follows: \begin{displayquote}‘\textit{That the causa discretionis, why some are converted, lies only and alone in man is an assertion which our theologians are not accustomed to make,  --rather do they all say with one voice that the cause why those are converted who are converted is not in man, but only and alone in God; but the cause why those who persist in their ungodliness are not converted is not in God, but only and alone in man!}'\end{displayquote}  Thus even Musaeus admits, and that as he says, together with all Lutheran theologians, that an inexplicable mystery is here.\footnote{L.u.W., 1883, p. 92 f.; Berichtigung, etc., p. 25.}\end{fancyquotes}

                That a mystery should be recognized and acknowledged here is of the utmost importance.  It belongs to the characteristics of a pure theologian.  He who acknowledges no mystery here, but has found a reasonable solution, is necessarily either a synergist or a Calvinist.  This is in the nature of the case.  “\textit{It is true}”, Walther writes, \begin{displayquote}“\textit{that when reason hears that some, without any contribution or merit of their own, are chosen to salvation by grace alone, then it can, if it wants to follow its own principles, not conclude otherwise than that the others are not saved for the reason that God has not chosen also them without any contribution or merit of their own by grace alone.  It is further true that when reason hears that those who are lost are lost solely by their own fault, then it can, if it wants to follow its own principles, not conclude otherwise than that the others who are saved attain salvation rather than the rest solely because they are better or have conducted themselves in a better way than the others}”.\footnote{L.u.W., 1884, p. 134.}\end{displayquote}  In order to avoid both deviations, Calvinism, as well as synergism, Walther therefore demands a refraining from all harmonizations and the unqualified acknowledgment of a mystery.  On one occasion he says very briefly: “\textit{He who finds no mystery here must be either a synergist or a Calvinist.  Tertium non datur}”.\footnote{Berichtigung, etc., p. 26.}  Walther cites approvingly the following words of \textbf{Guericke}: \begin{fancyquotes}The saved, so teaches the Lutheran Church, is saved alone by the grace of God {\scriptsize\textsc{(in Christ)}} without any merit of his own; the unsaved is unsaved by his own fault, because he continually resists divine grace; the reason that the resistance of the former against divine grace is finally broken, while that of the latter is not, is not the merit of the former, but is indeed the fault of the latter; the underlying inner disposition of man, insofar as it is good, comes indeed also from God alone, but insofar as it is bad, not from God; man with his dull sin-beclouded understanding is not able to explore this deepest depth of the divine working, and it is greater wisdom to acknowledge the divine mystery than blasphemously to solve it.\end{fancyquotes}

                Hereupon Walther continues, directing his words against the new theology: \begin{displayquote}“\textit{The new theology cannot get over wondering about this dilemma which confronts the old theology in its progressive development of doctrine it has found the easiest way in the world to thoughts; it says namely: that a number are converted and saved, while others are not converted and are lost, has its ground simply in the fact that the former faithfully employ the powers of grace bestowed before conversion unto their conversion and freely decide for grace, whereas the latter resist.  Thus it is not only clear why a number are lost, but also the mystery, why the others who are indeed originally in the same destruction are saved, is solved for human reason, namely because of their better conduct}.''\footnote{L.u.W., 1872, p.197.}\end{displayquote}

                Without the aid of synergism also the acceptance of the \textit{intuitu fidei} into the doctrine of the election of grace affords no reasonable solution.  Walther explains how the advocates of the \textit{intuitu fidei} convince themselves that they can very well solve the mystery why the elected alone for the sake of the mercy of God and the merit of Christ are chosen rather than others, namely because God had regard to the fact that it would be grasped and held fast by them in faith.  But, he continues, \begin{displayquote} ``\textit{thereby the mystery is solved only on the assumption that God did not Himself also decide to give faith to the elect, but they rather by virtue of their free will gave themselves to faith or at least as people who accommodated themselves to the divine order allowed God to work faith in them.  But this is simply nothing else that the most crass synergism}”.\footnote{Berichtigung, etc., p.38f.  In a lecture on May 6, 1881, Walther expressed himself upon this point as follows: \begin{displayquote}“The reply is made that by the ‘\textit{in view of faith}’ it is by no means intended to point out the cause which moved God to choose the elect, but that this expression is only used to forestall the assumption that election is an absolute, purely arbitrary act.  But if the \textit{intuitu fidei} shall not point out any cause, then what purpose shall it serve? \par For if it has been no cause for God the mystery remains so also the appearance as though election were an absolute and arbitrary act remains, and so it remains inexplicable why God chose just these foreknown believers, who nevertheless did not give themselves faith, but to whom God has given faith.  But if one says that the difference is just this, that the reprobate resisted, while the elect allowed faith to be given them, the elect person is thereby obviously made the cause of his election, which consists simply in the fact that he held still for God.  But according to God’s Word God must also first remove the resistance.  Hence, however our opponents may explain their \textit{intuitu fidei}, it either has no sense at all, or else faith is in a Pelagian way made a work of man, which God has regarded.''
\end{displayquote}} \end{displayquote}
 

                We are firmly convinced that if an understanding should ever come about between the church bodies which are now split into two camps on account of the doctrine of conversion and election, this will occur in no way than by an honest acknowledgment of the mystery of the \textit{discretio personarum} on the part of our opponents.  They also indeed speak of “\textit{inscrutable mysteries}” in the doctrine of conversion and election.  It has become the fashion to say: \begin{displayquote}{\footnotesize “It is self-evident that there are many inscrutable mysteries in the election of grace; we cannot comprehend God’s providential dealings either with regard to entire peoples or with regard to individual persons”}\end{displayquote}  But this has remained hitherto a mere mode of speech whereby people deceived themselves and others by a certain degree of outward conformity to the Lutheran way of speaking.  There must be an actual acknowledgment of that mystery which the Formula of Concord designates as such, namely: \begin{displayquote}{\footnotesize “One is hardened, blinded, given over to a reprobate mind, while another, who is indeed the same guilt, is converted again”.}\end{displayquote}  There must be the acknowledgment: \begin{displayquote}{\footnotesize  Although we know, on the one hand, why those who are lost are not converted and saved {\scriptsize\textsc{(namely, alone by their own fault)}}, and on the other hand, why those who are saved come to faith and remain in faith {\scriptsize\textsc{(namely, alone by God’s grace)}}, yet it remains here below a mystery for human reason why some rather than others are converted and saved, or, which is the same thing: in consideration of the fact that the grace of God is universal and that all men lie under the same guilt, in the same total corruption, it remains in this life a mystery for the human comprehension why only a part and not all men are converted and saved.}\end{displayquote} To explain this mystery for the human reason is the endeavor of the new theology; to this end its entire doctrine of conversion and election is directed.  For this purpose that status medius, the cooperation already in conversion, the theory of self-determination, etc., have been thought up.\footnote{L.u.W., 1872, p.293f., Note.}

                Solely in the same interest does Iowa here in this country set up the proposition: \begin{displayquote}{\footnotesize “That of two men who hear the Gospel the resistance and death is taken away in the case of the one, in the case of the other not so…, this has its basis in the free self-determination  of man”.}\end{displayquote}  In the same interest also Ohio says that a man’s conversion and salvation does not depend alone upon the grace of God but in a certain respect also upon the conduct of the man.  In the same interest both Iowa and Ohio cling so tightly to the proposition of the later dogmaticians that election took place “\textit{in view of faith}”, allowing themselves to substitute for “\textit{faith}” also “\textit{human conduct}”.  From the same standpoint Ohio, Iowa, and all who hold with them raise against Missouri the charge of Calvinism.  Missouri is charged with Calvinism not because it directly teaches Calvinism, but because this follows from the position of Missouri.  Missouri is charged with Calvinism because it acknowledges the mystery of the \textit{discretio personarum}, because it will not allow human conduct to stand alongside of the grace of God and the merit of Christ as an “\textit{explanatory cause}” why some are converted and saved rather than others. \par In short, if the opponents of Missouri would not occupy such a standpoint that they believe they must explain to human reason why some are converted and saved, resp. elected, rather than others, both their teaching concerning “\textit{human conduct}”, “\textit{self-determined}”, etc., would fall and also the charge that Missouri teaches Calvinism would be silenced.
%%% Local Variables:
%%% mode: latex
%%% TeX-master: "../main"
%%% End:
