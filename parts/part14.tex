\chapter{Conversion}
\hrule
\vspace{.30cm}
Having examined the point which according to Walther constitutes the main point of difference between the modern and the old sound Lutheran theology of conversion and election, we now undertake to present the main points of these doctrines themselves.  First, the doctrine of Conversion.
\vspace{.30cm}
\hrule
\vspace{1.25cm}
     \textbf{Luthardt} finds fault with the Formula of Concord as well as the orthodox Lutheran dogmaticians that they do not begin the cooperation of man already in conversion, but let it enter only after conversion.  Walther’s position, on the contrary, is the following: \begin{displayquote}{\footnotesize no kind of cooperation of man in conversion is to be admitted, neither from natural nor from so-called spiritual powers, but it must be maintained that God is alone active in conversion, but man is purely passive {\scriptsize\textsc{(mere passive)}}, merely \textit{subjectum convertendum}.}\end{displayquote}  If this is not maintained, if man is allowed to cooperate or contribute toward his conversion, then the characteristic feature of the Christian doctrine whereby it distinguishes itself from heathenism is  abandoned, then man is no longer saved by grace, then the doctrine of justification is subverted, then the assurance of the state of grace ceases.  Walther’s presentations with regard to the doctrine of conversion have the purpose of excluding synergism in every form, even the most subtle.

     Walther first rejects the teaching which openly accepts a cooperation of man toward his conversion or a preparation for it from natural powers.  Walther preserves the boundary between the kingdom of grace.  The distinction between these two spheres, he points out, “\textit{is not merely of degree but of kind}”.  Between nature and grace there is a gulf which only God’s almighty work of grace can bridge.  Hence there is no preparation for conversion on the part of natural man, as, for example, by a decent life, by “\textit{normal use of reason}”, by education and culture, etc.  In opposition to Kahnis, who attributes the rapid propagation of Christianity in the environment of the classical world in part to this environment itself, Walther says: \begin{displayquote}``\textit{In classical Athens Paul did not at all experience that people there were prepared for Christ more than others, and we are convinced that Dr. Kahnis experience nothing of the kind in classical Leipzig, but rather the reverse; as far, that is, as the Gospel is still preached there}.”\footnote{L.u.W., 1878, p.261—264.}\end{displayquote}  Yes, not only Scripture\footnote{Synodical report of the Northern District, 1873, p.47}, but even experience teaches that external worldly decency is no basis for conversion; “\textit{for often just the most vicious heathen have accepted the Gospel first of all}”.  Yes, \begin{displayquote}“\textit{outward worldly decency is often the most powerful hindrance to conversion.  It is for this reason no doubt that God withdraws his hand from many a man and allows him to fall into sin and shame in order that He may bring him to conversion}”.\footnote{L.c..45.}\end{displayquote}

     But there is also no cooperation toward conversion from so-called spiritual powers, or from powers conferred by grace.  This was indeed the position of the \textbf{Latermannian} synergists in the Seventeenth Century, and this is also the position of most modern Lutherans.  They say:  \begin{displayquote}{\footnotesize Indeed man can not cooperate toward his conversation from natural powers, but the man who lives under the sound of the Word, and stands under the influence of converting grace, can be active toward his conversion through the powers.}\end{displayquote} Iowa says: \begin{displayquote}{\footnotesize The man who is not yet converted but stands under the influence of converting grace can freely decide for conversion through the powers conferred upon him by grace.  Upon this “\textit{free decision}” depends his conversion and salvation.}\end{displayquote} Ohio says:  \begin{displayquote}{\footnotesize The man who is not yet converted can by God’s grace so conduct himself that upon him before others salvation is conferred.  Upon this “\textit{conduct}” depends his conversion and salvation.}\end{displayquote}  Thus it is asserted that the man who is not yet converted can through powers of grace be active to bring about his conversion.  It is also principally against these forms of synergism that Walther directed to his fight.

                Walther points out ever and again that this position involves a self-contradiction.  The right use if powers of grace implies a spiritual life-principle in man, or a man who can make the right use of powers of grace must already be converted.  He says: \begin{fancyquotes}If anyone says \begin{displayquote}“\textit{he ascribes to man a synergism toward his conversion not through his natural powers but only in the sense that he cooperates through powers conferred upon him by grace for this purpose}”,\end{displayquote} that is merely a theological sleight-of-hand.  For he who is himself able to effect something through powers of grace must either possess by nature the ability to put these powers of grace to use, or else he is already converted.\footnote{L.u.W., 1885, p. 109.}\end{fancyquotes}  In more detail Walther says on the same point: \begin{fancyquotes}Only after we are converted do we ourselves begin to work; the new man must first be born, then he begins to bestir himself, to speak, to do something; previously he does nothing at all, just as a child does nothing to bring itself to birth.  Hence man also cannot decide for himself {\scriptsize\textsc{(in conversion)}}. \par Many imagine conversion in such a way as though man found himself confronted with a cross-roads where the ways to heaven and to hell diverge; now man is given his choice which way he will go; if he goes the right way he will be converted, if he goes the wrong way he will be lost.  Thereby all honor is likewise taken from God; for if man can himself decide for the good, then there must be some good in him, and the decision itself would be a good work which he does before his is yet converted. --Those who hold this false doctrine of decision say indeed: \begin{displayquote}{\footnotesize Our doctrine takes no honor from God, for we do not say that man decides by his own natural powers, but we say that he does this with the powers of grace which are given him, and so nothing at all is ascribed to man;}\end{displayquote} --but they do not consider that only he can possess and use powers who is already alive.  Take a stock or stone and suppose that powers are blown into it – the stone would not trouble itself at all about the powers but would remain as before.  Powers presupposes a subject which uses the powers; and so man would have to be converted already in order to be able to convert himself; he would have to be already awakened in order to be able to convert himself; he would have to be already awakened in order to be able to awaken himself; he would have to be already renewed in order to be able to renew himself. \par No, as soon as a man is so far along that he can use the divine powers of grace he is also converted, then God has already decided and determined him, then He has already given him a new heart, then He has already regenerated Him through His Holy Spirit.\footnote{Report of the Western District, 1876, p. 67, 68.}\end{fancyquotes}  Walther says, with the old theologians who opposed the Latermannian synergism: \begin{displayquote}``\textit{Spiritual powers are not first given, that man may afterwards convert himself by means of them, but the bestowal of spiritual powers is in fact the conversion itself.}\footnote{L.u.W., 1872, p. 268.}\end{displayquote} If one says: \begin{displayquote}{\footnotesize The Holy Spirit so operates in liberating a man that a man can thereafter convert himself,}\end{displayquote} -- then Walther asks: \begin{displayquote}“\textit{Can a man be liberated and yet not be converted or regenerated?  The liberation of the man is itself the conversion or regeneration}”.\end{displayquote}

                As this doctrine is self-contradictory, so also it contradicts Scripture, the Lutheran Confession, and also experience.  According to Scripture conversion is “\textit{a great miracle which God performs}”, which God brings about by His good pleasure, and in which every cooperation of man is excluded.\footnote{Report of the Western District, 1876, p. 63, 65. Jeremiah 31:18; Philippians 2:13; Psalms 51:10; Isaiah 65:1; 2 Corinthians 4:6.}  Conversion according to Scripture is worked by God the Holy Ghost by grace alone for Christ’s sake.\footnote{Report of the Northern District, 1873, p. 43.56;  Romans 3:23-24; Ephesians 2:1 ff.; 2 Timothy 1:9; Colossians 2:12.}  In particular Walther refers to those passages of Scripture in which conversion is described as a new creation, an awakening from death, a new birth.  He says, for instance: \begin{fancyquotes}Holy Scripture compares conversion with creation, for we are called {\scriptsize\textsc{(after the change which has taken place in us through conversion)}} new creatures.  But what can the thing created do toward its own creation?  What did the world do toward its creation? For it was not yet there at all; so it could also do nothing.  What did Lazarus do toward his re-vivification? – for conversion is called a quickening in Holy Scripture – for he was dead; therefore he could also do nothing. \par Christ did it; He said: ‘{\color{red}\textit{Lazarus, com forth!}}' and Lazarus came forth.  Or what have we done toward our own birth?  Nothing, for all took place without us.  Only after we have been created, born, and quickened, our cooperation begins, not sooner.  Hence all who ascribe to man a cooperation toward his conversion thereby overthrow the entire Scriptural doctrine of conversion.  For, in the first place, we are entirely dead in sins, so that we can in no way cooperate toward our conversion, and, in the second place, the apostle says: \begin{displayquote}‘\textit{It is God which worketh in you both to will and to do of His good pleasure}’.\footnote{Report of the Western District 1876, p. 69.}\end{displayquote}\end{fancyquotes} – Walther offers the proof that according to the teaching of the Lutheran Confession the cooperation enters only after conversion, that in conversion man is merely subjectum convertendum, i.e., that he is purely passive {\scriptsize\textsc{(mere passive)}}, not active, e.g., in “\textit{L.u.W.}”.\footnote{L.u.W, 1872, p. 259 f., 290f.}  The champions of the teaching that man by virtue of grace is active toward his own conversion have indeed claimed that they were able to hold fast the pure passive of the Confession.  But Walther replies: \begin{displayquote}“\textit{To assume a synergism {\scriptsize\textsc{(cooperation)}} of the human will with divine grace not only after completed conversion but also during the act of conversion and still to be in agreement with the Confession of our Church is obviously a contradictio in adjecto. – For cooperation {\scriptsize\textsc{(which is activity)}} and passivity so completely exclude each other that it seems foolish even to waste a word on the matter}”.\footnote{L.u.W., 1872, - 289 f.}\end{displayquote} – Walther also appeals to the experience of Christians.  He writes, for instance: \begin{fancyquotes}We on our part can not only not understand how Prof. F. can regard this as Lutheran doctrine, but also not how any Christian who has come to true faith can so judge.\footnote{namely, that the conversion and salvation of definite individuals should depend upon their own free decision}  If we should say that we came to faith, while so many of our contemporaries, who, let us merely say, were not more depraved than we, remained in unbelief, for the reason that we freely decided for God with our own will: we would thereby have to deny our innermost Christian consciousness. \par Also all those who bear the unmistakable tokens of being truly believing Christians and who have communicated their experiences to us have always hitherto confessed that their having become believers truly did not have its basis in their own free decision but in nothing else than an incomprehensible eternal mercy of God in Christ.  All who with the poet could triumphantly exclaim: \begin{displayquote}{\footnotesize ‘Now I have found the firm foundation’}\end{displayquote} we have always heard confess with the same poet:

                  \begin{displayquote}
                    {\footnotesize It is that mercy never ending,

                    Which human wisdom far transcends,

                    Of Him who, loving arms extending,

                    To wretched sinners condescends;

                    Whose heart with pity still doth break

                    Whether we seek him or forsake.}\footnote{L.u.W., 1872, p. 243-244; Western Dist. 1876, p. 64-65}\end{displayquote}\end{fancyquotes}

                This teaching of a self-determination for grace underlies the assumption of a neutral state {\scriptsize\textsc{(status medius)}}, a state which is supposed to be intermediate between being unconverted and being converted.  There is supposed to be a state in which a man is indeed not yet converted but yet has been so far liberated by calling grace that he is able to be active toward his conversion, to decided for grace.  Walther calls this \textit{status medius} a fiction, while he at the same time carefully distinguishes between truth and error in the claims which are brought forward for the support of this neutral state.  Walther does not deny that impulses {\scriptsize\textsc{(Bewegungen)}}, and indeed powerful impulses precede conversion in most cases.  In this connection he often used the figure of a fortress which is to be stormed, whereby a great stir is called fort within the fortress.  So also in unconverted men powerful motions may take place during the preaching of God’s Word. \par Walther was accustomed to adduce the examples of \textbf{Felix}, \textbf{Agrippa}, etc.  The former trembled as Paul reasoned of righteousness, temperance, judgment to come {\scriptsize\textsc{(Acts 24:25)}}.  The latter was so moved by the preaching of the apostle that he said: “\textit{Almost thou persuadest me to be a Christian}”. \par But these motions in the still unconverted prove nothing for a \textit{status medius} or for a cooperation from spiritual powers before conversion.  There is still no life in man in connection with these motions.  \begin{displayquote}“\textit{The Holy Ghost is only working from without into man.  The soul of the man, although moved by the Holy Ghost, has not yet become the dwelling-place of the Holy Ghost}”.\end{displayquote}  No spark of spiritual life has yet been kindled in the man himself.  The impulses have not yet become the man’s own, that is to say, they do not come from a life-center {\scriptsize\textsc{(principium vitale)}} already existing in the man.  As soon therefore as the influence from without ceases the impulses also cease.  Walther was accustomed to use in this connection the figure of pressure upon gutta-percha.  \begin{fancyquotes}A \textit{gutta-percha} yields to the pressure of the finger, but as soon as the finger is removed immediately reoccupies its former space, so also a holy longing and yearning often arises in an unconverted man through the operation of the Holy Spirit without his being in the least active in it; but as soon as the Holy Ghost withdraws His hand this longing also vanishes.  Only when man has given in to the operations of God, when grace is no longer merely an influence working from without {\scriptsize\textsc{(gratia assistens)}} but has become indwelling in him {\scriptsize\textsc{(gratia inhabitans)}} can he cooperate.  He who teaches otherwise can only do it upon Pelagian premises.\footnote{Report of the Northern Dist. 1873, p. 51, 52}\end{fancyquotes}  Walther declares it to be very important that “\textit{the external and the internal working of the Holy Spirit}” be not confused the one with the other.  As long as in man great motions indeed occur, but are only the consequence of the external operation of the Holy Ghost, the man is still unconverted, still in a state of wrath, and no kind of cooperation, no ability to decide for grace, no good conduct by virtue of grace is to be ascribed to him.  But so soon as spiritual power has become man’s own, so soon as a spark of spiritual life has been kindled in man, and man can now make a decision, he is already converted.  We shall now cite a few more utterances of Walther relevant to this point.  He says: \begin{fancyquotes}The synergists after Luther’s death did not present their error in such a refined and subtle manner as did the \textbf{Helmstädt} synergists in the Seventeenth Century.  The course of synergism was the same as the course of error always is.  First came gross \textbf{Arianism}, then the finer semi-arianism; first gross Pelagianism, then the fine semi-pelagianism; first gross synergism, then the fine, so to speak, semi-synergism.  It sound quite fine when recent theologians say: \begin{displayquote}{\footnotesize When God gives unconverted man the power he can cooperate toward his own conversion.}\end{displayquote}  But it is not correct; for a dead man cannot use the powers conferred upon him as long as he does not have that power which is necessary to the use of such powers, as long, that is, as he does not have life in himself.  One can roll a dead body about and operate upon it electrically so that it opens its eyes, opens its mouth, and the like, but all this is only the consequence of powers operating upon it from without; only that one can move himself who has subjectively come into possession of the power.\footnote{Report of the Northern Dist., 1873, p. 52, 53.}\end{fancyquotes}  Furthermore: \begin{displayquote} “\textit{When the father say that one must not think of conversion in such a way as though a man could simply take it lying down, as though it took place as in a sleep, but much must take place in the understanding, will, and affections, this is falsely applied by recent theologians to the cooperation of man toward his conversion.  As the garrison of a fortress does not do anything toward shooting breaches in the walls and bulwarks and towards a setting the defenses on fire at various points, but will rather only close up the breaches and quench the flames, such is the situation also in conversion; in however lively a manner things may take place, yet it is only a life which is suffered, and man in all this is only a passive, not an acting participant.  But though he remains pure passive, he is not in this case like the sealing wax which neither knows nor feels anything of the impression of the seal, but man knows and perceives the work of the Holy Spirit upon him}.\footnote{l.c., p. 51.}\footnote{Walther’s frequently used picture of the fortress to be stormed which he carries out particularly in the {\scriptsize\textsc{Report of the Western Dist., 1876, p. 68, 69}}, has been used, especially on the part of the Iowa Synod, to charge Walther with teaching a most terrible conversion by force.  They have not tired of spreading about the world the report that the Missouri Synod follows Walther in teaching a “\textit{bomb and canon conversion}”.  From the connection it is entirely clear what the \textit{tertium comparationis} is in this figure, namely this, that man in no way comes to meet the activity of the Holy Ghost, but only resists it, and indeed resists until the heart of man is changed by the Holy Ghost, that is, converted.  But this is also the clear teaching of the Lutheran Confession. {\scriptsize\textsc{Formula of Concord, Art. II, par. 21 (Triglot, p. 889)}}: \begin{displayquote}“For man neither sees nor perceives the terrible and fierce wrath of God on account of sin and death, but ever continues in his security, even knowingly and willingly, and thereby falls into a thousand dangers, and finally into eternal death and damnation; and now prayers, no supplications, no admonitions, yea, also no threats, no chidings, are on any avail, yea, all teaching and preaching is lost upon him, until he is enlightened, converted, and regenerated by the Holy Ghost”.\end{displayquote}  Shortly before {\scriptsize\textsc{(par. 18)}} a “\textit{hostiliter repugnare}” is ascribed to man, and shortly after {\scriptsize\textsc{(par. 22)}} an “\textit{obstinate enmity against God}”.  That unconverted man only resists the Holy Ghost and indeed “\textit{hostilely resists}” can be no matter of wonder to anyone who maintains that in natural man there is nothing good by which he would in any way be ready to come to meet the Gospel.  But this is the very point in which modern theology, also in the Iowa and Ohio Synods falls short, as Walther has likewise pointed out.  The Iowan-Ohioan doctrine is based on the assumption that before the spiritual powers there is still something good in man.  They say indeed: \begin{displayquote}“\textit{Through powers of grace}” the still unconverted man can decide for or against grace.\end{displayquote}  But the “\textit{powers of grace}” are certainly not neutral, equally effective in either direction {\scriptsize\textsc{(indifferentes)}} toward conversion or turning away.  \begin{displayquote}“And so there must be a power in man before the powers conferred by the Holy Ghost, by which, with the help of assisting grace and the powers bestowed by the Holy Ghost, that which is necessary unto conversion is performed, and by which also the unwillingness to be converted is effected.  But this is Pelagianism and synergism itself''\end{displayquote} -- and so it comes to light, as soon as one looks into the matter more carefully, that also in connection with the phraseology that man freely decides by virtue of grace or that man conducts himself rightly by virtue of grace conversion is placed with regard to the decisive factor in the natural powers of man, or natural powers are attributed to man whereby he deals rightly with the “\textit{posers of grace}”.  Thus also this subtle form of synergism, that man converts himself by powers of grace, exposes itself as Pelagianism.  Walther says: \begin{displayquote}“Synergism is at bottom nothing else than papistical leaven; for the Papacy is nothing else than hierarchism on the one side and Pelagianism on the other.  Synergism or semipelagianism is only a more euphemistic expression, but in fact the same as Pelagianism. \par When the devil finds himself exposed he dons another garb and seeks through the subtle false doctrine to plunge people into gross heresy to the forfeiture of their salvation – but the final decisive question is just this: Who is the one who is to manage the powers conferred upon him from elsewhere? A dead person can do nothing with vital powers laid into the coffin unless he is first awakened to life.  Christ did not say to Lazarus, the young man at Nain, or the daughter of Jairus, before they were quickened: Here you have vital powers; now make use of them that you may live!  But He made them alive with His Word. – So the Iowans may talk as they will; they let it be known that they ascribe to the unconverted man power to make use of powers bestowed upon him”. {\scriptsize\textsc{(Report of the Northern Dist, 1873, p. 56, 57.)}}\end{displayquote}  That a cooperation toward conversion is ascribed to natural powers also comes to light at times in undisguised form.  So, e.g., when Ohio says that conversion and salvation does not depend upon grace alone, but in a certain respect also upon the conduct of man.  Now what does not depend upon grace depends upon the natural powers of man.  \textit{Tertium non datur!}  Hence if conversion and salvation should not depend only upon grace but besides and in addition also upon conduct, then this conduct must be based upon natural powers. \par Furthermore: that, in spite of all the talk of a conduct by virtue of grace and of a self-decision “\textit{by virtue of grace}”, nevertheless they have in mind a conduct and a self-decision by virtue of natural powers, is evident from the fact that by means of the ``\textit{conduct}” and the “\textit{self-decision}” they wish to explain to human reason why one man is saved rather than others.  Such a “\textit{basis of explanation}” {\scriptsize\textsc{(Erklärungsgrund)}} is obtained only if one lets the decisive conduct be effected purely by natural powers.}\end{displayquote}

 

 

     Thus Walther is determined to hold fast that neither before nor in conversion dies any cooperation of man take place.  During conversion powerful motions take place in man, but in connection with them man is not active, cooperative, but passive.  Hence for Walther transitive and intransitive conversion are not two different stages in the process of conversion, so that God should first convert man or give him powers unto conversion, in order that thereafter man might convert himself, but for him transitive and intransitive conversion coincide in fact.  He says: \begin{displayquote}“\textit{Transitive and intransitive conversion are merely different ways of looking at the same thing.  Man is converted when God converts him.  -- The ship turns when the steersman turns it.''}\end{displayquote}

     Walther also repeatedly dealt with the common objections, that man, if he in no wise cooperates toward his conversion, is in no wise active, does not decide, etc., but only suffers what God works in him, would be degraded to the level of a machine, that conversion would be a conversion by force, that the “\textit{moral element}” in conversion would be lost, etc.  Walther answers the objection, that man would sink to the level of a machine if he could not decide freely for or against grace, by reducing the opponents \textit{ad absurdum} and says: \begin{displayquote}``\textit{If a man is not degraded to the level of a machine when the so-called prevenient grace calls for the motions in man {\scriptsize\textsc{(motus inevitables)}} without man’s own decision or activity, which is admitted by the opponents, then this would also not be the case when converting grace works faith without the free decision or activity of man}.''\footnote{L.u.W., 1872, p. 296.}\end{displayquote}  The “\textit{conversion by force}” Walther repudiates as an insinuation of which synergists have always been guilty against confessionally loyal Lutherans.  Only then could one speak of a “\textit{conversion by force}” if the Lutherans taught a conversion in which no inner change took place in the understanding, will, and heart of man.  But the Lutheran doctrine is this: \begin{displayquote}{\footnotesize Although the human will is corrupt in the extreme and in no wise cooperates toward conversion, yet in it a total change takes place in and through conversion: in and through conversion it is changed from unwilling to willing.}\end{displayquote}  In this conversion consists.  “\textit{God creates the willingness and thereby and therewith God converts man}”.  The will of man is the subject in which conversion takes place.  Through conversion not the Holy Ghost but man becomes a believer.  In the charge of “\textit{conversion by force}” on the part of the champions of self-decision, etc., there lies an artifice.  They pretend that they want to insist upon absence of coercion in conversion, whereas in reality they want to secure in this way a cooperation toward conversion.  After Walther has granted over against Iowa that “\textit{man’s own free decision}” may be accepted \textbf{if} all that is meant thereby is “\textit{that man is not converted by coercion, but that in conversion also the will of man is moved to will and that it is man himself who believes}”, he continues: \begin{fancyquotes}But that Prof. F. with his ‘\textit{free decision}’ does not wish to assert only a freedom which is identical with the absence of coercion is unfortunately only too evident when he expressly writes: \begin{displayquote} ‘He, the natural man, receives in consequence of the operation of grace \textit{arbitrium liberatum}.  His will, enslaved by sin, is so far liberated that he can by his own will decide freely for or against God’.\end{displayquote}  Yea, in order that he may be correctly understood he makes Dr. Philippi’s words his own: \begin{displayquote}‘As, accordingly, a certain synergism of man in the use of the means of grace even before the beginning of the inner working of divine grace not only after completed conversion, but also during the act of conversion itself, only indeed no synergism of the natural free will but only a synergism of the will by grace.'\footnote{L.u.W., 1872, p. 258.}\end{displayquote}\end{fancyquotes}

      With regard to the saying of recent theologians: “\textit{Faith is free obedience}”, Walther remarks: \begin{displayquote}“\textit{Faith is indeed ‘free’, that is, uncoerced, but not a matter of ‘free choice and free determination’, as the moderns want to make it}”.\end{displayquote}  And, as regards the concern of the moderns that “\textit{ethics}” might suffer if man would not “\textit{freely decide}” for faith and faith accordingly would not be a “\textit{personal act}” {\scriptsize\textsc{(Selbstthat)}} of man, Walther again refers to the fact that also most of the moderns let “\textit{the first influence}” of grace come about without man’s cooperation or personal activity.  Now if through this occurrence “\textit{ethics}” is not overthrown, then it is also not overthrown through the occurrence of the conversion itself, even though God alone is active therein and man does not conduct himself actively but only suffers what God works in him.  Walther refers in this connection to the creation.  \begin{displayquote}“\textit{The will to good was created in Adam without his {\scriptsize\textsc{(Adam’s)}} cooperation, and yet this was not contrary to ethics}”.\end{displayquote}  Walther here breaks out in the words \begin{displayquote}“\textit{It is offense at Christ crucified, at the religion of grace, which makes men unwilling to let conversion take place without man’s cooperation}”.\end{displayquote}  Elsewhere Walther demonstrates that men have thought up the whole \textit{status medius}, in which man is supposed to be indeed not yet converted, but still through calling grace already enabled his conversion by good conduct, only for the purpose of solving the mystery that man is saved alone by grace and yet damned by his own fault.\footnote{L.u.W., 1872, p. 293, 294. Note.}
      When Walther repudiates the \textit{status medius} in this manner his answer to the question whether conversion takes place successively or in a moment is already evident.

 
%%% Local Variables:
%%% mode: latex
%%% TeX-master: "../main"
%%% End:
