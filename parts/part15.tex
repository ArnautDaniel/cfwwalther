\chapter{Election of Grace}
\hrule
\vspace{.30cm}
Walther in the year 1880 brought his doctrine of the election of grace to succinct expression in the well-known Thirteen Theses.  He himself says of these theses that they contain the doctrine in which he intends by God’s grace to continue until his death.\footnote{Lutheraner, 1880, p. 11.}
\vspace{.30cm}
\hrule
\vspace{1.25cm}
In the first four theses he directs his attack against Calvinism.  In opposition to Calvinism he teaches, \begin{displayquote}“\textit{that God has loved the whole world from eternity, has created all men for salvation and none for damnation, and earnestly desires the salvation of all men}”.\end{displayquote}  Moreover: \begin{displayquote}“\textit{that the Son of God has come into the world for all men, has borne, and atoned for the sins of all men, has perfectly redeemed all men, none excepted}”;\end{displayquote} Furthermore: \begin{displayquote}“\textit{that God earnestly calls all men through the means of grace, i.e., with the intention of bringing them through these means unto repentance and unto faith and of preserving them therein to the end and of thus finally saving them}”.\end{displayquote}  Hence he teaches finally also against Calvinism \begin{displayquote}“\textit{that no man is lost because God would not save him, or because God with His grace passed him by, or because He did not offer the grace of perseverance to him also and would not bestow it upon him; but that all men who are lost perish by their own fault, namely, on account of their unbelief, and because they have obstinately resisted the Word of and the grace of God to the end}”.\end{displayquote}  Although Walther thus holds fast universal grace in its full extent, he also teaches further {\scriptsize\textsc{(Theses 5)}} over against the \textbf{Huberian} error, that the election of grace is not universal but particular, that is, that it does not concern all men, but only the “\textit{true believers, who believe to the end or who come to faith at the end of their lives}”.  Hereupon follows Theses 6, \begin{displayquote}“\textit{that divine election is immutable, and hence, that not one of the elect can become reprobate and be lost, but that every one of the elect is surely saved}”.\end{displayquote}  Theses 7 and 8 deal with the knowledge of election.  Walther teaches that a Christian can and ought to be sure of his eternal gracious election, but teaches with regard to the manner of this assurance \begin{displayquote}“\textit{that it is folly and dangerous to souls, leading either to carnal security or to despair, when men attempt to become or to be certain of their election or their future salvation by searching out the eternal mysterious decree of God}”\end{displayquote} and insists to the contrary \begin{displayquote}“\textit{that a believing Christian should endeavor from the revealed Word of God to become sure of his election}”.\end{displayquote} Theses 9-11 treat, in thesis and antithesis, of what election is and is not as well as of the causes of it.  \begin{displayquote}{\footnotesize Election \marginpar{Theses 9} does not consist of the mere foreknowledge of God as to which men will be saved; also it is not the more purpose of God to redeem and save mankind, in which case election would embrace not only the temporary believers also, but all men generally; election is finally also not a mere decree of God to save all those who believe to the end.}\end{displayquote} In all these ways the election of grace is not correctly described. \begin{displayquote}{\footnotesize For \marginpar{Theses 10}  since the cause which moved God to choose the elect is solely His grace and the merit of Jesus Christ, and not any good thing which God has foreseen in the elect, even not the faith foreseen by God in them.}\end{displayquote}  Walther therefore believes, teaches, and confesses \begin{displayquote} “\textit{that \marginpar{Theses 11} election is not the mere foresight or foreknowledge of the salvation of the elect, but also a cause of their salvation and what pertains thereto}”\footnote{specifically also of faith itself.}\end{displayquote}  \textbf{Thesis 12} directs attention to the mysteries in the doctrine of election and the attitude which the Christian should take toward them.  Walther says \begin{displayquote}“\textit{that God has still kept secret and concealed much concerning this mystery and reserved it alone for His wisdom and knowledge, which no man can or should search out}”;\end{displayquote} -- hence he rejects \begin{displayquote}“\textit{the attempt to penetrate into what is not revealed and to harmonize with reason those things that seem to contradict our reason whether this is done in the Calvinistic {\scriptsize\textsc{(namely, by denial of the universal and earnest divine will of grace)}} or in the Pelagian-synergistic theories {\scriptsize\textsc{(namely, by the assumption of a better conduct on the part of the elect as a basis or explanatory basis of their election)}}}.''\end{displayquote}  In \textbf{Thesis 13} Walther declares, \begin{displayquote}“\textit{that it is not only neither useless nor even dangerous, but rather necessary and wholesome to present publicly also to our Christian people the mysterious doctrine of predestination, as far as it is clearly revealed in God’s Word}”;\end{displayquote} --  he does not agree with those \begin{displayquote}“\textit{who think that this doctrine must either be entirely concealed or must be reserved only for the disputations of the learned}”\footnote{The translation of the Thirteen Theses follows that given in the Concordia Cyclopedia, 1927, pages 511 and 512.}\end{displayquote}

                We thought as should place first this general overview of Walther’s doctrine of the election of grace on the basis of his very carefully drawn up Thirteen Theses.  Nevertheless we regard it as necessary to add Walther’s more detailed expositions on a number of individual points.  For Walther indeed directed his attention from the very beginning also to the doctrine of election, and had to devote the last ten years of his life chiefly to the controversy over the Lutheran doctrine of election.  Such points on which our readers will be pleased to receive more detailed information are the following: the relation of the faith resp. of the entire Christian status of the elect to their eternal election; election in the narrower and in the wider sense; election and the way of salvation in general possibility of knowing and being sure of election correct use and abuse of the doctrine of election, etc.

                In advance let us here refer once more to the real center of Walther’s position in the doctrine of election.  Let us here again expressly focus our attention on the point which forms the key to Walther’s position over against modern theology.  Modern theology asserts either one must grant that there is a better conduct in the saved whereby they distinguished themselves from the lost or else one has hopelessly gone over to the Calvinistic side.  It allows us in fact only the choice between synergism and Calvinism.  Luthardt for instance, says: \begin{displayquote}“\textit{If God would Himself effect the appropriation of salvation, the obedience of faith, conversion, -- the word being take in the sense of the present more Biblical usage -- then predestinationism would indeed be unavoidable}.''\footnote{Die Lehre vom freien Willen p. 276}\end{displayquote}  Luthardt wants to say: \begin{displayquote}{\footnotesize You can attribute to God’s operation only the possibility of faith, but to man himself must be ascribed the very decision itself the effecting of actual faith, otherwise you go over to Calvinism.}\end{displayquote}  In opposition to this position Walther’s demand is that every Christian and every theologian hold fast at the same time to both as inviolable truths, namely, that conversion and salvation depend alone upon God’s grace and not also upon the conduct of man and also that God’s grace is universal and earnest.  On the one hand it is to be taught the basis of election is alone God’s grace in Christ, and the conduct, self-decision, faith, etc. of man is not to be added to this basis as explanatory basis.  On the other hand it is to be held fast without any limitation that God’s grace is universal and earnest and that every one who is lost is lost solely by his own fault.  \par “\textit{As important as it is}”, says Walther, \begin{fancyquotes}that we should maintain that God contributes nothing to our being lost, so important is it that we should maintain that we contribute nothing to our being saved.  As important as it is that we attribute no blame to God in the loss of many men so important is it that we also do not take away from God the honor that it is He alone who saves us without any merit or worthiness of ours, by pure grace alone.\footnote{Evangelienpostille, p. 93.}\end{fancyquotes}

                In this framework move all the expositions of Walther concerning the doctrine of election.  Already in his \textbf{Gospel Postil} he proposes as the theme of a sermon on the election of grace the question: \begin{displayquote}“\textit{What must we do above all hold fast if we are to go astray neither to the right nor to the left in the doctrine of election}?”\end{displayquote} and answers: \begin{displayquote}“\textit{We must hold fast --\\ 1. that according to Holy Scripture he who is lost has not been appointed thereto by God, but is lost by his own fault, and\\ 2. That according to Holy Scripture he who is saved is not saved by any merit of his own, but by pure grace alone}”.\end{displayquote}  And one of the tracts written in the recent doctrinal controversy he closes with the words: \begin{fancyquotes}Do you, dear Christian, abide simply by that little text in which God the Lord Himself says: \begin{displayquote} ‘\textit{O Israel, thou hast destroyed thyself, but in Me is thine help}’ {\scriptsize\textsc{(Hosea 13:9)}}\end{displayquote} from this golden text turn aside neither to the right nor to the left; thus you are journeying upon the right road and the end of this your journey of faith will be eternal blessedness.\footnote{Die Lehre von der Gnadenwahl in Frage und Antwort, p. 59 [The Doctrine of Election in Questions and Answers, pg 159]}\end{fancyquotes}

                Walther is well aware that this position is not “\textit{rational}” {\scriptsize\textsc{(vernunft-gemäß)}}.  He acknowledges ever again and again that human reason, freely drawing its conclusions, if it rejects all synergism and does not base conversion or election on any human “\textit{conduct}” as the decisive factor is driven to the denial of universal grace, and on the other hand, if it wishes to maintain universal grace, its conclusions will lead it to synergism.  He says for instance: \begin{fancyquotes}When Holy Scripture teaches that those who are elected are elected by grace alone without any contribution of their own and on the other hand, those who are rejected are rejected on account of their own resistance and unbelief, then reason cannot do otherwise than find a contradiction here.  For it must conclude if one teaches that the basis of damnation lies in man then one must also admit that the basis of salvation and election lies in man; but if one teaches that the basis of salvation lies alone in the grace of God, while the basis of damnation lies alone in man then one must ascribe to God a double self-contradictory will or else surrender the universality of grace and with Calvin assert an absolute election and reprobation; and so only synergism or Calvinism is logically consistent.\end{fancyquotes}  But now Walther makes the demand upon every Christian and every theologian that he refrain from this conclusion however necessary and unavoidable it may seem to reason and believe both grace alone and also universal grace against every objection of reason.  “\textit{As we believe}” says he \begin{displayquote}“\textit{that the Father is true God, the Son is true God, the Holy Ghost is true God and yet in full earnest believe that there is only one God so we also believe that God alone does everything that the saved may be saved and yet at the same time we believe in full earnest that God wills to save all men and that whoever is lost is lost by his own fault on account of his unbelief and obstinate resistance}.”\footnote{February 3, 1882}\end{displayquote} This is also, as Walther shows, the position of the Lutheran Church.  Melanchthon, indeed, and all the synergists who followed him, once taught that not only the cause of rejection but also the cause of election lies in man, that the better conduct of the elect is the reason why they were elected rather than the others.  “\textit{But}”, explains Walther, \begin{fancyquotes}the only right way is that followed by our precious Confessions and those who hold strictly to its example.  They reject on the one hand the opinion \begin{displayquote}‘\textit{that not alone the mercy of God and the most holy merit of Christ, but that also in us there is a cause of God’s election\footnote{etiam aliquid in nobis causa sit electionis divinae} on account of which God has chosen us to eternal life}’\footnote{Mueller, p. 723, par. 88; Triglot, p. 1093};\end{displayquote} on the other hand they likewise reject with greater earnestness the following opinions: \begin{displayquote}\begin{itemize}\item That God is unwilling that all men repent and believe the Gospel. \item Also, that when God calls us to Himself, He is not in earnest that all men should come to Him. \item Also, that God is unwilling that everyone should be saved, but that some, without regard to their sins, from the mere counsel, purpose, and will of God, are ordained to condemnation so that they cannot be saved.\footnote{Mueller, p. 557, par. 17-19; Triglot, p. 837}\end{itemize}\end{displayquote}

                Since both are plainly taught in Scripture, we accept both, whether reason is able ‘\textit{to harmonize them}’\footnote{Mueller, p. 715, par. 53; Triglot, p. 1081} or not.  Regardless of reason’s conclusion that, if there is no cause of election in the elect and the only cause is God’s mercy and Christ’s merit, then the cause that so many do not come to faith and are lost must also lie in God, nevertheless our Confessions and those who follow it do not seek to harmonize this by cheap rationalizations either at the cost of the clear Scriptural doctrine of the universality of grace, or at the cost of the clear Scriptural doctrine of the bondage of the will, but they humbly acknowledge here a mystery insoluble in this life, in accordance with {\scriptsize\textsc{Romans 11:33-36}}, and take their reason captive under the obedience of Christ and His Word. \par As often as they come to the question, \textit{why}, since God must do all, God does not give faith to all men, they indulge in no rational speculations, but refer it to eternal life, where God will reveal this to us, and show us that His grace is nevertheless universal... Thus, then, should all stand who lay claim to be confessionally loyal Lutherans.\footnote{L.u.W., 1880, p. 261-270}\end{fancyquotes}

                Walther calls the road which leads between the errors regarding the doctrine of election “\textit{narrow indeed}”.  Only he can travel this road who has learned in the school of the Holy Ghost to refrain from making deductions which seem to reason to be absolutely necessary.  Hence Walther wrote even before the public outbreak of the controversy over the doctrine of election an article entitled: “\textit{What shall a Christian do when he finds that two doctrines which seem to contradict each other are both clearly and plainly taught in the Scriptures}?”\footnote{L.u.W. 1880, p. 257 ff.}  The answer reads: \begin{displayquote}{\footnotesize Accept both doctrines in simple faith and refrain from all rational deductions.  Thus through a correct doctrine of election the last remnant of rationalism is eliminated from theology.}\end{displayquote}  In an evening lecture {\scriptsize\textsc{(Nov. 10, 1882.)}} Walther spoke of the blessing which has come upon our Church through the recent election controversy.  He remarked: \begin{fancyquotes}Many view it as a misfortune that the controversy on election has broken out.  The good name of our Synod seems to be damaged, the Synodical Conference torn apart, the splendid program of our work has apparently been brought to a standstill.\footnote{Walther's note -- To be sure, only “\textit{apparently}”.  Through the doctrinal controversy, as soon became evident, the work of the Missouri Synod and the synods connected with it was not interrupted.  The Synodical Conference made up with surprising speed also the numerical loss which it suffered through the falling away of the Ohio Synod and the disaffiliation of the Norwegian Synod.}  But we must nevertheless thank God also for this controversy, for now a twofold truth first became fully clear: \begin{itemize}\item Whether people were truly earnest in teaching that man is really saved by grace, \item Whether they were entirely free from rationalism and really regarded God’s Word as their only light in spiritual matters.\end{itemize}\end{fancyquotes}

                As Walther teaches that only he who is free from rationalism can travel the right road in the doctrine of election of grace, so he also again pointed out in the recent controversy that the doctrinal position of the opponents has its basis in their rationalism.  He declares: \begin{displayquote}``\textit{...also our most recent opponents would not teach that election, conversion, and salvation are dependent upon the conduct of the elect if they did not suppose that only in this way they could maintain the universality of grace.  From the same source flows also the fact that they {\scriptsize\textsc{(the opponents)}} raise the charge of Calvinism against those whose only fault is that they allow two doctrines clearly revealed in Scripture to stand side by side without harmonizing them for human reason.}''\end{displayquote}  We add a few more utterances of Walther on this subject, although we have previously dealt very thoroughly with this point.  Walther says: \begin{fancyquotes}Our opponents deal with the doctrine of election as the Jews and the Calvinists have dealt with regard to other articles of faith.  The Jews say it is clearly and plainly written: \begin{displayquote}“\textit{Hear, O Israel, the Lord our God is One Lord!}\footnote{Deuteronomy 6:4}\end{displayquote} -- therefore the doctrine of the Christians that the Father is God, the Son is God, and the Holy Ghost is God, and that each of these three is a distinct Person, must be false, and indeed nothing but heathen polytheism.  The Calvinists say that Christ plainly and distinctly says: \begin{displayquote}“\textit{A spirit has not flesh and bones, as ye see me have}”.\footnote{Luke 24:39}\end{displayquote}-- therefore the doctrine of Lutherans that Christ is omnipresent also according to His human nature must be false, and nothing else than heretical Eutychianism\footnote{Eutychianism refers to a set of Christian theological doctrines derived from the ideas of Eutyches of Constantinople {\scriptsize\textsc{(c. 380 – c. 456)}}. `One formulation is that Eutychianism stressed the unity of Christ's nature to such an extent that Christ's divinity consumed his humanity as the ocean consumes a drop of vinegar.'}...  So our opponents set the doctrine of the particularism of election and the doctrine of the universality of grace or of the divine will of grace in opposition one to the other, whereas both are taught in the Scripture and there is no real contradiction between them.\footnote{L.u.W., 1883, p. 12-14.}\end{fancyquotes}  Walther says further: \begin{fancyquotes}Our opponents throw it up to us that our doctrine is an illogical, inconsistent Calvinism.  But they do not consider that the very essence of Calvinism consists in the fact that it puts the consequences which blind human reason draws from the Scripture on the same level with divine truth.  From the Scripture teaching that he who is saved is saved alone by grace without any cooperation, the Calvinist draws the consequence that he who is not saved for the reason that God did not wish to save him but had predestined him from eternity unto damnation.  From the Scripture teaching that the elect will assuredly be converted and saved, he draws the consequence that the elect are converted by irresistible grace.  From the Scripture teaching that only the elect are saved, he draws the consequence that those who are saved are not saved for that reason that God did not elect them.  From the Scripture teaching that only few are saved, and election is thus particular, he draws the consequence that grace, redemption, and earnest call, the power of the means of grace, are particular.  From the Scripture teaching that faith is a pure gift of God, without any contribution on man’s part, he draws the consequence that God does not wish to bring all men to faith.

                Because we maintain those doctrines of Scripture with the utmost earnestness, but reject and condemn all these rational consequences which are drawn from them, therefore our adversaries ascribe to us an inconsistent Calvinism on forcing it upon us and convicting us of it, yea, even claim that we secretly admit it.\footnote{L.u.W., 1883, p. 14.}\par – What prevents the opponents from understanding our “\textit{good Lutheran statements}” {\scriptsize\textsc{(which they recognize in our writings)}} is nothing else than their delusion, according to which they suppose that when one does not seek and find the cause of election in man, but alone in the grace of God and the most holy merit of Christ, as the Formula of Concord testifies\footnote{Mueller, p. 557, 723; Triglot, p. 837, 1093}, then one is a Calvinist and teaches the absolute predestination of Calvin.\footnote{Beleuchtung, p. 60.}\end{fancyquotes}
%%% Local Variables:
%%% mode: latex
%%% TeX-master: "../main"
%%% End:
