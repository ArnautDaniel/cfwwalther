\chapter{Election and Faith}
\hrule
\vspace{.30cm}
In what relation does the faith, which exists in time in those who are being saved, stand to their eternal election?  Does faith conceptually precede eternal election, so that those being saved must first have come to faith and have persevered in faith before God elected them to eternal life, or is the faith which those who are being saved here in time a consequence and effect of their eternal election?  This question was one of the controverted points in the recent controversy concerning the doctrine of the election of grace.
\vspace{.30cm}
\hrule
\vspace{1.25cm}
                Walther rejects the teaching that faith is to be placed before eternal election and teaches on the contrary that the faith as well as the entire Christian status of the elect in general flows from their eternal election.  Right at the beginning of the controversy he set up the proposition that \begin{displayquote}“\textit{God elected and ordained the elect children of God out of mere grace and mercy and alone for the sake of the most holy merit of Christ already from eternity unto salvation and to all which pertains thereto, thus also unto faith, unto repentance, and unto conversion, before the foundation of the world}”.\end{displayquote}  In this immediate connection he rejects the proposition that \begin{displayquote}“\textit{God in His election took account of anything good in man, namely, of the foreseen conduct of men, of their foreseen non-resistance, and of their foreseen persevering faith, and thus elected certain men in view of, with regard to, upon the basis of, or in consequence of this conduct, this non-resistance, and this faith of theirs, unto salvation}”.\footnote{Der Gnadenwahlslehrstreit, etc. p. 5}\end{displayquote}

                This doctrine concerning the relation of faith to the election of grace Walther demonstrates as the only one based upon Scripture and as the doctrine attested by the Lutheran Confessions and the Lutheran teacher of the Sixteenth Century.

                Above all he urges the Scripture proof.  He says in an evening lecture: \begin{displayquote}“\textit{Important as it is, when a controversy arise within our Church concerning any point of doctrine, that we should consult that which our Church teaches concerning it in her Confessions, it would nevertheless be entirely unlutheran, yes, papistical, if we should desire to base our faith upon the fact that our Church teaches thus or so, and if we would not above all see what the Word of God itself teaches about it.  The Word of God alone makes the heart sure, secure, and happy}”.\end{displayquote}

                After Walther has hereupon advised the students “\textit{to put together a collection of all the passages of Holy Scripture which treat of the election of grace}”, he proceeds \begin{fancyquotes}Simply compare all the passages of Scripture which treat of the election of grace, and you will soon see that according to Holy Scripture the election of grace is an incomprehensible mystery of divine love, grace, and compassion, the eternal source of our salvation, and the impregnable rock upon which our hope of blessedness rests.  As often as Holy Scripture speaks of election, just so often does it have the purpose of showing that God saw nothing in the elect which could have moved Him to elect precisely this person, but that it is pure grace when God has brought a man to Christianity, to faith, to righteousness, and finally to bliss and glory.\end{fancyquotes}  Walther here adduces the passages {\scriptsize\textsc{John 15:16,19; Romans 8:28-30; Ephesians 1:3-6; 2 Timothy 1:9}}, and adds: \begin{fancyquotes}So speaks the great God Himself concerning the election of grace.  Here election is first of all ascribed to divine grace; here, in the second place, all that the believing Christian has, enjoys, and hopes is traced back to this election as the source of all gifts of grace; here, in the third place, every share in bringing about his own salvation is denied to man and this honour is given to God alone; and finally, in the fourth place, nothing is asked of man but to praise and glorify God for this glorious grace.  Here there is nothing, absolutely nothing, said of God’s having considered or regarded anything in man which should have moved God to elect just him to grace and salvation.\end{fancyquotes}

                Thereupon Walther with great diligence furnishes the proof that this and no other is the doctrine attested in the Lutheran Confessions.  In particular he demonstrates that the Formula of Concord knows nothing of an election “\textit{in view of faith}” or of human conduct etc..  He writes in his tract “\textit{Die Lehre von der Gnadenwahl in Frage und Antwort}” as follows: \begin{fancyquotes}According to common knowledge it is currently taught that God has not chosen the elect out of the world and determined to make them His own; though Christ clearly says, {\scriptsize\textsc{John 15:19}},\begin{displayquote} ‘\textit{I have chosen you out of the world, therefore the world hateth you}’.\end{displayquote}  On the contrary is it taught that God first gave attention to how men would conduct themselves, which of them would desert the world and become believing children of God and remain so till the end, and then in consequence of this which God foresaw He chose such men unto sonship, unto sanctification, and unto salvation... Yea, they {\scriptsize\textsc{(the present advocates of this teaching)}} say that this doctrine is found also in our Lutheran Confessional writings, and specifically in the last confessional writing, in the Formula of Concord.  This is based at best on a deplorable self-deception.  There is nowhere even one word in our Lutheran Confessions to the effect that God in consequence of the foreseen persevering faith of certain men, and so in consequence of their foreseen right conduct, chose these same men unto sonship and salvation.  But in the eleventh article of the Formula of Concord the exact opposite is clearly and plainly stated: namely, that on the contrary election is a cause of our salvation and of all that belongs to its attainment, thus also a cause of faith and of conversion, which the Formula of Concord proves also from {\scriptsize\textsc{Acts 13:48}}, where it reads: \begin{displayquote}‘\textit{And as many as were ordained to eternal life believed}’.\end{displayquote}\end{fancyquotes}  The objection that the Formula of Concord here, where it represents eternal election as a cause also of faith, speaks of election in a wider sense is refuted by Walther with the remark that the Formula of Concord here designates the same eternal election as a cause of faith, concerning which it immediately before said that it does not extend over the godly and the wicked, but only over the children of God.\footnote{Die Lehre von der Gnadenwahl, p. 44,45.}  If one should here raise the further objection: \begin{displayquote}{\footnotesize The Formula of Concord says of the election of which it is treating that it is `\textit{a cause which procures...our salvation and what pertains thereto}’.  What then pertains to salvation?}\end{displayquote} Consequently the Formula of Concord is not treating of Dr. Walther’s election in the narrower sense, but of election in the wider sense. – Dr. Walther merely points again to the clear words of the Formula of Concord: \begin{displayquote}“\textit{The eternal election of God...is also from the gracious will and pleasure of God in Christ Jesus, a cause which procures, works, helps, and promotes our salvation and what pertains thereto}”.\end{displayquote}  Walther adds: \begin{fancyquotes}Does he {\scriptsize\textsc{(the one who raises the objection)}} not see that here through the words ‘\textit{in Christ Jesus}’ it is clearly indicated that the subject is not the redemption yet to be secured but the salvation already secured through the redeeming work of Christ, and how election is a cause which promotes the attainment and partaking of the salvation which has already been secured and what pertains thereto?!  As indeed par. 23 of the Formula of Concord\footnote{Mueller, 708; Triglotta, p. 1069} expressly says, God \begin{displayquote}‘\textit{has in grace considered and chosen to salvation each and every person of the elect who are to be saved through Christ, also ordained that in the way just mentioned He will, by His grace, gifts, and efficacy, bring them thereto}’.\footnote{Berichtigung, etc., p. 118)}\end{displayquote}\end{fancyquotes}  The point that the Formula of Concord always speaks only of election in the “\textit{narrower sense}” and calls this election a cause of the faith and the Christian’s status of the elect is expounded by Walther in a special tract under the title: “\textit{Der Gnadenwahlslehrstreit, das ist, einfacher, bewährter Rath für gottselige Christen, welche gern wissen möchten, wer in dem jetzigen Gnadenwahlslehrstreit lutherisch und wer unlutherisch lehre}”.\footnote{The Election Controversy, i.e. simple and tried counsel for godly Christians who would like to know who in the present election controversy is teaching Lutheran doctrine and who is not, St. Louis, Mo. 1881, 15 pages, On Walther’s further line of proof from the Formula of Concord one may compare “\textit{Die Lehre von der Gnadenwahl}”, etc., p. 50, Berichtung, p. 42, p. 76 ff.}

                Since it was claimed that all faithful Lutheran theologians who had entered upon the subject of the relation between faith and election had taught the intuitu fidei, Walther ever and again took occasion to demonstrate that this is not the fact, but that rather this form of teaching first gained entrance into our Church through Aegidius \textbf{Hunnius}.  He frequently cites Chemnitz, who says concerning the relation of the faith of the elect to their eternal election: \begin{displayquote}“\textit{Thus also God’s election does not follow after our faith and righteousness, but goes before as a cause of all this, for whom He foreordained or elected, them He also called and justified, Romans 8}”.\footnote{Beleuchtung, p. 17}\end{displayquote}  The same doctrine concerning the relation of faith to the election of grace Walther points out in Luther, Brenz, Urbanus Rhegius, Cyriacus Spangenberg, Lucas Osiander the elder, Körner, Timotheus Kirchner, etc.\footnote{Cf. Especially Walther’s article “\textit{Dogmengeschichtliches über die Lehre vom Verhältniss des Glaubens zur Gnadenwahl}”, L.u.W., 1880, -. 42 ff. “Berightung”, p. 81.}  After quoting the clear passages from the Formula of Concord and from Chemnitz, Walther makes the demand: \begin{displayquote}“\textit{Accordingly all those who reject the doctrine that faith follows the election of grace and that election precedes faith as a cause should in all honesty concede at least so much, that they are by no means fighting only against Missouri but also against Chemnitz and against the Formula of Concord which was written by him}”.\footnote{Die Lehre von der Gnadenwahl, etc. p. 53.}\end{displayquote}

                What position did Walther take toward the expression \textit{intuitu fidei} in particular?  Walther always took a negative position over against this expression.  Long before the outbreak of the controversy concerning the doctrine of election Walther calls \textit{intuitu fidei} an “\textit{unfortunately chosen terminology}”\footnote{L.u.W., 1872, p. 184}, which “\textit{strictly interpreted confirms an error which the {\scriptsize\textsc{(old)}} theologians themselves abhorred}”.\footnote{L.u.W., 1872, p. 139} -- the error, namely,\begin{displayquote} ”\textit{that the elect are chosen for the sake of their faith, that the faith of man is the ground, the cause, the condition of his election to salvation}”.\footnote{l.c., p. 132.}\end{displayquote}  Hence Walther does not wish to use the expression of the later theologians but to return to the simplicity of the Formula of Concord.  He writes, after a lengthy discussion of the efforts of the old theologians to bring the expression into harmony with the analogy of faith: \begin{displayquote}“\textit{We indeed believe that we can most easily avoid all the misunderstanding which is thereby so easily called forth if we entirely refrain from the new terminology of the dogmaticians of the 17th century and return to the simplicity of the Formula of Concord, which refuses to solve the mystery that results in this connection}”.\footnote{L.c., p. 140}\end{displayquote}  So Walther views the \textit{intuitu fidei} as an innovation in the Lutheran Church, as an unfortunate attempt at a “\textit{development of the doctrine of the Formula of Concord}”.\footnote{L.c., p. 193}  Accordingly it was entirely incorrect when it was asserted at the time of the controversy that Walther himself before the outbreak of the controversy had taught an election “\textit{in view of faith}”.\footnote{ It is difficult for Walther to have to confess that in this point he cannot agree with the teachers of the 17th Century whom he esteemed so highly.  He said in this very connection \begin{displayquote}“Nothing is more pleasing and delightful to us than to be able to agree with our fathers not only in faith, but also in expression, and nothing is farther from our desire than to part without urgent necessity  even in phrasibus from our old dogmaticians”. {\scriptsize\textsc{(L.u.W., 1872, p. 141.)}}\end{displayquote} Hence Walther is satisfied even with that expression of the old dogmaticians according to which God elected those concerning whom He foresaw that they would believe.  This expression which the dogmaticians of the 17th Century use promiscuously with \textit{intuitu fidei}, Walther understands as a description of the elect in the sense that only such are elect who in time come to faith in Christ and persevere in this faith.  He uses this expression in the sense of the well known utterance of Urbanus \textbf{Rhegius}: \begin{displayquote}“He who is ordained unto eternal life believes the Gospel and amends his life, for God calls him in His own time; one in youth, another in old age, according to His will; no elect person remains to the end in unbelief and sinful life... Even as God ordained Peter, Paul, and us other Christians unto salvation, so He has also foreordained and predestined their conversion, their Christian conversation, repentance, and good works, in which they must walk and attest their call and faith”.\end{displayquote}  But in “\textit{in view of their faith}” Walther will not use, because he cannot understand the expression otherwise than that thereby a ground of election is sought in man. {\scriptsize\textsc{(Beleuchtung, p. 32 f.)}}}

                Another charge was also repeatedly made against Walther with more apparent justification, the charge, namely, that Walther in his judgment of the old Lutheran theologians who used the \textit{intuitu fidei} and of the more recent champions of this expression {\scriptsize\textsc{(Iowa and Ohio)}} measured by a different standard.  While he would not condemn the old theologians as errorists on account of the expression \textit{intuitu fidei}, he did just this in the case of the Iowan and Ohioan spokesmen.  It is true that Walther did make this difference between the old and the aforenamed new proponents of the \textit{intuitu fidei}.  But he also gave his reason for doing so.  He judges the theologians of the 17th Century so mildly because they, in spite of their maintaining the “\etxtit{unfortunately chosen terminology}”, at the same time testified “\textit{that God in His election regarded nothing in man, but elected only from grace and compassion}”.\footnote{L.u.W., 1872, p. 134}
                Walther adduces in this connection the following declarations of Gerhard and Quenstedt. \par \textbf{Gerhard} writes: \begin{displayquote}``\textit{By no merits of man, by no worthiness of the human race, also not by His foreknowledge of good works or of faith was God moved to elect certain ones unto spiritual life, but this is to be ascribed entirely to His undeserved and unmeasurable grace}”.\end{displayquote}  The same Gerhard says further: \begin{displayquote}“\textit{We confess with loud voice that we hold that God found nothing good in the man to be elected unto eternal life, and that He had regard neither to good works, nor to the use of the free will, nor even to faith itself as though He should have been moved thereby or on this account to elect certain ones; but we say that only and alone the merit of Christ is that to the worthiness of which God had regard, and that He formed the counsel of election from pure grace alone}”.\end{displayquote}  So also speaks \textbf{Quenstedt}: \begin{displayquote}“\textit{God elected us not according to our works but from pure grace.  Also faith itself does not belong here, when it is looked upon as a condition, more or less worthy, whether in and of itself, or by virtue of its being counted worthy by the will of God.  Nothing of all this had any influence upon God’s choice, whether as a moving or impelling cause for His forming such a counsel, but it is to be ascribed only and alone to His grace, as the sainted Hülsemann writes}”.\footnote{L.u.W., 1872, p. 132, 133.}\end{displayquote}  In consideration of this position of the old theologians, whereby they in fact virtually retract again the \textit{intuitu fidei}, Walther is unwilling in spite of their retaining that expression, to designate them as errorists.  But he is compelled to regard the position of the Iowan and Ohioan spokesmen quite differently.  These say that the reason why some are converted and saved and elected unto eternal life rather than others lies in man.  Ohio says expressly that conversion and salvation depends not only upon the grace of God, but also in a certain sense upon the conduct of man.  Thus Walther saw in the spokesmen of the Synods of Iowa and Ohio people who covered up their synergism with the \textit{intuitu fidei} and used an incorrect expression employed by orthodox theologians to falsify the entire Christian doctrine, the doctrine, namely, that we are saved alone by grace for Christ’s sake.  Hence the differing judgement.\footnote{Cp. In this connection L.u.W., 1872, p. 325 f.  “Beleuchtung”, p. 13 ff.. “Berichtigung”, p. 29 ff., p. 149.}
%%% Local Variables:
%%% mode: latex
%%% TeX-master: "../main"
%%% End:
