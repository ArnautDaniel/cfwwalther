\chapter{Election -- Wide or Narrow}
\hrule
\vspace{.30cm}
We desired to conclude our presentation of Walther's doctrine concerning the election of grace by a more careful investigation of certain particularly important points. We have already offered a detailed treatment of two such points, namely, of the real center of Walther's doctrine and of the relation of faith to election.
\vspace{.30cm}
\hrule
\vspace{1.25cm}
A third point which deserves special consideration is the question whether there is such a thing as an election in a wider and in a narrower sense. In particular was the question debated in the recent doctrinal controversy whether the Lutheran Confessions in the Eleventh Article of the Formula of Concord treats of an election in a wider sense.

Already the later Lutheran dogmaticians assert that the Formula of Concord treats of an election in a wider sense. It is easy to understand how they came to make this assertion. The Formula of Concord certainly teaches that election is not merely an ordination to salvation but also to all which belongs to the attainment of salvation on the basis of Christ's merit, to the call, to conversion, to faith, to justification, to sanctification, to preservation in faith. All these particulars are expressly mentioned by the Formula of Concord as a consequence and effect of election\footnote{Solid. Decl. XI, par. 8, 44, 45ff.}.  This does not fit in with the doctrine adopted by the later dogmaticians that election took place "\textit{in view of persevering faith {\scriptsize\textsc{(intuitu fidei finalis)}}};" for according to this doctrine of the dogmaticians the objects of election are such persons as have already, according to God's foreknowledge on which election is supposed to be based, the entire way of salvation from conversion till the last breath of their earthly life behind them as a happily accomplished fact. In order not to place themselves in open opposition to the Confession, they usually say that the Confession uses the word election in a wider sense, in which connection they indeed overlook the fact that the Formula of Concord very emphatically protests against such an idea by explaining from the start that it is speaking of an election which "\textit{does not extend at once over the godly and the wicked, but only ever the children of God, who were elected and ordained to eternal life before the foundation of the world was laid}"\footnote{XI, par. 5}. With the closer investigation of the question whether the alleged "\textit{wider sense}" of the Formula of Concord is founded in Scripture the dogmaticians under discussion concern themselves but little; only in isolated cases do we find the direct charge that the Formula of Concord has an unbiblical concept of election\footnote{(So Caspar Loescher, whose statement is cited by Walther, Berichtigung, etc., p. 77)}. Quenstedt, on the other hand, is satisfied with the declaration that his concept of the election of grace, differing as it does from that of the Formula of Concord, is the only correct one\footnote{Theol.-did.-pol. III, 89}. Walther speaks in more detail concerning the exposition of the Formula of Concord on the part of the later dogmaticians in his \textit{Berichtigung, p. 76ff}.\footnote{Cf. also L.u.W., 26, 68, 167}

Walther himself teaches: \begin{displayquote}``\textit{There is an election of grace in only one sense, and that is the sense which is presented by the Formula of Concord on the basis of Scripture. That is the election of grace which extends not over all men but only over the children of God who are being saved, and which is not merely an ordination to the termination of the way of salvation, to blessedness, but also a cause of the entire Christian status through which God leads the elect unto eternal life.}\footnote{L.u.W., 26, 292f., 26, 72, 135f., 161f., 165, 166, 355}\end{displayquote}

In reply to the appeal to the so-called "\textit{eight points}"\footnote{Solid.Decl. XI, par. 15-22} as proof that the Formula of Concord teaches an election in a wider sense, Walther says: \begin{displayquote}"\textit{The eight points are adduced, inasmuch as God leads the elect to salvation on no other path and in no other order than He is willing to lead all men to salvation}."\end{displayquote} Or: \begin{displayquote}"\textit{In the eight points 'the manner is declared' in which God wants to bring the elect to salvation, aid, promote, strengthen, and preserve them}."\end{displayquote} That the eight points arc not to be understood in any other way Walther finds expressly testified in the Formula of Concord itself, namely, in the words preceding and following the eight points. He points out that only in this way is one guarded against the supposition, so discreditable to our Confession of our Church, that it right at the start defines election as something which has reference only to those who are being saved, and soon thereafter, without giving any indication to this effect, gives a wider sense to the word election.\footnote{Beleuchtung, p. 64ff., L.u.W., 26, p. 298ff}

Against this doctrine, that election extends only over the elect children of God and is the cause of their faith, the charge has been raised that it overthrows the universal will of grace, or – which amounts to the same thing, – that through this doctrine of election a special way of salvation for the elect is posited outside and apart from the universal way of salvation. We here come to a fourth controverted point, namely, how the doctrine of the election of grace is related to the universal way of salvation or to the universal will of grace.

In almost innumerable variations during the last ten to twelve years it has been objected against the doctrine – that the election of grace which extends only over the children of God is a cause of their {\scriptsize\textsc{(the elect's)}} conversion and perseverance in faith – that then there would be two ways of salvation, one for the elect, who obtain faith and salvation in consequence of their eternal election, and another for the rest of mankind which lacks the power to effect and preserve faith. According to the doctrine of election propounded by Walther, God is supposed to have "\textit{so arranged it}" by His election that the majority of men could not come to faith or at least could not remain in faith. – There is no doubt that through this objection many simple souls have been and are still being predisposed against the Scriptural and confessional doctrine of election. It did not help Dr. Walther at all that he ever and again unceasingly declared: \begin{fancyquotes}We believe, teach, and confess that no man is lost because God would not save him, or because God with His grace passed him by, or because He did not offer the grace of perseverance to him also and would not bestow it upon him; but that all men who are lost perish by their own fault, namely on account of their unbelief, and because they have obstinately resisted the Word and grace of God to the end, of which 'contempt for the Word of God the cause is not God's foreknowledge\footnote{vel praescientia vel praedestinatio}, but the perverse will of man, which rejects or perverts the means and instrument of the Holy Ghost, which God offers him through the call, and resists the Holy Ghost, who wishes to be efficacious, and works through the Word, as Christ says: \begin{displayquote}`\textit{How often would I have gathered you together, and ye would not!}'\footnote{Matthew 23:37}\end{displayquote} Hence we heartily reject and condemn the contrary Calvinistic doctrine.\footnote{The fourth of the 13 Theses, Lutheraner, 1880}\end{fancyquotes}  This protestation, as we have said, did not help Dr. Walther at all. In spite of it, many stuck by the assertion that Walther, resp. the Missourians, taught a double way of salvation. The universal will of grace and the doctrine that election is a cause of faith and of the entire Christian status of the elect cannot, said they, stand side by side. According to them the analogy of faith demands the surrender of this doctrine of the election of grace.

Over against this argumentation Walther first of all guards the correct principle. He calls attention to the fact that also the Calvinists appealed to the analogy of faith against the Lutheran doctrine of the Lord's Supper and claimed that the essential presence of the body and blood of Christ in the Lord's Supper, as taught by the Lutherans, conflicts with the truth clearly attested in Scripture that Christ's body is a true human body. But against this the Lutherans always asserted: \begin{displayquote}"\textit{The Scripture teaches both: that Christ's body is a true human body and that it is nevertheless truly distributed in the Lord's Supper; hence both must be believed and the one must not be placed in opposition to the other}."\end{displayquote} And so Walther demands that also in connection with the question whether the doctrine of universal grace, according to which God earnestly desires to save all men, and the doctrine of particular election, which is a cause of the faith and the entire state of grace of the elect, harmonize with one another, – that in this question the Scripture principle be held fast. The only question is whether the Scripture does not teach, just as clearly as it teaches universal grace, also this doctrine, that election pertains only to those who are being saved and is the cause of their faith and of their entire Christian status. This Walther teaches and most emphatically rejects the assertion that the passages of Scripture which treat of the election of the saved are obscure and hard to understand. Accordingly he demands: \begin{displayquote}``\textit{Both must be believed by one who wants to be a Christian and even an orthodox Lutheran. To correct one Scripture doctrine by another for the sake of one's reason, because the former seems obscure and contradictory, yea, entirely to cancel it on the pretext that obscure passages must be interpreted according to the clear, – this is a terrible abomination}.''\footnote{Beleuchtung, p.25ff., L.u.W., 29, 12ff.; 26, 264-270}\end{displayquote}

But after Walther has guarded the correct principle, he also demonstrates, through a closer investigation of the matter itself, that two different ways of salvation simply do not result from the doctrine that election is a cause of the faith and salvation of the elect. He shows: God leads the elect upon no other way of salvation than that upon which He earnestly wills to lead all men.\footnote{L.u.W., 26:296} The elect are by grace alone, for Christ's sake, through the Gospel, called, enlightened, sanctified, and preserved in time, and God has from eternity determined to deal with the elect on this basis and in this manner\footnote{L.u.W., 26:367}. Both the eternal counsel of election and also the execution of it in time correspond exactly to the universal way of salvation. Walther rejects as false the teaching of the Calvinists, that God has first elected to salvation in an absolute manner and then subsequently determined to redeem the elect through Christ and to endow them with faith. He writes: \begin{fancyquotes}We believe, teach, and confess that God did not first unconditionally and absolutely choose the elect unto salvation, as the Calvinists say, and then subsequently determine to give them faith as the means to obtaining salvation, but that God has at the same time elected them to all '\textit{which},' as our Confession says, '\textit{procures, works, helps, and promotes our salvation and what pertains thereto}' and so also indeed, and before all, to faith; as the Formula of Concord expressly says when it cites as proof from words just quoted, the text {\scriptsize\textsc{Acts 13:48}}, \begin{displayquote}`\textit{And as many as were ordained to eternal life believed}.'\end{displayquote} Hence we also believe, teach, and confess that according to God's Word the just God could not elect any man to salvation in an absolute manner, i.e. to say, if God had not first provided for his redemption and if He had not at the same time elected him to faith, i.e., if He had not at the same time determined to give him faith, for aside from Christ there is no salvation \footnote{Acts 4:12} and `\textit{without faith it is impossible to please God}'\footnote{Hebrews 11:6}. Hence when the Calvinists want nothing to do with an election '\textit{in view of faith},' that means something entirety different than when we reject this teaching. The Calvinists do this, as we said, because according to their teaching God has first elected to salvation in an absolute manner without regard to Christ or faith; we do this because God's Word teaches that God has decided to give us by grace not only salvation, but also faith, since the election unto salvation and unto faith coincides.\end{fancyquotes}  Walther therefore declares it to be a gross perversion of his doctrine when any one asserts that by it faith is excluded from the election of grace, and proceeds: \begin{displayquote}"\textit{We on our part rather regard faith as so necessary to salvation that we believe, teach, and confess that God, according to {\scriptsize\textsc{Romans 8:29-30}}, chose the elect first unto the Gospel call and thereby unto faith {\scriptsize\textsc{(not according to temporal sequence but according to the nature of the matter)}} and unto justification, and then unto salvation}."\footnote{Beleuchtung, p.19, 20}\end{displayquote}

In order further to evince that through the doctrine of eternal election as a cause of the faith and salvation of the elect no special way of salvation for the elect is posited, Walther ever and again points out that in connection with this doctrine of election we are confronted with no other mystery and no other difficulty than that which meets us in the doctrine of conversion and in general in the contemplation of the universal way of salvation in itself. If, e.g., human reason is allowed to draw its so-called necessary consequences, it concludes: \begin{displayquote}{\footnotesize If grace alone is the cause of faith and of preservation in faith, as Scripture testifies, and if nevertheless only a part of the human race lying in the same total depravity is converted and preserved in faith, then it is evident that in the case of the rest of mankind this grace has been either not at all or not sufficiently efficacious; there is, in spite of all the assurances of Scripture that God would have all men to be saved, no such thing as universal grace.}\end{displayquote}  Thus rationalizing human reason arrives at a double salvation from the premise of the simple concept of grace. Hence also the assertion of the modern rationalistic, synergistic theologians that the Formula of Concord would indeed fall into Calvinism if it should let actual faith be worked by the Holy Ghost\footnote{Cf. here the discussions in L.u.W., 1890, 349ff}. The Lutheran Church, on the other hand, in the clear knowledge that such conclusions are irrelevant deductions of reason, holds fast to the one revealed way of salvation, and says:\begin{displayquote}{\footnotesize  It is only and alone the work of God's grace that men are converted and saved, and it lies only and alone in the wicked obstinate resistance of man, and not in any lack of the gracious working of God in His Word, that men are not converted and saved {\scriptsize\textsc{(Hosea 13:9)}}.}\end{displayquote} The former, namely, the fact that those who are saved come to faith and are preserved in faith by grace alone, the Scripture traces back into eternity. Scripture says that God not only in time gives faith to those who are being saved and preserves it, but that He has already from eternity determined to do this for them. That is the election of grace. Therefore as little as one can raise the objection against the doctrine that God brings the saved to faith and preserves them in faith by grace alone, that thereby a double way of salvation is posited, so little can one raise this objection when the same effect is attributed to the election of grace, for the election of grace is nothing else than eternal grace viewed in relation to those who are saved. Here belong such utterances of Walther as the following: \begin{fancyquotes}If you, dear reader, are already by the grace of God standing in living faith, then let me further ask you; Did you perhaps give yourself faith?  –You will say: \begin{displayquote}{\footnotesize Ah, no; I could not do even the least thing toward my receiving through the Word of the Gospel a living faith, and I did not come to the Word, but the Word came to me.}\end{displayquote} -- Well!  Do you suppose then that you have just accidentally come to faith? –  You will doubtless answer: \begin{displayquote}{\footnotesize Ah, No; if I thought that I would indeed have to be a mere heathen; nothing takes place by chance.}\end{displayquote} – Well then; let me further ask you: To whom then do you owe it that you have through the Word of God come to faith? – You say: \begin{displayquote}{\footnotesize That I owe alone to the mercy of God and the most holy merit of Christ. It was God who opened my fast closed heart, as He did once for Lydia, that I gave attention to what I read and heard out of God's Word. I certainly did not deserve that in any way! On account of my many sins I rather deserved that God should neither have called me nor brought me to faith, but rather that He should have let me die and perish in my sins. My conversion is a mystery to me. Only so much I know, that I did nothing toward it.}\end{displayquote} – Do you suppose then that God first in time thought of bringing you to faith? then first, when your eyes were opened, and you recognized your wretchedness in sin and God's grace in Christ, came to faith, and became a different man? – you will say: \begin{displayquote}{\footnotesize How could I think that? For I know from God's Word that God has not only foreknown from eternity all the good which He does in time, but has also from eternity predetermined it.}\end{displayquote} – Let me then ask you just one more question: Do you also hope to be saved? – You will answer: \begin{displayquote}{\footnotesize Yes, such is my hope. If I did not hope that I would have to reject Luther's '\textit{Christian Questions};' indeed then I could not even with the entire holy Christian Church recite the Third Article in firm faith, in which it says: '\textit{I believe .... the life everlasting},' and I could not say with our Catechism: '\textit{I believe... that God will give unto me and all believers in Christ eternal life. This is most certainly true}.' And my dear Lord Jesus Christ says: \begin{displayquote} `\textit{My sheep hear My voice, and I know them, and they follow Me: and I give unto them eternal life; and they shall never perish, neither shall any man pluck them out of My hand}'.\footnote{John 10:27-28}\end{displayquote} So how could I doubt my salvation?}\end{displayquote} Just so, dear reader, – Behold, there have in brief words the entire doctrine of election that, and nothing else, is what the Formula of Concord teaches concerning election and what we teach with it.\footnote{Lehre von der Gnadenwahl, p. 58f.}\end{fancyquotes}

Permit us to add: \begin{displayquote}{\footnotesize In case anyone really holds with the Lutheran Church to both propositions, that the unbelief and damnation of those who are lost is to be traced alone to the obstinate resistance of man, while the faith and salvation of those who are saved is to be traced alone to the working of God's grace, then it can only be the result of intellectual confusion if such an one still claims that a double way of salvation is posited when it is said of the election of grace or of eternal grace that it stands in a causal relation to the faith and the entire Christian status of the saved.}\end{displayquote} To be sure, he who teaches that conversion and salvation do not depend alone upon the grace of God, but in a certain respect also upon the conduct of man, cannot do otherwise than regard the doctrine that the election of grace is the cause of the faith and preservation of the elect as a falsification of the universal way of salvation. For by this doctrine of election – to speak with the Formula of Concord – \begin{displayquote}``\textit{all opinions and erroneous doctrines concerning the powers of our natural will are overthrown, because God in His counsel, before the time of the world, decided and ordained that He Himself, by the power of His Holy Ghost, would produce and work in us, through the Word, everything that pertains to our conversion}".\footnote{Sol. Decl., XI, par. 44. Mueller, p. 714; Trigl. p. 1077}\end{displayquote}  The fact that our opponents see in this a falsification of the universal way of salvation or a special way of salvation for the elect aside and apart from the universal way of salvation is due to their holding in general a false doctrine of the universal way of salvation, specifically to their harboring the gross delusion that according to the universal way of salvation conversion and salvation does not depend upon the grace of God alone, but also upon the conduct of man, and that therefore a special way of salvation is posited for the elect when their conversion and election is made to depend not upon their conduct but alone upon the grace – the eternal grace – of God. As a matter of fact, the real situation is this: according to the universal way of salvation conversion and salvation depends upon the grace of God alone, and not – even in the thousandth part – upon the conduct of man, and according to the election of grace the case is not otherwise but exactly the same. By the election of grace according to which God has from eternity "\textit{not only before we had done anything good, but also before we were born}"\footnote{F.C., par. 88; Mueller, p. 723; Trigl. p. 1093}, endowed us with conversion, righteousness, and salvation\footnote{F.C. par. 45; Mueller, p. 714; Trigl. p. 1079}, the pure grace of God is only brought more clearly to light: "\textit{it establishes}," as the Formula of Concord says - \begin{displayquote}"\textit{very effectually the article that we are justified and saved without all works and merits of ours, purely out of grace alone, for Christ's sake}".\footnote{Formula of Concord XI, par. 43; Mueller, p. 713; Trigl. p. 1077}\end{displayquote}

The charge against Dr. Walther and the Missouri Synod, that with their doctrine of election, specifically with the doctrine that election is a cause of conversion and salvation, they assure a double way of salvation, will be silenced on the part of that sector of our opponents which knows what it wants only when {\scriptsize\textsc{(the opponents)}} have given up their false doctrine of the universal way of salvation.
\section{A Letter of Note} The confusion and delusion into which the leaders of the Iowa and Ohio Synods have driven the ignorant among their pastors, especially by the charge that Missouri teaches a double way of salvation, is evident from a document which has come into our hands. A pastor of the Iowa Synod, who in fighting "\textit{the Missourians}" in the State of Wisconsin, sent to a member of a congregation in Waushara County a writing in which we read: \begin{displayquote}``Dear Friend! On the third of July a man came to us who said that you desire from me proofs that the Missouri Synod in its writings teaches that God through His election is Himself to blame for the loss of part of mankind.\end{displayquote} \begin{displayquote}``Dear Friend, I assure you that this is undeniably evident from their writings, for in the fashion in which the Missourians teach the election of grace, there is nothing left for one portion of mankind than that they must through God's election go to hell; if they, the Missourians, do not directly say this, yet indirectly, that is, that it indisputably follows from their doctrine.-- But let us once again look at their statements concerning election is they stand in their own writings. -- I have in my heads a booklet by Pastor W. concerning the Missourian doctrine of election. I suppose that you have such a book also.\footnote{What he means is the tract of Dr. Walther:  The Doctrine Concerning Election Presented in Questions and Answers from the Eleventh Article of the Formula of Concord of the Evangelical Lutheran Church.} Now let us look first at Question No. 10, and especially at its answer. This reads thus: \begin{displayquote}`The eternal election of God not only foresees and foreknows the salvation of the elect, but is also, from the gracious will and pleasure of God in Christ Jesus, a cause which procures, works, helps, and promotes our salvation and what pertains thereto.'\end{displayquote} To this I tell you, my dear friend, if already the election of grace should now be the cause of my salvation, and that it procures and works everything, then I say that is false. Christ's merit and faith in it is the cause and all ground of my salvation, that is the right doctrine, no other. -- Let us look at Question and answer No. 11. Question: \begin{displayquote}`Is it then so important that the eternal election of God is a cause of our salvation and that it procures, works, helps, and promotes all that pertains thereto?' \end{displayquote} Answer: \begin{displayquote}`Yes, indeed, for upon this our salvation is founded that the gates of hell cannot prevail against it.'\end{displayquote} To this we say, it is important that we hear God's Word, believe on Christ, do not grieve the Holy Ghost, pray and work, then the Lord will by grace take us to Himself in heaven. That is important, very important. If the Missourians say the gates of hell cannot overthrow the election of grace and if my salvation is dependent upon it, then salvation can no more be lost to me. This is again false.''\end{displayquote} So far the writing. The pastor had apparently no idea that he had judged so severely the \textit{ipsissima verba} of the Formula of Concord which the leaders of our opposition might well take as a pattern for themselves.} \footnote{Translator's Note: The lame sentence structure and poor choice of words in the above quoted German letter is deliberately imitated by the translator. W.H.M.}
%%% Local Variables:
%%% mode: latex
%%% TeX-master: "../main"
%%% End:
