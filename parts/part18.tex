\chapter{Election -- Assurance of Faith}
\hrule
\vspace{.30cm}   One of the questions which was thoroughly discussed in the controversy on election was whether or not a Christian can and should be assured in faith of his election to salvation.  \par For Walther this point is “\textit{one of the most important}”.\footnote{Lutheraner, 1880, p. 25}
\vspace{.30cm}
\hrule
\vspace{1.25cm}
                The modern representatives of the intuitu fidei in decided disagreement with most of the later Lutheran theologians denied that a believing Christian can and should be sure of his election.  They gave voice to such utterances as the following: \begin{displayquote} “\textit{Whether I am indeed elected even in the stricter sense I do not know.  I should believe and hope that I am.}\end{displayquote}  The Christians find themselves “\textit{from day to day between fear and hope as on trial between two millstones}”.\footnote{Berichtigung, p. 120, 121.}

                Walther points out ever and again that his opponents in accordance with the nature of their doctrine cannot do otherwise than deny Christians the assurance concerning their election.  Uncertainty concerning is a necessary consequence of the synergism harbored by the opponents.  If election does not depend on the grace of God in Christ alone, but also upon the conduct of man, then the Christian will have to doubt until his death whether he is elect, since no man can know whether he will conduct himself rightly in the future.  We adduce a few statements of Walther concerning this point.  He writes: \begin{displayquote}“\textit{The sect of the Armenians teaches that man is converted through his own cooperation with grace by means of his own decision, and hence naturally teaches also that man must harbor doubt as to his election until his death, since he cannot know how he will conduct himself in the future}”.\footnote{Lutheraner, 1880, p. 27, 28.}\end{displayquote}  Applying this to his opponents, Walther says: \begin{fancyquotes} Since they {\scriptsize\textsc{(the opponents)}} teach that God in connection with election has followed the ‘\textit{rule}’, to elect those of whom He foresaw that they would conduct themselves rightly and remain faithful unto death, and since they naturally cannot know by their own reason and strength whether they will conduct themselves rightly in the future also and remain faithful unto death, they find themselves, as “\textit{Altes und Neues}”\footnote{Vol I. p. 10, so clearly says, ‘\textit{from day to day between fear and hope, as on trial between two millstones}’!} \footnote{Berichtigung, p. 120 f.}\end{fancyquotes}  Walther writes even more fully on the same point: \begin{fancyquotes}The doctrine of uncertainty of election and salvation is something entirely unheard of in the Lutheran Church.  But our opponents, with their doctrine of election, cannot do otherwise than deny all certainty of these on the part of the elect, just as the papists do.  The Lutheran Confession says that election is a cause of salvation and of all that pertains thereto; our opponents say that the attainment of salvation is on the contrary a sort of cause of election.  The Lutheran Confession teaches that faith which perseveres to the end depends upon election; our opponents say that election depends upon faith which perseveres to the end.  The Lutheran Confession most earnestly rejects the teaching that there is also in man a cause of election; our opponents say that God’s rule in connection with election is the conduct of man; that election took place in consequence of faith, the nonresistance of man, his permitting himself to be converted. \par According to the doctrine of our opponents, therefore, it is entirely impossible that a man should without a special divine revelation be sure of his salvation and election; for since according to this teaching his salvation rests in his own hands, in his perseverance, and he must admit that he could easily stumble, fall, and forever fall away, he has nothing which could make him sure of his salvation, and must therefore necessarily be in doubt.\footnote{Abendvorlesungen [Evening lecture], June 10, 1881}\end{fancyquotes}

                “\textit{Our opponents do indeed declare}”, says Walther in the same connection, \begin{displayquote}“\textit{that a believing Christian can and should indeed have conditional certainty.  But a conditional certainty is simply no certainty.  Or let them say themselves whether that is certainty when a general is sure of victory on the condition that he will defeat his enemies?….. It is evident that to hold something of this sort to be certainty is simply ridiculous}”\footnote{Abendvorlesungen [Evening lecture], June 10, 1881.}\end{displayquote}

                Now Walther himself teaches that a Christian can and should be sure of his election and salvation in faith.  Since election depends only upon God’s grace in Christ, therefore the Christians can and should in faith recognize his election from the Gospel, which reveals and assures to him the grace of God in Christ.  The teaching of the uncertainty of election and salvation is to Walther in itself a distinctive mark of false doctrine.  He calls this teaching, in the words already quoted above: “\textit{something entirely unheard of in the Lutheran Church}”.  He writes: \begin{fancyquotes}By this denial of the certainty of salvation the teaching of our adversaries is already condemned, even if there were nothing else against it.  For that a Christian should and may be sure of his election and salvation Holy Scripture teaches in innumerable places.  As often as the believers are called blessed in Holy Scripture, so often does Scripture preach this certainty and summon the believers not to doubt their coming salvation, however sad at heart they may be; hence Paul says: \begin{displayquote}‘\textit{we are saved by hope}’,\end{displayquote} whereby he testifies that the blessedness of the believers in this life does not consist in their already enjoying, feeling, and perceiving it, but in their hoping for it, that is in awaiting it with assurance.  For Christian hope is nothing else than assured faith directed toward that which is to come.\end{fancyquotes}  In a more thorough discussion of this point Walther says: \begin{fancyquotes}Prof. S. indeed says that the assurance of his election, which he in the second thesis quite correctly calls an ‘\textit{assurance of faith}’ has no foundation in Holy Scripture; but thereby he contradicts a good many Scripture passages which are as clear as daylight.  Christ says to the seventy disciples: \begin{displayquote}{\footnotesize ‘In this rejoice not, that the spirits are subject unto you; but rather rejoice, because your names are written in heaven’.}\footnote{Luke 10:20}\end{displayquote} To the apostles the Lord says: \begin{displayquote}{\footnotesize ‘Ye have not chosen Me..., but I have chosen you’}.\footnote{John 15:16}\end{displayquote} -- and soon after: \begin{displayquote}{\footnotesize ‘Because ye are not of the world, but I have chosen you out of the world, therefore the world hateth you’.}\footnote{John 15:19}\end{displayquote}  Hence, following Christ in this, also the holy Apostles comfort the believers in their congregations with the fact that they are elect.  After St. Paul among other things has treated the doctrine of election, he continues: \begin{displayquote}{\footnotesize ‘Who shall lay anything to the charge of God’s elect? – Who shall separate us from the love of Christ? Shall tribulation, or distress, or persecution, or famine, or nakedness, or peril, or sword?  As it is written, For Thy sake we are killed all the day long; we are accounted as sheep for the slaughter.  Nay, in all these things we are more than conquerors through Him that loved us.  For I am persuaded, that neither death, nor life, nor angels, nor principalities, nor powers, nor things present, nor things to come, nor height, nor depth, nor any other creature, shall be able to separate us from the love of God, which is in Christ Jesus our Lord’.}\footnote{Romans 8:33, 35-39}\end{displayquote}  So, moreover, the same Apostle assures the believers at Ephesus: \begin{displayquote}{\footnotesize ‘According as He hath chosen us in Him before the foundation of the world’.}\footnote{Ephesians 1:4}\end{displayquote}  Further to the believing Thessalonians:\begin{displayquote}{\footnotesize  ‘Knowing, brethren beloved, your election of God’.}\footnote{1 Thessalonians 1:4}\end{displayquote}  He further declares to them: \begin{displayquote}{\footnotesize ‘But we are bound to give thanks always to God for you, brethren, beloved of the Lord, because God hath from the beginning chosen you to salvation’.}\footnote{2 Thessalonians 2:13}\end{displayquote}  To the believing Colossians he writes: \begin{displayquote}{\footnotesize ‘Put on therefore, as the elect of God, holy and beloved, bowels of mercies’, etc.}\footnote{Colossians 3:12}\end{displayquote}  Moreover Peter in the first chapter of his First Epistle greets the believers to whom he writes with the words: \begin{displayquote}{\footnotesize ‘Peter, an apostle of Jesus Christ, to the elect strangers’.}\footnote{1 Peter 1:1,2}\end{displayquote} -- and testifies to them in the second chapter: ‘\textit{Ye are a chosen generation}’.  Who now dares to assert that these are all mere empty assurances, in which the believers could and should not take comfort in faith?  And we are here passing by all the passages in which salvation is promised to the believers and they are assured of its certainty; when, e.g., to cite only this one, the Lord says: \begin{displayquote}{\footnotesize ‘My sheep hear My voice, and I know them, and they follow Me; and I give unto them eternal life; and they shall never perish, neither shall any man pluck them out of My hand.'}\footnote{John 10: 27-28}\end{displayquote}\end{fancyquotes}
\divider \begin{fancyquotes}
                And the teaching that ‘\textit{no believer can be sure of his salvation}’ is as contrary to the Confession as it is contrary to the Bible.  For thus we read there: \begin{displayquote}{\footnotesize ‘This also belongs to the further explanation and salutary use of the doctrine concerning God’s foreknowledge to salvation: Since only the elect, whose names are written in the book of life, are saved, how can we know, whence and whereby we can perceive who are the elect that can and should receive this doctrine for comfort’.}\footnote{Art. XI, par. 25, Mueller, p. 709, Triglotta p. 1071.}\end{displayquote} -- Hence also against this error of Dr. S., that no believer should or can be sure of his election, and thus also of his salvation, we must herewith publicly and solemnly protest.  Otherwise we could no longer say with the whole Christian Church of all ages: ‘\textit{I believe the everlasting life}’, and could no longer teach our dear Christians, and even our children, to confess with the entire orthodox Evangelical Church: \begin{displayquote}{\footnotesize ‘In which Christian Church He daily and richly forgives all sins to me and all believers, and will at the Last Day raise up me and all the dead, and give unto me and all believers in Christ eternal life.  This is most certainly true’.  Yes, to the fifth of our Christian Questions: ‘Do you also hope to be saved?’}\end{displayquote} instead of answering with our Church: ‘\textit{Yes, such is my hope}’, we should then have to answer: ‘\textit{No, such is not my hope!}’\footnote{that is to say, not with the assurance of faith. Cf. ‘Altes und Neues.’, I, p. 235, Antithesis 2.}\end{fancyquotes}

                In what way can and should a Christian come to the certainty of his eternal election?  How Walther answers this question has already been suggested in the preceding discussion by the fact that Walther wishes the certainty which a Christian has concerning his election to be called a “\textit{certainty of faith}”.  Faith has to do with God’s revealed Word, with the Gospel of Christ.  The important thing is to look in faith to the Gospel of Christ.  He who believes the Gospel of the grace of God in Christ thereby recognizes also his election which has taken place by grace for Christ’s sake.  Whatever assures a Christian of the grace of God in general assures him also of his eternal gracious election.  In the further development of this thought Walther directs attention to the Eleventh Article of the Formula of Concord\footnote{Mueller, pp. 709-715; Triglotta pp. 1071-1079} and continues: Here it is taught \begin{displayquote} “\textit{That a believing Christian can and should be assured of his election neither from reason, nor from the Law, nor by any sort of appearance, much less through searching out the secret hidden abyss of divine foreknowledge, but above all from his call through the Word which announces the universal grace, then from his baptism, from the Lord’s Supper, from private absolution, and from the testimony of the Holy Spirit}”.\footnote{Berichtigung, p. 121}\end{displayquote} He who wishes to be sure of his election must in all earnestness walk the way upon which God will save the elect.  “\textit{That}” says Walther, \begin{displayquote} “\textit{is not the right counsel which tells you that you must just firmly convince yourself that you are elect... No! we must also walk the way of salvation.  The doctrine of election is no pillow for the flesh}”.\end{displayquote}  Walther again directs attention to the Formula of Concord.  He writes: \begin{fancyquotes}The Formula of Concord governs itself strictly according to {\scriptsize\textsc{Romans 8:28-39}}, where the way is described upon which God leads His elect to glory, from which the Formula concludes that he who sees that he is in this way should not doubt that he is elect, and hence may confidently join in the jubilation of the Apostle: \begin{displayquote}{\footnotesize ‘I am persuaded that neither death, nor life, etc., shall be able to separate us from the love of God, which is in Christ Jesus our Lord’.}\footnote{l.c., p. 121, 122}\end{displayquote}\end{fancyquotes}

                The exhortations addressed to the believers in Holy Scripture to work out their salvation with fear and trembling have been introduced against the assurance of salvation.  But these exhortations do not contradict the assurance so clearly taught in Scripture.  They are not directed against firm faith in the Gospel promises, but against carnal security; they have not the purpose “\textit{of making us uncertain, but of preserving us in our certainty}”.\footnote{Lutheraner, 1880 p. 27}

                The further objection has been raised that just this doctrine of the certainty of election misleads people into carnal security.  With reference to this objection Walther says: \begin{fancyquotes}Our adversaries say indeed that this doctrine only makes people secure.  But if that were so, then no man could be sure that he is standing in God’s grace.  Christ calls to the seventy disciples: \begin{displayquote}{\footnotesize ‘In this rejoice not, that the spirits are subject unto you; but rather rejoice, because your names are written in heaven’.}\end{displayquote}  Yes, after Christ had forewarned Peter of his deep fall, he added for his comfort: \begin{displayquote}{\footnotesize ‘But I have prayed for thee, that thy faith fail not’.}\end{displayquote}  Did Christ thereby make His disciples secure? No, it was just the assured faith in their election which roused them to contend most faithfully, even to the bloody death of martyrdom.  And as Christ dealt with His disciples in this respect, so the disciples afterwards dealt with the believers converted through them.  We find, for instance, that Paul expressly assures the believers at Ephesus, at Thessalonica, at Colosse, and Peter the believers living in the Diaspora, that they are elect; yea, Peter directly calls the totality of all believers ‘\textit{a chosen generation}’.  Paul even puts on the lips of all believers the triumphant song \begin{displayquote}{\footnotesize ‘Who shall lay anything to the charge of God’s elect?  It is God that justifieth.  I am persuaded, that neither death, nor life, etc., shall be able to separate us from the love of God, which is Christ Jesus our Lord’.}\end{displayquote}  Did the Apostles thereby make the believers secure?  No, thereby they rather placed the helmet of salvation upon their head and the shield of faith in their hands, to quench all the fiery darts {\scriptsize\textsc{(doubts)}} of the evil one.  So then it is beyond all doubt that he who teaches concerning the election of grace in such a way that the believers cannot become certain of it is teaching a false doctrine, for it is unbiblical.\footnote{Abendvorlesungen [Evening lecture], Oct. 28, 1881.}\end{fancyquotes}   Entering even further into the discussion of the state of heart of him who is assured in faith of his election, Walther writes: \begin{fancyquotes}He who knows upon what way alone God leads His elect unto salvation, namely by the way of repentance and conversion, of faith and sanctification, of cross and perseverance, – he has in this knowledge, we think, enough warning and admonition; for as soon as he willfully forsakes that way his assurance of salvation is lost, according to our pure doctrine.\end{fancyquotes}  Only the “\textit{godless}” think and say that they need not trouble themselves about their salvation if it does not rest in their hand but alone in God’s hand.  In the case of truly believing Christians the situation is altogether different; “\textit{the more sure they are in faith that they are elect and so will be saved, the more zealous they are in all good}”.  \begin{displayquote}“\textit{The situation is just the same with the doctrine of election as it is with the doctrine of justification.  When a godless man hears that we are justified before God and saved without the works of the law by faith alone then he immediately thinks: that is a shameful doctrine, destructive of all morality, for he who believes that will think: if God according to this doctrine does not take any account of good works, why then do I need to do good works?  Even in the time of the Apostles there were people who actually drew such conclusions from the doctrine of justification...  But St. Paul pronounces upon those who draw such conclusions the dreadful sentence: ‘Whose damnation is just’.{\scriptsize\textsc{(Romans 3:8)}}”}\end{displayquote}

                Those who claim that the believers cannot and should not be sure of their eternal election have also sought to support this assertion of theirs with the authority of Luther.  But in proving their point from Luther they have led their readers astray by quoting such passages from Luther in which the reformer warns against searching out the secret will of God, while the same Luther tells the believers to be fully assured in faith of their eternal election in so far as this is revealed in the Gospel of Christ.  Walther writes with regard to this point: \begin{fancyquotes}When pure and godly theologians, -- when, e.g., our Luther at times seems to speak against men’s assurance of their eternal salvation, this is above all directed against those who sought to become sure of their salvation by searching out the secret counsel of God, in order then to be rid of all further care for their salvation and of all earnest seeking after salvation; for in Luther’s time there were fanatics who believed that men could and should seek to become sure of their salvation through a special divine revelation, and that then they could and should be unconcerned about their salvation, because then they could not fall away.  Against such abominable fanatics Luther indeed had to cry out: \textit{Away with your assurance!  It is produced in you by the devil!}  The more uncertain of your salvation you become, the sooner you may yet be saved. \par And since Luther had himself for a long time been plunged, as it were, into hell, because he had wanted to search out God’s secret counsel concerning himself, and still been unable to search it out, he held it to be his duty to warn those who had come into severe temptation concerning predestination against sinking in this abyss and against scaling this height.  But thereby Luther not only did not retract his doctrine of eternal, sure, and unalterable predestination, but he also still less desired to propose the terrible teaching that a Christian must doubt his salvation and until death be suspended in uncertainty, as between heaven and hell.  Rather did Luther in one of his last writings, namely in his Exposition of Genesis, in the exegesis of the 26th chapter, so gloriously show how a man can become entirely sure of his salvation in the right way, that the heart of a godly Lutheran Christian leaps for joy when he reads it\footnote{Lutheraner 1880, p. 22.}\end{fancyquotes}  Walther offers in evidence several passages from \textbf{Luther}, -- among other these: \begin{fancyquotes}Why should you want to listen to the Gospel, say the Epicureans, since after all everything depends on predestination?  In this way Satan forcibly takes away the predestination, of which we were firmly assured by the Son of God and through the holy Sacraments, and makes us uncertain, whereas we were before entirely sure.  And when he assails the poor terrified consciences with this temptation, we sink in death, just as it almost happened to me, if Staupitz had not rescued me, when I suffered this very same temptation... Dr. Staupitz used to comfort me with these words, speaking to me thus: \begin{displayquote}{\footnotesize My dear man, why do you plague yourself so with these speculations and high thoughts?  Gaze upon the wounds of Christ and the blood shed for you; there predestination will shine forth.  Therefore one should hear the Son of God, who was sent into the flesh, become man, and was manifested for this purpose, that He might destroy the works of the devil {\scriptsize\textsc{(1 John 3:8)}} and make you sure of predestination.}\end{displayquote}\end{fancyquotes}

                We close these discussions concerning Dr. Walther’s doctrine of election with a few words of Walther which he penned in the midst of the controversy concerning this doctrine.  In his tract “\textit{Die Lehre von der Gnadenwahl in Frage und Antwort}”\footnote{c.f. p. 11} we read: \begin{fancyquotes}God has given to our Lutheran Church in America through the election controversy which has broken out the great task of contending for one of the most mysterious doctrines of His Word, to judge of which not rationalists, not idle, curious, ambitious spirits, not frivolous false Christians, but only true, enlightened Christians concerned for their salvation, humble, and trembling at God’s Word, are fit and competent.  This election controversy deals with the great and highly important questions: \begin{displayquote}{\footnotesize ‘To whom do those who come to faith, remain in faith, and are saved, owe this grace?  Do they owe this to themselves?  Or do they owe it at least in part to themselves?  Or do they owe this alone to the grace of God and the most holy merit of Christ?  Does the glory for our salvation belong to God alone?  Or is there also a cause thereof in man?  Does man by nature possess powers to cooperate to some extent in the work of his salvation, to decide for salvation, or at least to assent thereto, though feebly?  Or is every man by nature spiritually dead, and must God therefore do all by His grace?’}\end{displayquote} – Yes, the present controversy is concerned with these great truths, with the doctrine of salvation by grace alone, for Christ’s sake, alone, through God-given faith alone, not with theological hairsplitting but with the most important points of practical Christianity.  May God therefore have mercy upon our American Lutheran Zion, and help that no upright soul may go astray in this battle for the truth, but that all true children of God within our Church may finally gather under the good banner of our Confessions also in regard to this doctrine, and so become a light for many in the midnight hour of this last time of sore distress.  May God grant it for the sake of Jesus Christ , the universal Savior of all sinners and the eternal King of truth. \textbf{Amen}.\end{fancyquotes}
%%% Local Variables:
%%% mode: latex
%%% TeX-master: "../main"
%%% End:
