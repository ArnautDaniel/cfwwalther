\chapter{How to be a Theologian}
\hrule
\vspace{.30cm}
We have seen that Walther understood by theology the sufficiency to lead sinners to salvation by means of the Word of God. How then, is this sufficiency obtained, or: How does one become a theologian?
\vspace{.10cm}
\hrule
\vspace{1.25cm}
Walther answered this question repeatedly in his writings. And each time he had to answer it for the theological students in the class-room, he spent considerable time upon it.  Theology for Walther is a wisdom from above. And this not only in the sense that the theologian derives everything that he teaches only and alone from the divine revelation, but also specifically that the competence to know the divine revelation, to impart it, and thereby to lead men to salvation, is one wrought only by the Holy Spirit.  Just as no man can discover the material with which theology deals by way of speculation, so also no man can induce in himself the competence rightly to treat and to evaluate this material through human power and art, as, for instance, by following a specific "\textit{scientific method}."

The theological habitude, Walther says, ``\textit{is a supernatural one, not to be attained by human power and diligence}." \begin{displayquote}``\textit{There are certain natural gifts which serve the holy office: keen judgment, eloquence, etc. But these do not belong to the specific gifts of office which make a minister of the Church. St. Paul enumerates the latter in {\scriptsize\textsc{1 Corinthians 12}} and in {\scriptsize\textsc{Romans 12}}: Wisdom, knowledge, faith, discerning of spirits, prophecy, teaching, exhortation, ruling, etc.}" \end{displayquote}The Holy Ghost, who revealed the divine Truth in the Scripture, must Himself through this Truth create for Himself also the instruments who shall know it and communicate and apply it to others unto salvation. ``\textit{Only the Holy Ghost makes D.D.s\footnote{Doctors of Theology}}" remarks Walther in commenting on Luther's dictum\footnote{To the Christian Nobility}, as to how Doctors of the Holy Scripture come into existence in distinction from ``\textit{Doctors of Arts, of medicine, of Laws, of the Sentences}," etc.

Hence Walther also declares that in Luther's sentence ``\textit{oratio, meditatio, tentatio faciunt Theologum}" ``\textit{the only correct theological methodology}" is given.

In his \textsc{Pastoral Theology} he remarks, on page 6: \begin{displayquote}``\textit{To attain to the theological habitude three things are requisite, which are contained in Luther's well-known axiom:}\textbf{ Oratio, meditation, tentatio faciunt theologum}."\end{displayquote}

The \textsc{Oratio} is the humble and earnest prayer that God would give us by His Holy Spirit the right understanding of the Scripture and not let us plunge into it with our reason. For ``\textit{although the grammatical sense of Scripture is clear, yet the Holy Ghost must open up for us the living and salutary understanding of the Scriptures}," and the ``\textit{beginning}" of all theology is to despair of all one's own wisdom, unconditionally to subject one's own opinion to the Word of God, and to be willing to derive all knowledge in spiritual things from the Word of God.

But this no man is able to do according to his own natural disposition. Therefore one must persist with the Oratio, and so much the more in proportion as learning and natural gifts are the greater. \\\begin{fancyquotes}Competent knowledge and rich gifts are a grand endowment. But it should never be forgotten, the greater the knowledge and gifts, the greater the danger that one becomes self-confident, also in theology!\end{fancyquotes}

The \textsc{Meditatio}, that is the constant study of the Scripture, \begin{displayquote}``\textit{the delving deep into God's Word},"
\begin{displayquote}``\textit{to occupy one's self with God's Word in every way},"
\begin{displayquote}``\textit{not in the heart alone, but also externally work on and apply the oral speech and the lettered words in the Book}\marginpar{\scriptsize according to Luther}, \begin{displayquote}``\textit{as one rubs aromatic herbs that they may give forth their own precious scent}''.\marginpar{\scriptsize adds Walther}\end{displayquote}\end{displayquote}\end{displayquote}\end{displayquote}
\divider
That the \textsc{Tentatio} also belongs to ``\textit{theological methodology}" is established, for instance, by {\scriptsize\textsc{2 Cor. 1:3}} When Luther says: \begin{displayquote}``\textit{As soon as the Word of God blooms forth through you, the devil will visit you, and make a real doctor of you, and by his affliction will teach you to seek and love God's Word}'',\end{displayquote}  Walther adds, that is indeed a ``\textit{strange promotion to the doctorate}." But God observes this method: ``\textit{hence no student of theology should grieve if God sends him all manner of temptation}." He is intent on holding fast to this ``\textit{methodology}," although He is well aware that many smile over it as insufficient for our times.
\divider
The \textbf{oratio}, \textbf{meditatio}, \textbf{tentatio} of which Luther speaks, however, are to be found only in the regenerate. And so Walther further insists most emphatically that only one who has first become a true Christian can become a theologian. He writes: \\\begin{fancyquotes}No unbeliever, no natural man, no slave of sin, no non-Christian, no hypocrite, but only a believer, a regenerate and sanctified person, in short, only a true Christian can be a true theologian; as the Christian presupposes the man, so the theologian presupposes the Christian, and as faith includes knowledge, so theology includes faith.\end{fancyquotes}

``\textit{The Holy Scriptures}," he continues,\\\begin{fancyquotes}states this clearly and plainly. The apostle, speaking of the office of the Word, cries out: `\textit{Who is sufficient}'\footnote{2 Corinthians 2:16} and answers: \begin{displayquote}`\textit{Not that we are sufficient of ourselves to think anything as of ourselves; but our sufficiency is of God, who also hath made us able ministers of the New Testament}'\footnote{2 Corinthians 3:5-6}.\end{displayquote} As surely, therefore, as the sufficiency for office is bestowed by God alone, so surely is also the theological habitude, which alone renders one competent for the exercise of the office, bestowed only by God. The holy apostle says further: \begin{displayquote}`\textit{The natural receiveth not the things of the Spirit of God\footnote{Does not perceive and accept what is of the Spirit of God, or the revealed mysteries of faith} for they are foolishness unto him: neither can he know them, because they are spiritually discerned. But he that is spiritual judgeth all things}\footnote{1 Corinthians 2:14-15}'.\end{displayquote}

As surely, therefore, as a natural man does not understand spiritual matters, and can have no correct judgment concerning them, so surely can a natural man be no true theologian, whose chief concern is to judge concerning spiritual matters. Only a truly spiritual man can be a true theologian. An unconverted man can indeed carry theology, as teaching, in his understanding and memory as in a book, and also impart it to others; but, although he can convert others, yet he is himself by virtue of head-knowledge and oral profession no more a true theologian than a book which contains the doctrine of theology comprised in letters and words; he is nothing else than what the apostle says of such, ``\textit{a sounding brass and a tinkling cymbal}"\footnote{1 Corinthians 13:1}. While he teaches others the pure truth unto salvation, it is to himself still a closed book, a mystery which he does not understand, yea, a foolishness. While he preaches to others, he himself is a castaway\footnote{1 Corinthians 9:27}. He does not hold the mystery of faith in a pure conscience\footnote{1 Timothy 3:9}. He still belongs to the world, and hence cannot receive the Spirit of Truth.\end{fancyquotes}
\par

``\textit{Godliness}," remarks Walther elsewhere with reference to the same subject, ``\textit{is not merely advantageous for the theologian, but a conditio sine qua non}\marginpar{\scriptsize\textit{conditio sine qua non\\} the condition without which he is not a theologian}." He refers to {\scriptsize\textsc{1 Timothy 3:1-7 and Titus 1:5-9}}, where, in the description of a true theologian ``\textit{the gifts of office and of sanctification are taken together}." In one category with the ``\textit{apt to teach}" stand ``\textit{vigilant, sober, of good behavior, given to hospitality}." In the assertion that there is no illumination without conversion Walther sides with the Pietists against a few of the later ``\textit{orthodox}."

Walther then demonstrates by the various activities which are incumbent upon a theologian that these can be performed only by one who stands in living faith. ``\textit{It is indeed}," he says, \begin{fancyquotes}a exceedingly important doctrine of our church that the Word of God is of itself quick and powerful and does not first become quick and powerful through the piety of those who preach it. But from this is does not follow that it is a matter of indifference whether one who occupies the office of the ministry is a godly man. Especially on account of the proper distinction of the Law and the Gospel, which is so necessary in the sermon and in the private cure of souls\footnote{Privatseelsorge} it is indispensably necessary that the preacher himself possess true faith of the heart and has himself had spiritual experience.\end{fancyquotes}

In his \textsc{Pastorale} Walther cites \textbf{Luther's} words: \begin{displayquote}`\textit{I experience it myself, and see daily also in others, how difficult it is to distinguish the doctrine of the Law and the Gospel. The Holy Ghost must here be Master and Teacher, or no man on earth will ever be able to understand or teach it. Therefore, no pope, no false Christian, no fanatic\footnote{Schwaermer} is able to divide these two from one another}.'\end{displayquote}

In this connection he notes: \marginpar{\scriptsize\textit{doctrine de discrimine legis et evangelii} \\The Proper Distinction between Law and Gospel}  \begin{fancyquotes}The \textit{doctrine de discrimine legis et evangelii} can indeed be correctly grasped by the intellect without a living faith, but then one goes astray in the application. Furthermore, the unconverted preacher who in his inner heart seeks only bread, honor, and a good living, not the salvation of the souls entrusted to him, will neglect to reprove sins, since he fears that thereby he would make enemies and thereby lose the treasures which he seeks to gain. The unconverted preacher dare not draw clear a picture of a true or false Christian on the basis of God's Word, for he has to fear lest his hearers will say, `\textit{You yourself are not like that!}' or `\textit{That's just like you!}'\par In an unconverted preacher the faithfulness, the zeal, the daily care, and in his preaching the real spiritual fire will be lacking. No office has such great temptations to unfaithfulness as the ministry. The pastor can lounge about for six days, if he wishes, and often the congregation observes with pleasure that the pastor does not '\textit{come around}.' If he has good gifts he can still, with all his laziness preach in such a way that the people will think they are hearing something wonderful. The unconverted preacher chooses such subjects as he can easily treat, and avoids the difficult ones, regardless of how necessary it may be to treat them.\end{fancyquotes} 

Hence Walther as a theological professor always took pains not only to present the Christian doctrine clearly, but also to edify the hearts and sharpen the consciences of his seminarians. Probably the majority of his students will testify that they experienced rich advancement in their spiritual life through Walther's theological instruction. His entire presentation was both instructive and edifying. Individuals among his students first came to a living faith in Christ in his theological lecture hall.

But however strongly Walther, on the one hand, emphasized and reiterated to his seminarians the truth that only ``\textit{one standing in grace, only a regenerate man}" can be a theologian, he, on the other hand, also warned against the abuse of this truth on the part of the sects and fanatics. He said, ``\textit{One can also abuse the teaching that theology is a habitus practicus}," namely in the direction of contempt for thorough theological study, or to indifference and negligence in study. ``\textit{The Methodists imagine that as soon as they are converted they can be preachers}." Every theologian is a Christian, but not every Christian is a theologian! The theological habitude is bestowed by God alone, but by way of diligent study.
 Walther cited the words of \textbf{L. Hartmann}: \begin{fancyquotes} What Tertullian once correctly said of the Christians, that Christians not born but made\footnote{\textit{Christiani non nascuntur, sed fiunt}}, is also true of faithful ministers and teachers of the church, who have need of a long preparation and intensive study if they are to be competent to enter upon their exalted office. For here mere personal reputation or earnestness and holiness of life are not sufficient, much rather is theological knowledge required.\footnote{Pastorale ev., Nuremberg, 1697, p. 237}\end{fancyquotes} In this connection Walther remarks, \begin{displayquote}``\textit{Only the regenerate can become theologians, but theology is not, like the spiritual life, bestowed in a moment}.''\end{displayquote}

As Walther, therefore, strove to impart the most thorough theological training, and that particularly on account of the peculiar circumstances in which the Church of the Reformation is placed in this country, so he also to spur the students on to the greatest diligence in study. He was accustomed to call to their minds that men like \textbf{Chemnitz}, \textbf{Gerhard}, \textbf{Calov}, yea, even \textbf{Luther}, became the great theologians they were ``\textit{not through their great gifts, but through the tireless diligence which they applied}.''

Among the notes which have been made available to the present writer are found also the following, which we present in their original aphoristic form, as they make clear the thoughts which Walther developed for his students: \begin{displayquote}``\textit{Be wary of time! - Read with the pen! - Make excerpts! - Schedule your studies! - Divide up the day and the week! - Read only quality! - Don't read trivia! - Review everything from time to time! - Index things! - Priorities: first the Necessary, then the most useful, then the useful! - Read with a theological interest! - Do not cram for exams! - Don't read trash at all!}"\end{displayquote} Walther warned the candidates of theology not to set too modest a goal. No one should permit himself to be misled by the thought that he has only mediocre gifts, so as to content himself from the beginning with mediocre accomplishments. ``\textit{To be modest in setting your goal is a sinful modesty}.''

This is the way Walther understood his statement: \begin{displayquote}``\textit{Theology is a habitude wrought by the Holy Ghost, and drawn from the Word of God by means of prayer, study, and trial}."\end{displayquote} We cannot surrender this definition of the concept. It is the Lutheran definition, the one taken from God's Word. The danger that we should fall into a fanatical line of thinking, and imagine that every Christian is without further training capable and called to teach publicly, is rather remote.

Also the sects within recent years have at least partially recovered from this delusion and insist upon theological training. But by God's grace we must also bear in mind that mere training makes no theologians, but that rather the basis and beginning of all theological knowledge and ability is living faith in Christ, a genuine conversion.

Only young men who are in a state of spiritual life are capable of studying theology; only truly believing pastors are competent to administer their office. The orthodox Lutheran Church in this country is still greatly in need of pastors. The prospects are that this need will in the immediate future grow not less but greater. But the need can never become so great that we allow manifestly unconverted persons to be called into the ministry, contrary to the Biblical and Lutheran principle that only converted Christians should be preachers or can be the right kind of preachers.

That the orthodox Lutheran synods of this country may look back upon such richly blessed activity comes also from the fact that God has endowed them with pure doctrine and also a ministerium of true believers. If God grants and preserves to them this gift also in the future, then their blessed fellowship will remain and prosper. If we through our ingratitude and heedlessness should lose this gift, should we get a ministerium in large part spiritually dead, then that the fresh and happy activity in our fellowship should soon cease and also the external apostasy from the right doctrine would soon follow.

%%% Local Variables:
%%% mode: latex
%%% TeX-master: "../main"
%%% End:
