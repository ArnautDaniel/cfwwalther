\chapter{Inspiration \& Open Questions}

\hrule
\vspace{.30cm}
In Zöckler’s ``\textit{Handbuch der Theologischen Wissenschaften}”\footnote{2nd Ed., III}, besides the Reformed theologians, \textbf{Kohlbruegge}, \textbf{Gaussen}, and \textbf{Kuyper}, Walther in St. Louis is named as a champion of the old church doctrine of inspiration  ``\textit{on the Lutheran side}”.  As proof for this reference is made to an article in ``\textit{Lehre und Wehre}” which later appeared in pamphlet form, which was indeed not written by Walther the article: ``\textit{Was lehren die neueren orthodox sein wollenden Theologen von der Inspiration?}”\footnote{Lehre und Wehre, 1871, p. 33ff}; but the statement of the ``\textit{Handbuch}” is nevertheless correct.  Walther not only championed the old church doctrine of inspiration with fullest conviction throughout his whole career as a teacher of the church, but he also designated the yielding up of this doctrine as an apostasy in principle from the Christian religion.
\vspace{.10cm}
\hrule
\vspace{1.25cm}

In the very first volume of ``\textit{Lehre und Wehre}''\footnote{Lehre und Wehre, 1855, p. 248}, in a review of Kahnis’ work: ``\textit{Der innere Gang des deutschen Protestantismus}'', etc., reference is made to the following words which occur in this writing:

\begin{displayquote}``\textit{Protestantism stands and falls with the principle of the sole authority of Scripture.  But this principle is independent of the doctrine of inspiration taught by the old dogmaticians.  To take it up again as it was could be done only with hardening against the truth}.''\end{displayquote} Kahnis took a better position then than he did later.  At that time his name was still in good repute in the Lutheran Church.  Yet Walther even then remarked on the words just cited:

\begin{fancyquotes}We must confess that when we read these words we were terrified in our very heart.  Who wants to go along with a new theology which introduces itself as a development of the old Lutheran theology and then departs from the doctrinal type of our old church just in the very doctrine of the formal principle of theology, in the doctrine of the Holy Scripture, specifically of the \textit{ratio formalis scripturae}, of that which makes Scripture the Holy Scripture?\end{fancyquotes}\marginpar{\scriptsize \textit{ratio formalis scripturae}\\ the formal grounds for something; the essential attributes of matter as they appear in the mind or in a definition.} So wrote Walther in the first volume of ``\textit{Lehre und Wehre}''.  He also treats inspiration in the last Foreword written by him, in the Foreword to the 32nd volume of “\textit{Lehre und Wehre}”\footnote{Lehre und Wehre, 1886}.

                What doctrine of inspiration Walther held for the correct one he briefly shows in “\textit{Lehre und Wehre\footnote{Lehre und Wehre, 1875, p. 257 f.}}'' in three short citations from \textbf{Baier} and \textbf{Quenstedt}.  But he always treated this subject very exhaustively in the regular lectures, and finally in the academic year 1885--1886 in evening lectures.  Walther’s doctrine of inspiration may be briefly summarized as follows: Holy Scripture does not merely contain God’s Word, but is according to its entire compass God’s Word in the proper sense, because God spoke through the sacred writers or gave them\footnote{denselben eingegaben hat} the matters and words, so that now in Holy Scripture not the slightest error can occur, either in dogmatical or even in historical, geographical, and other such matters.  So one must believe, says Walther, concerning Holy Scripture, if one accepts ``\textit{what Holy Scripture says of itself}”.\marginpar{\scriptsize 2 Timothy 3:16; 2 Peter 1:20-21; 1 Corinthians 2:13; John 10:35;}  He regards the concept of the inspiration of Scripture as having been given up by all those \begin{displayquote}“\textit{who acknowledge only an inspiration of the ‘what’ and not the ‘how’ of the matters and not also of the words of Holy Scripture, or who assume degrees of inspiration giving precedence to one book before another, or who grant that Scripture may contain any error, that it condescends not only to the comprehension of simple people but also to their false notions}”.\footnote{Lehre und Wehre, 13, 100.}\end{displayquote}  With regard to those who confuse inspiration with enlightenment and transform inspiration into a mere preservation from error, so that we would still have at least an errorless Scripture, Walther remarked: \begin{fancyquotes}That seems harmless enough, and yet thereby the entire doctrine of inspiration is given up.  We need not merely truth, but divine truth.  We must have a word which has passed through the mouth of God, and consequently is glowing with divine power and penetration, immersed, so to speak, in the mind of God.  The simple truth works through the power of persuasion; not so the Word of God.\end{fancyquotes}  With regard to the expressions of the Church Fathers and the old Lutheran teachers, to the effect that the Holy Writers were like manuas, calami, notarii, tabelliones of the Holy Ghost, Walther remarks: \begin{displayquote}“\textit{Though more recent positive theologians {\scriptsize\textsc{(die Neuglaeubigen)}} may ridicule these expressions yet they express the teaching of Holy Scripture}”.\end{displayquote}  The variations of style which are found in the various books of Scripture Dr. Walther explained, together with the great majority of the old teachers, by the fact that the Holy Ghost used His instruments as He found them; for the “\textit{essence of inspiration lies not in new words but in the fact that words, which may indeed have been otherwise in common use, passed through the mouth of God, that God made them His own words}.”  Whether the Hebrew vowel points which appear in the current pointed Hebrew text were written in the text from the beginning, as the majority of the old Lutheran teachers supposed, Walther declared to be not a dogmatical but a critical question.  He, for his own person, held with Luther, who declared the traditional Hebrew system of vowel points to be the product of a later age.
\divider
We shall here give just one example of the way Walther refuted the objections raised against the church’s doctrine of inspiration.  It is well known that the recent theologians assert they gave up the old doctrine of inspiration in order to rescue the “\textit{divine-human character of Scripture, which the earlier theologians had overlooked.}\footnote{Handbuch der theologischen Wissenschaften, l.c.}  Walther said:
\begin{fancyquotes}Among the many objections which modern believing theologians raise against the doctrine of inspiration as taught by our old dogmaticians one of the most common is that this doctrine in its emphasis on the divine character of Holy Scripture does not do justice to its human side, yes, entirely abolishes this aspect.  As in the Apostolic age the sect of the Docetists denied that in Christ God had become a true man, and taught that the apparent human in Christ was only an appearance, in like manner, it is now said, the old Lutheran dogmatics did with the Bible; the old dogmatics, they assert, makes everything human in the Bible a mere appearance. \par  All this is simply not true.  Also the old dogmatics indeed acknowledges a human side of the Bible in a certain sense.  It acknowledges that the Bible was not, like the Ten Commandments, written directly by God’s own finger, but through men, namely, the apostles and prophets.  Also the old dogmatics further acknowledges that the Bible does not speak the language of heaven, of which St. Paul says he heard unspeakable words, but that the Word of God has clothed itself in our human language and human writing.  Yes, the old dogmatics admits that the Bible was written by the holy writers not in a state of ecstasy but with full consciousness, and that the Holy Spirit accommodated Himself to the language and the human style of each apostle and prophet. \par   The old dogmatics, however, and we with it, teaches that in Christ the Son of God became a true man, but without sin, and thus also in the Bible the Word of God became true human speech, but without error.  As therefore a man for the reason that he is without sin is still not a mere appearance of a man, but a real man, so also human speech which is without error is not for that reason a mere appearance of human speech, but truly human speech. \par  For what purpose then is the cry that the old dogmatics does not do justice to the human side of the Scripture?  The intention is none other than this:  Our error is to consist in the fact that we do not ascribe errors to Holy Scripture as to every other human writing, but that we hold it, among all books, to be the Book of Truth.\footnote{Evening Lecture on the Doctrine of Inspiration, December 18, 1885. (Manteufel Translation)}\end{fancyquotes}

                For what reason did Walther hold so firmly to the church’s doctrine of inspiration?  Before all else because this is the clear teaching of Scripture concerning itself.  But then also because, as already suggested, with the surrender of the church doctrine of inspiration also the truth that Scripture alone is the source and norm of Christian doctrine is surrendered.  It is inconceivable how a man like Kahnis, who has been labeled a “\textit{thinker}”, can put forward the proposition that the principle of Protestantism concerning the sole authority of Scripture is “\textit{independent}” of the old church doctrine of inspiration, that is, from the teaching that Scripture is the perfectly inerrant Word of God.  Everyone will at once be constrained to agree with Walther when he ever and again declares: \begin{fancyquotes}We must absolutely hold fast to the doctrine of inspiration taught by our old dogmaticians.  If we grant that in the Bible even the least error can occur, then man must undertake to separate truth from error.  Thereby man is placed over the Scripture and Scripture has therewith ceased to be the source and norm of faith.  Human reason is made the norma of truth and Scripture sinks to the level of a norma normata.  The slightest deviation from the old doctrine of inspiration introduces a rationalistic germ into theology and leavens the entire structure of doctrine.\footnote{In a lecture 1874-1875}\end{fancyquotes}

On the same subject Walther said, with reference to the controversy over the doctrine of inspiration recently provoked by the Dorpat professors \textbf{Volck} and \textbf{Muehleau}: \begin{fancyquotes} With the doctrine of inspiration stands and falls the truth, certainty, and divine authority of Holy Scripture and therefore of the entire Christian religion and church.  This is not just one doctrine alongside of others, but upon it rest all other doctrines as upon their foundation.  If Holy Scripture is not inspired of God, but brought forth by the will of man, then it is also no divine but a human Scripture.  But if one says: \begin{displayquote}{\footnotesize In all which Scripture reports and declares concerning the earning and attainment of eternal salvation it is of divine origin and therefore infallible in this respect; only in that which stands no necessary connection therewith, in the non-essential and incidental matters is it of human character and therefore not entirely errorless,}\end{displayquote} -- this does not help matters.  For then, by the assertion that human error is mingled with the divinely true content of Scripture, not a part but the whole of Scripture is rendered unsteady and unreliable and the reader is made the superior judge as to which component parts of Scripture contain the essential and which the non-essential, which the divine and which the human, which contain truth and which contain error or at least could contain error.  Then it would be a gigantic hoax and gross deceit that the Christian Church hitherto has always regarded Holy Scripture as the formal principle or as the pure source of all her Christian knowledge, as the inerrant rule and standard of all faith and life, and as the supreme and ultimate arbiter in all controversies concerning faith and religion.  Then one could no longer admonish a Christian as often as he opens his Bible to pray with Samuel \begin{displayquote}‘\textit{Speak, Lord; for Thy servant heareth}’,\end{displayquote} but would rather have to warn every Bible-reader not to surrender himself to Scripture with entire confidence, and admonish him to read Scripture with great caution and constant discrimination and to devote himself to the task of culling the divinely true from the midst of human error.\footnote{Evening Lecture on the doctrine of inspiration, November 27, 1885.}\end{fancyquotes}

                Hence Walther exclaims: “\textit{God have mercy upon His poor Christendom in this last age of distress and danger}”\footnote{L.u.W. 32, p. 77}, in which the Christians have their Bible taken from them, “\textit{the lamp unto their feet and the light unto their path to eternity, their rod and staff in the dark valley of tribulation, in short, God’s Word, and therefore their comfort in the anguish of sin, their hope in the night of their dying hour}”.\footnote{L.c. p. 76.}

                It is his desire therefore that “\textit{Lehre und Wehre}” shall also in future warn against the deniers of the inspiration of Holy Scripture ”\textit{as the worst false prophets of our time}.''

                He writes: \begin{fancyquotes}It is time indeed for every believing theologian, as he values his soul’s salvation, with the utmost earnestness to get into the fight for the highest treasure of Christians which God has given to men after the bestowal of His Son.

                Woe unto him who wants to be reckoned among the theologians and yet will not acknowledge that this above all is his vocation, to preserve unto the common Christian that upon which his faith, and hence also his salvation and blessedness, rests, the ‘\textit{foundation of the apostles and prophets, Jesus Christ himself being the chief cornerstone}’. \par Woe unto him who wants to be reckoned among the theologians and on the other hand  imagines that just for that reason he must as such contend above all that its full freedom remain assured unto science!  Just in this lies the deepest ground for the ever more complete apostasy of modern theology from the revealed divine truth and for the complete transformation of the Christian religion into a human science, namely, that modern theology no longer wishes to be a \textit{habitus practicus}, but the ‘\textit{scientific self-consciousness of the church}’ or ‘\textit{the ecclesiastical science of Christianity}’.\footnote{L.c. p. 6.}\end{fancyquotes}

%%% Local Variables:
%%% mode: latex
%%% TeX-master: "../main"
%%% End:
