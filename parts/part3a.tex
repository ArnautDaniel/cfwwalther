\chapter{Inspiration \& Open Questions II}

\hrule
\vspace{.30cm}
Dr. Walther had the same object in mind, namely, the guarding of the principle of Scripture, or holding fast to the truth that Holy Scripture alone is the source and norm of  Christian doctrine, also in the controversy on “\textit{open questions}”.  As human reason or science is made the norm for Christian doctrine through denying the church doctrine of inspiration, so “\textit{the church}” with its doctrinal decisions takes the place of Holy Scripture through the modern theory of open questions.  For in what sense did Pastor Loehe, the Iowans, and the authors of the Dorpat theological opinion\footnote{Gutachten} speak of “\textit{open questions}”?  As open questions they desired to consider such doctrines as, although revealed in Scripture, have not yet been decided by the Church in her Symbols or concerning which no agreement has yet been reached among orthodox theologians.\footnote{For the fact that those named really spoke on this sense of open questions, evidence is offered, e.g. in “L.u.W.” 14, 129ff.  Later indeed the Iowans declared that it never entered their mind to speak of open questions in this way.}  \par Among the doctrines declared to be such were the doctrine of the \textbf{Church}, of the \textbf{Ministry} and \textbf{Power of the Keys}, of a \textbf{Millennial Kingdom} still to be expected, of a future twofold Visible \textbf{Advent of the Lord}, and of a twofold bodily \textbf{Resurrection}, of \textbf{Sunday}, etc.

\vspace{.10cm}
\hrule
\vspace{1.25cm}


Also Walther acknowledged the existence of “\textit{open questions}”.  But in an entirely different sense.  He wishes to have the term “\textit{open questions}” used as synonyms with “\textit{theological problems}.”  Hence open questions are to him such as God’s Word leaves open questions which indeed arise in connection with the discussion of the Christian articles of faith, “\textit{but which find no solution in God’s Word}.\footnote{L.u.W., 14, 33.}  Walther insists most strenuously that open questions in this sense be acknowledged, and this for the very purpose that the Scripture principle may remain inviolate.  For if one should wish to “\textit{close}” a question which God’s Word leaves open then one would be adding to the Scripture.  He writes: \begin{displayquote}``\textit{What is not contained and decided in God’s Word must also not be equated with God’s Word and thus added to God’s Word.  But this would take place if orthodoxy should be made dependent upon any doctrine not contained in God’s Word and the denial of it should be given church-divisive significance.  Open questions in this sense are therefore all doctrines which are neither positively nor negatively decided by God’s Word, or such by the affirmation of which nothing which Holy Scripture denies is affirmed, and by the denial of which nothing which Holy Scripture affirms is denied.\footnote{L.u.W., 14,33.}''}\end{displayquote}
Among such open questions Walther, with the older theologians, reckons, among others, also the following:\begin{itemize}\item Whether Mary gave birth to other children after Christ\footnote{the Semper virgo};\item Whether the soul is imparted to every man through propagation from his parents, as flame from flame\footnote{per traducem, traducianism}, or through creative infusion\footnote{creationism};\item Whether the visible world will pass away on the last day according to its substance or only according to its attributes, etc.\footnote{L.u.W., 14, 34}\end{itemize}  On the other hand Walther insists most strenuously that nothing shall be declared an open question and treated as such which is clearly taught in God’s Word and thus decided by God’s Word.
\divider
                And in this connection it makes no difference whether the doctrine in question is fundamental or non-fundamental.  For here the Scripture principle comes into question, namely, whether all which God prescribes to men in Scripture to be believed is to be received by men in faith.  Walther writes: \begin{displayquote}“\textit{We can regard and treat no doctrine which is clearly taught in God’s Word or which contradicts God’s clear Word as an open question, no matter how subordinate or how far removed from the center of saving doctrine upon the periphery it may appear to be or actually is}”.\footnote{L.u.W., 14, 66}\end{displayquote}  And shortly thereafter: \begin{fancyquotes}We assert that in the orthodox church no justification can be conceded to any error against God’s clear Word, that in the orthodox church it may not be made optional to depart even in the least point from God’s clear Word, be it negatively or positively, directly or indirectly, and that every such departure from God’s clear Word, though it should consist in nothing more that the denial that Balaam’s ass spake, demands action on the part of the orthodox church against it, and that when all instructions, admonitions, warnings, and threats, and all exercises of patience have proved fruitless and ineffective to induce the person or group concerned to give up their contradiction against God’s clear Word, finally nothing else than expulsion or a separation can result.\footnote{l.c., p. 68.}\end{fancyquotes}

                Walther further explains how the Scripture principle comes into question here as follows: \begin{fancyquotes}What else is the assertion that such doctrines as are clearly contained in God’s Word could belong to the open questions than an assertion that one can indeed ‘diminish from’ God’s Word, need not always go according ‘\textit{to the law and to the testimony}’, that ‘\textit{a little leaven}’ of false doctrine does no harm and is therefore to be tolerated, that the Scripture can now and then ‘\textit{be broken}’, and one need not exactly ‘\textit{believe all that the prophets have spoken}’, that all Scripture is not so necessary and ‘\textit{profitable}’, and that it is indeed permitted to ‘\textit{break}’ much which is contained in the Scripture?  And yet more: suppose that all the passages cited\footnote{Deuteronomy 4:2; Deuteronomy 12:32; Isaiah 8:20; Revelation 22:19; Galatians 5:9; John 10:35; Luke 24:25; 2 Timothy 3:13,17; Matthew 5:18,19} and similar ones were not found in Holy Scripture, who would not even then, if he only really holds God’s Word to be God’s Word, have to find that theory unacceptable?  For if the Bible is God’s Word, then all the utterances contained therein are decisions of the exalted divine Majesty Himself.  Is it not terrible to declare that which the great God has decided to be still undecided? \par -- When the great God has spoken, to give man the liberty to contradict Him? -- where the great God has given His final judgement, to speak of the right of any creature to pas another judgement? -- to undertake a sifting of that which the eternal Wisdom and the eternal love has revealed for the salvation of men, and to say: This you must believe, confess, and teach, but that you may reject?\footnote{L.u.W., 14, 69.}\end{fancyquotes}
                
{\color{Black} \par If then, anyone says that doctrines are to be regarded and treated as still open because the orthodox church has not yet rendered her decisions upon them in her Symbols, or because there is not yet complete agreement concerning them among the teachers of the orthodox church, the Scripture principle of the Lutheran Church is thereby openly surrendered and crass papism is adopted.  Walther exclaims:} \begin{fancyquotes}From their point of view, then, any one has the liberty to accept or reject what God has revealed and decided in His Word as long as the Church has not yet spoken and rendered her decision; but as soon as the Church has spoken, all liberty has come to an end!}”\footnote{L.u.W., 14, 162.  Trans.: C.T.M., X, 8,588} \par It substitutes the Church for Scripture, man and his decisions for God and His divine decision.  And this substitution surrenders the foremost principle of true Protestantism and ascribes to our Church the principle of the antichristian Papacy, with all its errors and abominations.}”\footnote{l.c., p. 163.  Trans.: l.c., p. 589, corrected.}\end{fancyquotes}

{\color{Black} \par  The question whether a doctrine revealed in God’s Word is first raised to the dignity of a publicly acknowledged article of faith through the Symbolical decision of the Church, coincides with the question whether dogmas are gradually formed, or whether doctrines of the Word of God first become dogmas when they have passed through an ecclesiastical controversy and have become “\textit{symbolically fixed}”.  Walther’s utterance on this point takes account of the exact status controversiae and concedes what must be conceded:}
\begin{fancyquotes}It is true that the Word of God prophecies, and the history of the Church confirms, that the Church does not always stand before us in the same brilliant light of pure public preaching, that it rather, to use the figure of ancients, in this respect decreases and increases like the moon, that it experiences times of special gracious visitation and then again declines.

                \par But it is an error to say that the Church from century to century accumulates an ever-growing fund of divine teachings and according to the law of historical development arrives at constantly enhanced depths and riches of knowledge.  We admit that the Church all the time, through \begin{displayquote}{\footnotesize `men that arise in its midst and who speak perverse things to draw away disciples after them,\footnote{Acts 20:30} is compelled to formulate with increasing precision the pure doctrine which it possesses in order that the fraudulent errorists may be unmasked and false teachings be kept from creeping into it through ambiguous phraseology'}\end{displayquote} -- but this does not imply that the number of its dogmas grows; they are through this activity merely safeguarded ever more carefully against the danger of becoming perverted. 
                \par That Christ is with the Father, that the union of the divine and human nature in Christ took place, that “\textit{in, with, and under}” the bread and wine of the Lord’s Supper Christ’s body and blood are actually present, are given, and are orally received by worthy and unworthy communicants, -- these are, it is true, dogmatic expressions which were not found in the orthodox Church till the days of Arius, Nestorius, Eutyches, and Zwingli; but they are not new dogmas. \par Furthermore, we do not deny that through continued searching of the Scriptures by the Church some things are by and by cleared up which before, through imperfect acquaintance with the languages and history, had been unknown; we admit that in this manner the content of the various doctrines of faith at times is set forth and unfolded in a higher degree than before and that from this point of view we may indeed speak of a progress in knowledge.  But this by no means implies the gradual origin and increase of dogmas which modern theology teaches; we must rather say that through this course that which already is known receives new confirmation.\footnote{L.u.W., 14, 137. Translated: C.T.M, X, 7, 510, 511.}  In the first place it is not true that our dogmas come into existence gradually and that hence there are articles of faith \begin{displayquote}{\footnotesize `which are still in the process of formation, and others which as yet have either not at all or merely by way of beginning been drawn into the stream of events in which dogmas take shape’.}\end{displayquote}  It is not true that some articles of faith ‘\textit{have come down to us as undecided, unfinished question, incomplete structures, as open questions}’, because concerning these things ‘one does not yet find unanimous agreement’ in the Lutheran Church.  This theory, held and advocated with more or less emphasis by almost all modern theologians, though entirely unknown to the old orthodox theologians of our Church; as we view it, it is merely a daughter of Rationalism appearing in Christian dress, a sister of Romanism hiding behind a Protestant mask, and a fruitful mother of large families of heresies.  With respect to the Rationalists it is well known that they were the first to describe dogmas not as the unchangeable, divine, fundamental truths of Christianity but as doctrinal opinions which has arisen in a scientific process or which had been elevated by the various or which had been elevated by the various denominations to the position of ecclesiastical teaching and were considered authoritative in the respective age.
\par For this reason they strictly distinguished between doctrines of the Church and of the Bible... No proof is needed to show that Roman Catholics also teach the gradual rise of dogma; but a few years ago we beheld the spectacle of the present Pope’s declaring the teaching of the Virgin Mary’s immaculate conception, which before had been considered an open question, to be a dogma and now binding for all ‘\textit{believers}’, and just now\footnote{1868},  according to reports, the alleged heir of Peter’s episcopal throne is preparing to enrich his Church again through a new dogma by decreeing his own infallibility.  While modern Lutheran theologians are far removed from the position which would vindicate the right of the Roman Church or even the Pope to create new articles of faith, their theory that dogmas come into existence gradually, that on certain points a ‘\textit{unanimous consensus}’ arises, or that the Church has finally ‘\textit{pronounced}’ and ‘\textit{decided}’ with respect to such matters, is nothing but a sister of Romanism, having put on a Protestant mask. \footnote{L.u.W., 14, 133-136.  Trans.: C.T.M., X, 7, 507 and 508.}\end{fancyquotes} 

Of particular importance is the axiom championed by Dr. Walther: \begin{displayquote}“\textit{Every doctrine of the Bible is a doctrine of the Church.  He who hears the Scripture even from the humblest layman, he hears the Church, because the Church knows and confesses nothing else than the truth revealed in the Scripture.}\end{displayquote}  Walther writes:\begin{fancyquotes} What struggles it cost Luther to attain to this knowledge is well known...  Latter Luther finally realized that he had then really heard the Church when the humblest layman had convinced him with the Scripture.  Our modern Lutherans have returned to the condition of the Christians before the Reformation.  No matter what clear Scripture is brought them by a common Christian, they look upon this, in the language of {\scriptsize\textsc{(the theological faculty of)}} Dorpat, as merely \begin{displayquote}{\footnotesize ‘private and individual Christian convictions, however well grounded they may be, and the results, for the time, of conscientious and believing searching of the Scriptures’,}\end{displayquote} and await the decision of the Church, \begin{displayquote}{\footnotesize ‘because as yet there is no acknowledged standard for their ecclesiastical validity and the question of their Scripturalness is still an undecided point of contention’.}\end{displayquote}  Scripturalness is for them something to be decided not from the Scripture but by the Church.  That they should be hearing the Church when a miserable Missourian brings Scripture is to them a ridiculous idea.  For them the hearing of the Church requires first of all that the learned come together, discuss, dispute, and finally decide. \footnote{L.u.W., 14, 209.}
\end{fancyquotes} 

                Thus therefore Walther emphatically rejected the suggestion that only that is “\textit{Lutheran Church doctrine}” upon which our Church expresses herself in her Symbols.  No, every true Bible doctrine is Lutheran Church doctrine, even if it is not Lutheran Symbolical doctrine.  The Lutheran Church confesses in her Symbols by no means only those doctrines which, because of certain circumstances, she specifically mentions therein, but the entire Holy Scripture and all doctrines contained in Scripture.  \begin{fancyquotes}In regard to a heterodox Church that has set up a false principle and does not accept the Word of God as it reads, but insists on interpreting the Word either according to reason or according to tradition, the following statement cannot be upheld: \begin{displayquote}‘\textit{For her every doctrine of the Bible is a doctrine of the Church}’.\end{displayquote}  But this statement can be made of the truly orthodox Church and hence also of our dear Evangelical Lutheran Church.\end{fancyquotes}

                Hereupon Walther adduces passages of the Lutheran Confessions in which it is asserted that whosoever brings the Scripture, the Word of the prophets and apostles, causes the voice of the Christian Church to be heard.\footnote{L.u.W., 14, 208. Trans.: C.T.M, September, 1939, pp. 663-664} \begin{fancyquotes}That which truly belongs to the Church is always Biblical, and that which is truly Biblical always belongs to the Church, with a different (besonderen) faith; she does desire to be a part of the Church of the apostles and prophets, a part of the Bible Church.  She has indeed written Confessions and defined doctrines, not because they should contain her whole body of doctrine, nor because she had reached a decision only on those doctrines found in her Symbols, but because false churches and false teachers forced her to make clear-cut statements on certain doctrines.  Up to the present time she has seen no necessity for writing special Symbols on other doctrines.  All that she believes therefore is not found in her Symbols, but only in the Bible.  Her Symbols are not so much ‘\textit{the landmarks of spiritual development}’ as the boundary line separating her from certain falsehood.\footnote{L.c, p. 210.  Trans. L.c. pp. 664, 665}\end{fancyquotes}\divider \begin{fancyquotes}If  our Church makes claim only to Symbolical and not at the same time to canonical unity, as Gerhard calls it, i.e., to Biblical unity, then our Church is, we repeat it, not an orthodox Church, but a miserable sect, which does not bind itself to accept the whole Word of God but only certain doctrines thereof.  No matter how dear and valuable the incomparable Confessions of his Church are to every Lutheran, he does not permit them to become the Lutheran Bible, in which the whole faith of his Church is contained, while all other Biblical doctrines are nothing more than matters of ‘\textit{private and individual Christian conviction, however well grounded they may be}.’\footnote{L.c., p. 211. Trans.: l.c., p. 666}\end{fancyquotes}

                “\textit{It is indeed strange}”, Walther adds, \begin{displayquote}“\textit{that men who constantly speak against placing the Confessions above the Bible declare themselves bound as Lutherans only by those doctrines which are fixed Symbolically.  This fact makes it quite evident who those men are that actually stand on Scripture and believe in its supreme authority as well as in its clarity, and those who do not.}”\end{displayquote}\textbf{Pastor Hochstetter}, who took part in the colloquy arranged with the Iowa Synod in 1867 at Milwaukee, writes: \begin{fancyquotes}It was then first really clear to me \footnote{Pastor Hochstetter had recently come from the Buffalo Synod to the Missouri Synod} that the strength of the Missourian teachers lay not so much in their dependence upon the Symbols, as rather in their reverence for God’s Word!\footnote{Isaiah 66:2} There the maxim was: \begin{displayquote}{\footnotesize Everything is Church doctrine which is Bible doctrine, whether it is contained and established in the Symbols or not, if only it is in Holy Scripture}’.\footnote{Geschichte der Missouri-Synode, page 288}\end{displayquote}\end{fancyquotes}

%%% Local Variables:
%%% mode: latex
%%% TeX-master: "../main"
%%% End:
