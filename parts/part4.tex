\chapter{Lutheran Dogmaticians}

\hrule
\vspace{.30cm}
We have seen that Walther wanted the old church doctrine of inspiration maintained also for the sake of guarding the Scriptural principle of the Church of the Reformation.  We saw further that Walther, with the same end in view, when examining the theory of “\textit{open questions}” rejects every authority of the Church or its teachers to establish or validate dogmas.

\vspace{.30cm}
\hrule
\vspace{1.25cm}

                Despite this fact the charge was quite generally raised against Walther that his theology was a dead repristination\footnote{Restoration to an original state; renewal of purity.} of the doctrinal decisions of the old Lutheran Church and of the old Lutheran teachers.  The charge seems to have a certain justification if one looks only upon the outward form of most of the writings published by Walther.  For there is probably no Lutheran theologian who has so frequently cited Luther, the Lutheran Confessions, and the writings of the dogmaticians as Walther did.  He himself admits: \begin{displayquote}“\textit{The suspicion has indeed been cast upon us that our theology is a doctrinal traditionalism void of independent effort and a dead repristination}”\end{displayquote} -- because \begin{displayquote}“\textit{hitherto a continual supporting of our propositions with the testimonies of the older orthodox teachers of our Church has characterized our publications}.”\footnote{L.u.W., 21, 66.}\end{displayquote}

                But Walther very decidedly rejects that charge as unjustified.  As for the frequent citation of the teachers of the Church, he writes of himself and of those who have worked in a similar manner: \begin{fancyquotes}We are of the opinion that we have done it in a manner that whoever was open to conviction was also compelled to see that we did not follow those faithful teacher of our Church blindly but from personal conviction, that we do not mechanically repeat and imitate, but that we are their sons, so that at all times we could say: \begin{displayquote}‘\textit{I believed, therefore have I spoken}’.\end{displayquote}  They, the Confession and its confessors, have indeed been our leaders, but we have let ourselves be led by them into the Scripture, so that we have ultimately at all times and in all points been able to say: \begin{displayquote}{\footnotesize Now we believe, not because of your saying, for we have read for our ourselves and know that your doctrine is indeed the truth of God.  Incomparably valuable though the pure Confessions of our Church have been to us, still we never submitted even to them as to a doctrinal law imposed upon us, but we rather received them with joyous thanksgiving to God for His unutterable grace because we found in them our confession.  Our Lutheran Church in America had to fight many a strenuous battle with the proud sects, whom of course, we could not persuade with the testimony of our fathers, but whoever has witnessed these battles knows that God’s written Word has proved to be a victorious weapon also in our weak hands.\footnote{L.u.W., 21, 66, 67; translation in part from English version of Pieper’s Dogmatics, I, 165,166.}}\end{displayquote}\end{fancyquotes} That Walther, with all his citation of the Lutheran doctrinal fathers, held fast to the Lutheran Scriptural principle, to the principle that the canonical writings of the apostles and prophets, inspired by the Holy Ghost, are the only source of saving truth and the only judge in all doctrinal controversies, is evident even from the outward form of his writings and the many synodical essays which he presented.  However abundantly Luther, the Confessions, and the old Lutheran teachers may be given a voice in these writings, the Scripture proof is always placed first in the exposition of each point. \par Hence Walther always blamed the sainted Philippi the he, making a concession to the modern theological fashion, admits a threefold source from which Christian doctrinal theology is to draw its material\footnote{Baieri Compendium ed. Walther, Proleg. II. 91.}:
                \begin{enumerate}
                \item \textsc{The enlightened reason}
                \item \textsc{The dogmas of the Church}
                \item \textsc{The Scripture}
                \end{enumerate} 
                 Walther protests against such a coordination of Scripture and Church dogma when one is treating the “\textit{source}” of the Christian doctrine; the teacher of the Church shall definitely be left in their place as testes veritatis.

                What was then the reason why Walther, instead of expounding his whole subject chiefly in his own words, so predominantly let the old Lutheran teachers speak in his writings?  Also on this point he expresses himself.  “\textit{That we came forward in just this way}” – he remarks in the year \textsc{1857}--, \begin{fancyquotes} was forced upon us merely by the circumstances in which we were placed from the beginning and are still placed today.  We have unfortunately not enjoyed, as did our fathers, the unutterable benefit of fighting together with a great cloud of witnesses within our own church against her foes, but rather just those who with us bear the Lutheran name have been our most vehement opponents, who have tried to dispute our claim that our doctrine is that of the Evangelical Lutheran Church.  When we Lutherans in America again unfolded the good old banner of our Church and closed ranks around it, while roundabout us Zwinglianism, ‘\textit{enthusiasm},’ and rationalism were sailing under the Lutheran colors, the cry immediately went up:\begin{displayquote} \textit{Another new sect!} \end{displayquote}-- Some cried: \begin{displayquote}\textit{You are on the way to Rome!}\end{displayquote}-- Others:\begin{displayquote} \textit{You are unionistic!}\end{displayquote}-- Still others:\begin{displayquote} \textit{You are independents!} \end{displayquote}-- Others again: \begin{displayquote}\textit{You are Pietists, Enthusiasts, Donatists, Calvinist!}\end{displayquote}  -- And who can name all the sects that were said to have arisen and been reborn in us?  In short, we were said to be anything except what we expressly declared ourselves to be:  Confessors of the doctrine of the Reformation, Lutherans.  What could we do under these circumstances if we did not want to bear the name of a sect?  As long as men denied that we were true Lutherans, we were obliged to appeal constantly to our precious Confessions and the old faithful teachers of our Church as our witnesses.\footnote{L.u.W., 21, 66; translation in part from the English version of Pieper’s Dogmatics, I. 165.}\end{fancyquotes}
                So Walther himself speaks!  There is, however, yet another reason to be adduced in explanation of the form of Walther’s theological labors.  He believed that it was in the interest of the cause that he let his own words recede behind those of the old theologians.  It was his opinion that these could speak better than himself concerning the various doctrines.  We are firmly convinced that Walther was here involved in a bit of error.  Walther, as regards spiritual experience, theological learning, logical acumen, and the gift of presentation, certainly is not inferior to most of the old theologians of our Church; in our opinion he excels many of them in these respects.  In support of this judgment we appeal to the independent expositions of doctrine which Walther prefixed or affixed to the presentations of the old theologians.  Walther’s own expositions are not inferior to those of the old teachers in clarity and sharpness of conception, but frequently it is Walther’s presentation above all which first makes the matter really clear.

                Moreover, if one wants rightly to understand Walther’s attitude toward the teachers of the old Lutheran Church, he must take note of the following: Though Walther looked up to the theologians of the Old Lutheran Church with great veneration, yet he made a great distinction among them.  The theologians of the 17th century he regards as inferior to those of the 16th century.  He grants indeed that the former threw more light on certain points of doctrine and even gave a more precise formulation to some individual points .  But through the systematization of doctrine striven for during this period doctrinal purity already suffered diminution in some particulars.  Walther wanted a return to the theology of the 16th century, above all to the theology of Luther and the Lutheran Confessions.  He writes, also in the year \textsc{1875}:\begin{fancyquotes}The fact is that those who call our theology the theology of the 17th century do not know us.  Highly as we value the immense work done by the great Lutheran dogmaticians of this period, still they are not in reality; the ones to whom we have returned; we have returned, above all, to our precious Concordia and to Luther, whom we have recognized as the man whom God has chosen to be the Moses of His Church of the New. Covenant, to lead His Church out of the bondage of Antichrist, under the pillar of the cloud and the pillar of fire of the sterling and unalloyed Word of God.  The dogmatic works of the 17th century, though storehouses of incalculably rich treasures of knowledge and experience, so that with joy and pleasure we profit from them day and night, are nevertheless neither our Bible nor our confession; rather do we observe in them already a pollution of the stream that gushed forth in crystal purity in the sixteenth century.\footnote{L.u.W., 21, 67; translation from English version of Pieper’s Dogmatics II, 166.}\end{fancyquotes}  Walther desired chiefly to be a faithful pupil of Luther, “\textit{whose writings he confesses to have made his chief study}.”  In Luther he sees not just a theologian alongside others, but the one whom God Himself chose as the Reformer of the Church and the revealer of the Antichrist.  “\textit{Would it not},” he exclaims,~\\\begin{fancyquotes}be unspeakable ingratitude toward God, who sent us this man, if we would not hearken to his voice?  Then we should not have known the time of our visitation... God holds Christendom responsible for it if they do not recognize this man as the Reformer of the Church... Woe to the church if she will not use God’s instrument, but passes him by.  A church in which Luther’s writings are not studied by the pastors in the first place, and then at their incitement by the common Christians, certainly does not have Luther’s spirit, and Luther’s spirit is the pure evangelical spirit of faith, of humility, of simplicity.\footnote{L.u.W., 33, 305f.}\end{fancyquotes}
                
%%% Local Variables:
%%% mode: latex
%%% TeX-master: "../main"
%%% End:
