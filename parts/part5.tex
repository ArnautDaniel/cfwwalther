\chapter{``Scientific'' (Modern) Theology}

\hrule
\vspace{.30cm}

It remains for us to examine Walther’s position toward the theology of the present day.  He was a decided opponent of the modern “\textit{scientific}” theology.  Not as though he were an opponent of science in general.  In the Foreword to the Twenty-first Volume of “\textit{Lehre und Wehre}” he expressly defends himself against the “\textit{charge of contempt for science and a corresponding isolation from the intellectual activity of modern times}”.  He demonstrates that it is neither Biblical nor Lutheran, but fanatical, to despise science.

\vspace{.30cm}
\hrule
\vspace{1.25cm}

Here he breaks out into the following encomium upon science: \begin{fancyquotes}We have a lively appreciation of the fact that, with the exception only of God’s Word, science is of otherwise incomparable importance not only for the temporal prosperity of mankind, but also for the eternal welfare of the world, for church and theology, and that the contempt for this noble gift has always brought and must necessarily bring irreparable damage.  The spirit of Carlstadt, of the Anabaptists, and other fanatics, who despised science as something unnecessary, yea, dangerous and carnal, and instead boasted of inspirations of the ‘\textit{Spirit}’, has no place among us.  \par We are keenly aware, not only that all sciences enter into the services of sacred learning and can be applied thereto, but that in the absence of many of them, especially without thorough knowledge of the original languages of Holy Scripture, without knowledge of profane as well as sacred history, of the history of religion as well as church history, without knowledge of classical as well as of Biblical and ecclesiastical archeology, etc., a thorough and relatively comprehensive understanding of the Scriptures, and hence the attainment and preservation of pure Biblical doctrine, is not possible. \par We do not forget what unutterably valuable treasures of wisdom and experience the Christian Church through eighteen centuries until this present hour has stored up in writings in various languages or in a form which to the scientifically untrained reader is equivalent to an entirely foreign idiom, treasures which without scientific knowledge would all be lost to the church of today. \par We are keenly aware that it is only by way of general scientific studies through long years, and from youth up, that one may become a fully equipped theologian, and that it is only by this means that he can attain that practiced and acute mind, that habitus mentis, that intellectual readiness, which is absolutely essential as a condition \textit{sine qua non} for him who shall be able to establish and defend the divine truth against all kinds of opponents, not only to note and recognize every perversion of the truth and every antiscriptural error which arises against it in its full compass and destructiveness, but also to expose this to others and convince them thereof, to solve all linguistic, historical and logical difficulties and apparent contradictions in the Scriptures, to come to the help of all honest souls assailed by all kinds of doubts, to meet all objections of enemies of the truth, however hidden, in short, to clear up the muddy water of the adversaries’ sophistry and to defeat the foe, if possible, even with his own weapons. \par  We are not of the opinion that the church should flee into the wilderness, and for the sake of self-preservation isolate itself, shut itself off from all contact with the unbelieving world, let the enemies of the outside have everything their own way, give up to their own devices the antireligious cultured persons to whom the Gospel can be brought only in a certain form, and direct its efforts only toward uneducated people; no, we recognize it as our sacred duty to be made all things to all men, that we may by all means save some!  From our heart we agree with \textbf{Melanchthon} when he writes: \begin{displayquote}‘\textit{An unlearned theology is a veritable Iliad of evils}’.\end{displayquote}\end{fancyquotes}

                Walther himself in a footnote at this place refers to the fact that at the cornerstone laying of the college and seminary building at St. Louis he had shown in detail “\textit{that the church has always been a faithful upright friend and fosterer of art and science and according to her very essence and vocation must ever be so}”.  So contempt of science in general was not the reason why Walther took a position against the modern scientific theology. -- But the reason for this stand was also not the circumstance that this theology speaks in a new manner of divine things. \par Walther declares, \begin{displayquote}“\textit{as unyielding as he is determined by God’s grace to abide by the faith of the Apostolic Church and the Church of the Reformation in all points, as that which thoroughly agrees with Scripture, so little does he contend for the outward form in which this faith was presented in former times}.” \end{displayquote}  Yea, the form in which, for example, a number of the old Lutheran theologians present the Christian doctrine, the arrangement of the entire material according to the analytical method, and within the individual loci\marginpar{\scriptsize \textit{Loci} \\ The place where something is situated or occurs.} according to the causal method, was entirely contrary to his taste.  Rambach’s criticism of the “\textit{Aristotelian scholastic method}” he had made his own.\footnote{Baieri Comp. ed . Walther.  Proleg. P. 77.}

                Walther has quite a different complaint to make against the modern scientific theology.  What he has against it is that it places science in a false relation to theology.  Here we must first examine what relation Walther holds should obtain between science and theology.  It is evident from the above-quoted praise of science that he believes science should stand in a purely auxiliary relation to theology.  The knowledge of the original languages of Holy Scripture, of the text of Scripture, the knowledge of history and of antiquities should be applied for the purpose of better understanding the divine revelation in Scripture.  All intellectual training in general studies and particularly in logic should serve toward the end of clearly apprehending the divine doctrines as they are revealed in Scripture, of recognizing the opposing error, and of demonstrating its disagreement with Scripture.  But if science is not willing to serve in this manner indicated, but wants to rule, if instead of bringing the content of Scripture to light it wants to criticize, correct, and supplement it, in short, if science wants to sit in judgement upon the content of Scripture, then the God-pleasing relation of science to theology is entirely perverted.  Such a use of science is as unscientific as it is ungodly. \par Walther writes: \\\begin{fancyquotes} As necessary as we hold scientific knowledge, especially linguistic science, logic, rhetoric, and history, to be for searching out the content of Holy Scripture, we nevertheless want nothing to do with a science which, instead of being a maid and a learner over against Scripture, wants to play mistress and lord it over Scripture, which, instead of being serviceable for the discovery of the truth contained in Scripture, wants to sit in judgement and make decisions upon it, which instead of being corrected by Scripture wants to correct Scripture by its own principles, instead of remaining in its own sphere wants to elevate the rules which have relative validity in its own domain into universal principles and force their application also in the domain of Scripture.  Such a we hold to be as idolatrous as it is unscientific.\footnote{L.u.W. 21, 34.}\end{fancyquotes}  We place science, Walther further declares, not above theology, nor on the same level with it, but infinitely far below it. \begin{displayquote} “\textit{One passage of Scripture is for us incomparably higher and an immeasurably greater treasure than all the wisdom of this world}”.\footnote{l.c. pg 33}\end{displayquote}

                Hence in the case of “\textit{conflicts}” between Scripture and science, it is a \textit{priori} certain for him that science is in error. \begin{fancyquotes} No matter how confidently science may give out the results of its researches as absolutely certain truths, we hold not it, but Scripture to be infallible.  If the findings of scientific research contradict clear Scripture it is a \textit{priori} certain for us that they are nothing but certain error, even when we are not in a position to prove it such otherwise than by appeal to the Scripture.  Holy Scripture is certain for us under all circumstances, however great the conflict with the findings of science in which we may become involved by following this principle.  Hence, as often as we have to choose between science and Scripture, we say with Christ our Lord: \begin{displayquote} ‘\textit{The Scripture cannot be broken!}'\footnote{John 10:35}\end{displayquote} -- and with the holy apostle: \begin{displayquote}“\textit{Bringing into captivity every thought to the obedience of Christ}”\footnote{2 Corinthians 10:5}.\footnote{l.c. p. 35}\end{displayquote}\end{fancyquotes}

                Walther therefore requires of the theologian, in order that he may not pervert the relation between theology and science, that he hold the authority of Scripture as a \textit{priori firm} and unshakable.  Otherwise the theologian will make unjustifiable concessions, and instead of being useful to the church will harm the church with his work.  Walther expresses himself at some length regarding Biblical criticism and isagogics\marginpar{\scriptsize \textit{Isagogic} \\Introductory to the study of theology.}.  Of those who work in these disciplines he demands that they do not approach the Scriptures as doubter, but “\textit{with the presupposition that the written foundations upon which the Church of Christ rests stand unshakably firm.}” \par “\textit{A science}”, says he,\begin{fancyquotes}which first asks whether the foundations of the apostles and prophets is not perhaps, at least in part, a foundation of lies, we regard not as a Christian but a heathen science, of which nothing should be found in the Christian Church, except as an object to be combated and overthrown.
                But a science the aim, or at least result, of which is the unsettling of the basis upon which Christendom, as long as it exists, must stand and rest, we regard as nothing less than a weapon of the devil, and all those who employ it we regard as the devil’s servants. \par A Biblical criticism and isagogics which defeats the enemies of Scripture with their own weapons we esteem very highly; but if these disciplines make the slightest concession to the enemies on the interest of science against the basis upon which the Church stands, then we tread them under foot as traitors.  We do not wait for science to conquer the ground for us.  We already have it in possession, and it stands as firm for us, before all scientific investigation or examination, as our God who established it.  Whatever science may bring to light, it neither gives us faith nor takes it from us.\footnote{l.c., p. 36f.}\end{fancyquotes}

                Thus Walther defines the relationship between theology and science.  He finds that the modern scientific theology allows science to step out of its merely auxiliary position and makes it the mistress in theology.  “\textit{The maid has been promoted to mistress}”.\footnote{L.u.W., 18, 127.}  This theology, instead of defending the ground upon which the Christian Church stands, has demanded the surrender of this ground in the name of science.  It has designated the doctrine that Holy Scripture, because inspired of God, is God’s inerrant Word, as scientifically untenable.  That under these circumstances Biblical criticism and isagogics should still approach Scripture with holy awe is entirely impossible.  With the abandonment of inspiration, Scripture has become an object of criticism.  How much or how little of Scripture remains in force as divine truth depends upon the verdict of science which has set itself upon the judgement seat.  Instead of unhesitatingly placing itself upon the side of the Bible in any conflict between Bible and science, even the most positive representatives of modern theology grant in advance that in historical, geographical, natural historical and similar matters science may be in the right over against the Bible and in fact often is in the right.

                But also in the presentation of Christian doctrine itself, in dogmatics, modern theology has reversed the relation of science and theology.  Walther insists, with the old Lutheran theologians, that only the formal or organic use of reason is admissible.  The function of the theologian consists merely in deriving the individual doctrines from clear Scripture and arranging them in order.  “\textit{We agree}”, says Walther,  \begin{fancyquotes} fully with August \textbf{Pfeiffer} when he defines theology as follows: \begin{displayquote}‘\textit{Positive theology is nothing else than Holy Scripture arranged in a strict order according to a clear method into certain sections {\scriptsize\textsc{(loci)}}; hence no member, however small it may be, dare find a place in the body of doctrine which is not derived and supported from rightly understood Scripture}’.\end{displayquote}  And no less do we agree with Johann \textbf{Gerhard} when he writes:\begin{displayquote}‘\textit{The only principle of theology is the Word of God, and therefore what is not revealed in God’s Word is not theological}’.\end{displayquote}\end{fancyquotes} The proof for the correctness of the Christian doctrines is to be brought only and alone by adducing evidence that these doctrines are revealed in Holy Scripture.  No attempt is to be made to justify the mysteries of faith before the bar of human reason.  But modern theology -- in the interest of scientific method -- will not let Scripture be the source of theology, will not draw the Christian doctrines out of Scripture, but wants to derive and develop them out of “\textit{the religious faith of the Christians}”, out of “\textit{the Christian consciousness}”, out of “\textit{the enlightened reason}”. \par Only afterward will this theology institute an investigation of the Scripturalness of the independently discovered doctrine.  The old method, according to which the Christian doctrines were taken directly from Scripture is supposed to be “\textit{mechanical}”.  Walther sees in all this an apostasy from the principle of Christian theology.\footnote{L.u.W., 21, 225 ff.}  When it is pointed out that the Christian doctrines are drawn not from the unregenerate but from the enlightened reason, Walther answers:\begin{fancyquotes}Also the enlightened and regenerate reason can not be a principle of cognition beside the Scripture and coordinated with it, since it belongs to the essence of an enlightened and regenerate reason that it makes not itself but the Scripture its principle of cognition in matters of faith {\scriptsize\textsc{(2 Corinthians 10:5)}}, aside from the fact that a perfectly renewed and enlightened reason exists in no man here below. {\scriptsize \textsc{(Genesis 18:10-15)}}\footnote{L.u.W., 13, 99, Anmerk. 1}\end{fancyquotes}--But as the modern theology, in order to be scientific, will not simply take the Christian doctrines from the Scripture but from the Ego of the theologian, so it also, in the same interest, will not confine itself to demonstrating the correctness of the Christian doctrines by the appeal to Holy Scripture, but regards it as its peculiar function to elevate the Christian doctrines into “\textit{absolute truth}”, that is to demonstrate them to be truth also independently of Scripture, in short, to justify the Christian faith before the bar of reason.  Walther, to the contrary, holds that it is in conflict with the essence of the Christian articles of faith to try to rediscover them by way of speculation or even to demonstrate them a posteriori on the basis of reason.  The result would be the destruction of faith and of the articles of faith.  “\textit{However great a service}”, he writes: \begin{fancyquotes}may seem to be rendered thereby to Christian theology, we are nevertheless sure that such supposed demonstrations are not only nothing more than a deception, but also that, instead of explaining and proving the mysteries of faith, they rather alter and entirely destroy them according to their essential content, and only thereby produce the appearance of a demonstration and reproduction of the Christian mysteries of faith.  With our whole heart we hate all such apologetics, for it assumes that there is something surer than Gods Word, from which surer source the mysterious content of revelation can be derived by way of discursive thinking.

                But God himself says of His mysteries that they were kept secret since the world began, but now made manifest, and by the Scriptures of the prophets, according to the commandment of the everlasting God, made know {\scriptsize\textsc{(Romans 16:25-26)}}, that they are the contents of what is to human reason the foolishness of preaching, which natural man receives not, which is indeed foolishness to him, yea, that they are a light which God commanded to shine out of darkness.\footnote{1 Corinthians 1:21; 2:14; 2 Corinthians 4:6}
\end{fancyquotes}
                Walther on the one hand, is of the firm conviction that between Christian theology and true science, science \textit{in abstracto}, no real contradiction exists or can exist.  On the other hand, he does not hold it to be the function of a theologian, or to be at all possible, to harmonize theology with science as it presents itself \textit{in concreto}.  One must therefore desist from the attempt to point out to the world the harmony between Christian faith and science.  He writes:\begin{fancyquotes}We are firmly convinced that the present apostate world cannot be helped by the lie that the divinely revealed truth stands in the most beautiful harmony with the wisdom of this world, but only by the preaching of the divine foolishness, the old unadulterated, of which Paul and the history of the church of all ages and of each individual Christian bears witness that it is the \textit{power of God unto salvation to everyone that believeth; to the Jew first, and also to the Greek}. \par A man who has been won for Christianity by the demonstration that Christianity stands the sharpest test of science has not yet been won, and his faith is no faith.  The instruction which Christ’s servants have for “\textit{conquering the world for the Kingdom of Christ}” reads: \begin{displayquote} “\textit{Go ye into all the world and preach the Gospel to every creature.  He that believeth and is baptized shall be saved; but he that believeth not shall be damned.}'' \end{displayquote} In this instruction we hear nothing to the effect that they shall offer to the world the scientific solution of its questions: ‘\textit{How can these things be?}’ or ‘\textit{Whereby shall I know this?}’.  \par No, as ‘\textit{ambassadors for Christ}’, in the name of the great God they shall ‘\textit{testify}’ to the world ‘\textit{repentance toward God, and faith toward our Lord Jesus Christ}’; if they have done this, then they have fulfilled their mission to the world, and as many of their hearers as are ordained to eternal life will believe.\footnote{Acts 13:48.}  Men may depreciate such theology in this scientific age: but this \textsc{is} the theology of the prophets and apostles, by which we intend to abide until our death!\footnote{L.u.W. 21, 41f}
\end{fancyquotes}
                Because the modern theologians conceive of theology as the science of Christianity, therefore the Christian doctrines should now also constitute a whole in the sense of reason.  It shall be the task of theology to demonstrate how the individual doctrines harmonize with one another.  Walther, to the contrary, emphasizes that two doctrines, which indeed seem to exclude each other, but are nevertheless both clearly revealed in Scripture, are both to be held fast; the light of glory will bring us the solution of the seeming contradiction.  Walther treated this point in the article: \begin{displayquote}“\textit{What shall a Christian do when he finds that two doctrines which appear to contradict each other are both clearly taught in Scripture?}”\footnote{L.u.W. 26, 257 ff.}\end{displayquote} He concludes that article with the words of Luther: \begin{displayquote}“\textit{If it should be a matter of harmonizing we would retain no article of our faith.}”\end{displayquote}

                And what is the result, according to Walther, at which the modern theology has arrived by its desire {\small(making a science out of theology)} to elevate faith to knowledge to present Christian doctrine {\scriptsize\textsc{ (both as to the individual doctrines themselves and as to their connection)}} in such a way as to take account of the “\textit{intellectual requirements}” of the Christians and of the world?  The representatives of modern theology have asserted that they are merely teaching old truth in a new manner, and that, where alterations were to be made over against the former mode of presentation, this was required by the progress in scientific knowledge.  Walther, to the contrary, asserts that this theology does not merely present the Christian doctrines in a new manner, but entirely alters the content of them, that what this theology calls “\textit{progress}” is a surrender of the Biblical churchly doctrine and a retreat to old errors.  Walther brings the evidence for his assertion in the well-known article “\textit{What is the situation with regard to the progress in doctrine of the modern Lutheran theology?}” which article is continued through three volumes of “\textit{Lehre und Wehre}''\footnote{Volumes 21, 22, 24}.  By the extracts which he here adduces from the old dogmaticians and the writings of the chief representatives of the modern theology he intends to maintain “\textit{that the modern Lutheran theology is not a progress or further development of the old, but an entirely new, different theology-- a most decided apostasy from the latter.}”\footnote{21, 161}

                Elsewhere Walther summarizes his judgement upon the modern theology by declaring at the same time in what sense there can be and is a “\textit{progress in doctrine}” in the course of time: \begin{fancyquotes}Not a greater precision in the presentation of the old doctrine, not a richer proof thereof from the Scripture, not a previously unoffered triumphant demonstration that the newly arisen teachings have long ago been condemned by the old, certain, unshakably firm doctrine which has stood the test of time, but on the contrary entirely new teachings, not development but alteration, not establishment but correction, not defense but liquidation, destruction, surrender, and professed refutation of the old doctrine, and not only of one or the other non-fundamental doctrine, but of the fundamental doctrines of our Church, yea, a direct overthrowing of its foundation- that is what is now commended to us as development and progress, and this in our Lutheran Church, and which we are supposed to acknowledge as doctrinal development and doctrinal progress. \par It is as though the leading spokesmen also within the so-called Lutheran Church of our time, with very few exceptions, had made an unspoken agreement to distribute their efforts among the various Loci of our Lutheran doctrinal structure, and had assumed the duty the one to overthrow this and the other that article, in order finally to bring it about that each article should either be eliminated from Lutheran dogmatics, or at least essentially altered, so that an entirely new Christian religion, reconciled with the supposed results of scientific research, and acceptable to our progressive age, might come into being.\footnote{L.u.W. 21, 69}\end{fancyquotes}

                Hence, although Walther also acknowledges that the researches of the modern theologians “\textit{have in many departments brought and are still bringing to the Church a fund as rich as it is valuable}”\footnote{l.c., p.68} and that he desires to have “\textit{every real gain resulting therefrom}” fully utilized, he nevertheless until his end most emphatically warned against the entire character of the modern “\textit{scientific}” theology as “\textit{the transmutation of the Christian religions into a human science}”. \footnote{L.u.W., 32, 6}

                Having up to this point shown what Walther understood by theology, as well as the position he assumed toward the Scripture and toward the teachers of the Church, we intend in the following articles to present Walther’s position in the individual doctrines, especially those which have come into controversy in this country.

%%% Local Variables:
%%% mode: latex
%%% TeX-master: "../main.tex"
%%% End:
