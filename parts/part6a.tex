\chapter{The Church I}

\hrule
\vspace{.30cm}

It remains for us to characterize Walther’s position in certain individual doctrines which came into controversy.

\vspace{.30cm}
\hrule
\vspace{1.25cm}

                In the first place, however, we must take note that Walther was not a theologian who cherished and cultivated certain favorite doctrines and for their sake neglected other doctrines which are just as clearly revealed in God’s Word.  That has indeed been the habit of not a few men who have become famous in the church.  Thereby they revealed that they indeed stood at the head of a sect, but could not work in a truly churchly manner. \par No, Walther was a true church theologian, who with the greatest faithfulness sought really to teach and to maintain all which is entrusted to the Church in Holy Scripture.  Hence, although he, on the one hand, well knew how to distinguish between the individual doctrines with regard to their absolute necessity for the engendering and preservation of faith, yet, as his teaching in the theological seminary testifies, he held to all the doctrines of the Christian faith with the greatest diligence.\footnote{Cf. Pastorale, p.90 f}  Circumstances, nevertheless, brought it about that Walther had to devote very particular attention and labor to certain individual doctrines.  And to Walther’s position in these doctrines we shall turn our attention in what follows.

                The doctrine which, immediately after their arrival, not only occupied the attention of the Saxon immigrants but became a most vital question for them is the doctrine of the Church.  “\textit{We are no longer a church}”, was the thought in the hearts of many, when the man whom most of them followed with the utmost confidence as their leader and bishop, fell away, and thereby as with one blow the structure of the church which they had hitherto regarded as the true Church was demolished.  In this situation it was principally Walther who convincingly answered from the Scripture, the Confessions, and the writings of Luther, the question, what the Church is, and thus effectively warded off the confusion which threatened to disrupt the little congregation.
\divider
How Walther and the \textit{Missouri Synod} came to the doctrine of the Church, as it is set forth, for instance, in Walther’s book, “\textit{Die Stimme unserer Kirche in der Frage von Kirche und Amt}”, is a matter concerning which quite false views are current still today in Germany.  It is said that Walther fashioned the doctrine according to democratic American conditions.  But the exact opposite is the case. \par In the first place, the immigrants were still very little acquainted with “\textit{American church conditions}'', at the time when the question of Church and Minister was already decided among them.  And when at a later time they came into closer contact with these “\textit{American}” conditions, then it was not these which exercised a decisive influence upon them, but it was they who exerted a deciding influence upon the conditions.  Says Walther: \begin{displayquote}“\textit{We set ourselves with all our might against the abuses prevailing in American church circles. In many circles we succeeded in doing away with the hiring of pastors and the absolute power of the congregation}”.\footnote{We again call attention to the fact the we are citing Walther according to manuscript notes wherever we do not make specific reference to a printed writing.}\end{displayquote} To be sure, the conditions into which God placed the little flock of immigrants were the occasion which led to their recognizing the doctrine of the Church which they now championed as the true doctrine.  But this doctrine itself is not derived from the circumstances, but in time of intense temptation and great tribulation was achieved through the study of the Word of God, the Confessions, and especially the writings of Luther. \par Walther himself writes in the Foreword to “\textit{Kirche und Amt}”\footnote{Church and Ministry}:
\begin{fancyquotes}Willingly as we grant that the conditions under which we live here in America were of decisive influence in leading us to the vital recognition of the doctrine of Church and Ministry laid down in this book, so that we hold it fast as a precious treasure and now confidently confess it before the whole world: we must nevertheless decidedly reject the charge that we have bent and fashioned the holy pure doctrine of our Church in the interest of the conditions and circumstances surrounding us.  \par Since we are here living not under inherited ecclesiastical conditions, but are rather in a position which requires that we lay the foundation for such, and in which also we are able to lay it unhindered by anything already existing, these circumstances have therefore the rather impelled us with great earnestness to search for the principles upon which according to God’s Word and Confessions of our Church the polity of a truly Lutheran fellowship must rest, and according to which such polity must be formulated. \par The less the question arose: \begin{displayquote} {\footnotesize What can we retain without sin?}\end{displayquote} And the more we were occupied with the question: \begin{displayquote}{\footnotesize Who should it be in accordance with God’s Word and the principles expressed and demonstrated in our Church’s Confession?}\end{displayquote} -- so much the more urgent for us was the need of coming into the clear and arriving at a firm assurance of faith concerning the principles of the doctrine of the \textbf{Church}, \textbf{Ministry}, \textbf{Power of the Keys}, \textbf{Church Ordinances}, and the like.  We have not fashioned the doctrine of our Church according to our conditions, but have ordered these according to the doctrine of our Church.  To anyone who doubts this we confidently issue the summons: \begin{displayquote}\textit{Come and see!}\end{displayquote}  And he who with astonishment finds principles presented by us as principles and doctrines of the Lutheran Church which he has hitherto abominated as fanaticism, -- him we can confidently directly to the references which we have adduced in proof, and leave him the choice of either granting us the praise of Lutheran orthodoxy or denying it to the entire cloud of faithful witnesses from Luther down to a Baier and a Hollaz.\footnote{Kirche und Amt, 3rd Edition, Foreword, VIII; 4th Edition, Foreword VIII, IX.}\end{fancyquotes}

Over against the assertion that the doctrine of Church and Ministry expressed in our Confessions is “\textit{still undeveloped and unclear}” Walther says in the same Foreword: \begin{fancyquotes} We are of the firm conviction that the reason Lutherans are now divided over the important doctrines of Church and Ministry and all which is directly connected therewith is that they have disregarded and turned aside from the doctrine laid down in the public Confessions of our Church and developed in the private writings of her orthodox teachers.\par  We are of the firm conviction that our Church has not left the doctrines of Church and Ministry unexamined, so that they now still await development; much less has she in any manner obscured these doctrines or assigned them an unfitting place in the entire structure of doctrine, so that they must now still be readjusted. \par We are of the firm conviction that the great decisive conflict of the Reformation which our Church fought in the Sixteenth Century against the Papacy revolved about these very doctrines of the Church and Ministry which have now again come into question among us, and that the pure clear doctrine on this subject is a precious spoil which our Church won in that conflict.\footnote{Kirche u. Amt, V, VI}\end{fancyquotes}
\divider
                What is the Church in the proper sense?  Walther in his instruction in Dogmatics designated this \textit{a priori} as the main question and the determinative point in the entire \textit{locus doctrinae} concerning the Church and all that is connected therewith.  “\textit{The main thing is to know what the Church is properly and essentially}”.

                What the Church is, is something which was not known in the Papacy before the Reformation, nor was this knowledge desired.  A man who knew it and spoke out about it was burned at Constance.\footnote{See the citations from Aegidius Hunnius and Luther in Walther’s Baier III, 614, 619}  Through Luther it again became known what the Church is, and so well known that Luther could write in the Smalcald Articles\footnote{Smalclad Articles -- (Part III, Art. XII; Mueller, p. 324; Triglotta, pg 499}: \begin{displayquote}“\textit{Thank God, a child seven years old knows what the Church is, namely, the holy believers and lambs who hear the voice of their Shepherd.  For the children pray thus: ‘{\small I believe in one holy Christian Church}’}.”\end{displayquote}  In our time this children’s wisdom has again become almost as unknown to many who hear the Lutheran name as it was under the papacy.  To the question as to what the Church is, even such as have considerable reputation in the Lutheran Church give the most various answers, only not the simple and only correct answer, that the Christians are the Church. As essential parts of which the Church is supposed to consist the following are mentioned: \begin{itemize} \item Christ \item the means of grace \item godly and ungodly \item the office of the means of grace or  the order of teachers and learners \item the order of those who rule and \item those who obey in a definite ecclesiastical constitution.\footnote{Cf. The extracts from the writings of recent theologians, L.u.W., 16, 162 f.}\end{itemize}
                From these and other parts men constructed for themselves the “\textit{Church}”. To the most the Church is an “\textit{outward polity}”, an “\textit{institution}”, in which the Christians form a more or less essential component part, only that they are not themselves the Church. --  It is obvious that, with the existent confusion with regard to the concept of the Church, especially with the conception of the Church an “\textit{institution}”, the much lamented evils of the church cannot be rectified.  How shall one help the church if one does not know what the Church properly is.  If the Church were held to be what it is, the congregation of believers, then care would be directed principally to that thereby believers, children of God, are born and preserved, namely, the preaching of the pure doctrine, and that whereby faith is hindered and destroyed, namely, false doctrine, would be decisively opposed and removed.  \par But since the Church is held to be essentially an institution and a sum of ordinances and relations, care for the welfare of the church is consequently exhausted in the care for the maintenance or restoration of ordinances; yes, in this way everything which could disturb the ecclesiastical “\textit{institution}” {\scriptsize\textsc{(or establishment)}} is anxiously avoided.

                According to Walther the Church is the totality of believers, nothing more and nothing less.  Nothing more: for to the Church belongs no unbeliever or unregenerate person, even though such a one may be in the outward fellowship of the Church, yea, even occupy the highest offices in the same.  Not less: for all believers on the whole earth belong to the Church, whether they are in the visible fellowship of the orthodox Church, or are held under the sects and the Papacy.\footnote{Lutheraner, XI, 17,18}; also those who have been wrongly excommunicated, if they have faith, belong to the Church, as well as those who have not yet been formally received into the Church by Baptism, if they have already come to faith through the Gospel.  In short, only living faith in Christ is decisive of membership in the Church.  \par In Walther’s work, “\textit{Die Stimme unserer Kirche}”, the first two Theses “\textit{of the Church}” read this:\begin{displayquote}``\textit{The Church, in the proper sense of the term, is the communion of saints, that is, the sum total of all those who have been called by the Holy Spirit through the Gospel from out of the lost and condemned human race, who truly believe in Christ, and who have been sanctified by this faith and incorporated into Christ.  To the Church in the regenerated, no heretic}.''\end{displayquote}  Walther proves this with texts such as {\scriptsize\textsc{Ephesians 1:22-23; Ephesians 5:23-27}}, where Christ is called the Head of the Church and the Church is called Christ’s body, where the Church is described as “\textit{subject unto Christ}” and “\textit{sanctified}” and “\textit{cleansed}” by Him.  He remarks on {\scriptsize\textsc{Ephesians 1:22-23}}: \begin{fancyquotes}Since Christ, according to this text, is the Head of the Church and the latter is His body, the true Church, properly so called, is the sum total of all those who are united with Christ as the members of a body are with their head; and on {\scriptsize\textsc{Matthew 16:18}}: \begin{displayquote}`\textit{Upon this rock I will build My Church; and the gates of hell shall not prevail against it}'\end{displayquote} The Church, then, in the proper sense of the term, is built, as regards its members, on the rock of Christ and His Word.  Upon this rock, however, only he is built who by a living faith makes it his foundation.  -- Thus writes St. Paul\footnote{Romans 8:9;  Translation appears to be partially paraphrased.}: \begin{displayquote}`\textit{If any man have not the Spirit of Christ, he is none of His}’.  Now if a person does not belong to Christ, neither is he a member of the true Church, which is His spiritual body.\footnote{Kirche und Amt, 4th Edition, pp. 1,2, and 10.}\end{displayquote}\end{fancyquotes}
\divider
                To designate the relation in which the godless stand to the Church Walther liked to use the expression of \textbf{Gerhard}: “\textit{The godless are indeed in the Church \footnote{according to external fellowship} but not of the Church}”, and \textbf{Calov’s} word: \begin{displayquote}“\textit{Although the hypocrites are in that multitude in which the Church is, yet they are not properly in that multitude which is the Church}.”\end{displayquote}  Between believers and hypocrites, even if they are externally in the same fellowship, there remains always as great a difference as between Christ’s Kingdom and the devil’s kingdom.  According to Walther the reason why people make Christ, the means of grace, the office of the ministry, etc., essential component parts of the Church is because one represents that which is necessarily connected with the Church as the Church itself.  Against this “\textit{error so widely current in our time}” Walther extracts\footnote{L.u.W., 9, 284} the following from the “\textit{Mecklenburgische Theologische Zeitschrift}”: \begin{fancyquotes} That which cannot be separated from the Church, without which the Church cannot exist, which therefore in some way necessarily belongs to the Church, still is not included in the Lutheran concept of the Church as such, does not belong to that which makes up the Church, the communion of saints, Christendom, as such. \par Thus man cannot live without air and daily bread, but air and daily bread do not belong to the concept of man; the human race cannot exist without the earth upon which it lives, and without the heaven which arches itself above it, and without the sun which rises upon it with its light and warmth, nevertheless the concept of the human race is distinct from all this, does not coincide with the concept of the universe.  Christ, the Head of the Church, is inseparable from the Church, which is His body; the existence of the Church would be eliminated with her separation from the Head, from the Lord who dwells in her and works in her through the means of grace, yet Christ does not belong to the concept of the Church, which is the body of Christ, and as such distinct from the Head. \par The same holds true of the means of grace, of the Word and Sacraments.  Through them the Church receives her life from the Head, and without them the Church lacks the basis of her existence; nevertheless they do not belong within the scope of the Lutheran concept of the Church; inseparable from her they are yet distinct from her.  The means of grace have been given to the Church by the Lord, the Church has them, uses them, lives by them, in the Church they are administered in the service of the Lord, that the working of the Lord, through them, increasing and perfecting the Church, may ever go forward, but they themselves are not in any respect the Church.  Therefore the means of grace, rightly administered, are also designated as the \textit{notae} of the true Church.  They are called such not because in them a part of the Church, as it were, emerges into visibility, but because it is assured to faith according to the Word of God that the means of grace where they are rightly administered will not remain without fruit.  For Lutheran doctrine the questions: \begin{displayquote}{\footnotesize What is the Church?}\end{displayquote} And: \begin{displayquote}{\footnotesize Who belongs to the Church?}\end{displayquote}  Are indistinguishable; for the Church is the communion of believers.\end{fancyquotes}

                Since the Church is essentially the communion of believers, it is invisible.  Walther refers to the following texts: \begin{displayquote}“\textit{The Kingdom \marginpar{\scriptsize Luke 17: 20-21}of God cometh not with observation: neither shall they say, Lo here! Or, lo, there! For behold, the Kingdom of God is within you}.”  \\\par “\textit{the Church \marginpar{\scriptsize 1 Peter 2:5}is a spiritual house in which spiritual priests offer up spiritual sacrifices, acceptable to God; and hence is invisible}.”\footnote{Die evangelisch-lutherische Kirche die ware sichtbare Kirche, p. 11.} \\\par “\textit{the Lord alone\marginpar{\scriptsize 2 Timothy 2:19} knows them that are His; now, only those who are the Lord’s constitute the true Church; hence no man can see the Church}”.\footnote{Kirche und Amt, p. 15.}\end{displayquote}  Walther writes in the first volume of the “\textit{Lutheraner}”\footnote{Lutheraner, p. 83}: \begin{fancyquotes} The Church is not a visible institution like a state, but an invisible kingdom, a spiritual building erected by the Spirit of God in the hearts of men... \par  ...It is indisputable\footnote{John 18:36; Luke 17:20-21} that the true Church of Christ is, properly speaking, never visible.  It cannot be otherwise.  For since only truly believing regenerate Christians are members of the Church, no one can say: these or those people are the Church; for everyone should and can become and be sure, as concerns himself, that he is in Christ and Christ in him; but no one can be infallibly sure concerning another man whether he is a child of God, whether he is a living stone of the spiritual house of God or the Church.  Even as Solomon says: \begin{displayquote}‘\textit{God only knoweth the hearts of the children of men}.’\footnote{2 Chronicles 6:30}\end{displayquote} Hence we confess: \begin{displayquote}‘I believe a Church’, ``\textit{but faith is the substance of things hoped for, the evidence of things not seen}’.\footnote{Hebrews 11:1}”\end{displayquote}  And though the men who form the Church can be seen, yet, since they are seen as bodily men, not as spiritual men who belong to the house of the Church\footnote{1 Peter 2:5}, it still remains true that the Church, as a spiritual house built up of spiritual men, is invisible.\footnote{Kirche und Amt, p. 22; Lutheraner, I, 21.}  Hence the holy Christian Church here on earth is invisible at all times, not only in times when the Papacy ruled, but also in times when the light of the Gospel shines brightly upon the nations.\footnote{Kirche und Amt, p. 21.}

                Through the preaching of the Word and the administration of the Sacraments the Church is indeed recognized in its presence but not visible in its essence, even as the soul clearly manifests its presence in the body, but without itself becoming visible.\footnote{Lutheraner, VI, 9; I, 83; VIII, 42.}\end{fancyquotes}

%%% Local Variables:
%%% mode: latex
%%% TeX-master: "../main"
%%% End:
