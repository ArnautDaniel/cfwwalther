\chapter{The Church II}

\hrule
\vspace{.30cm}

The one holy Christian Church, as consisting only of the true believers, is and remains invisible as to its essence.  But, though the Church itself cannot be seen, yet the place where the Church is to be found can be designated.  The Church is wherever the seed of the Church is, namely, the Word of God and the Sacraments.  The means of grace are indeed not an essential part of the Church, but they are marks of the Church, and they are so because they are the means whereby alone the Church is established and preserved, as well as the treasure entrusted to her {\scriptsize\textsc{(the Church)}}, which she alone administers, guards, and transmits to others.\footnote{Lutheraner 1, 83.} \begin{displayquote} “\textit{As the star showed the wise men from the east the house in which the Christ-child lay, so the heavenly light of the Word of God shows the house in which Christ dwells, namely, the Church}”.\footnote{L.c.}\end{displayquote}
\vspace{.30cm}
\hrule
\vspace{1.25cm}
Walther expresses this in “\textit{Kirche und Amt}”\footnote{Thesis V, p. 52}, as follows: \begin{displayquote} “\textit{Although the true Church, in the proper sense of the term, is invisible as to its essence, yet its presence is perceivable, its marks being the pure preaching of the Word of God and the administration of the holy Sacraments in accordance with their institution by Christ}.”\footnote{Translation from “Walther and the Church”, p. 60}\end{displayquote}  From {\small \textsc{Mark 4:26-27, 14}} and {\small \textsc{Isaiah 55:10-11}} Walther takes the following instruction: \begin{fancyquotes}The Word of God is not only the seed from which alone the member of the Church are born, but also from it there certainly bud forth always, wheresoever this heavenly seed is sown, some ‘\textit{children of the kingdom}’, ‘\textit{without men’s knowing how}’, in accordance with the divine, undeceivable, and infallible promise.  Wherever, therefore, this seed is sown, there the Church indeed is not seen, but there we have an undeceivable mark {\scriptsize\textsc{(criterion)}} that the Church, that a group of true believers and saints in Christ Jesus, a congregation of children of God, exists.\footnote{Kirche und Amt, p. 53.  Translation: Walther and the Church, p. 61.}
\divider
 According to Holy Scripture, however, also the holy Sacraments, besides the Word of God, are the means by which the Church, the holy congregation of God, is to be founded, gathered, preserved, and is to be spread.\footnote{Matthew 28:18-20; Mark 16:16} Hence in every place where, besides the use {\scriptsize\textsc{(application)}} of the Word, Holy Baptism is administered, the portals of the Church are opened invisibly; there persons are found who believe and are saved; there the Lord is present with His grace; there we have an undeceivable mark that the Church exists in that place; there we must say with Jacob: \begin{displayquote}‘\textit{Surely the Lord is in this place, and I knew it not.  How dreadful is this place!  This is none other but the house of God, and this is the gate of heaven}’\footnote{Genesis 28:16-17}.\end{displayquote} Scripture says the same regarding the Holy Supper of the Lord.\footnote{1 Corinthians 10:17; 1 Corinthians 12:13}  Therefore where the Word of God is preached, Holy Baptism and the Sacrament of the body and blood of Jesus Christ are administered, there are members of the body of Jesus Christ.  There we must believe: Here is a holy Christian church.\footnote{L.c., pp. 53, 54.  Translation from Walther and the Church, pp. 61-62.}\end{fancyquotes}

                Hence also the Holy Scripture speaks not only of the Church in general {\scriptsize \textsc{(Matthew 16:18; Ephesians 1:22-23; 5:27)}},  but also of churches in particular places {\scriptsize \textsc {(1 Corinthians 16:19)}}, of the churches of Asia {\scriptsize\textsc{(2 Corinthians 8:1)}}, of the churches of Macedonia {\scriptsize\textsc{(1 Corinthians 1:2)}}, of the church of God at Corinth {\scriptsize\textsc{(Acts 8:1)}}, of the church at Jerusalem.  Furthermore, when Christ commands to feed His sheep {\scriptsize\textsc{(John 21:16-17)}}, and Paul to feed the church of God {\scriptsize\textsc{(Acts 20:28)}}, and Peter to feed the flock of Christ {\scriptsize\textsc{(1 Peter 5:2)}}, it is likewise presupposed that the believers can be found in certain places\footnote{Kirche und Amt, p. 56. Lutheraner 1, 83}.  These are local churches or particular churches.

                In what relation do the local churches stand to the \textit{una sancta}?  The sum total of the local churches {\scriptsize\textsc{(naturally with the addition of individual believing souls who are cut off from all church fellowship)}} makes up the one Church scattered over the whole earth.  To the words of \textbf{Baier}: \begin{displayquote} “\textit{The whole Church is related to the individual congregations of believers as a whole of the same kind, which has the same character and the same nature as its parts}”.\footnote{Locus de ecclesia, par. 19, note d.}\end{displayquote} Walther adds: \begin{displayquote}“\textit{as the drops in a pond are of the same character as the whole pond}”.\end{displayquote}  Just as the godless and hypocrites do not belong to the \textit{una sancta}, so also they form no part of a particular church, when one adheres to the proper significance of the word church.  Walther does not want the point “\textit{overlooked}” which is made by J.B. \textbf{Carpzov}\marginpar{\scriptsize\textit{Johann Benedict Carpzov II}\\ (24 Apr 1639 – 23 Mar 1699) was a German Christian theologian and Hebraist. He was a member of the scholarly Carpzov family.}: \begin{fancyquotes}\par...a group which consists of hypocrites and true upright believers is something entirely different from a group wherewith hypocrites are mingled.  The Church, properly so called, is not a group which consists of hypocrites and unholy persons, but it is a group wherewith hypocrites and unholy persons are mingled. So the Augsburg Confession at the beginning of the Eighth Article {\scriptsize\textsc{(in the Latin text)}} discriminatingly declares.\end{fancyquotes}  Also what the old theologian \textbf{Dannhauer}\marginpar{\scriptsize\textit{Johann Dannhauer} \\(24 Mar 1603 - 7 Nov 1666) was an Orthodox Lutheran theologian and teacher of Spener.} expresses as follows: \begin{displayquote}{\textit{``They {\scriptsize\textsc{(the hypocrites)}} are indeed not members of the invisible Church, also out of the true visible Church, and yet are members of the visible church insofar as they with others, the true members thereof, make up one whole.''}\end{displayquote} \par Finally \textbf{Calov} writes: \begin{displayquote}‘\textit{Although the hypocrites are in that group in which the Church is, yet they are not properly in the group which is the Church}’.\footnote{Die Rechte Gestalt, p. 4.}\end{displayquote}  Hence Walther defines a Lutheran local congregation as follows: \begin{displayquote}“\textit{An Evangelical Lutheran local congregation is a gathering of believing Christians at a definite place, among whom the Word of God is preached in its purity according to the Confessions of the Evangelical Lutheran Church and the holy Sacraments are administered according to Christ’s institution as recorded in the Gospel}.”\footnote{Die Rechte Gestalt, p. 1, Translation from Walther and the Church, p. 88.}\end{displayquote}  False Christians and hypocrites are only “\textit{mingled with}” the local congregation.  Walther ever and again reminds us that we must not imagine a “\textit{double church}”, namely one church which consists of believers only and another which is composed of believers and unbelievers, but -- so he expounds it -- the word church is used in a double sense, first, in the proper sense, for the invisible communion of believers, then, in an improper sense, for the visible communions of those who are gathered about God’s Word in which the believers find themselves.  But the visible communions are called churches only on account of the believers contained in them – thus synecdochically, -- not insofar as they are made up of believers and hypocrites\footnote{Kirche und Amt, Thesis VI, p. 63f.} --  \begin{displayquote}“\textit{The whole bears this glorious name merely on account of a part of it, to which alone this name belongs in the proper sense}.”\footnote{L.c., translation from Walther and the Church, p. 63.}\end{displayquote}

                The visible communions and particular churches are called churches synecdochically, but not by a misuse of the term.  The Scripture itself, although it clearly teaches that only the true believers are real members of the Church, nevertheless accords the name “\textit{church}” also to such mixed groups, as is evident from the fact that Paul calls those gathered about the Word in Galatia and Corinth “\textit{congregations}” or churches, although he testifies regarding the Galatians that most of them had lost Christ, and regarding the Corinthian congregation that it had many members who were contaminated in doctrine and life and had grievously fallen.\footnote{L.c.}  And as these visible communions rightly bear the name of churches for the sake of the believers contained in them, so they also possess all the power which Christ has given to His Church, but this likewise only for the sake of the believers contained in them, though they be but two.  All which those who are not believers, and thus do not belong to the Church and of themselves have no right to the power of the keys, do in the Church {\scriptsize\textsc{(preaching, administering Sacraments, choosing and ordaining ministers of the Church, etc.)}} they do only as instruments, as those delegated by the Church, that is, the true believers.\footnote{Kirche und Amt, Thesis VII, pp. 77ff.}  That Christ has given all spiritual power specifically to the local congregation, and that for the sake of the believers contained in it, is proved by Walther from {\scriptsize\textsc{Matthew 18:17-20}}.  \par He expounds this passage as follows: \begin{fancyquotes}\textit{Thus says the Lord}\footnote{Matthew 18:17}: \begin{displayquote}‘\textit{Tell it unto the church; but if he neglect to hear the church, let him be unto thee as an heathen man and a publican}’.\end{displayquote}  No proof is needed to show that the Lord in this passage is speaking of a visible particular, local, church, However, when immediately after those words the Lord proceeds thus: \begin{displayquote}‘\textit{Verily, I say unto you, Whatsoever ye shall bind on earth shall be bound in heaven, and whatsoever ye shall loose on earth shall be loosed in heaven}’\footnote{Matthew 18:18},\end{displayquote} He manifestly delegates with these words also to each visible local church the keys of the kingdom of heaven, or that church power which, in Peter, He had given to His entire holy Church in {\scriptsize\textsc{Matthew 16:19}}.  However, lest we imagine that this great power were given only to great, populous congregations -- \end{fancyquotes} He adds {\scriptsize\textsc{Matthew 18:19-20}}: \begin{fancyquotes}\begin{displayquote}‘\textit{Again I say unto you, That if two of you shall agree on earth as touching anything that they shall ask, it shall be done for them of My Father which is in heaven.  For where two or three are gathered together in My name...',\end{displayquote} -- two or three true believers, true children of God, true members of the spiritual body of Jesus Christ, the congregation would on account of them be a congregation of God and in legitimate possession of all rights and powers which Christ has acquired for, and given to, His Church}.\footnote{Kirche und Amt, p. 78.  Translation from Walther and the Church, p. 64.}\end{fancyquotes}

                Particular churches are of a twofold sort, orthodox or heterodox.  That church is orthodox in which the Gospel is purely preached and the holy Sacraments are administered according to the Gospel.  No more, e.g., not a certain polity or certain ceremonies instituted by men.  But also no less.  For that in a church or congregation the pure Word of God or the church confession is merely officially acknowledged does not yet make a church or congregation orthodox, but it is requisite that the pure Word actually prevail in its public preaching.\footnote{Die rechte Gestalt, pp. 2, 5}  The communions which have become guilty of a partial falling away from the pure doctrine of the Word of God are rightly called heterodox churches.  Such heterodox communions are called both churches and also sects, but in a different respect.  They  are called churches insofar as God’s Word and Sacrament are not entirely denied in their midst, but both are still essentially present, and hence true children of God are still to be found also in these communions.  But insofar as these communions persistently err in fundamental doctrines of God’s Word and have caused divisions in Christendom they are called sects, i.e., heretical communions\footnote{L.c., 18,24.}  The expression that heterodox communions, insofar as they still have God’s Word in essence and children of God are found among them, are to be called churches, caused the charge of a unionistic spirit to be directed against Walther {\scriptsize\textsc{(on the part of Grabau)}}.\footnote{Lutheraner 13, 195.}

                With regard to our judgment of heterodox churches and our position over against them a twofold truth is to be held fast.  First: also in heterodox, heretical congregations there are children of God.  The \textit{una sancta} extends beyond the bounds of the visible orthodox church.  Walther remarks: \begin{fancyquotes} The Lutheran Church is charged with claiming to be the only saving Church.  True Lutherans believe and teach just the opposite. \par When the holy apostle designates the Galatians who have been called as ‘\textit{congregations}’, or churches, addressing his epistle ‘\textit{unto the churches of Galatia}’,\footnote{Galatians 1:2} it follows without question that also in these communions there still remained a hidden seed of a Church of true believers.\end{fancyquotes}  From {\scriptsize\textsc{1 Kings 19:14-18}} we see that also where the priests of Baal were dominant, a holy Church of 7,000 elect, who were unknown even to the prophet Elijah, had been preserved.  People such as these adhere to Christ inwardly by a living faith, while outwardly they follow their false leaders because they do “\textit{not know the depths of Satan}”\marginpar{\scriptsize Revelation 2:24}.  They are like those 200 men who joined the insurgent Absalom and his rabble of rebels but “\textit{went in their simplicity and knew not anything}.”\marginpar{\scriptsize 2 Samuel 15:11}\footnote{Kirche und Amt, pp. 95,96.  Translation from Walther and the Church, p. 65.}  The Lutheran Church confesses this truth in the Preface to the Book of Concord.\footnote{L.c., p. 96.} \par Walther testified repeatedly: \begin{displayquote}“\textit{As long as I did not know this I did not want to be a Lutheran}”.\end{displayquote}  Yea, it is possible, and at times has actually occurred, that there has been no orthodox visible church, while according to the divine promise it is impossible that the one holy Christian Church should ever perish.\footnote{Die ev.-luth. Kirche, etc., pp. 47ff. [The True Visible Church, page 39 ff.]}

                But in the second place it is to be maintained: \begin{displayquote}{\footnotesize We are not to allow the distinction between a true visible or orthodox church and a heterodox church, or, which is the same, the difference between church and sect, to removed through the circumstance that there are children of God also in heterodox communions.  The outward form of a church, as God wills it, is orthodoxy.  God wills only a church which in all respects continues in Christ’s Word, which with regard to revealed doctrine speaks only one thing and is perfectly joined together in the same mind and in the same judgment.  Hence God has given permission to no Christian to belong to a communion in which false doctrine is taught, but has rather commanded every Christian to flee all false prophets, to avoid fellowship with heterodox congregations or sects, and to adhere only to the orthodox church.}\end{displayquote} This everyone is bound to do as he wishes to be saved.  These are truths almost universally forgotten in the church of our time, which Walther ever and again expounded and defended against all objections. – Regarding the point that God wills only an orthodox church he writes: \begin{fancyquotes}Christ says:\begin{displayquote} ‘\textit{If ye continue in My Word, then are ye My disciples indeed; and ye shall know the truth.}’\footnote{John 8:31-32}\end{displayquote}\begin{displayquote}‘\textit{The sheep hear His {\scriptsize\textsc{(the Shepherd’s)}} voice…follow Him…and a stranger will they not follow, but will flee from him.}’\footnote{John 10:3-5}\end{displayquote}  Since, then, the Church is the whole number of Christ’s disciples and flock of His sheep, therefore also only that is a true visible church in an unrestricted sense, or is as it should be, which in all things continues in Christ’s Word, hears His voice, follow Him in all things, and flees from the stranger who brings a different doctrine. \par St. Paul exhorts: \begin{displayquote}``\textit{Endeavor to keep the unity of the Spirit in the bond of peace.  There is one body, and one Spirit, even as ye are all called in one hope of your calling; one Lord, one faith, one baptism, one God and Father of all, who is above all, and through all, and in you all}''\footnote{Ephesians 4:3-6}\end{displayquote}  A true Church, as it should be, is only that in which not different faith, false and true, but unity of spirit reigns in faith and life, in Word and Sacrament.  Finally, the same apostle writes: \begin{displayquote}‘\textit{Now I beseech you, brethren, by the name of our Lord Jesus Christ, that ye all speak the same thing, and that there be no divisions among you; but that ye be perfectly joined together in the same mind and in the same judgement}.’\footnote{1 Corinthians 1:10}\end{displayquote}  A church, as it should be, is hence only that which with regard to revealed doctrine not only speaks one thing, but does this also in the same mind and in the same judgement.
\end{fancyquotes}
                If objection is made to this statement on the ground that such a church, orthodox in every respect, cannot exist, and that the communion which should claim to be such a church would be speaking in arrogant self-conceit, Walther answers: \begin{fancyquotes}Praise God, there is such a church, and that is the Evangelical Lutheran Church.  This we joyfully confess, and hold with full assurance of faith that our dear Church is the Church established by the Lord Christ and His apostles 1800\footnote{(now over nineteen hundred - Ed) [now over 2000 - BTL] ... years ago} years ago, and that for the reason that our faith, doctrine, and confession agrees in all respects most exactly with the Scripture, the words of Christ and the apostles.  The Lutheran Church is therefore not merely a real\footnote{(our old theologians call also erring churches real churches in contradiction to communions which are no churches, e.g., the Unitarian, F.P.)} but the true visible Church of God upon earth, insofar as ‘\textit{true}’ denotes nothing else than: as it should be according to God’s Word.\footnote{Synodical Essay for the Western District 1870.  Synodalbericht, p. 23.}\end{fancyquotes}  Walther offers the proof that the Lutheran Church teaches in accordance with the Word of God in all doctrines in the book: “\textit{Die Evangelische-Lutherische Kirche die wahre sichtbare Kirche auf Erden}”.  Walther finds the reason why it is regarded as arrogant, intolerable presumption, when we assert that the Lutheran Church is in the possession of the full truth, in the dominant spirit of unionism, in the theology of doubt which denies the clarity and majesty of Holy Scripture.  So he says,\footnote{l.c. pp. 24ff. [Essays For The Church, Vol. I, pgs 204-205 – BTL]}: \begin{fancyquotes}Our theologians of doubt want always merely to search for the truth, but never to have found it, and thereby place themselves by the side of the heathen philosophers, who were always searching for the truth, but never found it.  But since Christ and His Gospel has appeared upon earth, the eternal, full, saving truth is also upon earth, and indeed for everyone.  Would our adversaries indeed dare to accuse these apostolic congregations of arrogant self-conceit when they refuse the hand of brotherhood and sacramental fellowship to the seducing spirits who were trying to creep in, and of whose should poison the holy apostles had warned them by word of mouth or by letter, and declared to them: We have the truth and you have not the truth but a doctrine of devils?  They would not dare to do so.  But just that which they must allow to those congregations they will not grant to us.  Why not?  Because, they say, we have not the apostles but only Luther for our teacher.  But O foolish cavil, which reveals to us their unbelief in the Word of God! \par For have not we Lutherans still today this holy Word of God ‘\textit{pure, true and right, through His servants, written in Holy Scripture}’?  Does not St. Paul still speak to us in the Bible the very same truth which he then preached and wrote to those congregations?  Do we not therefore still have, even today, the eternal, full, undeceivable truth?  And would it not be a thoroughly false modesty to think that it would be arrogant and presumptuous to say: \begin{displayquote}{\footnotesize I have the truth, for I stand upon the rock of God’s Word, and I reject the contrary doctrine as Satan’s lie?}\end{displayquote} Thereby we do not ascribe to ourselves any personal infallibility, as people have maliciously remarked.  \begin{displayquote}{\footnotesize We Lutherans maintain that there is indeed an infallible truth, but only in the Word of God, and that we certainly possess it, as long as we stand upon the Word}.\end{displayquote}  For as surely as the Bible is God’s Word and inspired by the Holy Ghost, as surely as Christ is the Son of God and the mouth of eternal truth, so sure is it also that we, as long as we hold fast to the letter of Holy Scripture, cannot err.  We do not say that a Lutheran Christian cannot err in any one thing that is contained in Holy Scripture, but we merely assert that in all articles of faith, which are so clearly and plainly revealed for everyone in the Scripture, he has the full truth, so that he can gladly live and die upon it.  It is a grievous delusion of the errorists when they assert that only some doctrines of faith, as, e.g., the doctrine of the deity of Christ, are clearly and plainly revealed in Holy Scripture; but that others, as e.g., certain distinctive doctrines, are not, and that in these latter, therefore, the infallible truth is not attainable.  To this we say nay.  All doctrines of faith are revealed in Holy Scripture with utter clarity which cannot be misunderstood, and since our church confesses this doctrine she is the infallible mouth of God.\end{fancyquotes}

                To the objection that one can and should remain in heterodox communions, or at least have fellowship with them, sincere there are still Christians within them, Walther answers: \begin{displayquote}\textit{The Christians who adhere to the heterodox communions do this out of weakness in knowledge.  But those who are convinced of the partial apostasy of the church body to which they adhere and still remain in it belong not to the weak, but are either lukewarm, whom the Lord will spew out of His mouth, or are Epicurean mockers of religion who in their heart say with Pilate:} \begin{displayquote}{\footnotesize What is truth?}}\footnote{Thesis 5 and 6. Synodalbericht Westlichen Districts 1870, Essays For The Church, Vol, I, pg 212-213}\end{displayquote}\end{displayquote}  Walther further expounds this in \textit{Kirche und Amt}\footnote{Kirche und Amt, pp. 113ff.}: \begin{fancyquotes}Not a few, when they hear that the Church is wherever there are still essentials of the Word of God and the Sacraments, draw this conclusion: \begin{displayquote}{\footnotesize It is a matter of indifference whether one attaches himself to an orthodox or to a heterodox communion; for even if one joins a heterodox congregation, one is still in the Church and can still be saved.}\end{displayquote}  However this is an error...\par  It is true that many are saved who from lack of knowledge adhere outwardly to sects and yet abide in the true faith... Now a person who has learned to know the false doctrine of the sects and of their teachers and still adheres to them does not belong to the divine seed that lies concealed among the sects.  His fellowshipping the sect is not a sin of weakness, which can coexist with a state of grace.  Such a person wantonly acts contrary to the Word of God; for God commands us in His holy Word to flee and avoid false teachers and their counterfeit worship.\par  As little as the doctrine that Christians in a state of grace still have sins of weakness justifies those who for this reason imagine that they may knowingly and willfully continue in sin, as surely as those who sin trusting in grace are rather children of perdition, so little does the doctrine that even among the sects there are children of God justify those who, contrary to God’s command knowingly remain with them, and so surely such wanton participants in the corruption of the Word of Truth are children of perdition.\footnote{Translation from Walther and the Church, pp. 65ff.}\end{fancyquotes}

                If the attempt is made to excuse fellowship with the heterodox by saying that going out from them would only create more division, then this is based on a false conception of division and unity within the church.  Accordingly to {\scriptsize\textsc{Romans 16:17}} the false teachers are the ones who cause divisions and offenses in the church.  He therefore who has fellowship with the false teachers furthers the division, he who avoids them furthers the unity of the church.\par
                In short, we are never under any circumstances to cultivate church fellowship with heterodox churches and teachers.  “\textit{With the heterodox}”, says Walther, ``\textit{we can well confer and dispute, but we cannot have synodical fellowship}''. Hatred against false doctrine and therefore against church union with disunity in doctrine is characteristic of a true Lutheran, but it must indeed be hatred which flows from the fear of God.

%%% Local Variables:
%%% mode: latex
%%% TeX-master: "../main"
%%% End:
