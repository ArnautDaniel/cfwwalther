\chapter{The Ministry}

\hrule
\vspace{.30cm}
In what relation do Church and Ministry stand to one another?  That is the second main question which was investigated by Walther with attention to the various antitheses and misunderstandings.
\vspace{.30cm}
\hrule
\vspace{1.25cm}
                As objection was made to the doctrine that the Church is the congregation of saints and therefore in its essence invisible, that thereby the Church is attenuated into a mere Platonic idea, so also it was claimed that the doctrine of the ministry advocated by Walther did not do it justice.  Especially from the so-called “\textit{doctrine of commitment}”\footnote{Uebertragungslehre} did men take occasion to assert that Walther identified the universal priesthood of believers with the public ministry or at least did not properly distinguish them.

                But Walther, on the one hand, distinguishes clearly and sharply the office of the public ministry or the pastoral office from the priesthood which belongs to all believers, that he may then indeed, on the other hand, protest just as energetically against a false juxtaposition of the ministerial office and the Christian estate.

                Walther teaches first, in opposition to the fanatics of former and recent times, who upon their having come to faith claim that they have also at the same time been made public preachers of the Word, that no one becomes a public preacher either by natural birth or by spiritual rebirth.  Walther’s \textsc{Thesis I} on the Ministry in “\textit{Kirche and Amt}” reads: \begin{displayquote}“\textit{The holy ministry, or the pastoral office, is an office distinct from the priestly office, which belongs to all believers}.”\end{displayquote}  This he further expounds\footnote{l.c., p. 315}, \begin{displayquote}“\textit{that the spiritual priesthood which all truly believing Christians possess, and the holy ministry, or the pastoral office, are not identical; that neither is an ordinary Christian a pastor for the reason that he is a spiritual priest, nor is a pastor a priest for the reason that he holds the public office of a preacher}”.\end{displayquote}\begin{displayquote}  “\textit{The Christians have indeed become priests through their Baptism received in faith or grasped by faith, but not public teachers, preachers, pastors, bishops, etc}.”\footnote{The Buffalo Colloquy, p. 14}\end{displayquote}  In the Scripture proof for the above cited \textsc{Thesis I} we read: \begin{fancyquotes}Although Holy Scripture testifies to us that all believing Christians are priests\footnote{1 Peter 2:9; Revelation 1:6; 5:10}, nevertheless at the same time it teaches us explicitly that there is in the Church an office for teaching, shepherding, governing, etc., which does not belong to Christians by reason of their general Christian calling. \par For this it is written: \begin{displayquote}‘\textit{...Are all teachers}?’\hfill{\scriptsize\textsc{1 Corinthians 12:29}},\end{displayquote}\begin{displayquote} ‘\textit{How shall they preach\\ except they be sent}?’\hfill{\scriptsize\textsc{Romans 10:15}}\end{displayquote}  But not only the fanatical total identification of the status of a Christian with the office of a preacher does Walther exclude, but also the teaching of Höfling, according to which \begin{displayquote}“\textit{the distinction between clergy and laity... Belongs merely, though with inner necessity, to the human ordinances of ecclesiastical cultus}”\footnote{Grundsätze ev. Luth. Kirchenverfassung, Dritte Auflage 1853, p. 76.  Quoted, L.u.W., 16,174.  Also Buffalo Colloquy, p.13}.\end{displayquote}\end{fancyquotes} To the contrary Walther teaches, \textsc{Thesis II}: \begin{displayquote}“\textit{The ministry, or the pastoral office, is not a human ordinance, but an office established by God Himself}”.\end{displayquote} For not only is the office of the public ministry included in the apostolate and with it instituted by God, {\scriptsize\textsc{(Matthew 10; 28:18-20; Mark 16:15)}} but the mediately called teachers are represented in Scripture as given of God, {\scriptsize\textsc{(Acts 20:28; 1 Corinthians 12:28-29; Ephesians 4:11)}} and placed by the side of the holy apostles as brethren in office.\footnote{ Kirche und Amt, p. 193f.}{\scriptsize\textsc{(1 Peter 5:1; Colossians 4:7; Philippians 2:25; 1 Corinthians 4:1; 1:1)}} 
                \par If, then, the office of the ministry is a divine institution, then it is not an arbitrary office, but its character is such that the Church has been commanded to establish it and is ordinarily bound to it till the end of days.\footnote{Thesis III, p. 211.}  From {\scriptsize\textsc{Matthew 28:19-20}}\footnote{Matthew 28:19-20 -- “Go ye and teach all nations, baptizing them” etc. “And lo, I am with you always, even unto the end of the world”} it is evident that by the command of Christ the apostles’ ministry of preaching was to endure to the end of days.  Now, if this is to be the case, the Church must continually to the end of days establish the orderly public ministry of preaching and in this ordinance administer to its member the means of grace.\footnote{L.c. pp. 211-212}

                On the other hand the office of the ministry is not to be placed into an improper opposition to the estate of Christians. \begin{displayquote} “\textit{The ministry of preaching is not a peculiar order, set up over against the common estate of Christians, and holier than the latter, like the priesthood of the Levites, but it is an office of service}”\footnote{Thesis IV, p. 221.}\end{displayquote}  All believing Christians, and only these, are priests, or of priestly estate.\footnote{1 Peter 2:9}\\\\\begin{tabulary}{.88\textwidth}{JJ}  Among the believers of the New Testament in general there is no difference of order; & \\ & {\scriptsize\textsc{Galatians 3:28;  Matthew 23:8-12}} \\\\ those who possess the public ministry of preaching are not priests on that account or priests before others, but they are only the ministering persons among a priestly people.\footnote{Kirche und Amt, pp. 221,222.} & \\ & {\scriptsize\textsc{1 Corinthians 3:5; 2 Corinthians 4:5; Colossians 1:24-25}}\\\end{tabulary} \divider When Loehe says of the ministerial office: “\textit{The office stands in the midst of the congregation like a fruitful tree which has its seed in itself; it perpetuates itself}”, and when Loehe in consequence calls the preachers a “\textit{holy aristocracy}”, then Walther judges: \begin{displayquote}“\textit{Hereby Loehe manifestly makes the preachers an order, like the Levitical priesthood.  Loehe’s view is Romanizing error}”.\footnote{L.u.W, 16, 176-178.}\end{displayquote}

                But the most disputed question was and is the question concerning the origin of the office of the ministry \textit{in concreto} or the question: “\textit{How do the individual persons come into the office}?”

                That the office is granted or conferred by God is admitted on all sides, though this admission on the part of those who deny the divine institution and ordinance of the office of the ministry is meant in a somewhat different sense.  The question that is in controversy to this day within the Lutheran Church is the question, which are the human media through whom certain individual persons obtain the office of the ministry.  Loehe, as already noted, lets the ministerial office perpetuate itself, since he calls it a fruitful tree standing in the midst of the congregation, “\textit{which has its seed in itself}”.  Loehe further says of the ministry \begin{displayquote}“\textit{that it perpetuates and propagates itself from person to person, from generation to generation.  Those who have it pass it on, -- and he upon whom it is conferred by its incumbents has it also by divine commission... The office is a stream of blessing which flows from the apostles to their disciples, and from these disciples to their successors, and so down through the ages... Where the Lord’s office is to be propagated the Lord’s chosen servants, the bearer of His office, are in charge}”.\end{displayquote}  According to Loehe, then, the office of the ministry is committed by the ministerial order through ordination.  The Christian congregation may express “\textit{reasonable wishes}”, it may even be permitted to choose and call.  No one, however, comes into the office of the ministry by election and call of the congregation, but this is conferred solely by ordination at the hands of those “\textit{who were elders {\scriptsize\textsc{(preachers)}} before him}”.\footnote{L.u.W., 16, 178} Grabau called ordination at least one of the two feet upon which the office of the ministry stands.\footnote{Buffalo Colloquy, p. 26.}\par  Against this Walther teaches: \begin{displayquote}``\textit{the office of the ministry is conferred not by an order of the ministry, also not by a church government or a committee in the church, but by those to whom God originally and properly entrusted all spiritual power, blessings, and gifts in the Church, namely, by the believing congregation}.''\end{displayquote}  Hence Walther says in \textsc{Thesis VI} of the Ministry: \begin{displayquote}“\textit{The ministry of preaching is conferred by God through the congregation, as holder of all church power, or of the keys, and by its call, as prescribed by God}”.\end{displayquote}  Thus the question through whom the office if the public ministry is conferred goes back to the question: who on earth properly possesses all spiritual power? To whom on earth did Christ originally and properly entrust all spiritual blessings and hence also the office of the public ministry?  Walther answers: \begin{fancyquotes}Not to individual persons or a privileged order in the Church but to the Christian congregation.  Of the Christian congregation the apostle says, {\scriptsize\textsc{1 Corinthians 3:21}}:\begin{displayquote} `\textit{All things are yours; whether Paul, or Apollos, or Cephas, or the world, or life, or death, or things present, or things to come; all are yours; and ye are Christ’s}'.\end{displayquote} Here it is `\textit{clearly taught}': all that a Paul and a Peter possessed were nothing but goods from the treasury of believing Christians, or of the Church”.\footnote{K.u.Amt., p. 31.} \par Accordingly we read that even the apostle Matthias was not elected to his exalted office only by the eleven apostles but by the entire gathering of the assembled believers, about a hundred and twenty of whom were present {\scriptsize\textsc{Acts 1:15-26}}.\footnote{K.u.A., p. 245.}\end{fancyquotes}  At this point we cannot deny ourselves a reference to an exposition of Walther on {\scriptsize\textsc{1 Corinthians 3:21}} which is found in a sermon upon this text {\scriptsize\textsc{(Brosamen, p. 589)}}:
                \begin{fancyquotes}All is yours, says the Apostle.  Accordingly nothing is excepted, there is simply nothing which the believing Christians do not have by faith; and indeed what is hereby clearly attributed to them is not only the use and benefit of all things, but the very possession itself.  The Christians, accordingly, are not merely, so to speak, tenants and lessees of God’s property, but they are here declared to be the only rightful possessors, owners, and lords of all things; yes, while they still do not exactly enjoy all things in fact, yet they possess them all by faith. \par The apostle hereby calls out to them: Yours is all which God the Father hath created, yours is all which God the Son hath merited, yours is all which God the Holy Ghost hath wrought.  Yours is God Himself, yours the heaven, and yours the earth.  Yours are all treasures and means of grace and all fruits of the reconciliation and redemption; yours the liberty from sin, death, devil, and hell; yours all forgiveness established, yours all righteousness won; yours the divine sonship and all hope of eternal life; yours is the Word and the holy Sacraments; yours the keys of paradise and of hell; yours all offices and rights and powers which Christ hath purchased for sinners with His blood.  Yours is finally every gift and comfort of the Holy Ghost, in short, ‘\textit{all}’ says the apostle himself, ‘\textit{whether Paul or Apollos}’. \par  That the congregation of believers is the proper and only holder and bearer of all spiritual blessings, rights, powers, and offices which exist in the Church, is further, and principally, expressed in the fact that Christ, according to {\scriptsize\textsc{Matthew 16:15-19; Matthew 18:18; John 20:22-23}}, has given to the congregation of believers the keys of the Kingdom of Heaven.  For the expression “\textit{keys of the Kingdom of Heaven}” includes in itself all church rights and powers, every function, power, and authority, whereby are performed all things that are necessary for the Kingdom of Christ or the governing of the Church \footnote{K.u.A., pp. 42,43.}, especially also the office of the Word and the Sacraments.\footnote{L.c., p.38.}  When, furthermore, the communion of believers is called the bride of Christ\footnote{John 3:28-29; 2 Corinthians 11:2, etc.}, the thought is thereby expressed that this communion is also the true holder of the possessions of Christ, its Bridegroom. \par If in {\scriptsize\textsc{Galatians 4:26}} “\textit{Jerusalem which is above}”, that is, the Christian Church, is called “\textit{the mother of us all}”, then that whereby children of God are born, Word and Sacrament, and whereby Word and Sacrament are put to use, belongs to the Church.\footnote{K.u.A., pp. 30,31.}\par  Finally St. Peter writes to the believing Christians, {\scriptsize\textsc{1 Peter 2:9}}: \begin{displayquote}“\textit{Ye are a chosen generation, a royal priesthood, an holy nation, a peculiar people, that ye should show forth the praises of Him who hath called you out of darkness into His marvelous light}”.\end{displayquote}  Thus God has commanded the entire true holy Christian Church to proclaim His precious Gospel.  Hence, wherever a group of believing Christians or a true church exists this church has the command to preach the Gospel; but if it has this command then it naturally has also the power, yea, the duty, to ordain preachers of the Gospel.\footnote{K.u.A., pp. 31,33.} -- \\\par Now if the situation is indeed such that the congregation or church of Christ, i.e., the assembly of believers, possesses the keys and the priestly office immediately, being the bride of Christ and mother of all believers, to which everything that is in the Church belongs originally, then it is likewise the congregation, and it can be only the congregation, by which, namely, by its election, call, and commission, the ministry of preaching, which publicly administers the office of the keys and all priestly offices in the congregation, is conferred on certain persons qualified for the same.  The example of this truth placed before the Church for all times is to be found, among others, in the instance recorded in {\scriptsize\textsc{Acts 6:1-6}}.\footnote{K.u.A., pp. 245, 246.}\end{fancyquotes}\\\par Herewith Walther has proved the above cited \textsc{Thesis VI}.  He designates the relation which the church and office-holders in the church bear to the office of the ministry also thus: \begin{displayquote}“\textit{It is the doctrine of our Church in accordance with God’s Word that Christ gave the office and all blessings and powers earned by Him, just as He gave the Gospel, immediately to His Church, as the original and first possessor; so that the Church has the office not mediately, by virtue of Christ’s having conferred it upon certain persons in the Church, who then should perpetuate it and administer it indeed for the benefit of the Church.  Just the reverse: it is not the Church which has the office as mediated through the office-holders, but it is the office-holders who have the office as mediated through the Church, which as the communion of believers and saints, as the body of Christ, bears all this within herself}”.\footnote{K.u.A., p.33.}\end{displayquote}  Walther quotes \textbf{Luther} with the following emphases and insertions: \begin{displayquote}“\textit{The Christian Church alone has the keys, no one else, although the bishop or the pope can use them, as those who have been charged with this duty by the congregation.  A pastor exercises the office of the keys, baptizes, preaches, administers the Sacrament, and does other duties whereby he serves the congregation, not for his own sake\footnote{Pieper: That is, not on his own authority}, but for the congregation’s sake\footnote{That is, as one to whom it has been delegated by the congregation, who does it by command of the congregation}.  For he is the minister of the entire congregation, to which the office of the keys is given, even if he is a rascal.  For what he does in the stead of the congregation, that the Church does}”.\footnote{Die Rechte Gestalt, p. 18.}\end{displayquote}

                The keys of the Kingdom of Heaven, and therewith all spiritual power, belongs to each local congregation, the smallest as well as the largest, in like measure, as Christ expressly testifies, {\scriptsize\textsc{Matthew 18:17-20}}\marginpar{{\scriptsize “Tell it unto the church”…\\ “Where two or three are gathered together”, etc.}}  \begin{displayquote}“\textit{That a congregation, in order to possess and exercise all church rights, must be outwardly with other congregations and stand with them under one church government, and thus be dependent upon other congregations, is an error upon which the papacy is based}”.\footnote{Die rechte Gestalt, pp. 13-20}\end{displayquote}  That each congregation, also the smallest, has all church rights and all church power, that the entire church and any aggregation of congregations has not more power than the smallest local congregation, yes, no more power than the individual Christians, is evident from the fact that the Christians possess everything as Christians or believers, not insofar as there are more or less of them.\footnote{Die rechte Gestalt, p. 15.} \par Some have wanted to interpret the well-known words of the \textbf{Smalcald Articles}: “\textit{In addition to this, it is necessary to acknowledge that the keys belong not to the person of one particular man, but the {\scriptsize\textsc{(entire)}} Church}”, as though here nothing were said of the “\textit{congregation}”, but only of the “\textit{Church}”, and indeed of the “\textit{entire Church}”.  But Walther rightly remarks: \begin{fancyquotes}To distinguish {\scriptsize\textsc{(here)}} between congregation and Church is a pure invention!  The Smalcald Articles themselves promptly define the “\textit{Church}” which has all power as the local church or local congregation, when they go on to say \begin{displayquote}And Christ speaks in these words, ‘\textit{Whatsoever ye shall bind}’, etc., and indicates to whom He has given the keys, namely, to the Church: ‘\textit{Where two or three are gathered together in My name}’, etc.\end{displayquote}  When the Smalcald Articles speak of the entire Church they intend to say, as the context indicates: not only this or that {\scriptsize\textsc{(member)}}, but all members of the Church.\footnote{L.u.W., 16: 179.}\end{fancyquotes}

                So it is the congregation, as the holder of all church power, through whose call God confers the office of the ministry.  As regards ordination, it is an apostolical church ordinance.   Holy Scripture testifies that the holy apostles made use of ordination and that at that time the communication of glorious gifts was connected with the laying on of hands.  But ordination is not by divine institution.  For Scripture is silent regarding a divine institution of ordination.  \begin{displayquote}“\textit{But whatever cannot be proved by God’s Word as having been instituted by God cannot without idolatry be declared to be, and accepted as, an establishment of God Himself}.”\end{displayquote}  Hence ordination, as a good church ordinance, is indeed to be retained, for when it is joined with a prayer of the church, based on the glorious promises that have been specially given to the ministry of preaching, it is not an empty ceremony but is accompanied by an outpouring of heavenly gifts upon the believing recipient of ordination, but ordination has nothing to do with bringing about the essence of the ministerial office.  \begin{displayquote}“\textit{Our fathers testify\footnote{S.A., Of the Power and Jurisdiction of Bishops, Mueller, p. 342; Triglotta, p. 524} that the divine ordinance of the ministerial office is properly brought about through the call and election of the Church, that ordination does not create this work of God, but where it has already taken place publicly acknowledges, attests, and ratifies it}.”\footnote{K.u.A., p. 289.}\end{displayquote}  Quite otherwise \textbf{Loehe}, who regards ordination as being of divine institution and sacramental character, and makes it “\textit{not only a conditio sine qua non, but the only actual factor of the office}.”\footnote{L.u.W., 16, 178.} Also \textbf{Grabau} taught: \begin{displayquote}“\textit{Ordination itself is no adiaphoron or unessential matter.  It belongs to the obligatory divine order and has divine and apostolical command}.”\footnote{Buffalo Colloquy, p. 26.}\end{displayquote}

                After Walther has presented the relation of the spiritual priesthood and the office of the ministry to one another, and also the fact that the ministry, like all other spiritual blessings and powers, belongs originally to the believing congregation, which according to God’s ordinance and command commits the office to certain persons qualified for the same, he states in \textsc{Thesis VII} what the office of the ministry is in its essence: \begin{displayquote}“\textit{The holy ministry is the authority conferred by God through the congregation, as holder of the priesthood and of all church power, to administer in public office the common rights of the spiritual priesthood in behalf of all}.”\footnote{K.u.A., p. 315.}\end{displayquote}  The correctness of this thesis is evident from all of the above, as Walther also states for further confirmation in the following recapitulation: \begin{fancyquotes}A reminder may be in place here that Holy Scripture exhibits to us the Church, that is, the believers, as the bride of the Lord and the mistress of His house, to whom have been committed the keys and therewith the right and the access to all courts, sanctuaries, and treasures of the house of God and the authority to appoint stewards over it; furthermore, that every true Christian, according to Holy Scripture, is a spiritual priest and hence is entitled and called not only to use the means of grace for himself but also to dispense them to those who as yet have them not and hence do not as yet possess like priestly rights with himself. \par Scripture, however, teaches that, where all possess these rights, no one may arrogate these rights as inhering in him exclusively; but wherever Christians dwell together in a community, the priestly rights of all are to be administered publicly in the common interest only by those who have been called by the communion in the manner prescribed by God.  The incumbents, then, of the ministerial office in the Church are for this reason also called in God’s Word not only servants and stewards of God, but also servants and stewards of the church, or congregation, and are thus represented as persons who administer, not their own, but the rights, authorities, possessions, treasures, and offices of the Church, hence are acting, not only in the name of Christ, but also in the name and in the place of His bride, the Church of the believers.\footnote{Cf. The further exposition in the place cited.}\end{fancyquotes}

%%% Local Variables:
%%% mode: latex
%%% TeX-master: "../main"
%%% End:
