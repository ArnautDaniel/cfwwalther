\chapter{The Ministry II}

\hrule
\vspace{.30cm}
With this doctrine of the origin of the office of the ministry \textit{in concreto} and its implications many who want to be Lutherans have found themselves unable to agree.  And a reason for such disagreement they have pointed to the very expression “\textit{conferred}” as objectionable.  Walther never insisted on this expression as a \textbf{Shibboleth}\marginpar{{\scriptsize Shibboleth:\\A word, especially seen as a test, to distinguish someone as belonging to a particular nation, class, profession etc. of the correct doctrine.\\Typesetter's Note: See Judges 12:5-6}}  He showed, on the one hand, that this expression was not new but had been used by the old orthodox teachers.  On the other hand, he is willing to acknowledge as orthodox on this point everyone who holds that the congregation originally possess the office, and that it is not conferred by one minister upon another, but comes through the election and call of the congregation.
\vspace{.30cm}
\hrule
\vspace{1.25cm}
He remarked on this point in 1873: \begin{fancyquotes}It is continually objected against us, even on the part of best-intentioned critics, as by Pastor Lohrmann in Müden, that we seem to make a particular ‘\textit{form of the conferral theory our Shibboleth, and thereby threaten to decline into a peculiar separatistic position over against all the rest of the Lutheran Church upon earth}’. \par But, thank God, it is not so!  In whatever form other Lutherans may speak of the office and of its conferral we will offer them the hand of church fellowship if only they confess with us the office of the keys as it is laid down, over against the papacy, in our Confession, particularly in the Smalcald Articles, and thus do not deny that not the office-holders but the Church, originally possesses the keys or the office and confers it through the call, so that the pastoral office is not a privileged self-perpetuating order which exists alongside of the Church. \par But whoever denies this, or, although he makes a pretence of admitting it, nevertheless declares our doctrine to be fanatical, while he, for instance, hides behind the invisible Church as a whole, and thus shows that he fundamentally still holds an essentially different doctrine to be correct, with such indeed we cannot work together.\footnote{L.u.W., 19:366 f.}\end{fancyquotes}

        Against the matter itself it has been contended that one becomes involved in contradictions by the doctrine of a conferral of the office of the ministry on the part of the congregation.  It has been put this way: \begin{displayquote}{\footnotesize If the Christians confer the office of the ministry as something which they had before and which the minister is to conduct in their stead, they must all previously have been minister or pastors.}\end{displayquote}  This oft repeated objection is not exactly very clever.  For it ignores the most ordinary analogies. The American citizens through their vote confer the presidency of the United States upon a particular individual without any necessity of their having previously been presidents themselves.  But let us hear Dr. Walther.  He writes: \begin{displayquote}“\textit{We also assert that the calling Christians are not pastors but simply the priestly generation of the New Testament, in whom all ecclesiastical power of office originally rests, through the conferring of which upon certain persons for the public exercise of the same according to God’s ordinance these persons become something which the Christians are not, namely pastors; even as free citizens possessing the right of suffrage are not civic officials but simply the free citizens in whom all civic power of office originally rests, through the conferring of which upon certain persons for the public exercise of the same these persons likewise become something which the citizens are not, namely, civic officials}”.\footnote{L.u.W., 19: 365f.}\end{displayquote}

                Another form of this objection is as follows: \begin{displayquote}{\footnotesize Since the Christians are supposed to possess the office of the keys through Baptism and faith, they could not get rid of the office of the keys without the necessity of “\textit{washing away their Baptism}” and “\textit{rooting out their faith}”.}\end{displayquote}  Besides, the circumstance that the Christians bear the Gospel upon their lips would indicate that they still had the office of the keys.  Otherwise a division of the office of the keys would have to be assumed.  Then the question would arise: “\textit{According to what proportion and relation” the division should take place}.  Walther answers: \begin{fancyquotes}The solution of all the above named difficulties and contradictions in which the doctrine of conferral is supposed to involve its adherents lies simply in the fact that the ministers are servants of the congregation.  As the mistress of the house is not ‘\textit{stripped}’ of her power when she engages servants to whom she commits its exercise, so also the Church of the believers is not deprived of anything; with this difference, that, whereas it is at the option of the mistress of the house whether she chooses to engage such servants, the Church has a \textbf{mandatum divinum} {\scriptsize\textsc{(divine order)}} to this effect.  The question ‘\textit{according to what proportion and relation}’ the Christian has and holds the office over against the minister is answered by the Fourteenth Article of the Augsburg Confession.\footnote{L.u.W. 16:182.}\end{fancyquotes}

Concerning the relation of the office of the ministry to the ministry to other offices in the Church Walther teaches: \begin{displayquote}“\textit{The ministry is the highest office in the Church, from which, as its stem, all other offices of the Church issue}”.\footnote{K.u.A., Thesis VIII, p. 342. Walther and the Church, p. 27}\end{displayquote}  The correctness of this Thesis, which is found \textit{verbotenus} {\scriptsize (word-for-word, Ed.)} also in the Lutheran Confession\footnote{Apology, Art XV., Müller, p. 213. Triglotta, p. 327}, is clear already from the fact that the office of the ministry has the public administration of the keys of the Kingdom of Heaven, which comprise in themselves all ecclesiastical power.  So there can be no office in the Church which stands above the office of the ministry.  Rather is every other office in the Church merely an auxiliary office, which stands at the side of the office of the ministry, whether it be the office of such elders as do not labor in the Word and doctrine {\scriptsize\textsc{(1 Timothy 5:17)}}, or the office of ruling {\scriptsize\textsc{(Romans 12:8)}}, or the diaconate {\scriptsize\textsc{(office of serving, in the narrower sense)}}, or whatever other offices in the Church may be committed to certain persons for their particular administration.  \par Hence those who administer the office of the holy ministry in the Church are called in Scripture elders, bishops, overseers, stewards, etc., and the holders of a subordinate office are called deacons, i.e., servants, not only of God, but also of the congregation and of the bishop, and only of the latter in particular is it said that they take care of the Church of God and watch over all souls as they that must give account \marginpar{{\scriptsize 1 Timothy 3:1, 5,7; 5:17;\\ 1 Corinthians 4:1;\\ Titus 1:7;\\ Hebrews 13:17.}}  Thus also there can be no \textit{jure divino} {\scriptsize\textsc{(by divine right)}} superiority and subordination among those who hold the office of the ministry, but all are on the same level.  Any superiority or subordination is only of human right.\footnote{ K.u.A. p. 342 f.}

                With regard to the rights of the office of the ministry it is to be said that reverence and unconditional obedience is due to this office when the minister speaks God’s Word.  Upon this Walther most urgently insists.  He has been accused of having made the ministers servants of men, with whom the congregations could deal according to their own pleasure, through his teaching concerning the relation of the office of the ministry to the Christian estate.  This accusation is completely unjustified.  Walther from the beginning until his end never surrendered a jot or tittle of the rights which God’s Word ascribes to the office of the ministry.  But let us hear his own words: \begin{fancyquotes}Although the incumbents of the public ministry do not form a more holy order, distinct from the ordinary order of Christians, but merely exercise the universal rights of Christians, with the public and orderly administration of which they have been commissioned, still they are not servants of men on that account.  The principal efficient cause of the ordinance of the public office of preaching is God, the Most High, Himself.\par  This ordinance is not an arrangement which men in their wisdom have instituted for propriety’s sake and for salutary reasons, but it is an institution of the Triune God, The Father, the Son, and the Holy Ghost.  Therefore, when official authority has been conferred on a person by the congregation by means of a regular, legitimate call, that person has been placed over the congregation by God Himself, although it was done through the congregation.\footnote{1 Corinthians 12:28; Ephesians 4:11; Acts 20:28.}  The person installed is henceforth not only a servant of the congregation but at the same time a servant of God, an ambassador in Christ’s stead, by whom God exhorts the Christian congregation.\footnote{1 Corinthians 4:1; 2 Corinthians 5:18-20.} \par Accordingly, when a preacher is ministering God’s Word in his congregation, whether he be teaching or admonishing, reproving or comforting, publicly or privately, the congregation hears from his mouth Jesus Christ Himself and owes him unconditional obedience as to a person by whom God wants to make known His will to them and guide them to eternal life.  The more faithfully the preacher discharges his office, the greater must be the reverence of which the congregation deems him worthy.\footnote{K.u.W., p. 360f. “Walther and the Church, p. 80.}\end{fancyquotes}  Therefore also Walther from the beginning protested against the calling of ministers until further notice {\scriptsize\textsc{(auf Kündigung)}} which had become a rather general custom in America.  This he denounced as a shameful contempt of the divine order of the ministerial office and a degrading of the ministers to the position of servants of men.  The congregation can and should depose a minister from his office only when it is evident that the principal cause of the office of the public ministry, namely God Himself, has deposed him from office, that is, in cases where the minister has become guilty of false teaching or offensive life.  Walther says on this subject: \begin{displayquote}“\textit{Moreover the congregation has no right to take away his office from such a faithful servant of Jesus Christ; if it does so it thereby rejects Christ Himself in whose name he presided over the congregation.  Only then can the congregation remove an incumbent of the office from his office when it is evident from God’s Word that the Lord Himself has deposed him as a wolf or hireling}”.\end{displayquote}  In his “\textit{Pastorale}”\footnote{Pastorale, p. 41f} Walther treats in detail of the usage obtaining specifically in America, \begin{displayquote}“\textit{that the ministers are called only temporarily, that is either with the provision that they may be dismissed at will, or only for a specific term of one or more years, or ‘until notice’, so that at a specified interval from the day the notice is given they are to withdraw from office}”.\end{displayquote}  Walther’s judgement is that a congregation has neither the right to issue such a call nor is a preacher authorized to accept it. \begin{fancyquotes}Such a call conflicts, in the first place, with the divinity of a rightful call to an office of ministry in the Church, which is clearly attested in God’s Word\footnote{Acts 20:28; Ephesians 4:11; 1 Corinthians 12:28}.  For if God is really the one who calls the ministers, then the congregations are only the instrumentalities for the selection of persons for the work to which the Lord has called them\footnote{Acts 13:2}.  When this selection has now taken place, then the minister stands in God’s service and office, and no creature can depose God’s servant from his office or dismiss him unless it can be proved that God Himself has deposed him from his office and dismissed him \footnote{Jeremiah 15:19 compared with Hosea 4:6}, in which case the congregation does not really depose or dismiss the minister, but only carries out God’s deposition or dismissal which has become evident to it.
\par
        If the congregation nevertheless does this at its own pleasure, it then makes itself, instead of God’s instrument, a mistress of the office and usurps God’s own rule and economy...  But the minister who gives to a congregation the right thus to call and dismiss him at its pleasure {\scriptsize\textsc{(discretion)}} thereby makes himself an hireling and a servant of men.  Such a call conflicts also with the \begin{displayquote}“\textit{honor and obedience, which the hearers are to render to the holders of the divine office of the ministry in accordance with God’s Word;\footnote{Luke 10:16; Hebrews 13:17; etc.} for if the hearers really possessed that assumed fullness of power, then it would stand entirely in their own power to release themselves from the rendering of that honor and which God requires of them}.''\end{displayquote}

To be sure, the enjoining and commanding on the part of the ministers and the obedience on the part of the congregation extend only as far as God’s Word.  For anything which is not commanded in God’s Word the preacher may demand no obedience.  If he does this he usurps a lordship in the Church for his own person and overthrows the cardinal principle that the Christians are subject only to Christ but among themselves are brethren.

{\color{Black} Hence also the so-called \textit{constitutive ecclesiastical power}, that is, the power to arrange matters of indifference, belongs not to the minister, but to the entire congregation, that is, to the minister with the congregation.\footnote{Pastorale, p. 365 ff.}\end{fancyquotes}
\par The demand on the part of the preacher that by virtue of the Fourth Commandment he is entitled to obedience also beyond the Word of God is papistical error.  Walther sets up the Thesis in his “\textit{Kirche und Amt}”: \begin{displayquote}“\textit{The preacher may not dominate over the Church; he has accordingly no right to make new laws and to arrange indifferent matters and ceremonies arbitrarily}”.\end{displayquote}  In the “\textit{Proof from the Word of God}” he cites the passages, {\scriptsize
  \textsc{Matthew 20:25-26; Matthew 23:8; John 18:36}}, and continues:}


\begin{fancyquotes}We see from this that the Church of Jesus Christ is not a dominion of such as command and such as obey, but it is one great, holy brotherhood in which no one can dominate and exercise force. Now, this necessary equality among Christians is not abolished by the obedience which they render to the preachers when these confront them with the Word of Jesus Christ; for in this case, in obeying the preachers, they do not obey men but Christ Himself. \par Just as certainly, however, this equality of believers would be abolished and the Church would be changed into a secular state if a preacher would demand obedience also when he presents to the Christians, not the Word of Christ, who is his and all Christians’ Lord and Head, but something which by virtue of his own understanding and experience he considers good and appropriate.\par  Hence the moment there is a discussion in the Church about matters indifferent, that is, such as are neither commanded nor forbidden in God’s Word, the preacher may never demand unconditional obedience for something which appears best just to him.  In such a case it is rather the business of the entire congregation, of the preacher together with the hearers, to decide the question whether what has been proposed should be accepted or rejected.  It is, however, due the preacher, by reason of his office of teacher, overseer, and watchman, to guide the deliberations that have been instituted, to instruct the congregation regarding the matter, to see to it that in settling indifferent matters and arranging order and ceremonies of the church nothing is done in a trifling manner and nothing harmful is adopted.\footnote{K.u.A., p. 370f. – Walther and the Church, p. 81 f.}\end{fancyquotes}  The holy apostles forbid the preachers to lord it over the people, that is, the congregations: {\scriptsize\textsc{1 Peter 5:1-3; 2 Corinthians 8:8; 1 Corinthians 7:35}}. \begin{fancyquotes}When the holy apostles, notwithstanding these statements, among other things write this: \begin{displayquote}‘\textit{The rest will I set in order when I come}’\footnote{1 Corinthians 11:34},\end{displayquote} it is evident from the foregoing that they made arrangements in regard to indifferent matters not by way of commands but by offering their advice and with the consent of the entire congregation.\end{fancyquotes}  As is well known, the recent Romanizing Lutherans ascribe to the ministers the power to make ordinance in the Church on their own authority alone, for which they appeal partly to passages such as {\scriptsize\textsc{Hebrews 13:17}}: “\textit{Obey them that have the rule over you, and submit yourselves}” \par So \textbf{Grabau}: \begin{fancyquotes}Lutheran Christians know that when God’s Word says, \begin{displayquote}‘\textit{Obey them that have the rule over you, and submit yourselves}’,\end{displayquote} it deals not only with preaching, but with all good Christian matters and occasions which God’s Word brings with it and requires, and which pertain to the good government of the churches and also to Christian welfare in life and work, and that honor, love, and obedience according to the Third and Fourth Commandments is demanded... Here the required obedience is in every respect a matter of conscience; but through the Holy Ghost also a willing and cheerful obedience on account of the believing recognition of what is good in the grace of Jesus Christ.\footnote{Colloqium, p. 20}\end{fancyquotes} ---  and in part adduce such passages as {\scriptsize\textsc{1 Peter 2:13}}: \begin{displayquote}“\textit{Submit yourselves to every ordinance of man for the Lord’s sake}”.\footnote{So Superintendent Münchmeyer.  L.u.W., 16, 184.}\end{displayquote}  With regard to the first passage Walther says with the Apoligy: \begin{displayquote}``\textit{Here nothing is said of the ordinances of men, but of teaching the Word of God. So also this passage does not establish a rulership apart from the Gospel}”.\footnote{K.u.A., p. 373.}\end{displayquote}  With regard to the application of {\scriptsize\textsc{1 Peter 2:13}} Walther says: \begin{displayquote}“\textit{To understand under ‘ordinances of man’ in this place the arrangements made by a preacher is a perversion which exceeds all bounds}”.\footnote{L.u.W., 16, 184.}\end{displayquote} This passage speaks of the ordinances of civil government in secular affairs!
\divider
In de-limiting the sphere of authority between congregation and office of the ministry Walther thoroughly examined in particular two points.  They are the questions: \begin{displayquote}{\footnotesize\textit{“To whom belongs the right to impose excommunication?”}} {\footnotesize \&} {\footnotesize\textit{“Who has the right to pass judgement on doctrine?”}}\end{displayquote}  Both questions had to be discussed in connection with the controversy with Pastor Grabau.\footnote{Cf. Buffalo Colloquy, p. 21,22.}

        With regard to the first question Walther insists: \begin{displayquote}“\textit{The preacher has no right to impose and execute excommunication alone, without a previous verdict of the entire congregation}”.\footnote{Thesis IX, C.; K.u.A., p.383. Walther & the Church., p. 83.}\end{displayquote}  Walther, as is characteristic of him, first gives fitting emphasis to the rights of the ministerial office.  For him it is certain “\textit{that the power of the keys in the narrower sense, namely, the power to loose and to bind”, and then hence according to the Word of the Lord and His sacred ordinances the public execution of excommunication belongs to, and must remain with, the incumbent of the public ministry}”.\par  Nevertheless, “\textit{according to the express prescription and order of the same Lord, the investigation preceding the execution of excommunication and the final judicial verdict must come from the entire congregation, that is, from the teachers and hearers}”.\footnote{Matthew 18:15-20.} \par  After citing this passage Walther continues: \begin{fancyquotes}Evidently here Christ, as our Confessions put it, gives the highest jurisdiction to the church, or congregation, and wants a sinner in the congregation to be regarded as an heathen man and a publican, and the awful judgement of excommunication to be executed upon him, only after several fruitless private admonitions and after he has been admonished in vain also publicly, in the presence of, and by, the whole congregation, and therefore his expulsion from their fellowship has been unanimously resolved upon by them and has been executed by the preacher of the congregation. \par In accordance with this procedure, then, even Paul would not excommunicate the incestuous person at Corinth without the congregation, but, in spite of his having declared this great sinner worthy of excommunication, he wrote the congregation that this must be done by them `\textit{when they were gathered together}’.\footnote{1 Corinthians 5:4}\footnote{K.u.A. p. 384. \& Walther and the Church. P. 83}\end{fancyquotes}  Hence Walther also passes the judgment: “\textit{An excommunication which has been resolved by a mere majority to the exclusion of the minority, not unanimously, with even the silent consent of all members, is illegitimate and invalid}”.\footnote{Pastorale, p. 348.}

        But also here Walther is very careful not to go beyond the rightful bounds.  An excommunication which has been imposed by a presbytery or consistory with the knowledge and consent of the people he declares to be valid and legitimate.  He remarks\footnote{ K.u.A., & Walther and the Church, l.c.}: \begin{displayquote}“\textit{It will go without saying that what the congregation through ‘many’ and ‘before all’\footnote{2 Corinthians 2:6; 1 Timothy 5:20} did at the time of the apostles can be validly and legitimately done also where the ruling congregation is represented by a presbytery or consistory, composed of clergymen and laymen, so that the presbytery or consistory alone renders the verdict of excommunication, provided only that is done with the knowledge and consent of the people}”.\end{displayquote}  And yet Walther most decidedly advises against the introduction of this arrangement in our American congregations.  And this he does also for the reason that the right to exclude impenitent sinners may not in this manner get away from the congregations altogether, as it has for the most part come about in the State Churches.  As concerns the right to judge doctrine, “\textit{no-proof}”, says Dr. Walther, “\textit{is needed}” that also this belongs to the office of the public ministry. ``\textit{According to divine right the function of passing judgment on doctrine belongs to the ministry of preaching}”.  Indeed, without this function the preachers could not at all discharge their office.  It is certainly the duty of the office of the public ministry not only to present the correct doctrine, but also to expose, refute, and warn against the false doctrine, if it is to achieve its purpose of leading souls, in spite of all sorts of seduction, unto final salvation.  But by the establishment of the special office for passing judgment on doctrine this right has not by any means been taken away from laymen.\footnote{Loehe and Grabau wanted to grant a seat and voice in ecclesiastical tribunals and councils (synods) to pastors only.  The latter says: “\textit{You shall leave the judgment of doctrine to those to whom according to the Twenty-eighth Article (?) of the Augsburg Confession it properly belongs}”. (Zweiter Synodalbrief. Colloqium, p. 22.)}  \par Rather does Scripture make the exercise of this right their most sacred duty.  This is proved, first, by all those passages of Holy Scripture in which this judging is enjoined also upon ordinary Christians.  For instance, thus writes the holy Apostle Paul: \begin{displayquote}“\textit{I speak as to wise men’ judge ye what I say.  The cup of blessing which we bless, is it not the communion of the blood of Christ}?”\footnote{1 Corinthians 10:15-16.}\end{displayquote}  Again: \begin{displayquote}“\textit{Try the spirits whether they are of God}”\footnote{1 John 4:1.  Cf. 2 John 10-11; 1 Thessalonians 5:21}\end{displayquote} The proof is furnished, furthermore, by all those passages in which Christians are exhorted to beware of false prophets, such as {\scriptsize\textsc{Matthew 7:15-16; John 10:5}}, in such passages in which they are praised for their zeal in testing doctrine {\scriptsize\textsc{(Acts 17:11)}}.  \par Lastly, we have an account in the Acts of the Apostles stating that at the first apostolic council laymen were not only present but also spoke, and that the decisions reached on this occasion were made by them as well as by the apostles and elders and were sent in their name as well as that of the apostles.  Hence there is no doubt that laymen have a seat and voice in church jurisdiction and at synods with the public ministers of the Church.\footnote{K.u.A., p. 298 f. Walther and the Church, p. 85 f.} To take away or even to diminish this right of the laymen is an accursed church-robbery and has as its consequence that it becomes impossible any longer to withstand the intrusion of false doctrine.\footnote{K.u.A. p. 400 f.}

