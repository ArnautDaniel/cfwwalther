\chapter{Church Government}

\hrule
\vspace{.30cm}
Concerning the Church and Church-Government Walther teaches: “\textit{With the keys of the kingdom of heaven every Evangelical Lutheran local congregation has the entire church power which it needs, that is, the power and authority to perform everything that is requisite for its government}”.\footnote{Die rechte Gestalt, etc., p. 24, Walther and the Church, p.21}
\vspace{.30cm}
\hrule
\vspace{1.25cm}
Also the so-called constitutive power, that is, the ordering of all things which are not ordered through God’s Word {\scriptsize\textsc{(adiaphora)}} belongs to the congregation itself, not to the pastor, and not to persons outside the local congregation.  The local congregation possess the supreme jurisdiction in its own sphere.\footnote{Pastorale, p. 365}  The jurisdiction which persons outside the local congregation have over it and its pastors is only of human right.\footnote{Rechte Gestalt, etc. p. 30}.  All congregations and pastors have of themselves equal ecclesiastical power, and no congregation is of itself superior or subject to another congregation, nor is any pastor of himself superior or subject to another pastor.\footnote{L.c., p. 212}  An organizational joining together of a number of congregations into a larger church body, e.g., by means of a synod with powers of visitation, a church council, a consistory, a bishop, etc., is not of divine but of human authority, and hence not absolutely necessary.\footnote{Pastorale, p. 393 f.} \par Every congregation may maintain its independence.  That every congregation is of itself independent is pure Lutheran teaching, not separatistic, as it is frequently called today.  Separatistic teaching is that each congregation shall be and remain independent\footnote{Rechte Gestalt, etc. p. 22}.  That a local congregation, in order to possess and exercise all church powers, must be externally connected with other congregations and together with them stand under one church government, and thus be dependent on other congregations, is an error upon which the papacy is founded.\footnote{L.c., p. 19f., Pastorale, p. 393}.  Moreover upon this assumption we should never be sure how large a church body would have to be in order to possess all church power.  But it is not so, for every local congregation possesses with the keys also all church power.  As no one dare impose anything upon an individual Christian against his will, so also the same rule holds with an individual congregation.  Synods, consistories, or church councils can have only advisory power over against the individual congregations. \par Every congregation must also retain the right at any time to withdraw from its connection with a larger church body, and to retrieve the rights delegated to others {\scriptsize\textsc{(e.g., consistories)}}, just as it may otherwise undertake such alterations in matters of indifference as may appear advisable to it.  Those who wish to establish a church government which by divine right stands over the individual congregations, and upon which therefore the individual congregations should be dependent, thereby deny the word: “\textit{One is your Master, even Christ, and all ye are brethren}”, and want to introduce another authority than the authority of God’s Word into the Christian Church.

                \begin{fancyquotes}They rob Christ’s Church of that liberty which He has obtained for her with so costly a price, with His own divine blood, and degrade this free Jerusalem which is above, in which only kings, priests, and prophets exist, this kingdom of God, this heavenly kingdom of truth, to a political institution, in which a man must submit to every human ordinance.  They aspire to the royal crown of Christ, the true and only King, they make themselves kings over His kingdom; they drive Christ, the true and only King, they make themselves kings over His kingdom; they drive Christ, the true and only Master, from His seat and set themselves up for in His Church; they strive to sever Christ, the true and only Head, from His body, the Church, and assume the authority of heads to His spiritual body.  They exalt themselves above the holy Apostles, and arrogate to themselves powers which are clearly denied them in the Word of God, nay, which God has conferred upon no man whatever, no creature, not even angels and archangels.\footnote{Brosamen, p. 523. Translation in June 15, 1949, Okabena Lutheran, pp. 4,5}\end{fancyquotes}  Hence church polity is a matter of indifference only so long as it does not deprive the Christians of their Christian rights bestowed upon them by Christ.\footnote{Brosamen, p. 496. K.u.A., p. 371.}

                Nevertheless, so Walther further declares, every congregation should be ready to unite with other orthodox congregations when there is opportunity for such union and this tends to serve and promote the glory of God and the upbuilding of His kingdom.  Every congregation should on its part endeavor to keep the unity of the Spirit, and provide that the gifts of the Spirit shall be manifest unto the common good, and that in every way the purposes of the Kingdom of God in general should be furthered.\footnote{ (Rechte Gestalt, etc. p. 212 ff. “Walther and the Church”, p. 115} The congregation will achieve these ends if it unites with other congregations into a larger church body, when, for instance, it enters into a synodical fellowship with other congregations \begin{displayquote}“\textit{for mutual fraternal consultation, inspection, and assistance, and for a united cooperation in spreading the kingdom of God}”.\footnote{Brosamen, p. 524. June 15, 1949, Okabena Lutheran, p. 6}\end{displayquote}  In his “\textit{Pastorale}”\footnote{Pastorale p. 69}, therefore, Walther calls the following to the attention of the pastors: \begin{displayquote}“\textit{After his ordination a pastor who has entered into the office should at his first opportunity join an orthodox synod.  If he should fail to do so when opportunity offers, he would thereby betray a sinfully separatistic, schismatic spirit, contrary to {\scriptsize\textsc{Ephesians 4:3; 1 Corinthians 1:10-13; 11:18-19; Proverbs 18:1}}}”.\end{displayquote}  And in another place, after he has first rejected the idea that a church-government organization of a number of congregations into a larger church body is of divine right and therefore absolutely necessary, Walther remarks: \begin{displayquote}“\textit{Nevertheless a preacher who, insisting upon his freedom, would with his congregation remain independent, even though an opportunity were offered him to join an orthodox synod, would thereby act contrary to the purpose of his office, the welfare of his congregation, and his duty to the church at large, and would reveal himself as a separatist}”.\footnote{Pastorale, p. 397.}\end{displayquote}  A Pastor should therefore endeavor to prevail upon his congregation, if it is still without synodical connection, to join a synod.  To be sure, a pastor must do this only by way of patient instruction, pointing out the true character of a synod.  Walther writes with reference to this point: \begin{fancyquotes}The pastor is indeed to endeavor to prevail upon his congregation to join the synod, but great caution is to be exercised in this endeavor; the congregation is first to be instructed concerning the significance of a synod, and is to be given time, in order that it may not form the opinion that it merely a matter of leading it with burdens diminishing its freedom, tricking it out of its church property, an subjecting it to the yoke of a so-called spiritual government.  Rather it is to be shown that this is purely a matter of its own welfare and its duty to care for its children and posterity and for the Kingdom of God in general, and finally, that a synod desires to be merely an advisory, auxiliary body, not a body which exercises dominion over the individual congregations.\footnote{Pastorale, p. 400f.}\end{fancyquotes}

                The proof that an ecclesiastical fellowship can very well exist, do the work of the church, and gloriously flourish on the basis of these principles is offered by the example of the Missouri Synod itself.\footnote{Editor would remark-Old Missouri itself, as long as she held to the basic principles, from which she has definitely and violently departed as of this date.}  Walther says in his synodical address of the year 1848: \begin{fancyquotes}One thought, however perhaps agitates the minds of us all, with some in a greater, with others in a lesser degree, and induces the anxiety that our transactions might easily remain fruitless; I mean the thought that our constitution, which forms the basis of our synodical connection, invests us with no other power besides that of deliberation, it confers upon us no authority but that of the Word and persuasion.  Pursuant to our constitution we have no right to issue decrees, to enact laws or orders, or in any way to deliver a judicial decision in matters imposing any duty upon the congregation, so that they should be forced absolutely to submit.  Our constitution by no means constitutes us a kind of consistory, or highest tribunal of our congregations. \par  On the contrary, it assures to them the most perfect liberty, nothing excepted save the Word of God, faith, and love.  According to our constitution we do not occupy a position above our congregations, but we stand among them and by their side.  Does it not seem, then, as though it were made utterly impossible for us to exercise a thorough-going, salutary influence upon our congregations?  Do we not, by reason of the relations entered into, run the hazard of wearying ourselves with labors that might but too easily prove utterly fruitless, as none are compelled to comply with our resolutions?\end{fancyquotes}  Walther answers this question with \textbf{No!} and then treats the question: \begin{displayquote}``\textit{Why shall and can we pursue our work joyfully, although we are possessed of no power but that of the Word}?''\end{displayquote}  He shows that Christ has given His servants no other power than the power of the Word, but that his power is also fully sufficient for the building of the Church.  Says Walther: \begin{fancyquotes}When a preacher is invested only with the power of the Word, but with its full power, and where the congregation receives his word as God’s Word, whenever he delivers to them the Word of Christ, just there the minister stands in the proper relation to his congregation; he performs his duties not as a hired mercenary, but as an ambassador of the Most High; not as a servant of men, but of Christ, teaching, exhorting, and rebuking in the place of Christ.  Just there the apostolic exhortation is obeyed: \begin{displayquote}‘\textit{Obey them that have the rule over you and submit yourselves}’\end{displayquote}  But the more a congregation perceives that the person who is over them in the Lord desires nothing but that the congregation exercise subjection to Christ and His Word; the more it sees that he does not desire to lord it over them, nay, that he even watches over the liberty of the congregation with a jealous eye, the more willing will it become to listen to his salutary proposals, even in matters left free by God... \par This very same expectation of salutary influence our synodical body may entertain, if it seeks to effect its objects by means of no authority save that of the Word.  Of course we shall have to encounter struggles and contests, but it will not be those little discouraging struggles for obedience to human ordinances, but high and holy struggles for the Word of God, and hence for the honor and kingdom of God.  And the more our congregations learn to know that we desire to exercise no authority over them but the divine power of the Word which saves all who believe in it, the more readily will our advice and counsel find an open door with them.  It is true, all that do not like the Word will separate themselves from us; but to those who love it our communion will be comfortable refuge; and in sanctioning resolutions they will not bear them as a strange burden imposed upon them by outward force, but will regard them as a blessing and a gift of brotherly love; they will adopt, defend, and keep them as their own proper possession.\footnote{Brosamen, p. 518-527.  Translated in June 15, 1949, Okabena Lutheran, pp. 2,7. A history of nearly fifty years confirms these words of Walther (1847-1890.}\end{fancyquotes}