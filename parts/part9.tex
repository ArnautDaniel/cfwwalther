\chapter{Church and State}

\hrule
\vspace{.30cm}
With regard to the relation of the Church to the State Walther teaches that the Church should be independent of the State, that is, that it should govern itself in all respects. ``\textit{As important as it is}”, he writes, in \textit{Die Rechte Gestalt}\footnote{Die Rechte Gestalt, p. 5 f.}, \begin{displayquote}“\textit{that the government of a land in which the orthodox Church has its dwelling should also belong to the Church, as great a blessing as this can be for the Church, nevertheless the separation of the Church from the State is not a defect or an irregularity, but the correct and normal relation which the Church should always bear toward the State}.”\end{displayquote}
\vspace{.30cm}
\hrule
\vspace{1.25cm}
                As proof for his position Walther appeals in the first place to the fact that “\textit{according to God’s Word Church and State are entirely separate domains and hence not to be mingled the one with the other}”.\footnote{John 18:36; 2 Corinthians 10:4; Matthew 22:21; Luke 12:13-14}  The most complete exposition by Walther concerning the utter dissimilarity of Church and State and the consequent separation of Church and State is contained in a synodical address on {\scriptsize\textsc{John 18:36-37}}.  Here Walther says: \begin{fancyquotes}Church and State are, according to God’s Word, as different from each other as heaven and earth.  The State is a kingdom of this world, hence an earthly kingdom; the Church, however, is ‘\textit{not from hence}’, not an earthly kingdom, it is as the Lord so often says, the ‘kingdom of heaven’ upon earth.  The State is an external, physical, visible kingdom, the Church an internal, spiritual, invisible kingdom, for, as Christ says with plain words, \begin{displayquote}\textit{‘the kingdom of God cometh not with observation; neither shall they say, Lo here! Or lo there! For, behold, the kingdom of God is within you’.}\end{displayquote}  The State has as its members all who allow themselves to be taken up externally into its association, bad and good, ungodly and pious, unbelievers and believers, non-Christians and Christians; the Church, on the contrary, has only those as members who are Christ’s sheep, who hear His voice and from their hearts believe on Him.  The State has for its purpose only the earthly welfare of men, protection of the body, property, and honor of its citizens, and external quietness, peace, discipline, and order in this world; the Church, on the contrary, has for its purpose the peace of men with God, protection against sin, death, devil, and hell, eternal righteousness, eternal life, and eternal blessedness.  The State has as its norm the light of nature or of human reason; the Church has the light of the immediate divine revelation embodied in the Holy Scripture.  The State has for laws those which it makes itself; the Church gives no laws, but only urges the eternal laws of God.  The State reproves only the outward evil deed; the Church reproves also the ungodly attitude of the heart.  The State allows everything which its earthly purposes demand or at least permit;\footnote{See ``Walther's Note''} the Church allows only what God in His Word declares allowable.  The State commands on its own authority and hence demands obedience to its commands on the basis of its official power; the Church commands nothing on its own authority and demands obedience only to the commands of Christ.

                The State has as its means and weapons, the bodily sword and external power of compulsion; the Church has the sword of the Spirit, namely the Word of God and the power of conviction through this Word.  The State has as its component parts government and subjects, those who command and those who obey; in the Church all are equal and subject one to another by love alone; even as Christ says in plain words to His disciples: \begin{displayquote}\textit{‘One is your Master, even Christ; and ye all are brethren.  It shall not be so among you: but whosoever will be great among you, let him be your minister’.}\footnote{Brosamen, p. 498.}\end{displayquote}\end{fancyquotes}\footnote{Walther's Note --- Here Walther remarks in a note: \begin{displayquote}\textit{Moses in his political laws had to allow divorce even aside from the case of adultery {\scriptsize\textsc{(Deuteronomy 24:1)}} because of the hardness of heart of the Jews {\scriptsize\textsc{(Matthew 19:7-9)}}; but the prophets reproved the use of this license by those who wanted to be members of the Church, according to {\scriptsize\textsc{Malachi 2:14-16}}.\end{displayquote}} We add a weighty utterance of Walther concerning this point from the Report of the {\scriptsize\textsc{Western District, 1885, p. 21 [Essays for the Church, Vol. II, pg 273 ]}}\begin{displayquote}  “\textit{Let it be noted that our Church does not teach that the secular government has no right to allow anything, that is to declare it exempt from punishment, which God has forbidden.  The State has that right, to be sure.  Even Moses, as a political lawgiver, allows much which the prophets forbid.  The government does not have only Christians under it, who are ruled by the Word of God; it is also not to rule the State, which is no institution for the salvation of souls but for the protection of body and property, according to the Word of God, but in accordance with reason.  But a prohibition of God does not lose its binding through the allowance of the government.  When the government, for instance, licenses sinful amusements, divorces on invalid grounds, the conducting of saloons, a Christian can make no use of this allowance.  The government must allow such things because of the ‘hardness of heart’ of its subjects in order to prevent rebellion, murder, and manslaughter.  Hence when the Pharisees, to adorn their false doctrine of divorce, submitted to Christ the question:\begin{displayquote}‘Why did Moses then command to give a writing of divorcement, and to put her away?’\end{displayquote} Christ answered:\begin{displayquote} ‘Moses because of the hardness of your hearts suffered you to put away your wives: but from the beginning it was not so}’.  {\scriptsize\textsc{(Matthew 19:7-8)}}\end{displayquote}\end{displayquote}}}Now, because Church and State according to God’s Word are so fundamentally different-- ``\textit{their entire character and nature are different, different are the requirements of their members, different their aim, norm, rule, their commands and prohibitions, their freedoms, their power, their means, the mutual relationships of those who belong to them, in short, their entire quality}” -– therefore the Church can neither be governed according to civil principles nor the State by ecclesiastical principles, that is to say, State and Church must remain unmixed, or, the Church shall be independent of the State.

                Walther further sets up the following statements concerning the relation of Church and State: \begin{displayquote}``\textit{Government officials, if they are believers, are also in the Church, but not as officials with their laws and their external authority, but as Christians and brothers, and hence equal in power and privilege with all other church members, even if they are princes, kings, or emperors}.''\footnote{Matthew 23:8; Luke 22:25-26; Galatians 3:28}\footnote{Die Rechte Gestalt, p. 8; Brosamen, p. 500. Walther remarks on the last passage in a note: “\textit{Even in the middle of the Fourth Century the ancient teacher of the church Optatus of Mileve wrote: ‘The State is not in the Church, but the Church is in the State’}.”}\end{displayquote}

                The civil government has indeed the duty over against the Church to guard it in its freedoms and rights against all outward force, to afford the Church as a society in the State the same protection which all other societies in the State enjoy.  In this way the civil government in our land fulfills its duty toward the Church.  “\textit{Our civil government here}” – says Walther\footnote{Brosamen, p. 507} – \begin{displayquote}“\textit{is indeed, as Isaiah prophesied, a nursing father and nursing mother also of our Church, for it powerfully protects us here in accordance with its office against all outward force, against the bloodthirstiness of the Antichrist and his minions, as against the murderous desires of the atheists of this last age of apostasy}”.\end{displayquote}  And persons in civil authority have this obligation in double measure when they are themselves members of the Church, as indeed every Christian should place his gifts into the service of Christ and His kingdom.\footnote{Report of the Western District, 1885, p. 28.} For as the rich man serves the Church with his riches, and the artist with his art, so should also persons in civil authority, if they are Christians, serve the Church with their power and reputation.\footnote{Pastorale, p. 368. Western District, p. 27.}  This is also the meaning of the Smalcald Articles when they say: \begin{displayquote}“\textit{Especially the chief members of the Church, kings and princes, ought to guard the interests of the Church, and to see to it that errors be removed and consciences be rightly instructed}”.\footnote{Mueller, p. 339; Triglotta, p. 519}\end{displayquote} --which words, as the word “\textit{especially}” already shows, speaks of a general duty of Christians\footnote{Western District, p. 29}, and ascribe to the princes not so much rights and powers over the Church, but rather duties toward it, and so much the greater as their station in life has more opportunity than others to lend a helping hand to the Church.\footnote{Rechte Gestalt, p. 8.}  For the protection which the government has to afford also to the Church is not to be extended or rather perverted as though the secular government had also the right to rule the Church.  Walther says: \begin{fancyquotes}The secular government has neither the right nor power to usurp rulership over the Church nor to attempt by compulsion to force upon men the true faith, or what it holds to be the true faith.

                Christ not only declares Himself to be the one who alone has authority in His Church and exercises it through His Word, but He also denies to all others any authority whatever in His Church.\footnote{Matthew 23:8. \& Brosamen, p. 520.}\end{fancyquotes} \begin{displayquote}“\textit{The dogmaticians of the Seventeenth Century have here departed from Scripture and the Confessions in favor of the State Church and call it \textbf{Gallionism} when one denies to the secular government as such the right to judge \textit{ex officio} concerning true and false doctrine}”,\end{displayquote} whereas -- \begin{displayquote}“\textit{the Holy Spirit has undoubtedly had this history {\scriptsize\textsc{(of Gallio, Acts 18:12-16)}} recorded, among other reasons, for the very purpose of letting us know that the secular government as such can pronounce no judgment in matter of doctrine}.\end{displayquote}  After Walther has expounded Baier’s doctrine of the power of secular government in the Church, he continues: \begin{fancyquotes}Secular and church government can hardly be worse confounded and confused with each other than our dear Baier dies here, contrary to the clear testimony of our Church in its basic Confession.  What applies only to the Church of the Old Testament which according to God’s will was to be bound up with the State until Christ’s coming, is here transferred to the Church of the New Testament, and what belongs to a David, a Josiah, etc., is here without distinction attributed to all princes and supreme secular authorities, and so a manifest \textbf{Caesaropapism} is established!  May God have mercy!\footnote{Western District, pp. 30-37.}\end{fancyquotes}

                In particular shall the secular government not attempt with outward force to compel men unto the right faith.  This is contrary to God’s will {\scriptsize\textsc{(John 18:36-37)}}; even the Jews in the Old Testament were to compel no one to adopt their religion; a war which is waged for the propagation of religion cannot please God.  \begin{displayquote}“\textit{Only when waged for the protection of the persons, of the confessors of a religion against its persecutors, can a religious war under certain circumstances be pleasing to God}”.\end{displayquote}  And as the application of outward forces is against God’s will, so also it works harm to the Church.  The Church in this way either wins hypocrites, since outward force cannot change the soul and make it obedient to the faith, or else it entirely repels the unbelievers.  \begin{displayquote}“\textit{Unbelievers indeed seek to justify their rejection of the Christian religion by pointing to the blood {\scriptsize\textsc{(supposedly)}} shed by the Church.  And they rightly assert that a church which makes use of such methods for its extension and preservation cannot possibly be the true Church}”.\footnote{L.c., p. 31-37.}\end{displayquote}  May then the secular government never employ force against ecclesiastical organizations?  It may do so in only one case.  And that is, when erring ecclesiastical organizations adopt, or at least practice, principles which are dangerous to the State.  So, for instance, the State had plenty of reason to proceed against the Pope as an errorist with principles dangerous to the State.  But apart from this case the secular authority has neither the right nor the authority to put its power of coercion into execution against false faith or false worship, pr what it holds to be such.\footnote{L.c., p. 42 ff.}

                Besides the principle that persons belonging to the government are not as such in the Church\footnote{What is true of the secular government is true of secular estates in general.  Walther expounds the matter as follows: \begin{displayquote}The Church indeed consists of men of various stations in life, but the domestic and civil estates do not as such belong in the Church, but are ordained by God alongside the Church.  The estates are not as such in the Church nor do they possess special rights in the Church.  When we say that the Church consists of people of all stations in life, this must be understood to mean that no station, however secular it may appear to be, can deprive Christians of their spiritual and priestly character and their share in the rights and privileges of the Church.{\scriptsize\textsc{(Rechte Gestalt, p. 11)}}\end{displayquote}}  Dr. Walther places the other principle “\textit{that the members of the Church are obliged to render obedience to the State not as Church but as citizens and subjects}''.\footnote{Die Rechte Gestalte, p. 7.10; Brosamen, p. 500.}  This latter, indeed, Walther very strongly emphasizes.  He says that a subject must obey the civil government, no matter what it may command, if only he is not thereby compelled to act against his conscience.  But our government is that which actually has power over us.  Whether it has come into office legitimately, whether it is godly, whether it is of our faith, are questions which here do not come into consideration.  He who is not subject to the government which has power over him not merely against man but against God Himself, whose ordinance the government is.  It is not just a disturbance of the public peace when one rebels against the secular government, but it is, properly speaking, a warring against the divine Majesty.  We must be subject for God’s sake and conscience’s sake.\par  Hence we must honor the powers that be not only with outward demonstrations of respect but in our hearts.  And everyone, specifically also every Christian and every preacher, is bound as a citizen to be subject to the secular government; it is anti-christian when popes and priests do not want to be subject to secular jurisdiction.\footnote{Western District, p. 15, 16.}
\divider
                On the other hand, Walther emphasizes just as strongly that Christians as Christians or as members of the Church are subject to no secular authority, but solely to Christ as their only Master who has made His will known to them in Holy Scripture.  If, therefore, the government commands something which God has forbidden or forbids something which God has commanded, the Christians must be disobedient to the government, which in this case has become guilty of a shameful usurpation, in order to remain obedient to God and keep their conscience undefiled.\footnote{Western Dist. P. 21.}  In all spiritual matters a Christian may not submit to be commanded by any man, also not by the secular government, because in the conscience of a Christian God alone rules through His Word. \begin{displayquote}``\textit{Also {\scriptsize\textsc{(the)}} we Lutherans annually celebrate the so-called National Day of Thanksgiving which our governors and presidents recommend should be celebrated; but we would not do this if they should ever by virtue of their office command it}”.\footnote{L.c. p. 32.}\end{displayquote} In the previously mentioned synodical address Walther says: \begin{fancyquotes}The secular authorities indeed rule over the members of the Church, but not in so far as they, being Christians, belong to the Church, but in so far as they, being men, belong to the State; hence also the State does not rule over the Church itself and over the conscience, faith, and worship of the Christians, but only over their mortal body and earthly goods.  Hence Christ declares: \begin{displayquote}‘ \textit{Render unto Caesar the things which are Caesar’s, and unto God the things that are God’s}',\end{displayquote} -- and thereby draws for all times a strict boundary and dividing-line between the kingdom of God and that of Caesar, between Church and State.\end{fancyquotes}

                This doctrine may indeed seem to contradict more than a thousand years of the Church’s history.  Not from history, however, but from God’s clear Word is to be learned that which is right with regard to the Church.  Even the Lutheran Church itself has been from the beginning till the present day, especially in the land of its origin, connected with the State, or a State-Church.  But this was only the consequence, in part of deplorable circumstances at the beginning, in part of the heedlessness of the appointed watchmen, but in no way a fruit of the doctrine of Luther and the Church which bears his name, the Evangelical Lutheran Church.\footnote{Brosamen, p. 500-503.} {\color{black}\par Moreover even history itself raises its voice loudly against the coupling of the Church with the State.  For great as was the blessing which true Lutheran princes brought to the Church by conducting the office of the territorial episcopate which had devolved upon them purely for the best interests of the Church, at peril of the loss of land and people, yea, to the endangering of their liberty and live, yet far greater has been the calamity which has come upon the Church through the unhappy mixing of Church and State.

                Walther shows in a portrayal as lively as it is historically true how the Church has been pressed almost to death in the arms of the State.  He says:} \begin{fancyquotes}The first consequence {\scriptsize\textsc{(of the unhappy mixing of Church and State)}} was that the Christian congregations lost all the rights and privileges so dearly won for them by Christ, so that hardly any of them remained.  Their right to call, install, and depose their own teachers and preachers, their right to determine ecclesiastical ceremonies and ordinances and to decide all matters of indifference in the Church, and again to abolish, alter, increase or diminish them, their right to exercise Church discipline upon all members in matters of doctrine and life, -- all these rights were almost wholly lost in the State-Church. \par If the territorial lord was worldly-minded he hindered through his like-minded officials all wholesome church discipline, forced the ministers of the Church to give that which was holy to the dogs and cast their pearls before swine, to solemnize marriages in conflict with God’s Word, to accept ungodly persons as sponsors in Baptism, to bury with Christian honors those who had lived as despisers of Word and Sacraments, and the like.  But if the territorial lord fell away from the true religion also outwardly, then he used his power as territorial bishop and prince to draw his people after him in his apostasy; for now he would depose and banish the faithful teachers in church and school and forced upon the congregations in their place belly-serving and fanatical errorists, eliminated pure books for church and school and introduced corrupted books instead.  The farther they traveled along this road the more fully they lost not only the correct practice but with it also the correct doctrine and knowledge, namely, that whatever power the territorial lord might have in the church, it flowed not from divine right either ecclesiastical or civil, but, if at all, only from human and hence at any time retrievable right.\par  Finally it went so far that the principle was enunciated: \begin{displayquote}‘\textit{To whomsoever belongs dominion over a land, his is also the religion of the land}’\end{displayquote} --so that now people began to regard the Church as properly a State institution, its ministers as State officials, and all subjects of the State as members of the State Church--.  What corruption in doctrine and life in this way intruded into the church and what distress of conscience was thereby inflicted upon sincere ministers of the Church and godly laymen cannot be expressed in words.  Here and there even the right to escape the tyranny over conscience by emigration was denied to the oppressed.  What has finally then become of the State-Churches? -- Citadels they are in which the enemies of the Church hold sway, from whose bastions the snow-white banner of the pure confession has been torn down and in its place the varicolored banners of heterodoxy, syncretism, and manifest unbelief now flutter in the breeze.\footnote{Brosamen, p. 503, 504.}\end{fancyquotes}

                Hence Walther summons us to acknowledge it as a great blessing of God that the Lutheran Church here in America is completely independent of the State and enjoys the freedom given her by Christ.